\chapter{Текстовый редактор VI}

Редакторы для Unix делятся на 2 группы - редакторы командного стиля(vi, emacs, joe, ed) и меню-ориентированные (mcedit, kwriter, kword).

Редакторы командного стиля обычно работают в \emph{консольном (текстовом)} режиме. Все действия в них выполняются подачей прямых управляющих команд, закрепленных за определенными сочетаниями клавиш. Мышь и меню в них, как правило, не используются.

Редактор \verb+vi+ присутствует как стандартный в любой Unix-подобной системе\footnote{В этом качестве он внесен в стандарт Single Unix Specification}. Существует несколько редакторов основанных на vi: vim, elvis. 

Современные клоны vi (vim к примеру) обладают очень большой функциональностью, скрытой за аскетичным интерфейсом. Редактор vi изначально создавался как кросс-платформенный, который обязан работать на любых типах терминалов и виртуальных консолей. Все действия в нем можно осуществить не покидая основной, алфавитно-цифровой, части клавиатуры.

\emph{Примечание}. Далее мы будем рассматривать редактор vim. Однако все описанные команды можно будет применить в любом vi-совместимом редакторе.

\section{Режимы работы}

В vi существует три принципиально различных режима работы:
\begin{itemize}
\item [-] Командный режим (command mode)
\item [-] Режим ввода (edit mode)
\item [-] Режим построчного редактирования (ex mode)
\end{itemize}

\emph{Командный режим} включается по умолчанию при запуске vi. В этом режиме нажатия клавиш \emph{не приводят} к вводу символов, а интерпретируются как внутренние команды перемещения по тексту и редактирования. Поэтому попытка немедленно начать ввод текста (как в DOS/Windows) ни к чему не приведет.

\emph{Примечание}. Если вы не знаете, в каком режиме находитесь, то нажмите клавишу \verb+ESC+ для перехода в командный режим.

Создание текста в командном режиме невозможно. Для этого нужно перейти в \emph{режим ввода}. Для этого служат команды (командного режима!) \verb+a+ (от append - после текущей позиции курсора)  и \verb+i+ (от insert - перед текущей позицией курсора). В режиме ввода нажатия клавиш приводят к вводу обычных символов, позволяя создавать новый текст или редактировать существующий.

Возврат в командный режим осуществляется нажатием клавиши \verb+escape+.

Для операций с документами (файлами) предназначен \emph{ex-режим}. Он вызывается командой \verb+:+ командного режима. После этого дается команда ex-режима. Например:
\begin{itemize}
\item открыть существующий файл (\verb+:e имя_файла+) 
\item вставить файл в позицию курсора (\verb+:r имя_файла+)
\item записать файл (\verb+:w+), в том числе под другим именем (\verb+:w имя_файла+)
\item выход из сохраненного файла (\verb+:q+)\label{viexit}
\item выход с предварительным сохранением файла (\verb+:x+)
\end{itemize}

Когда вы находитесь в ex-режиме, то в нижнем левом углу экрана появляется \verb+:+.

Возможно совмещение команд ex-режима. Например \verb+:wq+. 

Команда ex-режима отправляется на выполнение нажатием клавиши \verb+Enter+ после чего происходит возврат в командный режим.

\emph{Примечание}. Попытка загрузить новый файл (командой \verb+:e+) или завершить работу редактора (командой \verb+:q+) при несохраненном старом файле вызовет ошибку.

\section{Получение помощи}

Получить справку можно используя ex-команду \verb+:help+.

Подразделы справки выделены значками \verb+|раздел|+. Справку по ним вызывается через \verb+:help раздел+.

Очень полезным является учебник по vim - \verb+vimtutor+. С его помощью можно освоить основные навыки использования vim.

\section{Запуск и остановка редактора}

\verb+Vi (vim)+ может быть запущен из командной строки, с именем файла или без такового. Если указано имя файла, то редактор открывает его\footnote{Если файл не существует, то создается новый}.

\verb+Пример: $ vi ~/texts/newtext.txt+

Команда vim без имени файла откроет редактор vim и выведет заставку.

Для выхода из редактора нажмите \verb+:q+ или \verb+:wq+ (см. "Режимы работы" \ref{viexit}). 


\section{Перемещение по тексту}

В командном режиме существуют следующие команды\footnote{Обычно работают и стрелки на клавиатуре, но не стоит полагаться на них}:
\begin{itemize}
\item \verb+h+ - курсор влево на 1 символ
\item \verb+l+ - курсор вправо на 1 символ
\item \verb+j+ - курсор вниз на 1 строку
\item \verb+l+ - курсор вверх на 1 строку
\end{itemize}

Также, есть расширенные команды, действующие с блоками текста.
\begin{itemize}
\item \verb+w и W+ - перемещение вперед на "маленькое слово"\footnote{Отдельное слово, отделенное пробелом, знаками препинания, +, -} и т.н. "большое слово"\footnote{Обязательно отделенное пробелом} 
\item \verb+b и B+ - перемещение назад на "маленькое слово" и "большое слово" 
\item \verb+0 и $+ - на начало и на конец строки
\item \verb+( и )+ - на начало предложения и его конец 
\end{itemize}


Вообще, для многих команд vi характерно наличие парных элементов - в нижнем и верхнем регистрах одной клавиши (\verb+e+ и \verb+E+, \verb+w+ и \verb+W+); действие второй команды из пары как бы расширяет действие первой.

Команды навигации vi могут использоваться с численными аргументами.

Например команда \verb+5h+ переместит курсор на 5 символов влево (считая символ в позиции курсора), а команда \verb+3B+ на 3 "больших" слова назад. 

Для перемещения на конкретную строку, можно использовать следующую команду ex-режима: \verb+:N+, где N - номер строки.


\section{Ввод и редактирование текста}

Для создания текста необходимо перейти в режим ввода.

Для этого служат следующие команды:
\begin{itemize}
\item \verb+i и I+ - ввод в позиции курсора или в начале строки
\item \verb+a и A+ - ввод после курсора или в конце строки
\end{itemize}

Текст можно изменять и в режиме ввода, используя \verb+DEL+ и \verb+BACKSPACE+, но часто удобнее использовать команды редактирования.

Команды редактирования предназначены для изменения существующего текста без перехода в режим ввода:
\begin{itemize}
\item \verb+x+ - удаление одиночного символа
\item \verb+dd+ - удаление строки
\item \verb+dw+ - удаление слова
\item \verb+d)+ - удаление предложения
\end{itemize}

Как и команды перемещения, команды редактирования можно использовать с численными аргументами. Так команда \verb+5dd+ удалит текущую строку и еще 4 строки ниже ее, а \verb+3dw+ удалит три слова считая текущее.

\section{Копирование и вставка}

В vim для этих целей существует отдельный режим - выделения, Visual Selection (см. \verb+man vim+). Однако в большинстве случаев мы можем обойтись стандартными командами:
\begin{itemize}
\item \verb+p+ - вставить из буфера.
\item \verb+yy+ - скопировать строку в буфер
\item \verb+yw+ - скопировать текущее слово в буфер
\item \verb+y)+ - скопировать предложение
\item \verb+y}+ - скопировать абзац
\end{itemize}

\emph{Примечание}.В буфер также попадает все удаленное с помощью команд x, dd, dw и им подобных. Таким образом эти команды могут служить для копирования с удалением.

Все вышеперечисленные команды могут использоваться с численным аргументом. Например: \verb+3p+ - 3 раза вставить содержимое буфера.

\section{Откат действий}

Действие ошибочно введенных команд редактирования может быть отменено командой \verb+u+ (сокращенно от undo). Повторное нажатие - отмена предыдущего действия, и так далее. Для возврата (redo) ошибочно отмененной операции используется \verb-control+r-.

\section{Поиск и замена}

Для поиска по тексту служит команда \verb+/+ (прямой слэш). При вводе этого символа в командном режиме в нижней строке появляется символ \verb+/+, после которого вы можете ввести образец для поиска. Это может быть текстовая строка или \emph{регулярное выражение} (см. \ref{regexp} ). После нажатия \verb+ENTER+ в тексте будут подсвечены \footnote{Это справедливо только для vim} все возможные вхождения строки поиска и курсор перейдет к первому доступному найденому фрагменту вниз по тексту.

Для поиска следующих вхождений строки поиска, существует команды \verb+n+ (вниз по тексту) и \verb+N+ (вверх по тексту).

Для поиска и замены текстовых фрагментов, в том числе и с использованием регулярных выражений, предназначена команда ex-режима \verb+:s+(substitute). Формат команды: \newline
\verb+:#s/pattern/string/опция+\newline
где \verb+#+ - интервал строк (через \verb+,+ или \verb+;+ - см. \verb+:help cmdline-ranges+).

\emph{Примечание}. опции в стандартном vi не поддерживаются.

Часто употребимые опции: \verb+c+ - подтверждение каждой замены, \verb+g+ - замена всех вхождений в строке.

\emph{Примечание}. Поиск и замена в vi возможны только для последовательности символов, составляющих 1 строку. Заменяющая последовательность символов тоже должна образовывать 1 строку.
 

\section{Вызов внешних команд}

Редактор vim часто назывют средой, так как он позволяет полноценнно работать в системе, не выходя из редактора.

Из vi можно запускать внешние программы с помощью команды ex-режима \verb+:! cmdlline+:
\begin{verbatim}
Пример
:! ls -l

:! man bash
\end{verbatim}

Так работать гораздо удобнее, так как нет надобности постоянно входить и выходить из редактора. При запуске командной строки ее вывод будет сохранен на экране до нажатия пользователем клавиши \verb+ENTER+.

