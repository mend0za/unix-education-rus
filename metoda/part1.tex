\chapter{Общие сведения о ОС UNIX}

\section{Вместо предисловия}

Это пособие - почти 100\% плагиат. Многие главы почти не содержат авторского текста. Это связано с тем, что автор трезво оценивает свои знания и навыки и считает, что гораздо полезнее собрать весь этот материал в одно издание из разрозненных источников, чем заново писать его. Тем более что самописная версия пособия будет заведомо хуже.

Это учебное пособие первоначально писалось для студентов 3 курса специальности информатика. Курс Операционные Системы и Среды (далее ОСиС) базируется на материале курсов ОАиП и КПиЯП и требует знаний языка Си и основ алгоритмизации.

Основная задача курса ОСиС - подготовка специалистов, овладевших пользовательским интерфейсом, архитектурой и программированием Unix-подобных систем (факультативно - элементы администрирования). 

  
\section{Введение}

Что такое Unix? Это семейство операционных систем (ОС), обладающих сходной архитектурой и интерфесом с пользователем. 

Unix как явление зародилось в начале 70-х годов и развивается до сих пор.

Основные современные варианты UNIX: Linux, BSD (FreeBSD, NetBSD, OpenBSD), AIX, HPUX, Solaris, SCO. 

Важнейшие современные стандарты, обеспечивающие целостность семейства UNIX:
\begin{itemize}
\item [-] POSIX - Portable Operating System Interface
\item [-] ANSI C (c89 и с99) 
\item [-] Opengroup\footnote{Opengroup - организация, занятая выработкой единых стандартов на Unix-системы (www.opengroup.org). Владелец торговой марки Unix.} Single Unix Specification Version 3 (далее SUSv3). 
\end{itemize}
Примечание. Далее под словом Unix мы будем подразумевать все Unix-подобные операционные системы, если не названа конкретная система.

\subsection{Основные черты}

\begin{itemize}
\item  Код на Си - позволяет переносить и изменять ОС.
\item  Многозадачная многопользовательская ОС.
\item  Наличие стандартов - основой всего семейства являются одинаковая структура и ряд стандартных интерфейсов.
\item  Простой, но мощный пользовательский интерфейс (командная строка).
\item  Eдиная древовидная файловая система. Через интерфейс файловой системы осуществляется доступ к данным, терминалам, принтерам, дискам, сети и даже к оперативной памяти.
\item  Большое количество программного обеспечения.
\end{itemize}

\subsection{Структура системы}

Классическая архитектура UNIX двухуровневая:
\begin{enumerate}
\item  Ядро - управляет ресурсами компьютера и предлагает программам базовый набор услуг (системные вызовы).
\item  Системные программы (управление сетью, терминалами, печатью), прикладные программы (редакторы, утилиты, компиляторы и т.д.). 
\end{enumerate}

\subsubsection{Функции ядра}

\begin{itemize}
\item  \emph{инициализация системы} - загрузка и запуск ОС
\item  \emph{управление процессами и потоками} 
\item  \emph{управление памятью} - отображение адресного пространства на физическую память, совместное использование памяти процессами
\item  \emph{управление файлами} - реализует понятие файловой системы, дерева каталогов и файлов 
\item  \emph{обмен данными между процессами} 
	\begin{enumerate}
	\item  выполняющимися внутри одного компьютера
 	\item  в разных узлах сетей передачи данных
 	\item  а также между процессами и драйверами внешних устройств
	\end{enumerate}
\item  \emph{программный интерфейс (API)} - обеспечивает доступ к возможностям ядра со стороны процессов пользователя через системные вызовы, оформленных в виде библиотеки функций на Си.
\end{itemize}

\subsubsection{Системные вызовы}

Ядро изолирует программы пользователя от аппаратуры. Все части системы, не считая небольшой части ядра, полностью независимы от архитектуры компьютера и написаны на Си.

\emph{Системные вызовы} - это уровень, скрывающий особенности конкретного механизма выполнения на уровне аппаратуры от программ пользователя. Для программиста, системный вызов - это функция (определенная на Си), которую он вызывает в своей программе. Все низкоуровневые операции осуществляются через системные вызовы.

\subsubsection{Подсистемы ядра}

\begin{enumerate}
\item  Файловая подсистема. Обеспечивает унифицированный доступ к файлам:
	\begin{itemize}
	\item контроль прав доступа к файлу;
	\item чтение/запись файла;
	\item размещение и удаление файла;
	\item перенаправление запросов  к периферийным устройствам, соответствующим модулям подсистемы ввода/вывода.
	\end{itemize}
\item  Подсистема управления процессами.
 
	Запущенная на выполнение программа порождает один или несколько процессов (задач). \\
Подсистема контролирует:
	\begin{itemize}
	\item создание и удаление процессов
	\item распределение системных ресурсов (памяти, вычислительных ресурсов) между процессами
	\item синхронизация процессов
	\item межпроцессное взаимодействие
	\end{itemize}

  Специальная задача ядра  \emph{планировщик процессов} разрешает конфликты процессов в конкуренции за ресурсы. 
\item Подсистема ввода/вывода. Выполняет запросы файловой системы и подсистемы управления процессами для доступа к периферийным устройствам (дискам, лентам, терминалам). Обеспечивает буферизацию данных и взаимодействует с драйверами устройств.
\end{enumerate}

\subsection{ОС UNIX для пользователя}
\subsubsection{Пользователь}

С самого начала ОС UNIX разрабатывалась как интерактивная система. Чтобы начать работу, нужно войти в систему, введя со свободного терминала (консоли) свое \emph{имя} (account name, логин) и \emph{пароль} (password).

Человек, зарегистрированный в системе и, следовательно, имеющий учетное имя, называется \emph{зарегистрированным пользователем системы}.

Регистрацию новых пользователей обычно выполняет администратор системы. Пользователь не может изменить свое учетное имя, но может установить или изменить пароль. Пароли находятся в отдельном файле в закодированном виде.

Все пользователи так или иначе работают с файлами. Файловая система имеет древовидную структуру. У каждого зарегистрированного пользователя есть \emph{домашний каталог}. К нему он имеет полный доступ. К другим каталогам доступ обычно ограничен.

\subsubsection{Интерфейс пользователя}

Традиционный интерфейс - \emph{командная строка}. После входа в систему для пользователя осуществляется запуск одной из \emph{командных оболочек}. Общее название \emph{shell (eng. - оболочка)}, так как они являются внешним окружением ядра системы. Оболочка - это интерпретатор комманд (встроенных и внешних) и  обладает мощным встроенным языком shell scripts, позволяющим писать сложные программы. 

\subsubsection{GUI и Unix}

Графический интерфейс (GUI) не является необходимым в Unix и рассматриваться не будет. Подавляющее большинство обычных операций выполняется без его участия.
