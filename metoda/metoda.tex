% some formating
\documentclass[11pt,a4paper,twoside]{book}

\usepackage[colorlinks,hyperindex,plainpages=false,pdfencoding=auto,unicode]{hyperref}
\usepackage{keyval}
\hypersetup{pdfauthor={Vlad Shakhov}}
\def\pdfBorderAttrs{/Border [0 0 0] } % No border arround Links
\makeatletter
\newcounter{chapter.H}
\setcounter{chapter.H}{0}
\makeatother

\usepackage{cmap}
\usepackage[utf8]{inputenc}
\usepackage[T2A]{fontenc}
\usepackage[russian,english]{babel}
\usepackage{amssymb}
\pagenumbering{arabic}

\usepackage{polyglossia}   %% загружает пакет многоязыковой вёрстки
\setdefaultlanguage[spelling=modern]{russian}  %% устанавливает главный язык документа
\setotherlanguage{english} %% объявляет второй язык документа

\setmainfont{CMU Serif}           %% задаёт основной шрифт документа
\setsansfont{CMU Sans Serif}      %% задаёт шрифт без засечек
\setmonofont{CMU Typewriter Text} %% задаёт моноширинный шрифт

\title{Учебное пособие по курсу Операционные Системы и Среды (ОСиС)}
\author{Влад 'mend0za' Шахов}
\date{последнее обновление - \today}

\begin{document}

\frontmatter

\maketitle

Copyright \copyright Владимир 'mend0za' Шахов, 2002-2007
\bigskip

Каждый имеет право воспроизводить, распространять и/или вносить изменения в настоящий Документ в соответствии с условиями GNU Free Documentation License, Версией 1.2 или любой более поздней версией, опубликованной Free Software Foundation;

данный документ содержит следующий Текст, помещаемый на первой странице обложки "Учебное пособие по курсу ОСиС";

данный документ не содержит неизменяемых секций;

Копия настоящей Лицензии включена в раздел под названием "GNU Free Documentation License".

\bigskip

\tableofcontents

\mainmatter

%введение
\chapter{����� �������� � �� UNIX}

\section{������ �����������}

��� ������� - ����� 100\% �������. ������ ����� ����� �� �������� ���������� ������. ��� ������� � ���, ��� ����� ������ ��������� ���� ������ � ������ � �������, ��� ������� �������� ������� ���� ���� �������� � ���� ������� �� ������������ ����������, ��� ������ ������ ���. ��� ����� ��� ���������� ������ ������� ����� �������� ����.

��� ������� ������� ������������� ��� ��������� 3 ����� ������������� �����������. ���� ������������ ������� � ����� (����� ����) ���������� �� ��������� ������ ���� � ����� � ������� ������ ����� �� � ����� ��������������.

�������� ������ ����� ���� - ���������� ������������, ���������� ���������������� �����������, ������������ � ����������������� Unix-������ (������������� - �������� �����������������). 

  
\section{��������}

��� ����� Unix? ��� ��������� ������������ ������ (��), ���������� ������� ������������ � ���������� � �������������. 

Unix ��� ������� ���������� � ������ 70-� ����� � ����������� �� ��� ���.

�������� ����������� �������� UNIX: Linux, BSD (FreeBSD, NetBSD, OpenBSD), AIX, HPUX, Solaris, SCO. 

��������� ����������� ���������, �������������� ����������� ��������� UNIX:
\begin{itemize}
\item[-] POSIX - Portable Operating System Interface
\item[-] ANSI C (c89 � �99) 
\item[-] Opengroup\footnote{Opengroup - �����������, ������� ���������� ������ ���������� �� Unix-������� (www.opengroup.org). �������� �������� ����� Unix.} Single Unix Specification Version 3 (����� SUSv3). 
\end{itemize}
����������. ����� ��� ������ Unix �� ����� ������������� ��� Unix-�������� ������������ �������, ���� �� ������� ���������� �������.

\subsection{�������� �����}

\begin{itemize}
\item ��� �� �� - ��������� ���������� � �������� ��.
\item ������������� ��������������������� ��.
\item ������� ���������� - ������� ����� ��������� �������� ���������� ��������� � ��� ����������� �����������.
\item �������, �� ������ ���������������� ��������� (��������� ������).
\item E����� ����������� �������� �������. ����� ��������� �������� ������� �������������� ������ � ������, ����������, ���������, ������, ���� � ���� � ����������� ������.
\item ������� ���������� ������������ �����������.
\end{itemize}

\subsection{��������� �������}

������������ ����������� UNIX �������������:
\begin{enumerate}
\item ���� - ��������� ��������� ���������� � ���������� ���������� ������� ����� ����� (��������� ������).
\item ��������� ��������� (���������� �����, �����������, �������), ���������� ��������� (���������, �������, ����������� � �.�.). 
\end{enumerate}

\subsubsection{������� ����}

\begin{itemize}
\item \emph{������������� �������} - �������� � ������ ��
\item \emph{���������� ���������� � ��������} 
\item \emph{���������� �������} - ����������� ��������� ������������ �� ���������� ������, ���������� ������������� ������ ����������
\item \emph{���������� �������} - ��������� ������� �������� �������, ������ ��������� � ������ 
\item \emph{����� ������� ����� ����������} 
	\begin{enumerate}
	\item �������������� ������ ������ ����������
 	\item � ������ ����� ����� �������� ������
 	\item � ����� ����� ���������� � ���������� ������� ���������
	\end{enumerate}
\item \emph{����������� ��������� (API)} - ������������ ������ � ������������ ���� �� ������� ��������� ������������ ����� ��������� ������, ����������� � ���� ���������� ������� �� ��.
\end{itemize}

\subsubsection{��������� ������}

���� ��������� ��������� ������������ �� ����������. ��� ����� �������, �� ������ ��������� ����� ����, ��������� ���������� �� ����������� ���������� � �������� �� ��.

\emph{��������� ������} - ��� �������, ���������� ����������� ����������� ��������� ���������� �� ������ ���������� �� �������� ������������. ��� ������������, ��������� ����� - ��� ������� (������������ �� ��), ������� �� �������� � ����� ���������. ��� �������������� �������� �������������� ����� ��������� ������.

\subsubsection{���������� ����}

\begin{enumerate}
\item�������� ����������. ������������ ��������������� ������ � ������:
	\begin{itemize}
	\item�������� ���� ������� � �����;
	\item������/������ �����;
	\item���������� � �������� �����;
	\item��������������� ��������  � ������������ �����������, ��������������� ������� ���������� �����/������.
	\end{itemize}
\item���������� ���������� ����������.
 
	���������� �� ���������� ��������� ��������� ���� ��� ��������� ��������� (�����). \\
���������� ������������:
	\begin{itemize}
	\item�������� � �������� ���������
	\item������������� ��������� �������� (������, �������������� ��������) ����� ����������
	\item������������� ���������
	\item������������� ��������������
	\end{itemize}

  ����������� ������ ����  \emph{����������� ���������} ��������� ��������� ��������� � ����������� �� �������. 
\item���������� �����/������. ��������� ������� �������� ������� � ���������� ���������� ���������� ��� ������� � ������������ ����������� (������, ������, ����������). ������������ ����������� ������ � ��������������� � ���������� ���������.
\end{enumerate}

\subsection{�� UNIX ��� ������������}
\subsubsection{������������}

� ������ ������ �� UNIX ��������������� ��� ������������� �������. ����� ������ ������, ����� ����� � �������, ����� �� ���������� ��������� (�������) ���� \emph{���} (account name, �����) � \emph{������} (password).

�������, ������������������ � ������� �, �������������, ������� ������� ���, ���������� \emph{������������������ ������������� �������}.

����������� ����� ������������� ������ ��������� ������������� �������. ������������ �� ����� �������� ���� ������� ���, �� ����� ���������� ��� �������� ������. ������ ��������� � ��������� ����� � �������������� ����.

��� ������������ ��� ��� ����� �������� � �������. �������� ������� ����� ����������� ���������. � ������� ������������������� ������������ ���� \emph{�������� �������}. � ���� �� ����� ������ ������. � ������ ��������� ������ ������ ���������.

\subsubsection{��������� ������������}

������������ ��������� - \emph{��������� ������}. ����� ����� � ������� ��� ������������ �������������� ������ ����� �� \emph{��������� ��������}. ����� �������� \emph{shell (eng. - ��������)}, ��� ��� ��� �������� ������� ���������� ���� �������. �������� - ��� ������������� ������� (���������� � �������) �  �������� ������ ���������� ������ shell scripts, ����������� ������ ������� ���������. 

\subsubsection{GUI � Unix}

����������� ��������� (GUI) �� �������� ����������� � Unix � ��������������� �� �����. ����������� ����������� ������� �������� ����������� ��� ��� �������.

%командная строка
\chapter{��������� ������}

\section{�������� �������� � �������}

��� ������� �� ��������, �������� ���������� ��������� �������� - ���� � ���������� ������. ��� ����� ������ ������ �.�. \emph{��������� ������}, ���������� ����������� � �����.

� �������� ����������� ��������� ������ ����� ���������� \verb+$+.

����� ����� ����� � ������, ������������ �������� ����������� �� ����. ����� ��������� ������, �� �������, ������� \verb+exit+ ��� \verb+logout+. 

����������� ������ ����� ���������� �������������� ����������. ��� �������� � ������� \emph{���������� � ������}.\\
����� ��� �������:
\verb+������� [-�����] ��������1 ... ��������N+

���������� ������� - \verb+echo+. ��� ������ ������� �� ����� ���� ���������.

\emph{����} - 1 ��� ����� ����, ����� ������� �������� \verb+-+(����� ��� �����). ������ ����� ��������� �������� ������ �������.

\emph{��������} - ������, ������������ �������.
\begin{verbatim}
������ 
$kill -9 1023
kill ��� ������� (���������� ��������)
-9  ����� (����������� ���������� ��������)
1023 ����� �������� (����������� �������) 
$ - ������ ����������� ����� (� ������ �������� - ������)
\end{verbatim}

����� ����� �������, ������� ���������� ����� � ��������� ��������, �������� ���������� \emph{PATH}\label{path}. ���� ������� �������, �� ��� ����� ��������. ������, ���� PATH �� �������� ������� ��������, ����� ������� ������ ���� (��. \ref{fullpath}) � ���������.

\emph{����������.} ������� ������� �� ������ � PATH. ��������� �� ���� ������������ ��������� �������:\\
\verb+./program+
 
\section{���������� �������}

����� Unix ������� �������� �������� �������� �������. ��� ���������� \verb+man+ (�� ����. ����� manual). 

������� ����������� ��������� �������: \verb+man ������+\footnote{����� �������� \emph{������ ������} ��������� ������}. ��� ����������� ������������ ������� \verb+"�����", "����"+. ��� ������ ������� \verb+q+.

\verb+man+ ������� �� ��������� ��������, ������� �������������\footnote{����� ��������� ������ �� �������, ������� ����� ������� ��������������� ������ ����� ����. �������� ������ �������� �� ������� ����� � ������ \cite{Rob} � \cite{Pet}}:
\begin{itemize}
\item 1 -  �������,  �������  �����  ����   �������� �������������
\item 2 -  ��������� ������ (�������, ����������� �����)
\item 3 -  ������������ ������ (������� �� ��������� ��������� ������ ����������������)
\item 5 -  ������� ������ � ����������
\end{itemize}

������ ���� \verb+bash(1)+ ����������, ��� ������� � ������� \verb+bash+ ���������� � ������� 1 (�������). ������ ��� ������� � ��������� �������� ���������� ���������. ����� ��� ��������� ������ ���� ���� ������� ������.

\begin{verbatim}
������� ������������� man
$ man echo
 
��������� ������� �� ����� ������� man
$ man man

������� �� ����������� �������
$ man 2 open

��� ������ � ����� ��������� �� ���� ��������
$ man -a mount
\end{verbatim}

��������� ������� �������� ���� ����������� ������� ������\footnote{� �������� ��� �������� ������ ���������� �������� GNU}, ��� ���������� � ������� ����� \verb+--help+ ��� ������� ��� ����������
\begin{verbatim}
�������
$ file
Usage: file [-bciknvzL] [-f namefile] [-m magicfiles] file...
Usage: file -C [-m magic]

$ split --help
Usage: split [OPTION] [INPUT [PREFIX]]
Output fixed-size pieces of INPUT to PREFIXaa, PREFIXab, ...; default
PREFIX is `x'.  With no INPUT, or when INPUT is -, read standard input.

  -b, --bytes=SIZE        put SIZE bytes per output file
  -C, --line-bytes=SIZE   put at most SIZE bytes of lines per output file
  -l, --lines=NUMBER      put NUMBER lines per output file
  -NUMBER                 same as -l NUMBER
      --verbose           print a diagnostic to standard error just
                            before each output file is opened
      --help              display this help and exit
      --version           output version information and exit

SIZE may have a multiplier suffix: b for 512, k for 1K, m for 1 Meg.

Report bugs to <bug-textutils@gnu.org>.

\end{verbatim}


\section{����������� �� �������� �������} 


���� � Unix - ��� ������ �����. ���������� ������������ "Unix - ��� �����".

��� ����� ����� �������� �� ����� ��������, ������� ����� ������ � ����������. ����� �� ����� ������� �������� - ��� ����� ����� ��������� �� ��� ���\footnote{����� ������� ���� ��� ������� � ������ �� ����� �����}. ������ ������������ ����������� ���������� ����� ������, ����� ��������� �� ���. ������� \emph{�� ��������� ���� ������ � �� ������������}. �������� ���� \verb+example.txt+ ����� ���� ����������� (����������), � ����� � �� ����. � \verb+/bin/sh+ ��� ������� �����������.

������ � ����� ���� ����������� �������� 
\begin{itemize}
\item����� � ������ ����� ��������� �� ��, ��� ���� - �������, �������� \verb+.profile+
\item���� \verb+.+ - ������ �� ������� ������� (��. ������ ������������� �����)
\item���� \verb+..+ - ������ �� ������������ �������
\end{itemize}

���� �� �� ������ ��� �����, ��������� ������� \verb+file+.
\begin{verbatim}
$ file /home/work/OSIS/LAB/RUsak/demon1.c 
/home/work/OSIS/LAB/RUsak/demon1.c: ASCII C program text, with very long lines

$ file /bin/bash
/bin/bash: ELF 32-bit LSB executable, Intel 80386, version 1, dynamically linked
\end{verbatim}

Unix ��������� �������� � ��������� ����� � ������. �������� \verb+name1.txt.125+ � \verb+NAME1.TXT.125+ - ��� ������ �����.

�������� ������� �� Unix �������� ������. ��� ������ ���, ��� ����� ��������� � ������ ����� ���������� ��������� - \emph{������ ���������}. � Unix ����������� ������� "����" � "����� ����������". 

�������� - ����� � ������� ������ ��������� � \verb+/mnt/floppy+, �� ����� ���� � � ����� ������ �����.

������� ������ �������� �������� ������� (������), ������������ \verb+/+ . 

������������ ��������� �������� ������ ����  \verb+/+ .

\emph{����} - ���������������� ������ ���������, ������� ����� ������, ����� ���������� ����� ��� ��������.

���� 2 ���� �����: \emph{����������} � \emph{�������������}.\\
\emph{����������}\label{fullpath} - ������ ���� ������������ ��������� ��������.
\begin{verbatim}
������� ���������� �����:
/etc/init.d/apache
/bin/sh
/usr/local/share
\end{verbatim}
\emph{�������������} - ���� ������������ �������� �������� \label{relativepath}
\begin{verbatim}
������� ������������� �����:
./a.out
laba1/text.cpp
../index.html
\end{verbatim}

\subsection{������� �����������}

\begin{itemize}
\item \verb+cd+ - ����������� ����� ����������. ����� �������������� � 2-� ���������:
\begin{verbatim}
����������� �� ���������� ����
$cd  /usr/local 

��������� � �������� �������
$cd
\end{verbatim} 

\item \verb+pwd+ - �������� ������� �������
\begin{verbatim}
$pwd 
/home/user/OSiS/metoda
\end{verbatim} 

\item \verb+ls+ - �������� ���������� ��������(����� � �����������). ����� ��������� ������ � ������� ������
\begin{verbatim}
��� ���������� - ����� ����������� �������� ��������
$ ls
literatura.aux  Makefile    metoda.dvi  metoda.tex  part1.aux  part2.aux
literatura.tex  metoda.aux  metoda.log  metoda.toc  part1.tex  part2.tex

C ���������� - ����� ����������� ����� ��������
$ ls /usr
bin   doc  games    kerberos  libexec  man                sbin   src  X11R6
dict  etc  include  lib       local    OpenOffice.org1.0  share  tmp

� ������ "-l" - ������ ������ ������ (� ��� �����������)
$ ls -l /home/user/OSiS
����� 710
drwxrwxr-x    2 user     user         1024 ��� 11 14:11 Lectures
-rw-rw-r--    1 user     user       123686 ��� 12 12:53 lectures.rar
-rw-rw-rw-    1 user     user         1789 ��� 11 15:44 lhr10.log
-rw-r--r--    1 user     user            0 ��� 11 15:44 lkypc.pdf
-rw-rw-r--    1 user     user       438112 ��� 20  1999 lshort.dvi
-rw-r--r--    1 user     user       112747 ��� 12 12:53 lshrtdvi.zip
drwxrwxr-x    2 user     user         1024 ��� 14 18:50 metoda
-rw-rw-r--    1 user     user        40960 ��� 11 14:06 metoda.doc
\end{verbatim} 

\end{itemize}

\subsection{����������� ������ ���������(���������, CD-ROM)}

����� ���������� ���������� (��������, CD-ROM, ������ ���������, ������� ����), ����� ������� �����, ���� ����� ������������ ��� ����������. ��� ����� ���������� \emph{����� ������������}. ����� ������������ - ��� ������� �������. ����� ����������� (� �������� Unix - \emph{������������}) ������� ����� ��������� �����, ������������� �� ����������. ������ ������������ ��������� � ������������ - ������� \verb+mount(8)+.

������ ����������� ���������� ������������ �� ������� \verb+/mnt+. �������� \verb+/mnt/cdrom+, \verb+/mnt/floppy+.

���������� (\emph{���������������}) ���������� ������������ ���������� �� ������ ���������. ������� \verb+umount(8)+.

\section{�����������, ��������, ����������� ������ � ���������}
\begin{itemize}
\item \verb+cp+ - �����������
\begin{verbatim}
����������� ������ ����� � ������
$ cp src.file /tmp/dest.file1

����������� ���������� ������ � ������ �������
$ cp *.tex metoda.dvi /mnt/floppy

����������� ���������
$ cp -r metoda /archive/old.Docs
\end{verbatim}

\item \verb+rm+ - �������� ������ � \verb+rmdir+ - �������� ���������\footnote{�������� ������ ��� \emph{������} ���������}
\begin{verbatim}
������� ��������
$ rm part1.aux intro.temp

�������� ���� ������ �� �������� 
$ rm /tmp/*

�������� ��������
$ rmdir oLD.stupid.dir
\end{verbatim}

\item \verb+mv+ - �����������(�������������)
\begin{verbatim}
����������� ���� ���� � ������
$ mv src.file /tmp/dest.file1

����������� ���������� ������ � ������ �������
$ mv *.tex metoda.dvi /mnt/floppy

����������� ���������
$ mv metoda /archive/old.Docs
\end{verbatim}

\item \verb+mkdir+ - ������� �������
\begin{verbatim}
c������ 1 ��� ����� ���������
$ mkdir /tmp/dir.12345
$ mkdir empty.DIR DiRecTorY
\end{verbatim}
\end{itemize}

\section{���������� � �������}

\begin{itemize}
\item \verb+ps+ - ������ ���������� ���������
\item \verb+who+ - ������ �������������, ���������� � ������� 
\item \verb+date+ - ������� ���� � �����
\item \verb+w+ - ����� ���������� � �������
\end{itemize}
\begin{verbatim}
�������:
�������� �� ������� �������
$ ps
 3642 pts/1    00:00:00 bash
 4548 pts/1    00:00:00 ps

�������� ����������� ������������
$ ps -u user
 1766 ?        00:00:04 xterm
 1853 pts/2    00:00:13 vim
 3642 pts/1    00:00:00 bash
 4553 pts/1    00:00:00 ps

��� ��������
$ ps -ef (��� BSD � Linux - ps ax )
UID        PID  PPID  C STIME TTY          TIME CMD
user      1766  1176  0 06:45 ?        00:00:04 xterm -title Terminal
user      1768  1766  0 06:45 pts/2    00:00:00 bash
user      1853  1768  0 06:56 pts/2    00:00:13 vim part2.tex
	<��������� - ������ 52 ������>

$ date
��� ��� 15 10:34:17 EEST 2002

$ who
root     tty1     Jun 15 10:24
work     tty2     Jun 15 10:24
user     pts/1    Jun 15 09:18

$ w
 10:35am  up  4:14,  3 users,  load average: 0.09, 0.04, 0.01
USER     TTY      FROM              LOGIN@   IDLE   JCPU   PCPU  WHAT
root     tty1     -                10:24am 11:25   0.02s  0.02s  -bash 
work     tty2     -                10:24am 11:11   0.03s  0.03s  -bash 
user     pts/1    -                 9:18am  1.00s  0.14s  0.01s  w 
\end{verbatim}

\section{������� ����� ��������������}
\begin{itemize}
\item \verb+write ������������+ - ������� ���������.\footnote{<ctrl+d> �������������� �������� ��� "����� �����"}
\begin{verbatim}
$ write stud11
������� ����� ���������
������ <ctrl+d>
\end{verbatim}
\item \verb+talk ������������+ - ������������� ��� 
\item \verb+mail+ - ����������� �����
\begin{verbatim}
������� ������
$  mail stud7 //���������� ������������
Subject: test
������� ����� ���������
������ <ctrl+d>

$ mail user@tut.by //���������� ������������
.....

$ mail // ����������� ���� �����
.....
\end{verbatim}
\end{itemize}

\section{��������, ��������, ����������� ������}

\begin{itemize}
\item \verb+cat+ - ����� ����������� ����� �� ����� 
\item \verb+more+ - ��������� �������� ����� �� ��������
\item \verb+less+ - �������� ������ (����� �� ������� \verb+q+)
\end{itemize}

\begin{verbatim}
�������
����� �� ����� �����
$ cat file

����������� �����
$ cat file >file2

����������� 2-� ������ � ������
$ cat file1 file2 >end.file

������������ ����� ����� �� �����
$ cat large.file |more 

���� ����� � ����������
$ cat >new.text
<���������� �����>
<ctrl+d>

\end{verbatim}

\section{����������� ������} 

�������������, Unix - ��� �������� �������� ���������� ��������� ������, ����������� �����-�� ���� ��������� ������. �� ��������� ���� ������ ����� �����, �� ���� ��������� ��������� � �������. 

\emph{�������} - ��� ���������, ��������������� ��� ��������� ������ ��� ��� ���� ��������.

����� ��� ����������� � ������ � ������� ���������, ������� \emph{���������}. �������� ������������ �������� \verb+|+. ��� �������� ���������: �������� ����� �� ��������� �������� ���� ����� �� ���� ��������� ������ �� ���������\footnote{�������� ����� ����������� ��������� ���}. ���������� ������: \verb+$ cat myFILE|more+\footnote{�� ��� ������� cat ���������� �� ���� more. � more ��������� ����� � �������� �������� �������.}. � Unix ���� ����� �������� \emph{���������������}. ��������� � ��������������� - � \ref{redirect}.

� ����� ������, ������� ������ ���� \emph{����������� ����} � ����� �� \emph{����������� �����}. ���� �� ���� ���������������, �� ������ ��������� ����������, � ������� - �����.
\begin{center}
����� ����������� �������
\end{center}
\begin{itemize}
\item \verb+grep+\footnote{���������� ����� ��������� grep-������ : grep, egrep, fgrep} - ����� � ����� �� �������
\\ \verb+������: $ ls /usr/include | grep "stdlib.h" +
\item \verb+sort+ - ���������� �����������
\item \verb+wc+ - ���������� �� ����� (���-�� ����, ����, ����� � �.�.)
\item \verb+head+ � \verb+tail+ - �������� ������(head) � �����(tail) �����
\item \verb+tee+ - ������������� ����� � ���� � �� �����
\\ \verb+������: $ ls -1 /tmp | tee all.tempfiles+
\end{itemize}

\section{������ ����������� ������}

�������� \verb+bash+ �������� 3 �������� ���������� �������������\footnote{��������� �� ������������� BASH - � \cite{Asp}, \cite{Pet}, man bash}, ��������� ������ � ��������� ������ ������� � ������:
\begin{enumerate}
\item �������������� ����� � ������

\emph{�������������}: ������� 1 ��� ����� ��������� �������� ������� � ������ \verb+TAB+. ���� �������� ������� ��� ����������� �������, �� ����������� ����� ��������� �������������. ���� ���������� ����� 1 ���������� �������, �� ��� ��������� �������� \verb+TAB+ �� ����� ����������� ������ ��������� ������. ����� ����� �������� ��������� ���� � ���������� ���������� �������.
\begin{verbatim}
$ la<TAB>
lambda      last        lastb       lastlog     latex       latex2html
$ last<TAB>
last     lastb    lastlog  
$ lastb<ENTER>
lastb: /var/log/btmp: No such file or directory
Perhaps this file was removed by the operator to prevent logging lastb info.
\end{verbatim}
��� ������ ������� ������������ ���������� PATH (��. \ref{path}). 

����� ��� �� �������� �������������� ��� �����: ��������� ������� ���� � ����� ������� <TAB> ���������� ���������� ����.\\

\emph{����������}. �������������� �� �������� ��� ������ ������� (\verb+ls --he<TAB>+ - ����� ���������������) � ��� ���������� �������� \verb+man+.
\item ������� ������

��� ��������� ������� - \verb+history+. ��� ��������� � ����������� ������ ������� - \verb+!�����+. ����� ������������ ������� � ������� ������ \verb+"�����"+ � \verb+"����"+. 
\item �������������� ��������� ������

�������� � ������� ������ \verb+"������"+ � \verb+"�����"+. �� ������ ������ - \verb-ctrl+a-, �� ����� - \verb-ctrl+e-.
\end{enumerate}

%оболочка (shell)
\chapter{Shell (��������)}
\section{������� ��������}

\emph{�������� (��������� ��������������, ����������, shells)} ������������ ����� ������������� ������� ����� ������������� � ��. ��� ����������� ��������� ������, ��������� �������������� ���������� ������, ������ � �������� �������.

Shell  ��� ������� ���������� ���������. ��� �� �������� ������ ����, � ������� ����� ���� �������� �� ����� ������, ��������, �� ������� ��� ��������� ��������. �� �������� ����� ���� �������� ������ �������� (��� ����� ��), ��� ���� �������������� �����������.

���� shell: sh, csh, ksh, zsh, tcsh, ash � ������.

� ����� ����� ��������������� \emph{Bourne Shell - sh}. ��������, ����������� � Bourne Shell ���������� ��� ��� ������ Unix. �� ������������ ������� ����� �������������� \verb+bash+ (Bourne Again SHell).

\section{Bourne Shell}

\emph{�������� (������)} �������� ������������ ����� ��������� ����, ������� ������ ���������� ������������������ ��������. �������� ����� ��������� ����� ������������������ ������ (��� ���������� ������ ��������, ��� � ������� ������ UNIX, � ����������� ��� ��� ���), ������� �������� ��� ������ ���������� ����� ���������.

��� Shell ����� �������, ��� ��� � ��������� � ���� ����������������.

������ ������������ � ��������� ����� ����� �� ���������� �� ���������� �������� ����������� �������.

�������� ������ �������: ��� ������� ������� �� ��������� ������ ����������� ����� ��������������, ��� �������� ������ ������ ������ (���-����� ������������ ��� ������ ��� �������).

\emph{������� �������}: �� �����, bash ���.\\
��� ������� �� �����, ���� ������� ������� ����� ������� X (eXecutable - �����������). ���������  �� ��������� ����� ��������� � \ref{attrib} 


\subsection{��������� ��������}

\verb+#+ - ��, ��� ������� �� ���(� ��� ����� � ������ \verb+#+), �������� ������������. ����������� ����� �������� ��� ������ ��� ��������� �� ��������.\\
\verb+\+ - ����������, ��� ������ ����������� �� ��������� ������ �����.

����� �������� ��������� ������ � 1 ������, ����������� \verb+;+ \label{tz}.

\section{����������}

�������� ���������� - ������, ������� ���������� �������������.\\
\verb+V1 = 5; v2 = "string"+ \footnote{�������� ���������� ����� ���������� �� ������� � ������}\\
���������� ����� �����  ����������� ��������, ������� ������������ ��������.\\
\verb+V3 = `pwd`+

\begin{center}
��������� ��������
\end{center} 
\begin{itemize}
\item \verb+$���_����������+ - � ��� ����� ������������� �������� ����������.
\item \verb+$(���_����������)+ -  �������� ���������� �� ����������� ��������.
\begin{verbatim}
������: 
$ echo result = $(v1)2 +
result = 52+
\end{verbatim}
\end{itemize}

� shell ���������� ��� \emph{����������������} ����������\footnote{������ ������ ���������� � ���������������� ���������� ����� ����� � man bash}:
\begin{itemize}
	\item \verb+HOME+ - �������� ������� ������������
	\item \verb+PATH+ - ���� ������ ����������� �������� 
	\item \verb+MAIL+ - ������ ��� ����� � ������ ������������
	\item \verb+PS1,PS2+ - ��������� � ��������� ����������� shell (������ \verb+$+, ������� ������� � �������� - ��� ��������� �����������).
\end{itemize}

� shell ���������� ��� ����������, ������� ������������ ��������� �� ���� ���������� ��������. ��� ��� ���������� \emph{����������} ����������:
\begin{enumerate}
	\item \verb+$0, $1, $2, ... ,$9+  �������� ����������, ������������ ������� �� ��������� ������. 
	\item \verb+$0+  - ��� ������ �������.\\
		\emph{����������}: ������� ����� ����������������  �������� ��� ������� �� \verb+$0+ � ���������� �������� ���������.
	\item \verb+$#+ - ����� ����������, ���������� �������;
	\item \verb+$*+ - ��� ���������, ���������� �������. ������������ ����� ������ �����, ����������� � �������.
\end{enumerate}

���������� ��� ���� ������� ��� ������������ ����������:
\begin{itemize}
	\item \verb+' '+ - ����������������  �����������,  ��������: \verb+v4='$v1'+  ��������  \verb+$v1+, � �� 5.
	\item \verb+" "+ - ����������� ����� ������������� �������� \verb+\+ � \verb+$+. ��������: \verb+v5=$v1+. ���������� \verb+v5+ ���������� 5.
		������ ��� �������������� �������� ������������. 
		\begin{verbatim}
		�������� ������������
        		v6=string
        		v7="string"
        		v8='string'
		���� ���������� ���������. ������� ������ ���� ���������� ������.
		\end{verbatim}
	\item \verb+` `+ - ���������� ������� ������ ������. ��������� ���������� ������� ����� �������� ����������.\\
	\verb+������: $list=`ls -a`+
\end{itemize}

�� ��������� ��� ���������� ��������, �� ���� ����������, ���� ����������� ������. ����� ������� �� ����������� (��� ������� shell), ���� ������ �� ��� ������ export. \verb+��������: export v1+

��� ������ �������������� ���������� ����������� ����� ������ ������.

��� �������� ���������� ������������ \verb+unset ������_����������+. \verb+������: $ unset $v1 $v3 $v4+

������� \verb+set+ ������� ������ ���� ������������� ���������� shell.


\section{��������������� �����/������}.

\label{redirect}������ ���������, ���������� �� shell, �������� ��� �������� ������ �����/������, ������� �� ��������� ������������� � ����������. ������ �������� ��������(\emph{�����������}):
\begin{itemize}
\item 0 ����������� ����� �����, ������������� � �����������
\item 1 ����������� ����� ������, ������������� � �������
\item 2 ����������� ����� ������, ������������� � �������
\end{itemize}

����������� ������ Unix ���������� ������ ����������� ������, ������� ��� ���� ������ ����� ������������ ���������������.
\begin{center}
���� ���������������
\end{center}
\begin{itemize}
	\item \verb+>file+ - ����� ������ ���������������� � ����. \verb+������: cat file1>file2+
	\item \verb+>>file+ - ������ �� ������ ������ ����������� � ����.
	\item \verb+<file+ -  ��������� ������ ��� ������������ ����� �� �����.
	\item \verb+p1|p2+ - �������� ������ ��������� �1 �� ���� ��������� �2 (\emph{�������� ��� ������������� �����}). \verb+������: cat spisok | wc l+
	\item \verb+n>file+ - ������������ ������ � ������� \verb+n+ � ����.
	\item \verb+n>>file+ - ������������ ������ � ������������ \verb+n+ � ����, �� ������ ����������� � ����� �����.
	\item \verb+n>&m+  - ���������� ������ � ������������� \verb+n+ � \verb=m=. 
	\item \verb+<<str+ - ����������� \emph{"���� �����"}. ���������� ����������� ����� ����� �� ��������� ������ \verb+str+ �� ����� � ����� �������� ��� �� ���� ���������.
\end{itemize}
\begin{verbatim}
�������: 
1)ls -al | wc 1>&2 1>>wc.out
2)run 2>/dev/null  ���������� ������ ������.
\end{verbatim}


\section{�������(wildcard's, �������������� �������)}

�������� ��������� ������ ����������� ����. \emph{�������������� ������} ���������� ��������� �� ����� ������, ���� ���-�� � �������� �������� ��� ������. ��� ������� � �������, ����� ������ ����� ��� ���������� ������� ��������� ������ �� ������������� �������.
\begin{itemize}
\item \verb+*+  - �������� ����� ���������� �������� (����� ���� � 0) , � ����� �����.
\item \verb+?+  - �������� ����� ������ � ����� �����.
\item \verb+[�������]+ -  ������ ����� ������ �� ���������.\verb+[a-c1-3]+ ���� ��� �  \verb+[abc123]+
\item \verb+\�+  - ������ ������ � ��������� (\emph{����������}) \label{slash}, ���� �  ��� ����������(\verb+\,',",`,# � �.�.+).
\end{itemize}
\begin{verbatim}
�������: 
$ ls [a-d]* //��� �����, ������������ �� a,b,c,d
$ ls x*y // ���, ������������ �� x � ����������� �� y
$ ls *\ ? // ������������� ������ - ������
\end{verbatim}

\section{�������� ���������� ������}

��������� ������� ����������� - \emph{�������� ����������}. ��� ���� ������: ����� ���� �������� ������� �� ���������� ���������� ����������. ���� ��� ���������� ����������� ������ \verb+';'+ (��. \ref{tz}). �� ��� ������������� \verb+';'+ ������������������ ������ ������ �����������, ��� ����������� �� ����������� ������ ��������� ������ (������ � ���). 

��� ��������� �������� ������, � Bourne Shell ���������� ��������� �����������: 
\begin{itemize}
\item \verb+p1&&p2+  ����������� �1, ���� ������ (��� �������� 0\footnote{���� �������� ��� shell - �������������� �� (��. \ref{shellreturncode})}), �� ����������� �2
\item \verb+p1||p2+ - ����������� �1, ���� �������� (��� �������� �� 0), �� ����������� �2
\item \verb+p1&+ - �1 ����������� � ������� ������, � shell �� ���� ��������� ������ �1 (��. \ref{jobcontrol}, �������� �������). �������� ����� ������� ����������� �� ����.
\item \verb+(p1;p2;)+ - ������� ����������� ��������������� � ����� ��������.  
\item \verb+{p1;p2;}+ - ������� ����������� ��������������� � ������� shell.
\end{itemize}
\begin{verbatim}
�������: 
1)(ps; who) | more
2)mount | wc -l > mounts.number &
\end{verbatim}

\emph{����������}: �������� � ������� ������ �� ����� ������������ ����������� ���� � �����, ������� �� ���� ��������������. ����� ������ �������� ��� ���������.


\section{�������� ���������}

���������
\begin{verbatim}
if   ������� 		
then 
else 	
fi 					 
\end{verbatim}

� shell true (0)  � false (�� 0) ����� �������� �������� �� ��������� � ��.\label{shellreturncode}. \footnote{����� �������� ���� �������� ��������� ������ ������� � man � ������ "EXIT STATUS" ��� "DIAGNOSTIC" ��� "RETURN CODE"}

�������� ��������� ����� ���������� � �������, �������� ;.

�������� ����� ���� ��������� �������. ����� ������������ \verb+test+ � �����������. �������� ����������� ��������:
\begin{itemize}
\item test s ����  - �������� �� ������ ����� �������� �� 0
\item test r ����  - �������� �� ���� ��� ������
\item test f ����  - ���������� �� ���� � �������� �� �� �������
\item test d ����  - ���������� �� ���� � �������� �� �� ���������
\end{itemize}

����� ���������� ��� \verb+test+, ��������� �� �� �������� � \verb+[ ... ]+.

\emph{����������}. ����� \verb+[, ], if+ ����������� ������ ������ �������!

\begin{verbatim}
������: ��������� ������ ������������:
1)if test f $HOME/file.txt
then
            echo �� ����!
     fi
2)if [ -f $HOME/file.txt]
then
            echo �� ����!
     fi
3)  test f $HOME/file.txt && echo �� ����!
\end{verbatim}

\subsection{��������� �����}
 
\begin{itemize}
\item \verb+������1 = ������2+  �������� �� ���������
\item \verb+������1 != ������2+  �������� �� �� �����
\item \verb+-n $����������+  true, ���� ������ ����� ��������� �����
\end{itemize}
\begin{verbatim}
�������:
1)if [ $v1 = abc ]; then; echo ����������!
2)if [ -n $empty ]; then; echo �������������
\end{verbatim}

\subsection{��������� �����}

����������� �������� \verb+$x+(�������� ����������) ��� �����\footnote{��� ����������� ��������� �������� � �����������}:
\begin{enumerate}
\item \verb+$x eq $y+   true, ���� ��������� �����
\item \verb+$x ne $y+  true, ���� ��������� �� �����
\item \verb+$x gt $y+  true, ���� �������� x ������ �������� y 
\item \verb+$x ge $y+  true, ���� �������� x ������ ���� ����� �������� y
\end{enumerate}

\verb+������:  if [ $# eq 2 ]; then; echo 2 ��������� +

\subsection{������� ���������}

\begin{enumerate}
\item \verb+!���������+	- ���������
\item \verb+���������1 a ���������2+	- ���������� �
\item \verb+���������1 o ���������2+	- ���������� ���
\end{enumerate}
\begin{verbatim}
������: 
1)if [ !\( $x eq $y\) ]
2)if [ $a ne 3 a $b lt $c ]
3)if [ $x = $y a \( $n lt 0 o $m gt 30\) ]
\end{verbatim}
\emph{����������}: ������ ������������ (��. \ref{slash}), ��� ��� ��� ����� ����������� ����� ��� ������ (���������������� ���������� � ����� ���������� shell).


\section{�����}

� ����� shell ���� ��������� ����� ������. ����� ����������� �� ���:\verb+for+, \verb+while+ .
\begin{itemize}
\item ���� \verb+for+\footnote{BASH ������������ ����� ����� � ����� ��: for ((i=1; $i<10; $i++)) - c 2 ������������ � ������������ ��������. } ����������� ������� ���, ������� ���� � ������. var ��������������� ��������� �������� �� ������. ������ ����� ������������� �������, ��� ����� ������� (`�������`) ��� � ������� ��������.
\begin{verbatim}
for ���������� in ������
do
....
done
\end{verbatim}
\item ���� \verb+while+ �����������, ���� ������� �� ������ ������.
\begin{verbatim}
while �������
do
...
done
\end{verbatim}
\end{itemize}
\begin{verbatim}
�������:
while sleep 60
do
 who | grep mary
done

for user in `who`
do
  echo � ���� $user �����!
done

for i in * ; do echo $i; done  #���������� ls.
\end{verbatim}

�������� �����  ����� ���������� �� ��������� ������� ��� �������� ';'. ��� ����� ������ ������������.

\emph{����������}. ���� ������� � ��������� ������ ����, �������� ��������� ��� ������ �� ������� ������ (�������) �� ��� ������� \verb+ENTER+ �������� ��������� ������� ��������� ������� �� ��������� ������.

���������� \emph{����������}(��. \ref{internalcmd}) ������� \verb+break+ ���  ������ �� �����.

\section{�������}

��� �������������� ������� ������������ ����� ���������� �������:
\begin{verbatim}
���_������� ()
{
   �������
}
\end{verbatim}
��������� � �������� ���������� - ��� � �������.
\begin{verbatim}
������: ���������� � ����������� ��� ��������
mcd ()
{
		cd $*
		PS1 = `pwd`
}
\end{verbatim}

������� ����������� �� \emph{����������} (� ��������) \label{internalcmd} � \emph{�������}. ������ ���������� ������� �� ������� �������� ������ ��������.
 
���������������� ���������� �������: \verb+cd+, \verb+pwd+, \verb+echo+, \verb+exit+, \verb+set+, \verb+unset+.


\section{���������� �������������� ��������}

\emph{������ ���������}. � shell ����������� ������ ������������� ����������\footnote{��� ��������� ����� � ������� ���������� ����� ������������ ��������� bc}!

\verb+�xpr ������+  ����������� ������ � �����. \verb+��������: expr 23+.

����������� ��������: \verb=+, -, *, /, %= (������� �� ������). �� ��������� �������.
\begin{verbatim}
�������: 
1)a = `expr $a + 3`
2)b = `expr 2 \* 3` - ������ \ �������� ����������� �������� *.
\end{verbatim}
\emph{����������}. ����� � �������� ����������� ���������.

% vi
\chapter{��������� �������� VI}

��������� ��� Unix ������� �� 2 ������ - ��������� ���������� �����(vi, emacs, joe, ed) � ����-��������������� (mcedit, kwriter, kword).

��������� ���������� ����� ������ �������� � \emph{���������� (���������)} ������. ��� �������� � ��� ����������� ������� ������ ����������� ������, ������������ �� ������������� ����������� ������. ���� � ���� � ���, ��� �������, �� ������������.

�������� \verb+vi+ ������������ ��� ����������� � ����� Unix-�������� �������\footnote{� ���� �������� �� ������ � �������� Single Unix Specification}. ���������� ��������� ���������� ���������� �� vi: vim, elvis. 

����������� ����� vi (vim � �������) �������� ����� ������� �����������������, ������� �� ���������� �����������. �������� vi ���������� ���������� ��� �����-�������������, ������� ������ �������� �� ����� ����� ���������� � ����������� ��������. ��� �������� � ��� ����� ����������� �� ������� ��������, ���������-��������, ����� ����������.

\emph{����������}. ����� �� ����� ������������� �������� vim. ������ ��� ��������� ������� ����� ����� ��������� � ����� vi-����������� ���������.

\section{������ ������}

� vi ���������� ��� ������������� ��������� ������ ������:
\begin{itemize}
\item[-] ��������� ����� (command mode)
\item[-] ����� ����� (edit mode)
\item[-] ����� ����������� �������������� (ex mode)
\end{itemize}

\emph{��������� �����} ���������� �� ��������� ��� ������� vi. � ���� ������ ������� ������ \emph{�� ��������} � ����� ��������, � ���������������� ��� ���������� ������� ����������� �� ������ � ��������������. ������� ������� ���������� ������ ���� ������ (��� � DOS/Windows) �� � ���� �� ��������.

\emph{����������}. ���� �� �� ������, � ����� ������ ����������, �� ������� ������� \verb+ESC+ ��� �������� � ��������� �����.

�������� ������ � ��������� ������ ����������. ��� ����� ����� ������� � \emph{����� �����}. ��� ����� ������ ������� (���������� ������!) \verb+a+ (�� append - ����� ������� ������� �������)  � \verb+i+ (�� insert - ����� ������� �������� �������). � ������ ����� ������� ������ �������� � ����� ������� ��������, �������� ��������� ����� ����� ��� ������������� ������������.

������� � ��������� ����� �������������� �������� ������� \verb+escape+.

��� �������� � ����������� (�������) ������������ \emph{ex-�����}. �� ���������� �������� \verb+:+ ���������� ������. ����� ����� ������ ������� ex-������. ��������:
\begin{itemize}
\item ������� ������������ ���� (\verb+:e ���_�����+) 
\item �������� ���� � ������� ������� (\verb+:r ���_�����+)
\item �������� ���� (\verb+:w+), � ��� ����� ��� ������ ������ (\verb+:w ���_�����+)
\item ����� �� ������������ ����� (\verb+:q+)\label{viexit}
\item ����� � ��������������� ����������� ����� (\verb+:x+)
\end{itemize}

����� �� ���������� � ex-������, �� � ������ ����� ���� ������ ���������� \verb+:+.

�������� ���������� ������ ex-������. �������� \verb+:wq+. 

������� ex-������ ������������ �� ���������� �������� ������� \verb+Enter+ ����� ���� ���������� ������� � ��������� �����.

\emph{����������}. ������� ��������� ����� ���� (�������� \verb+:e+) ��� ��������� ������ ��������� (�������� \verb+:q+) ��� ������������� ������ ����� ������� ������.

\section{��������� ������}

�������� ������� ����� ��������� ex-������� \verb+:help+.

���������� ������� �������� �������� \verb+|������|+. ������� �� ��� ���������� ����� \verb+:help ������+.

����� �������� �������� ������� �� vim - \verb+vimtutor+. � ��� ������� ����� ������� �������� ������ ������������� vim.

\section{������ � ��������� ���������}

\verb+Vi (vim)+ ����� ���� ������� �� ��������� ������, � ������ ����� ��� ��� ��������. ���� ������� ��� �����, �� �������� ��������� ���\footnote{���� ���� �� ����������, �� ��������� �����}.

\verb+������: $ vi ~/texts/newtext.txt+

������� vim ��� ����� ����� ������� �������� vim � ������� ��������.

��� ������ �� ��������� ������� \verb+:q+ ��� \verb+:wq+ (��. "������ ������" \ref{viexit}). 


\section{����������� �� ������}

� ��������� ������ ���������� ��������� �������\footnote{������ �������� � ������� �� ����������, �� �� ����� ���������� �� ���}:
\begin{itemize}
\item \verb+h+ - ������ ����� �� 1 ������
\item \verb+l+ - ������ ������ �� 1 ������
\item \verb+j+ - ������ ���� �� 1 ������
\item \verb+l+ - ������ ����� �� 1 ������
\end{itemize}

�����, ���� ����������� �������, ����������� � ������� ������.
\begin{itemize}
\item \verb+w � W+ - ����������� ������ �� "��������� �����"\footnote{��������� �����, ���������� ��������, ������� ����������, +, -} � �.�. "������� �����"\footnote{����������� ���������� ��������} 
\item \verb+b � B+ - ����������� ����� �� "��������� �����" � "������� �����" 
\item \verb+0 � $+ - �� ������ � �� ����� ������
\item \verb+( � )+ - �� ������ ����������� � ��� ����� 
\end{itemize}


������, ��� ������ ������ vi ���������� ������� ������ ��������� - � ������ � ������� ��������� ����� ������� (\verb+e+ � \verb+E+, \verb+w+ � \verb+W+); �������� ������ ������� �� ���� ��� �� ��������� �������� ������.

������� ��������� vi ����� �������������� � ���������� �����������.

�������� ������� \verb+5h+ ���������� ������ �� 5 �������� ����� (������ ������ � ������� �������), � ������� \verb+3B+ �� 3 "�������" ����� �����. 

��� ����������� �� ���������� ������, ����� ������������ ��������� ������� ex-������: \verb+:N+, ��� N - ����� ������.


\section{���� � �������������� ������}

��� �������� ������ ���������� ������� � ����� �����.

��� ����� ������ ��������� �������:
\begin{itemize}
\item \verb+i � I+ - ���� � ������� ������� ��� � ������ ������
\item \verb+a � A+ - ���� ����� ������� ��� � ����� ������
\end{itemize}

����� ����� �������� � � ������ �����, ��������� \verb+DEL+ � \verb+BACKSPACE+, �� ����� ������� ������������ ������� ��������������.

������� �������������� ������������� ��� ��������� ������������� ������ ��� �������� � ����� �����:
\begin{itemize}
\item \verb+x+ - �������� ���������� �������
\item \verb+dd+ - �������� ������
\item \verb+dw+ - �������� �����
\item \verb+d)+ - �������� �����������
\end{itemize}

��� � ������� �����������, ������� �������������� ����� ������������ � ���������� �����������. ��� ������� \verb+5dd+ ������ ������� ������ � ��� 4 ������ ���� ��, � \verb+3dw+ ������ ��� ����� ������ �������.

\section{����������� � �������}

� vim ��� ���� ����� ���������� ��������� ����� - ���������, Visual Selection (��. \verb+man vim+). ������ � ����������� ������� �� ����� �������� ������������ ���������:
\begin{itemize}
\item \verb+p+ - �������� �� ������.
\item \verb+yy+ - ����������� ������ � �����
\item \verb+yw+ - ����������� ������� ����� � �����
\item \verb+y)+ - ����������� �����������
\item \verb+y}+ - ����������� �����
\end{itemize}

\emph{����������}.� ����� ����� �������� ��� ��������� � ������� ������ x, dd, dw � �� ��������. ����� ������� ��� ������� ����� ������� ��� ����������� � ���������.

��� ����������������� ������� ����� �������������� � ��������� ����������. ��������: \verb+3p+ - 3 ���� �������� ���������� ������.

\section{����� ��������}

�������� �������� ��������� ������ �������������� ����� ���� �������� �������� \verb+u+ (���������� �� undo). ��������� ������� - ������ ����������� ��������, � ��� �����. ��� �������� (redo) �������� ���������� �������� ������������ \verb-control+r-.

\section{����� � ������}

��� ������ �� ������ ������ ������� \verb+/+ (������ ����). ��� ����� ����� ������� � ��������� ������ � ������ ������ ���������� ������ \verb+/+, ����� �������� �� ������ ������ ������� ��� ������. ��� ����� ���� ��������� ������ ��� \emph{���������� ���������} (��. \ref{regexp} ). ����� ������� \verb+ENTER+ � ������ ����� ���������� \footnote{��� ����������� ������ ��� vim} ��� ��������� ��������� ������ ������ � ������ �������� � ������� ���������� ��������� ��������� ���� �� ������.

��� ������ ��������� ��������� ������ ������, ���������� ������� \verb+n+ (���� �� ������) � \verb+N+ (����� �� ������).

��� ������ � ������ ��������� ����������, � ��� ����� � � �������������� ���������� ���������, ������������� ������� ex-������ \verb+:s+(substitute). ������ �������: \newline
\verb+:#s/pattern/string/�����+\newline
��� \verb+#+ - �������� ����� (����� \verb+,+ ��� \verb+;+ - ��. \verb+:help cmdline-ranges+).

\emph{����������}. ����� � ����������� vi �� ��������������.

����� ����������� �����: \verb+c+ - ������������� ������ ������, \verb+g+ - ������ ���� ��������� � ������.

\emph{����������}. ����� � ������ � vi �������� ������ ��� ������������������ ��������, ������������ 1 ������. ���������� ������������������ �������� ���� ������ ������������ 1 ������.
 

\section{����� ������� ������}

�������� vim ����� ������� ������, ��� ��� �� ��������� ����������� �������� � �������, �� ������ �� ���������.

�� vi ����� ��������� ������� ��������� � ������� ������� ex-������ \verb+:! cmdlline+:
\begin{verbatim}
������
:! ls -l

:! man bash
\end{verbatim}

��� �������� ������� �������, ��� ��� ��� ���������� ��������� ������� � �������� �� ���������. ��� ������� ��������� ������ �� ����� ����� �������� �� ������ �� ������� ������������� ������� \verb+ENTER+.


% regexp и sed
\chapter{Регулярные выражения. sed}

\label{regexp}
\emph{Регулярные выражения (regular expression или regexp)} -  специальные строки символов, которые задаются для поиска совпадающих фрагментов. Иначе говоря это способ описания наборов букв. 

Простейшим набором является слово, но регулярное выражение может включать и глобальные символы, заменяющие другие символы. Все UNIX-программы, осуществляющие поиск в тексте, используют регулярные выражения. Если слово или фраза описаны регулярным выражением, говорят, что они соответствуют регулярному выражению. 

\emph{Регулярные выражения} - мощное, гибкое и эффективное средство обработки текстов. Универсальные шаблоны регулярных выражений сами по себе напоминают миниатюрный язык программирования, предназначенный для описания и анализа текста. При дополнительной поддержке со стороны конкретной утилиты или языка программирования регулярные выражения способны вставлять, удалять, выделять текстовые данные любого вида и выполнять практичяески любые операции над ними.

Регулярные выражения расширяют принципы \emph{метасимволов (шаблонов или wildcards)}. 

Некоторые программы используют регулярные выражения в чистом виде (grep, egrep). Но чаще всего регулярные выражения используются внутри специальных языковых конструкций, т. н. "оберток". 

\section{Структура регулярных выражений}

Регулярное выражение состоит из двух типов символов. Специальные символы называются \emph{метасимволами}. Все остальные символы (то есть обычный текст), называются \emph{литералами}.

Регулярные выражения можно рассматривать как самостоятельный язык, в котором литералы выполняют функции слов, а метасимволы - функции грамматических элементов. Слова по определенным правилам объединяются сграмматическими элементами и создают конструкции, выражающие некоторую мысль.

Для примера: существует утилита \verb+grep+\footnote{Утилиты семейства grep предназначены для поиска текста по шаблонам регулярных выражений}. При запуске программе \verb+grep+ передается регулярное выражение и список просматриваемых файлов. Она сопоставляет regexp с каждой строкой файла и выводит только те строки, в которых было найдено совпадение.
\begin{verbatim}
$ grep 'cat' file1.text 
\end{verbatim}
Если в нашем выражении ($\ulcorner cat \lrcorner$ \footnote{Для grep рекомендуется заключать regexp в кавычки, так как некоторые метасимволы имеют для оболочки специальные значения и выражение может работать некорректно}) не используются метасимволы, оно фактически превращается в стредство "простого поиска текста". Будут найдены и выведены все строки файла, содержащие три стоящие подряд буквы \verb+c, a+ и \verb+t+. Среди них будут выведены строки, в которых встречается слово (к примеру) \verb+vacation+. Даже если в строке нет слова \verb+cat+, последовательность букв \verb+c+ \verb+a+ \verb+t+ в слове \verb+vacation+ все равно считается успешно найденной. Необходимо только наличие указанных символов.

\section{Правила для регулярных выражений}

Существует всего два универсальных правила для регулярных выражений:
\begin{enumerate}
\item Предпочтение отдается тому совпадению, которое начинается раньше.
\item Квантификаторы (см. \ref{requant}) всегда работают максимально. Если некоторый элемент может совпадать переменное число раз, механизм всегда пытается найти максимальное число повторений.
\end{enumerate}

Признаки хорошо написанного регулярного выражения:
\begin{itemize}
\item регулярное выражение должно совпадать там где нужно и нигде более
\item регулярное выражение должно быть понятным и управляемым
\item оно должно быть эффективным ( быстро приводить к совпадению или несовпадению в зависимости от результатов поиска)
\end{itemize}

\section{Диалекты регулярных выражений}

В разных программах регулярные выражения выполняют разные функции, поэтому наборы метасимволов и другие возможности, поддерживаемые программами, также различаются.

К примеру, диалекты регулярных выражений в sed, perl и grep имеют значительное число отличий между собой. Более того, различные варианты grep тоже могут использовать разные диалекты. 

Далее изложение будет придерживаться диалекта \verb+sed+, с указанием отличий от других реализаций регулярных выражений.
 
\section{Метасимволы}

Существует несколько типов метасимволов, выполняющих разные функции. Значение некоторых из них различно в разных частях выражения (или зависит от контекста).

\subsection{Начало и конец строки}

$\ulcorner\verb+^+\lrcorner$ (крышка) и  $\ulcorner \$ \lrcorner$(доллар) представляют собой начало и конец проверяемой строки.

Примеры:
\begin{enumerate}
\item \(\ulcorner \verb+^cat+ \lrcorner\) находит все строки, в начале которых находиться \verb+cat+.
\item \(\ulcorner \verb+^cat$+ \lrcorner\) находит все строки, которые состоят только из \verb+cat+
\item \(\ulcorner \verb+^$+ \lrcorner\) пустая строка
\end{enumerate}

Особенность $\ulcorner \verb+^+ \lrcorner$ и $\ulcorner \$ \lrcorner$ в том, что они совпадают с определенной \emph{позицией} строки, а не с символами текста.

\subsection{Символьные классы}

\subsubsection{Совпадение с одним символом из нескольких возможных}

При помощи конструкции $\ulcorner[\ldots]\lrcorner$, называемой \emph{символьным классом} (character class), можно перечислить символы, которые могут находиться в данной позиции текста. 

Примеры:\label{examplegray}
\begin{enumerate}
\item $\ulcorner gr[ea]y\lrcorner$. Это обозначает "найти символ \verb+g+, за которым идет \verb+r+, за которым следуют \verb+e+ или \verb+a+ и все это завершается символом \verb+y+".
\item $\ulcorner[Ss]eparate\lrcorner$. возможная смена регистра в первой букве
\end{enumerate}

Количество символов в классе может быть любым. Например, класс $\ulcorner[123456]\lrcorner$ совпадает с любой из перечисленных цифр.

В контексте (внутри) символьного класса \emph{метасимвол символьного класса} \verb+-+ обозначает интервал символов; так выражение $\ulcorner[1-6]\lrcorner$ эквивалентно предыдущему примеру. Классы $\ulcorner[0-9]\lrcorner$ и $\ulcorner[a-z]\lrcorner$ обычно используются для поиска цифр и символов нижнего регистра соответственно.

Символьный класс может содержать несколько интервалов, поэтому класс $\ulcorner[0123456789abcdefABCDEF]\lrcorner$ записывается в виде $\ulcorner[0-9a-fA-F]\lrcorner$. Интервалы также можно записывать вместе с литералами: $\ulcorner[0-9\verb+_!.?+A-Z]\lrcorner$ (совпадет со всеми цифрами, буквами в верхнем регистре и знаками подчеркивания, точки, восклицательного и вопросительного знаков).

\emph{Примечание 1}. Дефис выполняет функции метасимвола только внутри символьного класса - в остальных случаях он совпадает с обычным дефисом в строке.

\emph{Примечание 2}. Правила, определяющие состав поддерживаемых метасимволов (и их функции) внутри класса и за его пределами, полностью различны.

\emph{Примечание 3}. Дефис не интерпретируется как метасимвол, если он находиться на первой позиции класса, например $\ulcorner [\verb+-./+] \lrcorner$.

\subsubsection{Инвертированые символьные классы}

Есле вместо $\ulcorner[\ldots]\lrcorner$ используется запись $\ulcorner[\verb+^+\ldots]\lrcorner$, класс совпадает с любыми символами \emph{не входящими} в приведенный список. Пример: $\ulcorner q[\verb+^+u]\lrcorner$.

Префикс  \verb+^+ инвертирует список - вместо того, чтобы перечислять символы, принадлежащие классу, перечисляются символы, не входящие в него.

\emph{Примечание}. Инвертированный класс означает "совпадение с символами не входящими в список", а не "несовпадение с символами, входящими в список". Поэтому инвертированный класс удобно рассматривать как сокращенную форму записи для обычного класса, включающего все символы, \emph{кроме} перечисленных.

\subsection{Один произвольный символ}

Метасимвол $\ulcorner . \lrcorner$ (точка) представляет собой сокращенную форму записи для символьного класса, содержащего \emph{все} символы. Применяется в тех случаях, когда в некоторых позициях регулярного выражения могут находиться произвольные символы.

Пример: пусть надо найти дату, которая может быть записана в формате \verb+07/04/76+, \verb+07-06-76+ или \verb+07.06.76+. Самый простой вариант - $\ulcorner 07.04.76 \lrcorner$. Но такое выражение будет совпадать и со строкой \verb+19 207304 7639+. Выражение $\ulcorner 07[-./]04[-./]76 \lrcorner$ обеспечевает более точное совпадение, но его труднее читать и записывать.

При построении регулярных выражений часто приходиться идти на компромис с точноcтья за счет знания текста. Если вы уверены, что в тексте $\ulcorner 07.04.76 \lrcorner$ наверняка не вызовет нежелательных совпадений, то этим вариантом вполне можно воспользоваться. 

Замечание. \emph{Знание целевого текста - важный фактор, обеспечивающий эффективное использование регулярных выражений}

\subsection{Выбор}
\subsubsection{Одно из нескольких выражений}

\label{reor}
Очень удобный символ $\ulcorner \verb+\|+ \lrcorner$ \footnote{Для Perl и egrep - $\mid$ } обозначает "или". Он позволяет объединить несколько регулярных выражений в одно, совпадающее с любым из выражений-компонентов. 

Например, $\ulcorner Erik \lrcorner$ и $\ulcorner Bobby \lrcorner$ - два разных выражения, а  $\ulcorner Erik\verb+\|+Bobby \lrcorner$ - одно выражение, совпадающее с любой из этих строк. Подвыражения, объединенные этим способом, называются \emph{альтернативами} (alternatives).

Конструкция выбора всегда является высокоуровневой (то есть обладающей очень низким приоритетом).

Вернемся к примеру 1 из \ref{examplegray} $\ulcorner gr[ae]y \lrcorner$. Это выражение можно записать также в виде $\ulcorner gray \backslash \mid grey \lrcorner$ или даже $\ulcorner \verb+gr\(a\|e\)y+ \lrcorner$\footnote{Для диалекта sed. Для egrep или perl это будет gr(a$\mid$e)y.}. Здесь круглые скобки \verb+\(+ и \verb+\)+ отделяют конструкцию выбора от остального выражения. Без скобок $\ulcorner gra\mid ey \lrcorner$ будет означать "$\ulcorner gra \lrcorner$ или $\ulcorner ey \lrcorner$".

Выражение внутри скобок может быть как угодно сложным, но "снаружи" оно воспринимается как единое целое.
 
\emph{Примечание}. Не путайте конструкцию выбора с символьным классом. Класс $\ulcorner abc \lrcorner$ и конструкция выбора $\ulcorner \verb+\(a\|b\|c\)+ \lrcorner$ фактически обозначают одно и то же, но это не для общего случая. Символьный класс совпадает ровно с одним символом, каким бы длинным или коротким не был список допустимых символов. С другой стороны, конструкция выбора может содержать альтернативы произвольной длинны, совершенно не связанные друг с другом длиной текста: $\ulcorner \verb+\(1.000.000\|million\|thousand*thousand\)+ \lrcorner$. В отличие от символьных классов, конструкции выбора не могут инвертироваться.

\subsection{Границы слов}

Одна из распространенных проблем заключается в том, что искомое слово встречается внутри других слов. Для явного указания начала и конца слова используются \emph{метапоследовательности} $\ulcorner \backslash< \lrcorner$ и $\ulcorner \backslash> \lrcorner$

Как и якоря \verb+^+ и \verb+$+, эти метапоследовательности не соотвествуют конкретным символам.

Примеры:
\begin{enumerate}
\item $\ulcorner \verb=\<cat\>= \lrcorner$ - найти отдельное слово \verb+cat+
\item $\ulcorner \verb=\<free= \lrcorner$ - найти слово, начинающееся с \verb+free+, к примеру \verb+freeware+
\item $\ulcorner \verb=ed\>= \lrcorner$- найти слова, заканчивающиеся на \verb+ed+
\end{enumerate}

\emph{Примечание} Сами по себе символы $\ulcorner < \lrcorner$ и $\ulcorner > \lrcorner$ метасимволами не являются. Они приобретают особый смысл только в сочетании с обратным слэшем \verb+\+.  

\section{Квантификаторы} 

\label{requant}
\emph{Квантификаторы} регулируют количество экземпляров повторяющегося элемента. Сами по себе, квантификаторы не являются щаблонами символов в тексте, но поставленные после символа или выражения в скобках, указывают, сколько раз может повторяться этот символ или выражение.

Квантификаторы руководствуются критерием максимального совпадения и пытаются найти совпадение как можно большей длины.

\subsection{Необязательные элементы}

Метасимвол $\ulcorner \verb+\?+ \lrcorner$ \footnote{Для Perl, egrep - ?.} (вопросительный знак) означает "необязательный символ". Он ставиться после символа, который может находиться в данной позиции текста, но наличие которого не требуется для успешного совпадения. Вопросительный знак относиться \emph{только} к символу, расположенному непосредственно перед ним.

Пример: $\ulcorner \verb+colou\?r+ \lrcorner$  

Пример2; Пусть нам надо найти дату, содержащую четвертый день месяца. На английском, это будет выглядеть так: \verb+4+ или \verb+4th+ или \verb+fourth+ - $\ulcorner fourth\mid 4\mid 4th \lrcorner$. Вторую половину выражения можно сократить до $\ulcorner \verb+4\(th\)\?+ \lrcorner$. Получим $\ulcorner \verb+fourth\|4\(th\)\?+ \lrcorner$.

Таким образом, квантификатор $\ulcorner \verb+\?+ \lrcorner$ может присоединяться и к выражениям в скобках.

\subsection{Повторение}

\begin{itemize}
\item Метасимвол $\ulcorner \verb=\+= \lrcorner$ \footnote{Для perl и egrep $+$}обозначает "один или несколько экземпляров непосредственно предшествуюших элементов".
\item Метасимвол $\ulcorner * \lrcorner$ обозначает "любое количество экземпляров элемента (в том числе и нулевое)".
\end{itemize}

Иначе говоря, $\ulcorner * \lrcorner$ означает "найти столько экземпляров сколько возможно, но при необходимости обойтись и без них". Конструкция $\ulcorner \verb=\+= \lrcorner$ имеет похожий смысл, но при отсутствии хотя бы одного экземпляра сопоставление завершается неудачей.

Примеры:
\begin{enumerate}
\item $\ulcorner \verb=^[0-9]\+= \lrcorner$ - строка, начинающаяся с одной или более цифр 
\item $\ulcorner \verb=^[0-9]*=\$ \lrcorner$ -  строка, содержащая в себе только цифры(может быть и пустой) 
\item $\ulcorner .* \lrcorner$ - любое количество любых символов
\item $\ulcorner * \lrcorner$ и $\ulcorner + \lrcorner$ могут следовать за скобками: $\ulcorner \verb=\(th\)i\+= \lrcorner$ - одно и более сочетаний букв \verb+th+ подряд
\item $\ulcorner _\sqcup \verb=\+= \lrcorner$ - один или более пробелов
\end{enumerate} 

\subsection{Интервал}

Конструкция вида $\ulcorner \ldots \verb+\{min,max\}+ \lrcorner$\footnote{В egrep и perl она выглядит как \dots\{min,max\}} называется \emph{интервальным} квантификатором.

Например, выражение $\ulcorner \ldots \verb+\{3,12\}+ \lrcorner$ совпадает до 12 раз, если это возможно, но может ограничиться и всего 3 совпадениями. Запись \verb+\{0,1\}+ эквивалентна метасимволу \verb+\?+, \verb+\{1,\}+ - \verb=+=.

\section{Круглые скобки и обратные сcылки}

Мы уже знакомы с двумя применениями круглых скобок:
\begin{itemize}
\item ограничение области действия \verb+|+ (см. \ref{reor})
\item группировка символов для применения квантификаторов (см. \ref{requant})
\end{itemize}

Существует еще одно применение круглых скобок. Круглые скобки могут "запоминать" текст, который совпал с подходящим в них подвыражением.

\emph{Обратные ссылки} позволяют искать новый текст, который совпадает с другим текстом в предшествующей части регулярного выражения, причем на момент написания выражения этот текст \emph{неизвестен}.

Круглые скобки "запоминают" текст, а специальный метасимвол $\ulcorner \backslash 1 \lrcorner$ представляет этот текст (каким он бы не был) в оставшейся части регулярного выражения.

В выражение можно включить несколько пар круглых скобок и ссылаться на совпавший текст с помощью $\ulcorner \backslash 1 \lrcorner$ , $\ulcorner \backslash 2 \lrcorner$, $\ulcorner \backslash 3 \lrcorner$ и т.д. Пары скобок нумеруются в соответствии с порядковым номером открывающей скобки справа налево.

Пример. Пусть нам надо найти повторяющиеся слова. Если известно конкретное слово, то можно включить его в шаблон, например, $\ulcorner the\ the \lrcorner$. Но все пары слов проверить таким образом невозможно. Нам надо найти одно "обобщенное" слово, а потом указать искать то же самое. Заменим $\ulcorner the \lrcorner$ регулярным выражением для обобщенного слова - $\ulcorner \verb=[A-Za-z]\+= \lrcorner$ и запомним его в круглых скобках -  $\ulcorner \verb=\([A-Za-z]\+\)= \lrcorner$. Добавим выражение для пробелов - $\ulcorner _\sqcup \verb=\+= \lrcorner$. Теперь объединим полученные выражения и добавим обратную ссылку  $\ulcorner \verb=\([A-Za-z]\+\)= _\sqcup \verb=\+\1= \lrcorner$. И последнее - обозначим границы слов\footnote{Иначе будут найдены не только повторяющиеся слова, но и сочетания, когда буква, завершающая слово, является первой для следующего.} -  $\ulcorner \verb=\<\([A-Za-z]\+\)= _\sqcup \verb=\+\1\>= \lrcorner$.

\section{Экранирование}

Чтобы включить в выражение символ, который совпадает с метасимволом, необходимо выполнить \emph{экранирование}. Экранирование выполняется с помощью символа \verb+\+.

Например: метасимвол "точка" $\ulcorner . \lrcorner$ совпадает с любым символом. Чтобы получить обычную точку, надо записать $\ulcorner \verb+\.+ \lrcorner$, которая называется "экранированной" (escaped) точкой.

Экранирование может выполняться со всеми стандартными метасимволами, кроме метасимволов символьных классов.



\section{sed}

\emph{sed (sequential или stream editor)  неинтерактивный (поточный) редактор текста}. Он служит для выполнение анализа и преобразования текста. Фактически sed - это продвинутый текстовый фильтр. Имеет свой входной язык, тесно связанный с регулярными выражениями.

Очень удобен для использования в скриптах оболочек как средство обработки текстов.

Далее будет описываться GNU sed.

sed можно использовать двумя основными способами:
\begin{verbatim}
sed [-n] [-e] 'команды редактирования' входной_файл
\end{verbatim}
\begin{center}и\end{center}
\begin{verbatim}
sed [-n] -f сценарий входные_файлы
\end{verbatim}

Чаще используется первый способ.

Параметры:
\begin{itemize}
\item -f cmdfile  прочитать сценарий из файла
\item -n  блокирование вывода, кроме явно задаваемого из сценария 
\end{itemize}

Если команд несколько, то они разделяются \verb+;+.

Входные файлы: редактируемый входной поток. Если не указывать имя файла, то sed будет работать со стандартным вводом. Результат выводится в стандартный вывод и обычно перенаправляется в файл или конвейер. Если входных файлов несколько, то они объединяются в один буфер, с которым и идет потом работа. 

Строки в буфере пронумерованы. Если sed применяется к нескольким файлам, то номера строк будут продолжаться. Если первый файл содержит 200 строк, то адресом первой строки следующего файла будет 201. 

\emph{Примечание}. входные файлы \emph{не изменяются}.

\begin{center}
Схема работы:
\end{center}
Входные файлы (stdin) считываются в область шаблонов (pettern buffer), после этого к буферу последовательно применяются команды, и затем результат выводится для сохранения или дальнейшей обработки.

\emph{Внимание:} крайне не рекомендуется перенапрвлять результат в исходные файлы. Это приводит к непредсказуемым результатам.

\subsection{Общий вид команды}

\verb+[адрес1[, адрес2]] функция [аргументы]+

\emph{Функция}: представляет собой букву команды. Единственный обязательный параметр. Например \verb+'p'+.

\emph{Адрес}: может быть номер строки, регулярное выражение, \$ (последняя строка).

Если в адресе задается регулярное выражение, то оно задает все строки, соответствующие регулярному выражению. Регулярное выражение берется в \verb+/+ (прямой слэш), то есть \verb+/regexp/+.

Если не заданы адреса, то обрабатываются все строки буфера.

\emph{Адресный интервал} - это пара адресов, разделенная \verb+","+ и включающая все строки, начиная со строки, соответствующей первому адресу, до строки, соответствующей второму адресу включительно. Если второй адрес  раньше первого, то обрабатывается только первая строка, соответствующая первому адресу.

При добавлении символа \verb+!+ после адреса смысл меняется на противоположный: обрабатываются все строки, не лежащие в интервале.

Примеры: \verb+1,4; 1,$; 2,6!+.

\subsection{Команды sed}

Описаны только часто используемые команды.

\subsubsection{Замена} 

\verb+s/regexp/replacement/flags+. 

\verb+s+ - буква команды (замена, подстановка - substitute), \verb+regexp+ - строка поиска, то есть то, что заменится на \verb+replacement+. 

Флаги:
\begin{itemize}
\item \verb+g+  заменить все вхождения
\item \verb+w+ file  записать изменения в файл
\item \verb+p+  после замены вывести строку на экран (обычно используется с ключом sed \verb+-n+).      
\end{itemize}
\begin{verbatim}
Примеры: 
1. $ sed 's/sun/moon/g' myfile
2. $ sed '1,4 !s/sun/moon/g' myfile
3. $ echo Жужжали бабочки | sed -n 's/\(жж\)\(али\)/Гуж<&>\2\1/p'
# Результат: ЖуГуж<жжали>алижж бабочки
4. $ sed '/^Example/,/ED$/s/first/second/g' 
# если несколько совпадающих строк, то редактируются все вхождения.
5. $ sed 's/Sunday/Monday/gw' changes
# все Sunday заменяются
\end{verbatim}

\subsubsection{Удаление строк}
\verb+d+
\begin{verbatim}
Примеры: 
1. $ sed '4,5 d' file
2. $ sed '/sun/ !d' file.txt 
# удаление всех строк, кроме содержащих sun.
3. $ sed '/sun/,/moon/ d' myfile
# удаляется диапазон от первой строки, содержащей sun
до первой строки, содержащей moon
\end{verbatim}

\subsubsection{Вывод на экран}
\verb=p=  

Обычно используется с \verb+sed -n+ (иначе строки будут выводиться два раза).
\begin{verbatim}
Примеры: 
$ sed -n '/stroka/i !p'.
$ sed -n ' 1,4 p'
# вывести строки с 1 по 4 включительно
\end{verbatim}

\subsubsection{Трансляция символов} 

\verb+y/source_chars/dest_chars/+  

Замена символов по принципу "один к одному" (строки должны быть одной длины).
Пример:\verb+ $ sed 'y/abc/ABC/' file+

\subsubsection{Запись в файл} \verb+w file+ - пишет буфер в файл.

\subsubsection{Вставка файла} \verb+r file+ - вставка в выходной поток файла. Если его нет, то вставляется файл нулевой длины (без ошибки).

\subsubsection{Вставка строк} 
\begin{itemize}
\item \verb+адрес a+  помещает за обрабатываемой строкой
\item \verb+адрес i+  выводит до указанной строки
\end{itemize}

\emph{Примечание}: a и i разрешают использовать только один адрес.
\begin{verbatim}
Пример:
$ cat script
  3 a\
	Здесь добавлена\
	строка
$ who | sed -f script
 root
 stud1
 stud10
        Здесь добавлена
        строка
 stud11
\end{verbatim}                    

Символ экранирования \verb+"\"+ необходим, чтобы скрыть все символы конца строки кроме последнего.


% файлы и файловая система
\chapter{Файловая система ОС UNIX}

С точки зрения пользователя в ОС UNIX существует два типа объектов: файлы и процессы.

 Все данные хранятся в виде файлов, доступ к периферийным устройствам осуществляется через чтение/запись  в специальные файлы. 

При запуске программы ядро загружает соответствующий исполняемый файл, создает образ процесса и передает ему управление.

Во время выполнения процесс  может считывать или писать данные в файл. С другой стороны, вся функциональность ОС определяется выполнением соответствующих процессов. 

Таким образом, понятия файловой системы и процессов тесно взаимосвязаны.

\section{Базовые сведения о файловой системе}

В UNIX файлы организованы в виде \emph{древовидной структуры} (дерева), называемой \emph{файловой системой} (FS или file system).

\emph{Каждый файл имеет имя}, определяющее его расположение в дереве FS.

Корнем дерева является \emph{корневой каталог} (root directory), имеющий имя "/".

Имена всех файлов, кроме "/", содержат \emph{путь - список каталогов, которые надо пройти, чтобы достичь файла}.
Все доступное файловое пространство объединено в единое дерево каталогов, корнем которого является каталог "/". Таким образом, полное имя любого файла начинается с "/". Полное имя файла не содержит идентификатора устройства (HDD, CD-ROM или удаленного компьютера в сети), на котором он фактически находится. Символ "/" является разделителем в структуре каталогов.

Каждый файл имеет связанные с ним \emph{метаданные} (хранящиеся в индексных дескрипторах - \emph{inode}), содержащие все характеристики файла и позволяющие ОС выполнять операции над ним.
 	
Метаданные хранят \emph{права доступа}, \emph{владельца-пользователя} и \emph{владельца-группу}, указатели на дисковые блоки, хранящие данные. \emph{В метаданных нет} сведений об \emph{имени файла}.

\section{Типы файлов}

В UNIX существует шесть типов файлов, различающихся по строение и поведению при выполнении операций над ними:

\subsection{Обычный файл (regular file)}

Это наиболее общий тип файлов, содержащий данные в некотором формате. Для ОС это просто последовательность байт. Интерпретация содержимого производится прикладной задачей. \newline
Пример: текстовый файл, двоичные данные, исполняемый файл. Их можно просматривать командами \verb+cat имя+ и  \verb+less имя+.	

\subsection{Каталог (directory)}

Это файл, содержащий имена находящихся в нем файлов, а также указатели на метаданные этих файлов, позволяющие ОС производить операции над ними.
 
Каталоги определяют положение файла в дереве файловой системы, так как сам файл не содержит информации о своем местонахождении. Каталоги образуют дерево.

Пример:  
\begin{displaymath}
\verb+Номер inode+ 
\left(
\begin{array}{ll}
	\verb+10245 .+ \\ 
	\verb+12432 ..+ \\ 
	\verb+ 8672 file1.txt+ \\ 
	\verb+12567 first+ \\ 
	\verb+19678 report+ 
\end{array} 
\right)
\verb+Имя файла+ 
\end{displaymath}

Для работы с каталогами используются команды: \verb+ls+ с ключами \verb+-a+ и \verb+-l+, \verb+cd+, \verb+mkdir+, \verb+rm+, \verb+rmdir+, \verb+mv+.

Первые два байта в каждой строке каталога являются единственной связью между именем файла и его содержимым. Именно поэтому \emph{имя файла в каталоге называют связью}. Оно связывает  имя в иерархии каталогов с индексным дескриптором и, тем самым, с информацией.

\subsection{Специальный файл устройства (special device file)}

Обеспечивает доступ к физическому устройству. Различают символьные и блочные файлы устройств. Доступ к устройствам происходит путем открытия, чтения/записи в специальный файл устройства. \emph{Символьные файлы} позволяют небуферизованный обмен данными (посимвольно), а \emph{блочные} - обмен пакетами определенной длины - блоками. К некоторым устройствам доступ возможен как через символьные, так и через блочные файлы.

Для создания файлов устройств используется команда \verb+mknod+.

\subsection{FIFO или именованный канал (named pipe)}

Используется для связи между процессами. Подробно будет рассмотрен при описании системы межпроцессного взаимодействия (см. \ref{fifo}). 

\section {Связь (ссылка)}


\subsection{Жесткая ссылка}

Связь имени файла с его данными называется \emph{жесткой ссылкой} (hard link). 
Имена жестко связаны с метаданными и, соответственно, с данными файла, в то время, как файл существует независимо от того, как его называют в файловой системе. Такая система позволяет одному файлу иметь несколько имен в файловой системе.
\begin{verbatim}
Пример: 
$ pwd
/home/stud1
$ln first /home/stud2 second 
# создание жесткой ссылки.
\end{verbatim}
Все жесткие ссылки на файл абсолютно равноправны.  
 
Файлы \verb+first+ и \verb+second+ будут отличатся только именем в файловой системе. Изменения, внесенные в любой из этих файлов, затронут и другой, так как они ссылаются на одни и те же данные. Даже при переносе файлов в другой каталог все равно они будут жестко связаны.

\begin{displaymath}
\left(
\begin{array}{lcl}
	 \verb+/home/stud1+ 	&  & \verb+/home/stud2+ \\
 	\verb+10245 .+		&  & \verb+12563 .+ \\ 
	 \verb+12432 ..+ 	&  & \verb+12432 ..+ \\
	 \verb+8672  file1.txt+ &  & \verb+12672 a.out+ \\
	 \verb+12567 first+ 	& \longrightarrow \verb+12567(inode)+ \longleftarrow & \verb+12567 second+ \\
	 \verb+19678 second+ 	&  \downarrow  & \verb+9675  dir1+ \\
				& \verb+Данные файла+ &
\end{array}
\right)
\end{displaymath}

Файл существует в системе до тех пор, пока существует хотя бы одна жесткая связь, указывающая на него, то есть пока у него есть хотя бы одно имя. Например, простое удаления файла \verb+second+ не удаляет данные. Их можно достать через \verb+first+. 

В выводе команды \verb+ls -l+ вторая колонка показывает количество жестких связей файла.

Таким образом, жесткая связь не принадлежит к особому типу файлов, а является естественной формой связи имени файла с его метаданными.

Жесткие ссылки можно создать командой \verb+ln+ (link).

\subsection{Символическая ссылка}

Особый тип связи - символическая связь, позволяющая косвенно адресовать файл, в отличие от жесткой, обращающейся напрямую.
Символическая ссылка содержит в себе имя файла, на который ссылается, а не его данные.

Физическое расположение файлов различно. Размер \verb+symfirst+ - длина имени файла, на который ссылается символическая связь.
ОС работает с \verb+symfirst+ не так, как с обычным файлом: при обращении к нему появятся данные \verb+first+.

\subsection{Сокет (socket)}
Используются для межпроцессного взаимодействия. Будут подробнее рассмотрены в соответствующей теме (см. \ref{socket}.

\section{Структура файловой системы}

Все Unix-системы имеют сходную систему расположения и именования файлов и каталогов. Использование общепринятых имен файлов и структуры каталогов в UNIX-подобных ОС облегчает работу и перенос. Нарушение структуры ведет к нарушениям в работе.

Корневой каталог \verb+"/"+ является основой FS. Все остальные файлы и каталоги располагаются в рамках структуры, порождаемой корневым каталогом.

\emph{Абсолютное или полное имя} файла определяет точное местонахождение файла в структуре файловой системы. Начинается с \verb+"/"+ (в корневом каталоге) и содержит полный путь подкаталогов, которые нужно пройти, чтобы достичь файла.

\emph{Относительное имя} определяет местонахождение файла через текущий каталог. Никогда не начинается с \verb+"/"+.

\emph{Каталог-предок} - это тот, который содержит другой каталог. Две точки (\verb+..+) как имя каталога всегда относятся к каталогу, содержащему текущий каталог. Корневой каталог не имеет предка. Каталог, находящийся в другом каталоге, называется \emph{каталогом-потомком или подкаталогом}. К текущему каталогу можно обратиться по имени \verb+"."+. Например, \verb+./file1+.

\emph{Домашним или начальным каталогом} называется область, котрая выделяется каждому пользователю и в которой он может хранить свои файлы и программы.
К своему домашнему каталогу пользователь может обратиться по имени \verb+~+ (тильда). Например, \verb+~/file.txt+.

\subsection{Основные каталоги}
\begin{enumerate}
\item \verb+/bin+ - наиболее часто употребляемые файлы и утилиты.
\item \verb+/dev+ - содержит специальные файлы устройств, являющиеся интерфейсом доступа к периферийным устройствам. Может содержать подкаталоги, группирующие устройства по типам. Например, \verb+/dev/dsk+ - доступ к дискам.
\item \verb+/etc+ - системные конфигурационные файлы и утилиты. Иногда утилиты отсюда выносятся в \verb+/sbin+ и \verb+/usr/sbin+.
\item \verb+/lib+ - библиотеки Си и других языков программирования. Часть библиотек - в \verb+/usr/lib+. 
\item \verb+/lost+found+ - "каталог потерянных файлов", то есть потерявших свое имя при сбое, но существующих на диске.
\item \verb+/mnt+ - для временного связывания (монтирования) физических файловых систем к корневой для получения единой структуры.
\item \verb+/home+ - каталоги пользователей.
\item \verb+/usr+ 
\begin{itemize} 
 \item \verb+/usr/bin+ - утилиты;
 \item \verb+/usr/include+ - заголовочные файлы Си;
 \item \verb+/usr/man+ - справочная система;
 \item \verb+/usr/local+ - дополнительные программы;
 \item \verb+/usr/share+ - файлы, разделяемые между различными  программами.	
\end{itemize}
\item \verb+/var+ - временные файлы сервисных подсистем (печати, почты, новостей).
\item \verb+/tmp+ - каталог временных файлов. Обычно открыт на запись для всех пользователей системы.
\end{enumerate}

\section{Атрибуты файлов}

\subsection{Владельцы файлов}

Группой называется определенный список пользователей системы.
Пользователь может быть членом нескольких групп, одна из которых является первичной, а остальные - дополнительными.

\verb+/etc/passwd+ - список всех пользователей и их первичных групп;
\verb+/etc/group+ - список всех групп и их дополнительных пользователей.
В UNIX любой файл имеет двух владельцев:
\begin{enumerate}
\item владельца-пользователя
\item владельца-группу.
\end{enumerate}
При этом владелец-пользователь не обязательно принадлежит владельцу-группе. 

Команда \verb+ls -l+ выводит информацию о владельцах в третью и четвертую колонки. Для изменения владельцев используются команды:\newline
\verb+chown новый_влад. имя_файла+. Например: \verb+chown sys something.doc+.\newline
\verb+chgrp  новый_влад. имя_файла+. Например: \verb+chgrp adm something.doc+.

Сменить владельца-пользователя может либо текущий владелец, либо администратор (root). Сменить владельца-группу может либо владелец-пользователь для группу, к которой он сам принадлежит (POSIX), либо администратор.

\subsection{Права доступа к файлам}

У каждого файла существуют атрибуты, называемые правами доступа.
В UNIX существует три базовых типа доступа:
\begin{enumerate}
\item 	\verb+u+ (user) для владельца-пользователя
\item 	\verb+g+ (group) для владельца-группы         
\item  	\verb+o+ (other) для всех остальных 
\item 	\verb+а+ (all - объединяет 3 предыдущих класса). Для всех классов пользователей
\end{enumerate}

В каждом из этих классов установлены три основных права доступа:
\begin{enumerate}
 \item \verb+r+ (read) право на чтение           
 \item \verb+w+ (write) право на запись           
 \item \verb+x+ (execute) право на выполнение  
\end{enumerate}

В первой колонке вывода команды \verb+ls -l+ можно просмотреть установленные права.
\begin{verbatim}
Пример: 
$ ls -l 
 - r w - r - - r w x     1   stud1    students  ...  example.program
 0 1 2 3 4 5 6 7 8 9
0 - тип  файла: - обычный; d каталог; l символическая ссылка; 
    c,b символьный/блочный файл устройств.
1-3 - права доступа для владельца-пользователя.
4-6 - права доступа для владельца-группы.
7-9 - права доступа для остальных.
\end{verbatim}

Права может изменять владелец-пользователь и(или) администратор.
Для изменения прав доступа используется команда \verb+chmod+:
\begin{displaymath}
\verb+chmod+ \quad
\left[ \begin{array}{c}
 u \\ g \\ o \\ a
\end{array} \right]
\left[ \begin{array}{c}
+ \\ - \\ =
\end{array} \right]
\left[ \begin{array}{c}
r \\ w \\ x 
\end{array} \right]
\quad  \verb+файлы+ \qquad
\begin{array}{l}
+\ \verb+добавить права к текущим+ \\
-\ \verb+отнять права от текущих+ \\
=\ \verb+обнулить права и присвоить новые+
\end{array}
\end{displaymath}

\begin{verbatim}
Пример:
$ chmod a+w text  
# добавить разрешение писать всем пользователям;
$ chmod go=r text 
# установить только одно право на чтение для всех кроме владельца-пользователя;
$ chmod g+x-r program 
# добавить для группы право на выполнение и отнять у нее право читать;
$ chmod u+w, og+r-w text2;
\end{verbatim}

Возможно также задание прав через числовой формат в восьмеричной системе счисления.
% битовая структура 	                                          

\verb+Пример: chmod 666 *+.

\subsection{Значение прав доступа}

Для обычных фалов - очевидно: право на чтение надо, чтобы прочитать файл, право на запись, чтобы иметь возможность файл изменить, а право на выполнение, чтобы запустить программу или скрипт.

\emph{Примечание}. Для успешного запуска скрипта необходимо установить атрибут r, чтобы командный интерпретатор мог построчно считывать текст скрипта.

Для каталогов и символических связей интерпретация прав доступа проводится по-другому.

Права символических ссылок совпадают с файлом, на который она указывает. На самой ссылке стоит \verb+777+ (всем все) и это не имеет значения. 

Для каталогов \verb+r+ позволяет получить имена (и только имена) файлов, находящихся в данном каталоге. \verb+X+ позволяет "выполнить" каталог, то есть заглянуть в метаданные  и получить полную информацию о каталоге.
\begin{verbatim}
Пример:  
$ chmod u+r-x dir1
$ ls dir1      - выполнится
$ ls -l dir1  - Permission denied
$  cd dir1     - Permission denied (надо х).
\end{verbatim}

\verb+r+ и \verb+x+ для каталога действуют независимо (одно не требует другого).
\begin{verbatim}
Пример:  
$ mkdir dark_dir
$ chmod a-r+w dark_dir
$ ls dark_dir  -выполниться
$ ls -l        -нет  
$ cat file1       
# yes (заранее зная имя файла, можно обратиться к нему).
\end{verbatim}

Атрибут w должен быть установлен для того, чтобы можно было изменять каталог: создавать и удалять файлы.
Для удаления файла из каталога достаточно иметь установленный атрибут w для каталога, в котором он находился, а права файла при этом не учитываются.

\subsection{Последовательность проверки прав}

\begin{enumerate}
	\item если вы администратор (root), доступ разрешен. Права не проверяются.
	\item если операция запрашивается владельцем, идет проверка его прав. В соответствии с ними ему разрешается выполнение операции или нет.
	\item если операция запрашивается пользователем, входящим в группу, владеющую файлом, идет проверка его прав. Соответственно, он либо получает разрешение, либо нет.
	\item аналогично для всех остальных пользователей.
\end{enumerate}
\begin{verbatim}
Пример: 
----rwr--  2  stud1   students ... file1  
stud1 в доступе будет отказано, но он, как владелец, 
может в любой момент сменить права доступа. 
\end{verbatim}

\subsection{Дополнительные атрибуты файла}

Для обычных файлов:
\begin{itemize}
	\item \verb+t+ - \emph{"sticky bit"} (бит липучка)- сохранить образ выполняемого файла в памяти после выполнения (устаревший аттрибут)
	\item \verb+s+ - set UID, \emph{SUID} - установить права у процесса, как у запущенного файла, а не как у пользователя, запустившего программу (по умолчанию)
	\item \verb+s+ - set GID, \emph{SGID} - то же для группы
	\item \verb+1+ - блокирование - в каждый момент времени с файлом может работать только одна задача
\end{itemize}

Для каталогов:
\begin{itemize}
	\item \verb+t+ - пользователь может удалять только те файлы и каталоги, которыми владеет или имеет право на запись;
	\item \verb+s+ для создаваемых файлов группа-владелец наследуется от каталога-предка (а не от первичной группы пользователя, создающего файл).
\end{itemize}

Дополнительные атрибуты также устанавливаются с помощью \verb+chmod+.

 
% процессы
\chapter{��������}

\emph{�������} - ��� ��������� ������������� ���������.

��������� - ������������ ������, ���� �� ���������, ��������� ����� ���� ����������� ����.

��� ������� ���������  �� ���������� ��  ������ ������� \emph{���������} ��� ����� ���������� ������, ���� ��������� ������� ������, ����������� ������� � ����������� �����/������ � ��������� ��������� ��������.

������� ������� �� ����������, ����������� �����������, ������ � ���������� � ����������� ������, �����, ��� ����������� ������, �������� ����� � ������ ��������.

��������� ����� �������� ����� ������ ��������. ������������ ����� ��������� ��������� ����������� ����� ���������. ��������, ���������� BASH - ������������ ����� �������������. ����� ������� UNIX - ������������� ��.

���������� �������� ����������� � ������ ���������� ������ ����������, ������� ������� �� �������� ���������� ������ ���������� ������� ��������. ������� ��������������� �� ������ ������� � ������, �� ��� �� �������� ����� ������ � ����. 

�������� ����������� ���� �� �����. � �� �� �����, �������� ����� ����������� ������������ ���� � ������ ������� � ������� ������� �������������� �������������� \emph{(IPC)}.

���� IPC:
\begin{enumerate}
\item �������
\item ������
\item ����������� ������
\item ��������
\item ���������
\item �����.
\end{enumerate}

\section{���� ���������}

\subsection{��������� ��������}

\emph{��������� ��������} �������� ������ ���� � ������ ����������� � ����������� ������. ��������� �������� �� ����� ��������������� �� �������� � ���� ����������� ������ � ����������� ������ ������� ��� ������������� ���� �������.

\begin{verbatim}
�������:
1)��������� ��������
2)��������� ������
� ������.
\end{verbatim}

����������� ���������� � ������ ��������� ��������� ��������� � ����, ����� �������, ��� ����� ���������� � �������� � ������, �� ��������� ����� (����). 

������� \verb+init+ ����� ����� ������� � ���������, ���� �� ����������� �� �����. �� ����������� ���� ��������� �������. \verb+init+ ����������� ������ ����� �������� ���� � ��������� ��� ��������� ����������������� ������ �� ����������.

\subsection{������}

\emph{������} - ��������������� ��������, ����������� ������� �������, � ����������� � ������� ������. ��� �� ������� �� � ����� ���������������� ������� � �� ����� ��������������� ����������� �������������. ������������ ������ ��������� ���������: \begin{itemize}
\item ������
\item �������� �������
\item ������������� �������
\item �����
\item web-�������
\item ����.
\end{itemize}

\subsection{���������� ��������}

\emph{���������� ��������} - ��� ��������� ��������. ��� �������, ����������� � ������ ����������������� ������. 

\verb+������: ls, BASH.+

���������������� �������� ����� ����������� ��� � �������������, ��� � � ������� ������, �� ����� �� ����� (����������) ���������� ������� ������ ������������. ��� ������ �� ������� ��� ���������������� �������� ����� ����������.

\emph{����������}: ������������� �������� ���������� ������� ����������, �, ���� ����� ������� �� �������� ����������, ������������ �� ����� �������� � ������� ������������. (����� �������, ����� ���� ����� ������� ������ ��������� �� ����� �������������� ��������.)

\section{�������� ���������}

�������� ��������� �� ���������� ��������� ������� ��������.

�������� ��������� ��������: \verb+ps -ef+.

\subsection{������������� ��������}

������������� �������� - \emph{Process ID (PID)} - ������ ������� ����� ���������� �������������, ����������� ���� ������� ��������� ��������.
��� �������� ������ ��������, ���� ����������� ��� ��������� ��������� �������������. ���������� PID - �� ������������, �� ���� PID ������ �������� ������, ��� PID ��������, ���������� ����� ���. 

����  PID ������ ������������� ��������, ��������� ������� ������� ����������� ��������� � ���� �����������.

����� ������� ��������� ������ - ���� ����������� ������� �� PID.

\subsection{������������ �������}

������������� ������������� �����c�� -Parent Process ID (PPID) -  PID ��������, ����������� ������.

\subsection{��������� ��������}
��������� �������� (nice number) - ������������� ��������� ��������, ����������� ������������� ��� ����������� ����������� �������. ��� ������ �����, ��� ������ ��������� (nice - ��������, �� ���� ��� ����� "��������" �������, ��� ������ �� ��������� CPU).

����������� ������������� �������� - ��������� ����������: ����������� ���������� ����� �� ����� ����������. ������������� - ���������, �� ����� ���������� ��������������� ��� ������������� � ������� nice.

\subsection{������������ �����}
������������ ����� (TTY) - �������� ��� ��������������, ��������������� � ���������. 

\emph{����������}. ������ �� ����� ���������������� ���������. 

\subsection{�������������� �������������}

�������� (RID) �  ����������� (EUID) �������������� ������������. RID - ������������� ������������, ������������ ���� �������. EUID ������ ��� ����������� ���� ������� ��������� ��������� �������� (� ������ ������� � �������� �������.)
������ RID=EUID, �� ���� ������� ����� �� �� �����, ��� � ������������, ����������� ���. RID!=EUID, ����� �� ��������� ���������� ��� SUID. ����� EUID=UID, �� ���� ������� �������� �� �� �����, ��� � � ��������� ������������ ����� (��������, �������������).

\subsection{�������������� �����}

�������� (RGID) � ����������� (EGID) �������������� ������. RGID=GID ��������� ������ ������������, ������������ �������. EGID ������ ��� ����������� ���� ������� ������������ �� ������ ������� ������. �� ��������� RGID=EGID, ����� SGID, �������������� �� �������, ����� EGID=GID ������-��������� �������.

\section{��������� ���� ���������}

������� � UNIX ��������� ��������� ������� \verb+fork(2)+. 

�������, ��������� ����� fork(2), ���������� \emph{������������}, � ����� ��������� - \emph{��������}. ����� ������� �������� ������ ������ ����������� ��� ��������. 

\emph{����������}. ����� ������� ����� �� �� ���������� � ������, ��� � ��������. ����� ����, ���������� ������������� � ��������� �������� � ����� � ��� �� �����������, ��������� �� ��������� ������� fork. ������������ �� ������� - ������������� PID.

������ ������� ����� ������ ��������, �� ����� ����� ��������� ��������.

��� ������� ������, �� ���� �������� ����� ���������, ������� ������ ������� ����� \verb+exec(3)+. ��� ���� ����� ������� �� �����������, � ����������� ��� ������ �������� ��������� ���������� ����� ����������� ���������. ��� �� ����� ����������� �������� ���������� ���������, ���������� ����������� ������� �����/������ � ������, � ����� ��������� ��������.

� UNIX ������ �� ���������� ����� ��������� ����� ������ � ����������� ������ ��������. ����� �������, ������� ������� ��������� fork, �������� �������� �������, ������� ����� ��������� exec, ��������� ������� ������������ �������. ����� ��������� ������� ���������� fork-and-exec. 

������ ��������, ����� ���������� ������ ������ fork ��� ������������ exec. � ���� ������ ����������� ��� ������������� � ��������� ��������� ������ ��������� ���������� ��������� ��� ������������� � ��������� ���������. (fork ���������� PID ������������ �������� � ������������ � ���� - � ��������.)

��� �������� ��������� ����� ����� fork. ������ �������������� ���� �� fork-and-exec, ���� � ������� exec.
������������ ���� ��������� �������� init ��� �������������� ���������.


\section{�������}

������� �������� �������� �������� ����������� � ������������� ������-���� �������. ������ ����� ���� �� ������ �������� ������� ��� �� ���� �� ������-���� ��������.
 
������� - ���������� ����� IPC. 

��������, ��� ������� �� ���� �������� ���������� ������ SIGFPE, � ��� ������� Ctrl+C �� ��������� �������� �������� ���������� ������ SIGINT.

��� �������� �������� ������������ ������� \verb+kill+: \\
\verb+$ kill sig_no pid+, ��� \\
sig\_no - ����� ��� ���������� �������� �������;\\
pid - ������������� ��������, �������� ���������� ������.

������������ ����� �������� ������� ������ ��� ���������, ���������� ������� �� ��������, �� ���� RID � EUID ��������� � UID ������������. ������������� (root) ����� �������� ������� ���� ���������.

\begin{verbatim}
������: ������� ������� ��������, ������ ��� ����������� � ������� ������
$ back_fone_prog &
$ kill $! (�� ��������� ���������� SIGTERM, ����� 15).
\end{verbatim}

��� ��������� ������� ������� ����� ����������� ��������� �������:
\begin{enumerate}
\item ������������ ������.\newline
	\emph{���������:} �� ������� ������������ ��������� ��������� �������, ��������, SIGFPE.
\item �������� �� ���������. ������ ��� ���������� ������.
\item ����������� ������ � �������������� ���������� ���. ��������, �������� SIGINT �������� ������� ��� tmp-����� � ��������� ��������� ����������.
\end{enumerate}

\emph{����������:} SIGKILL � SIGSTOP ������ �� �����������, �� ������������.

�������� ��������, ����� ������� �� ��������� �� SIGKILL:
\begin{enumerate}
\item ��������-�����. ���������� �� ����������, �� �������� ������ � ��������� ������� ���������.
\item ��������, ��������� ����������� ������� NFS. ��������, �������� ������� ������ � ���� ���������� ����������, ������� ��� ����������. �������� ������ �������� SIGINT ��� SIGQUIT. 
\item �������, ��������� ���������� �������� � �����������, ��������, ��������� ����� ��� �������������������� ������� CD-ROM �� ��������� �����.
\end{enumerate}

������� ������������ �� ������ ��� ���������� ������ ���������, �� � ����� ����� ������������� �������� ��� ����������. (��� �� ��������� � SIGKILL � SIGSTOP, ������ ��� �� ������ �����������.)
� �������, ��������� ������ - proxy servers, smtp (pop, imap), ����, bind ��� ��������� ������� SIGHUP ������ ���������� ���� ���������������� ����� � ������������. 

% Unix - среда программирования
\chapter{Среда программирования Unix}

\section{Unix-way программирования}

\section{Unix как единая среда разработки (IDE)}
Из высказывания в ru.unix.prog  Алексея Махоткина из ru.unix.prog FAQ :
\begin{quote}
 >> Q: Какие есть IDE (integrated development environments) под Unix?  Hу\\
 >> чтобы компилятор, среда редактирования, отладчик и прочее - были все\\
 >> вместе?\\
 
\dots

UNIX сам по себе является Integrated Development Environment.

В "обычных" IDE есть бинарник-интегратор, который вызывает в лучшем случае
внешние утилиты, а в худшем случае -- свою реализацию каждой функцию из DLL
или прямо зашитую в бинарник.

В UNIX таким бинарником-интегратором является shell (Emacs считается
shell'ом в данном случае).  Для выполнения каждой функции вызываются
специально написанные динамически выполняемые модули, такие как make, cc,
ld, и т. д.

Преимущество в этом такое же, как преимущество математических функций
высшего порядка перед "обычными" функциями.

Hапример, функция "отслеживать зависимости" чаще всего реализуется с
помощью make, но можно также легко использовать, скажем, cook, или же
переключаться между GNU Make и BSD Make по вкусу.  Точно такая ситуация с
используемыми редактором, компилятором, etc.  Более того, сам по себе shell
является "функцией высшего порядка", и легко может быть заменен.

Кроме того, так как пространство функций практически неограниченно, то IDE
"Unix" обеспечивает также заранее не предусмотренные функции высшего
порядка, например, различную автогенерацию кода, поддержку тестирования и
т. п.
\end{quote}

Другими словами, командная строка Unix (shell) и является IDE для Unix. Подобный подход позволяет не зацикливаться на програмных решениях одного производителя (разработчика). Любой компонент, воспринимаемый как часть IDE (компилятор, отладчик, компоновщик, редактор, ассемблер, утилиты сборки и тестирования проекта, система контроля версий) может быть заменен на другой. Эти компоненты мы условно можем назвать \emph{Инструментальные средства Unix}.

\section{Низкоуровневый доступ к системе}

Базой UNIX-системы является компилятор Си (сс), библиотека libc  и ядро.

Все версии UNIX предоставляют строго определенный, ограниченный набор входов в ядро ОС, через которые прикладные задачи получают доступ к базовым услугам UNIX. Эти точки входа называются \emph{системными вызовами (system calls)}.

Системный вызов определяет функцию, выполняемую ядром ОС от имени процесса, выполнившего вызов. Syscall является интерфейсом самого низкого уровня взаимодействия прикладных процессов с ядром.

Язык системного программирования - Си. Характерная особенность Unix - ассемблер практически не используется. Более того, часть прерываний и регистров просто недоступна из program space. Ассемблер применяется только для написания драйверов устройств и ядра (платформенно-зависимые их части). 

Библиотека \emph{libc} - это набор интерфейсов к системным вызовам и различных функций, работающих поверх системных вызовов. Для программиста различие между \emph{системным вызовом} и \emph{библиотечной функцией} лишь в том, как они взаимодействуют с ядром. Системный вызов сразу уходит в пространство ядра и там выполняется. Библиотечная функция выполняется в пространстве процесса (хотя конечно может и делать системные вызовы в ходе выполнения).


\section{Принципы разработки программ для Unix}
\begin{quote}
Чтобы плавать, надо плавать. Мао Цзе Дун (из красных книжечек председателя Мао)
\end{quote}
За 30 лет существования вокруг Unix сформировалась своя культура: слэнг, традиции и опыт, передаваемый между поколениями программистов. Принципы - это набор философских высказываний, подводящий итоги и суммирующий опыт тысяч человеко-лет разработки.

Из Tao Of The Unix Programming (by Eric S. Raymond):
\begin{itemize}
\item Правило модульности: Пишите простые части, соединяемые ясными интерфейсами.
\item Правило ясности: Ясность лучше чем изощренность.
\item Правило соединения: Проектируйте программы чтобы они могли взаимодействовать с другими программами.
\item Правило разделения: Отделяйте алгоритмы от механизмов и интерфейсы от движков.
\item Правило простоты: Разрабатывайте просто. Используйте сложные конструкции только тогда, когда без этого не обойтись.
\item Правило умеренности: Пишите большую программу, только если ясно, что больше ничего не поможет.
\item Правило прозрачности: Пишите наглядно, чтобы сделать просмотр и отладку программы легче.
\item Правило надежности: Надежность - следствие ясности и простоты.
\item Правило представления: Храните знания в данных, а программная логика должна быть надежной и тупой.
\item Правило наименьшего удивления: Когда разрабатываете интерфейсы, делайте их как можно более предсказуемыми.
\item Правило тишины: Когда программе нечего сказать нового, она должна молчать.
\item Правило восстановления: Когда программа должна ошибиться, она должна шумно и долго об этом вопить. Настолько часто, насколько это возможно.
\item Правило экономии: Время программиста - дорогое, экономьте его, взваливая как можно больше задач на компьютер (автоматизация).
\item Правило генерации кода: Избегайте ручного кодирования. Пишите программы для генерации других программ всегда, когда возможно.
\item Правило оптимизации: Делайте прототипы перед полировкой кода. Заставьте код работать, прежде чем оптимизировать.
\item Правило многообразия: Не доверяйте всем претензиям на "единственно верное решение"
\item Правило рамширяемости: Проектируйте с прицелом на будущее. Оно может наступить быстрее чем вы думаете.
\end{itemize}

Философия Unix в одном уроке.

\emph{keep It Simple, Stupid - оставь это простым, тупица.}



% Инструментальные средства 
\chapter{Инструментальные средства разработчика}

\section{Компилятор Си}

Компилятор языка Си (C compliler или cc) - неотъемлемая часть системы. С него начинается разработка любой версии Unix или перенос на новую платформу существующей. Ядро и базовые утилиты системы написаны на Си\footnote{Более того, сам язык Си был создан для разработки ядра Unix. Авторы языка Си являются также первыми разработчиками Unix.}.

Компилятор реализован как утилита командной строки. Он вызывается командой \verb+cc+. Существует большое количество системных компиляторов Cи \footnote{В SUSv3(POSIX) опредено общее подмножество ключей и параметров, которые должен поддерживать компилятор Cи. Этому набору следуют все системные компиляторы. Естественно, что каждый из них имеет и свои дополнительные параметры.} (gcc в Linux и FreeBSD, собственные компиляторы в большинстве коммерческих версий Unix), поэтому \verb+cc+ будет указывать на компилятор по умолчанию для нашей системы. 

\begin{verbatim}
типовые ключи компилятора
$сс foo.c
компилирование и сборка (линковка) программы из 'foo.c' 
создаётся исполняемый файл 'a.out'

$cc -c foo.c
только компилировать. Будет получен объектный модуль 'foo.o'

$cc -o exec_foo foo.c
создается исполняемый файл 'exec_foo' (вместо 'a.out')

$cс foo.c bar.o
слинковать 2 объектных файла в исполняемый ('a.out')
\end{verbatim}


\section{make}

\emph{Make} - стандартное средство, применяемое для сборки программных проектов. Является универсальной программой для решения задач автоматической генерации и изменения файлов с учетом зависимостей.

Схема работы: make читает файл с описанием проекта (makefile) и, интерпретируя его содержание, выполняет определенные действия.

Makefile - текстовый файл, описывает отношения между файлами проекта и действия, необходимые для его сборки.

\subsection{Запуск}

Make является утилитой командной строки и запускается командой \verb+make+. При запуске \verb+make+ проверяет наличие файлов \verb+makefile+, \verb+Makefile+ и если и запускает на обработку первый из найденных Make-файлов. Можно явно указать какой Makefile использовать ключом \verb+-f+.

\subsection{Формат и использование make-файлов.}

Основной элемент - \emph{правила (rules)}.

\begin{verbatim}
Общий вид:
<цель 1> <цель 2> ?<цель n>:<зависимость 1> <завис-ть 2>?<завис-ть n>
        <команда 1>
        <команда 2>
        ?
        <команда n>
\end{verbatim}

\emph{Цель (target)} - некий желаемый результат, способ достижения которого описан в правиле. Цель может быть именем файла.

\emph{Примечание}. Перед командами вставляется табуляция, чтобы \verb+make+ отличал их от целей.

\begin{verbatim}
Пример1: цель как имя файла
iEdit: main.o Editor.o
         gcc main.o Editor.o -o iEdit
Пример описывает, как можно получить исполняемый файл из объектных модулей.
\end{verbatim}

Цель может быть именем некоторого действия, тогда правило описывает, как совершается указанное действие.
\begin{verbatim}
Пример2: цель как имя действия
clean:
        rm *.o iEdit
\end{verbatim}

Такие цели называют \emph{абстрактными} (phony targets) или псевдоцелями (pseudo targets).

\emph{Зависимость (dependency)} - это 'исходные данные', необходимые для достижения указанной в правиле цели. Это предварительные условия достижения цели. Зависимостью может быть имя файла или имя действия. В примере1 main.o и Editor.o - зависимости. Файлы должны существовать, чтобы можно было собрать iEdit.
\begin{verbatim}
Пример3:
clean_all: clean_obj
        rm iEdit
clean_obj:
        rm *.o
 Для достижения сlean_all необходимо выполнить действие clean_obj.
\end{verbatim}

\emph{Команда} - действия, которые надо выполнить для обновления или достижения цели. Перед командой должен быть символ табуляции (код 9). Так make определяет команды.

Типичный makefile, который содержит несколько правил, у каждого правила есть некоторая цель и зависимости.
\begin{verbatim}
Пример4: 
1. 	iEdit: main.o Editor.o
2.              gcc main.o Editor.o -o iEdit
3.       main.o: main.cpp
4. 	        gcc -c main.cpp
5.	Editor.o: Editor.cpp
6.	        gcc - Editor.cpp
7.	clean:
8.	        rm *.c
\end{verbatim}

Смысл работы - достижение главной цели (default goal). Если  цель - имя действия (абстрактная), то выполняется действие. Если главная цель - имя файла, то make строит самую свежую версию. Главная цель обычно задается как параметр make: make iEdit, make clean. Если make вызывается без параметров, то в качестве главной берется первая встреченная цель. (В примере  это iEdit). Обычно задают цель \emph{all} как цель по умолчанию.

Алгоритм работы:
\begin{enumerate}
	\item выбор главной цели
	\item достижение цели
	\item обработка правил
	\item обработка зависимостей
\end{enumerate}

Достижение цели - проверяет зависимости и потом определяет, надо ли запускать команды.
При вызове make iEdit определяет, что главная цель - iEdit. Правило ее достижения - строки 1,2. Обрабатывая правило iEdit, определяем, что зависит от main.o и Editor.o. Для этих зависимостей существуют правила (3,4) и (5,6). main.o зависит от main.cpp. Если нет еще объектного файла, но существует файл .срр, то запускается компиляция. Аналогично и для Editor.o. Для clean зависимостей нет и make сразу переходит к выполнению.

\emph{Инкрементная сборка} - перекомпилируется только то, что было изменено. Для файлов .с и .срр обычно указываются как зависимости .h файлы.

\subsection{Переменные make.}

Присвоение: \verb+имя = строка+ (можно с пробелами). 

Получение значения переменной: \verb+$(имя)+. Значение - текстовая строка, может содержать ссылки на другие переменные.
\begin{verbatim}
Пример: 
obj_list = main.o Editor.o 
# присвоение; 

$(obj_list) 
# получение значения

1)dir_list = . .. src/include
all:
        echo $(dir_list) 
2)optimize_flags  = -03
compile_flags = $(optimize_flags) -pipe
all:
        echo $(compile_flags)
Результат: -03 -pipe
3)program_name = iEdit
obj_list = main.o Editor.o TextLine.o
$(program_name) : $(obj_list)
        gcc $(obj_list) -o $(program_name)
\end{verbatim}

\emph{Примечание}. Значение переменной вычисляется в момент использования.

Часто используемые переменные:
\begin{enumerate}
\item \verb+CC+ - указать компилятор по умолчанию.
\item \verb+CFLAGS+ - параметры компиляции
\item \verb+LDFLAGS+ - параметры линковки объектных файлов
\end{enumerate}

\subsubsection{Автоматические переменные}

\emph{}
\begin{itemize}
\item \verb+$^+ - список зависимостей, разделённых пробелами
\item \verb+$@+ - имя цели (файла). Если у нас несколько целей (см \ref{PatternRules}), эта переменная принимает значение той цели, для которой выполняется шаблон в конкетный запуск
\item \verb+$<+ - имя первой зависимости
\end{itemize}
\begin{verbatim}
Пример:
$(program_name):$(obj_name)
        gcc $^ -o $@
\end{verbatim}

\subsection{Шаблонные правила}
\label{PatternRules}
Шаблонные правила (implicit или pattern rules) применяются к группе файлов.

Синтаксис: 
\begin{verbatim}
.<расширение_файлов_завис.> .<расширение_файлов_целей>:
        <команда 1>
        <команда 2>
        ?
        <команда n>	
\end{verbatim}
\begin{verbatim}
Пример:
.cpp .o:
      gcc -c $^
\end{verbatim}


\section{Системы управления версиями. CVS}

Очень часто над программой работает больше одного человека. Выходят различные версии программ.  И существует потребность как то упорядочивать внесение изменений и дополнений. Для этого служат системы управления версиями. Из используемых сейчас можно назвать CVS, Subversion, RCS, Monotone, Arch, GIT и SourceSafe. 

\verb+CVS+ - \emph{Conhurent Versions Systems} (система управления конкурирующими версиями). 

\subsection{Репозиторий}

\emph{Репозиторий CVS (или хранилище)} хранит полную копию всех файлов и каталогов под управлением CVS,  включая все сделанные

Обычно Вы никогда не получаете доступ к файлам в CVS напрямую. Используются команды CVS для получения копии в "рабочий каталог" и далее работа идёт над копией. 
После внесения изменений - юзер вносит изменения в репозиторий. После этого, в хранилище сохраняется информация о сделанных изменениях, времени внесения изменений и другая подобная информация.

Для указания, какой из репозиторев используется - применяется переменная окружения CVSROOT, либо явно указывается с помощью ключа -d.

Примеры:
\begin{verbatim}
$ CVSROOT=/var/cvs; export CVSROOT
$ cvs checkout module/project
или
$ cvs -d /var/cvs module/project
\end{verbatim}

Кроме локальных репозиториев - очень часто используются удалённые (сетевые). Для них необходимо указать адрес и (иногда) - способ доступа.
\begin{verbatim}
$ cvs -d server1:/root checkout sdir1
\end{verbatim}

Подробнее об этом - в \ref{cvsbook} или в её русском переводе на http://opennet.ru

\subsection{Получение рабочей копии исходников}


\subsection{Сохранение результатов и версионирование}

\subsection{Коллективная работа над проектом}



\section{Библиотека Си (libc)}

libc содержит 2 части: \emph{системные вызовы} и \emph{библиотечные функции}. 

Системные вызовы определены, как функции языка Си (независимы от фактической реализации в ядре). В UNIX каждый системный вызов имеет соответствующую функцию (или функции) с тем же именем, хранящуюся в стандартной библиотеке Си. Функции из библиотеки выполняют преобразования аргументов и вызов соответствующего кода ядра. Таким образом, библиотечный код - только оболочка, фактические инструкции находятся в ядре.

Функции общего значения - также часть библиотеки, но не являются системными вызовами. Функции общего назначения и системные вызовы - основа среды программирования UNIX.

В отличие от других библиотек, libc линкуется с каждым приложением, написанном на Си.

Информация о системных вызовах и функциях содержится в 2 и 3 разделах \verb+man+ соотвественно. В различных системах различный набор системных вызовов, поэтому некоторые функции могут быть реализованы как библиотечные в одной системе и как системные вызовы в другой. 

libc полностью включает в себя библиотеки, определенные в ANSI C (stdio, math, assert). Как следствие: один из основных методов сделать программу переносимой - это написать ее на ANSI C. Такая программа будет компилироваться и работать на всех unix-системах.

Подробнее интерфейсы libc будут рассмотрены в главах, посвященных архитектуре и межпроцессному взаимодействию.


% IPC.Процессы и сигналы
\chapter{�������� � �������.IPC}

� UNIX �������� ����������� � ����������� �������� ������������ �
����������� ���� �� �����, ����� ������� ������� � ��������
����������� ������� ��������� ���� �� �����.

�� ���������� ������������� �������������� ���������. ��� �����
���������:
\begin{itemize}
\item ���������� �������� ��������������
\item ��������� ������������� ������� ������ �������� �� ������
\end{itemize}

�������������� ������ ��������� ������:
\begin{enumerate}
\item �������� ������
\item ���������� ������������� ������
\item ���������
\end{enumerate}

������ �������� �������������� ���������� ��������� � ������
������������� ������� ������.

\subsection{���� IPC}
\begin{itemize}
 \item �������
 \item ������
 \item FIFO (����������� ������)
 \item 0x08 graphic
 \item 0x08 graphic
 \item ��������� (������� ���������)
 \item ��������
 \item ����������� ������
 \item ������
\end{itemize}
       
\subsection{�������}

���������� ��� IPC. ��������� ���������� ������� ��� ������ ���������
� ����������� ���������� �������.

������ ��������� - ����� ������� ����������� ������������ ������
���������. � ������ ������ ���� ���� ���������� �������������. �����
������ - �������, PID �������� ��������� � ID ������. ������ ������
����������� ��������� �� ��������.

������� ����� �������� ������ � ������� ����.

����������� �������� - ������� ����� ���� ������ � ����������, �������
���������� �����������. ��� �������� ������ ����� ���� � ��� ��
����������� ��������.

����������� ���� ���������� /dev/tty- ������ � ����������� ����������
��������. ������� ��� ����� ���������������� �������������� ������� ��
����������� ������������ �������, ������� ����� ���� ��������� ���
������ ���������.

������ - �������� ������ ������������ ��������� ��� �����������
���������� �������.

���� ��� ���� � ������������� ��������: ��������� (�����������),
�������� � ���������. � ���������� - �������� ��������.

����������: ������� �� ����� �������������, �� ���� � ����� ����������
������ ������� ���������� ��������� ����� ������ ����������� �������.

������� ���������� ��������:

  * ������ �������� (��������, ������� �� 0)
  * ������������ ���������� (������� ������ Del, Ctrl+Z, Ctrl+C,
    ���������� ���������)
  * ������ ��������
  * ���������� ��������� (��� ��������� ���������������)
  * ����� (���������� ��������� ����)
  * ����������� (������� ����������� ���������� � ����������
    ����������)
  * ������.

��� ��������� ����� ��������� ��� ��������:

 1. ��������� ������� �� ������ (���������)
 2. ������������ ������� - ������������ ��������� �� ����� ����������
    ����������� �������� ����
 3. ������� �������.

����������: ������� �� ����� ��������������� ���������� ����������.

������� ������� ��������� ������������� ��� (��������, SIGINT),
������� ���������, ��� ���� ������ ������������ ������ ����� ����.
����� �������� ���������� � <signal.h>.

������� ��������:

#include<sys/types.h>

#include<signal.h>

int kill (pid_t pid, int sig);

������: kill (7421, SIGTERM).

������� ����� �������� ������� ������ ����.

��������� ������ PID - pid=getpid();

��������� ������ PPID - ppid=getppid();

�����������: EUID ��� RID ��������, ����������� ������, ������
��������� � EUID � RID ��������-��������. ��� �������������� �����
����������� ���.

��� ��������� ������ kill ���������� -1 � ���������� errno
������������� ��������: EPERM (������ ������� ������ ��������), ESRCH
(������ �������� ���) , EINVAL (sig �������� �������� ����� �������).

����� ��������� PID:

Pid==0 - ������ ���������� ���� ��������� ������, � �������
����������� �������, ��������� ������;

Pid==-1 � ���� EUID �� ��������������, �� ���������� ���� ���������,
RID ������� ����� EUID ����������� ��������, ������� � ��� (���� ���
RID=EUID);

Pid==-1 � EUID ��������������, �� ������ ���������� ���� ���������,
����� ��������� ���������;

Pid<0 � �� ����� -1 - ���������� ���� ���������, ������������� ������
������� ����� �� ������ PID, ������� ��������� �������, ���� �� �����
������ � ��� ������.

������� ������ ����:

#include<signal.h>

int raise (int sig) - ����������� �������� ���������� ������. � ������
������ ���������� 0. ��������, raise (SIGKILL).

setitimer - ��������� ������ �������� (3 ����).

#include<unistd.h>

unsigned int alarm (unsigned int secs);

secs - ����� � ��������, �� ������� ��������������� ������. �����
��������� ������� �������� ���������� SIGALRM.

������: alarm(60).

���������� ������� alarm(0).

����� ������� �� �������������. ����� ���������� �������� ����������.

#include<unistd.h>

int pause (void) - ���������������� ���������� �������� �� ���������
������ �������, ����� ������������ ������ � alarm.

                  ���������� � ��������� ����������.

����������� �������� �������� ���������� ���������� (normal
termination). ������ �� ����� ��������� ������� exit.

��������� ������� ( SIGABRT, SIGBUS, SIGQUIT, SIGILL � ������)
���������� ��������� ���������� �� ������� �������� ����������,
��������� � �.�. � core (dump).

                    ����� �������� (�� ��������).

<signal.h>

SIGABRT - ���������� �������� (abort), ���������� �������� ��� ������
���������� ������ abort. Core dump.

SIGALRM - ������ �������.

SIGBUS - ���������� ������ �� ����. ��������� ����������.

SIGCHLD - ������� ��� ���������� ��������� ��������. ������������.

SIGCONT - ����������� ������ �������������� �������� (�������� ���
SIGSTOP).

SIGHUP - ������������ �����, ���������� ���������, ������������ �
������������ �����, ��� ���������� ��������� ��� ��� ���������� ������
������ ������ ������ ������.

SIGILL - ������������ ������� ����������. Core dump.

SIGINT - ���������� ���������� ��������� (�trl+C). ���������� ����
��������� ������.

SIGKILL - ����������� ����������� ��������. �� ���������������.

SIGPIPE - ������� ������ � ����� ��� �����, ��� �������� �����������
������� ��� �������� ������.

SIGPROF - ������ �������������� �������.

SIGQUIT - ���������� ���������. ����� �� SIGINT, �� ����������
���������.

SIGSEGV - ������������ ����� ������. ��������� �����.

SIGSTOP - ������ ��������. ���������� ���������. ������ �����������.

SIGSYS - ��������� ��������� �����.

SIGTERM - ����������� ������ ����������. ������������ ��� �����������
���������� ��������.

SIGTSTP - ������������ ������ ��������� (�trl+Z). ����� �� SIGSTOP, ��
����� �����������.

SIGTTIN - ������� ����� � ��������� ������� ���������. ���������
��������.

SIGTTOU - ������� ������ �� �������� ������� ���������. ���������
��������.

SIGURG - ����������� � ����� ������ ������� ������������ ������.

0x08 graphic
0x08 graphic
SIGUBR1

SIGUBR2

SIGVTALRM - ����������� ������.

SIGXCPU - ���������� ������ ������������� �������. Core dump.

SIGXFSZ - ���������� ������ �� ������ �����.

                           ������ ��������.

����� �������� - ��� ������ ��������, ������� ���������� ��������
���������� ������.

��� sigset_t � <signal.h>, ��� ������ ��������� ����������� ����
��������, ������������ � �������.

����� �������� - ���� �� �������, ������ ��������, ���� �� �������
������, �������� �����������.

������������� ������:

#include <signal.h>

int sigemptyset(sigset_t *set);

int sigfillset(sigset_t *set);

���������� � �������� ��������:

int sigaddset(sigset_t *set, int signo);

int sigdelset(sigset_t *set, int signo);

������:

sigset_t mask1, mask2;

sigemptyset(&mask1);

sigaddset(&mask1, SIGINT);

sigaddset(&mask1, SIGQUIT);

sigfillset(&mask2);

sigdekset(&mask2, SIGCHLD);

                         ���������� ��������.

����� ����������� ������ ����� ������ ��������� ��������:

#include<signal.h>

int sigaction(int signo, const struct sigaction *act, struct sigaction
*oact);

signo - ������, ��� �������� �������� ��������.

act - ���������� ����������.

oact - ���� �� NULL, �� � ��� ��������� ���������� ������ ����������.

struct sigaction {

void (*sa_handler)(int); //������� �����������

sigset_t sa_mask; //�������, ����������� �� ����� ��������� �������

int sa_flags; //�����, �������� �� ��������� �������

viod (*sa_sigaction)(int, siginfo_t*, void*);

};

����������:

- sa_handler:

 1. SIG_DFL - ��������� ��������� �� ���������.
 2. SIG_IGN - ��������� �������������. �� ����� ���������� ��� SIGSTOP
    � SIGKILL.
 3. ����� �������, ����������� �������� ���� int (sa_handler=f1). ���
    ����� ���������� ��� ��������� �������, � signo ���������� ���
    ��������.

���������� ���������� ������� �� ������ ����� ���������, � �����
�������� �� ��� ���������� ����� ���������� � �����, � ������� ����
��������.

  * sa_mask - ������� �� ����� ������ ����� �������������� �� �����
    ���������� �������-����������� (sa_handler).
  * sa_flags - ��������� ��������� �������:

 1. SA_RESETHAND - ����� �������� �� ����������� ������� ���������� ��
    ��������� SIG_DFL.
 2. SA_SIGINFO - ����������� ���������� �������������� ���������� �
    ������ sa_handler ������������ sa_sigaction.
 3. SA_RESTART - ������ ����������� �������� ���������� ������.

               ���������� ������� (���������� ������).

#include<signal.h>

void (*signal(int sig, void(*disp)(int))) (int);

sig - ����� �������.

disp - SIG_DFL, SIG_IGN ��� �������-����������.

������: signal(SIGINT, SIG_IGN);

                        ������������ ��������.

int sigprocmask(int how, const sigset_t *set, sigset_t *oset);

- how - �����������, ����� �������� ���� ���������:

 1. SIG_MASK - ���������� ������������ �������� �� ������;
 2. SIG_UNBLOCK - ������ ������������ �������� �� ������;
 3. SIG_BLOCK - ���������� ������ � ������� �����������.

  * set - ����� ��������.
  * oset - ������ �����. ���� �� ����� NULL, �� � ���� ���������
    ��������.

������:

sigset_t set1;

sigfillset(&set1);

sigprocmask(SIG_SETMASK, &set1, NULL);

//����������� ������� - �������������

sigprocmask(SIG_UNBLOCK, &set, NULL);

����������: ����� ����������� ������ � ������� ������������ ����.



% IPC.Каналы и FIFO
%\chapter{IPC. ������ � FIFO}

\section{������}

\emph{����������� �����} (��� ������ �����\footnote{� ������� ������������ ����� ����� ��������� ������������ �����}) ������ ��� ������������ ������������� �����, ����������� ���� ������� � ������.

������ �������� ���� � ������ shell. ��� ����� ���������� ��� UNIX.

\verb+������: who | wc -l+

\subsection{�������� � ��������}
\begin{verbatim}
#include<unistd.h>

int pipe(int filedes[2]);
\end{verbatim}

\verb+filedes+ - ������ �������� ������������ �� ���� ����� �����: \verb+filedes[0]+ - ���������� ��� ������ �� ������, � \verb+filedes[1]+ - ��� ������ � �����.

���������� �������� -1, ���� ��������� ������ ��������.

\verb+int close(filedes);+

���������� 0 � ������ ������. ������ �� ���������.

\subsection{������/������ ����� �������������� �������� ����/�����}

\begin{verbatim}
#include<unistd.h>

ssize_t read(int filedes, void* buffer, size_t n);

ssize_t write(int filedes, const void* buffer, size_t n);
\end{verbatim}
\verb+filedes+ - �������� ����������;

\verb+buffer+ - ��������� �� ������ ��� ���������, �/�� ������� ����������
������;

\verb+n+ - ���������� ����, ������� ����� ���������/��������;

\verb+size_t+ - ���������� ������� �����������/���������� ����.

\begin{verbatim}
������:

#include<unistd.h>
#include<stdio.h>

main()
{
	const int MSGSIZE 16
	char* msg "Hello world #0";
	char inbuf[MSGSIZE];
	int p[2], j;
	pid_t pid;

	if (pipe(p) == -1) exit(1);

	if (pid=fork())
	{
		close(p[0]);
		write(p[1], msg, MSGSIZE);
		break;
	} else 
	{
		close(p[1]);
		read(p[0], inbuf, MSGSIZE);
		printf("%s\n", inbuf);
		wait(NULL);
	};
	exit(0);
}
\end{verbatim}

\emph{���������:} ��� ������ fork() ����������� ����������� ��������� ���������� �� ��������, ������� ������ ������������ ��� ����� ������ ����� ������������ ����������.

����������� ������ ������ �� POSIX - 512 ����. �������� ������� ���, �� ���� ����� ����� ���� ������������ �������. ��� ����������� ���������� ����� ����� ������������ ������ ������ ��� ������.

\subsection{����� ������/������ � �������� ������}

���� write ���������� ��� ������, � ������� ���� �����, �� ������ ���������� � ���������� ���������� �������.

���� ��� write ��������� ������������, �� ������ �� ���� �� ��� ���, ���� �� ����������� �����.

���� write ����� ������ ����, ��� ����������� ���� � ������ �����, �� ������������ ������������ ����� ������, � ����� ��������������� �� ������������ ����� ��� ���������� ������.

������ write ����� ���������� �������� (��������), � ������ ���������� �� ���� ����������� ��������. �� ��� ������� ������ ������ ������ ���������� ��������.

���������: ���� ��������� ��������� ����� � �����, �� ����� ���� ������������ ����������.

��� read �����������, ������ �� �����. ���� �� ����, �� read ����������� �� ��� ���, ���� � ������ �� �������� ������. ���� ������ ����, ���������� ������� �� read, ���� ���� ���� ��������� ������ ������, ��� �����������.

\subsection{�������� �������}

2 ������:
\begin{enumerate}
\item ���������� �� ������:

���� ���������� ������ ��������, � ������� ����� ������ �� ������, �� ������ �� ����������. ���� ����� ��������� �� ����������, � ����� ��� ���� ����, �� ����� �������� ������� ������� ���� ������.

���������������� � ��������� �������� ��������� ������ � ������ �� read � ������� ���������.  

\item ���������� �� ������:

���� ���� ������ ��������, �� ������ �� ����������. ���� ��������� ������ ���, �� ���������� ������ SIGPIPE ���� ���������, ��������� ������ � �����. ���� ������� �� ������������� ���, �� �� ���������� ���� ������, ���� �� ���� �������� ��������� ��������������, �� ����� ����������� write ������ �������� -1 � ���������� errno ����� ��������� �������� EPIPE.
\end{enumerate}

\subsection{������ � ������ ��� ������������}

��� ������������� read � write ������ ����� ���������� ������������, ������� ����� ���� ������������. ��������, ��� ��������� ��������� ������ ��� ������ ���������� ������� �� ��� ���, ���� � ����� �� ��� �� �������� ������.

���������� ��� �������: \verb+fstat+ � \verb+fcntl+. ������ �����.

\verb+fstat+ - � ����� �� ������������ ����� ���������� ������� ���������� �������� � ������, �� ��� ����� ���������� ���������.

\verb+fcntl+ - ������ ��� ���������� ��� ��������� �������, ����� ��������� ������ �������.
\begin{verbatim}
#include<sys/types.h>
#include<unistd.h>
#include<fcntl.h>

int fcntl(int filedes, int cmd, ...);
\end{verbatim}
filedes - �������� ����������;

cmd - ����������� ������� (�������� �� ������ � ����� <fcntl.h>);

��� �������� ��������� ������ �� ��������� � ������� �� cmd.
\begin{verbatim}
������: ������ ����������

fcntl(filedes, F_SETFL, O_NONBLOCK);
\end{verbatim}

\subsection{������������� select}

���� ��������� ����� ������������ ��� ������ � ���������� �������
������������. ������������ ����� ��� ������� ������, ����������, FIFO
� �������.

������: ��������� ������� � ���������� �������� (��������). �� ������
���������� � ���������, ����� ������������ � ���������� � ������� ����
����������, ������ ���������, ��� ����� �� ���������. ���� ����������
��������� ����� � ��������� �������, �� ������ ������ ����� ��� ����
����� ������� ��� ��������� � ���������� ������� (��������, ��
�����������).

\begin{verbatim}
���������:
#include<sys/time.h>

int select (int nfds, fd_set* readfds, fd_set* writefds, fd_set*
errorfds, struct timeval * timeout);
\end{verbatim}

���������:

\emph{nfds} - ������� �����������, ������� ������������ ������� ��� �������.
��������, ���� 0,1,2 - ����������� � ������� ��� ��� ����� �
������������� 3 � 4, �� nfds=5. ���� �������� ����� ������ ������ ���
����� ��������� \verb+FD_SETSIZE+ �� \verb+<sys/time.h>+. ��� ����� �������������
����� ������������, ������� ����� ������������ select.

\emph{readfds, writefds, errorfds} - ��������� �� ������� �����, � �������
������ ��� ������������� ��������� �����������. ���� ��� �������, ��
��� �������� ������� select � ����� �����������. ����� ����������
������ ��������.

\emph{readfds} - ����������, ��� �������� ������ ������� ����������� ������;

\emph{writefds} - ����������, ��� �������� ������ ������� ����������� ������;

\emph{errordfds} - ����������, ��� �������� ������ ������� �����������
��������� ������ ��� �������������� �������� (��������, ������������
������ �� ����).

���������� ����������� ��� fd_set � ������� (��� �������, �
����������� �� ����������) ��� ������ � �������.

timeout - ��������� �� ��������� timeval:

#include<sys/time.h>

struct timeval {

long tv_sec; //�������

long tv_usec; //������������

};

 1. ���� ��������� timeval ����� NULL, �� select ����� ������������ ��
    ��������� ������������� �������;
 2. ���� ���� ��������� �������, �� select ���������� �����������;
 3. ���� ���� ���������, �� ������� ���������� ����� ��������
    ���������� ������ � �����������.

������� ��� ������ � fd_set:

#include<sys/time.h>

void FD_ZERO (fd_set* fdset); - ������������� ������� �����

void FD_SET (int fd, fd_set* fdset); - ��������� ���� �����������

void FD_IZSET (int fd, fd_set* fdset); - ��������, ���������� �� ���

void FD_CLR (int fd, fd_set* fdset); - ������� ��� ����������� � �����

select ����������:

-1 � ������ ������;

0, ���� ����� ��������� ��������;

����� - ���������� ������������ ������������.

�����������: ��� �������� ����������������� ������� ����� readfds,
writefds, errorfds � �������� ���� �� ���������� � ������� ���������
��������� �����������. � ��������� ����������� ���������� ����� �
�������� timeout, ��� ��� ��� ��������� ���������� ���-�� ���������.

������: �� ��� ����������� (�� pipe)

#include<sys/types.h>

#include<unistd.h>

#include<fcntl.h>

int fd1, fd2, fd3;

fdset readset;

fd1 = open("file1", O_RDONLY);

fd2 = open("file2", O_RDONLY);

FD_ZERO(&readset);

FD_SET(fd1, &readset);

FD_SET(fd2, &readset);

int res = select (5, &readset, NULL, NULL, NULL);

if (res > 0)

{

if FD_ISSET(fd1, &readset) printf("������ fd1");

if FD_ISSET(fd2, &readset) printf("������ fd2");

};

                     ������ � exec (����������).

who | wc -l

0x08 graphic
0x08 graphic
0x08 graphic
0x08 graphic
0x08 graphic
0x08 graphic

0x08 graphic
0x08 graphic
0x08 graphic
0x08 graphic
0x08 graphic
0x08 graphic
0x08 graphic

0x08 graphic
0x08 graphic

0x08 graphic
0x08 graphic
0x08 graphic

0x08 graphic
0x08 graphic
0x08 graphic
0x08 graphic
0x08 graphic
0x08 graphic

0x08 graphic
0x08 graphic
0x08 graphic

0x08 graphic

                      FIFO (����������� ������).

� ������� �� ������ - ��������� � �� ��� ��������� ��� ��������
������� UNIX. FIFO ����� ������, ���������, ����� �������.
�����������, �����������, ���������, ��� ������� ����.

read � write ����� ���� ��� � ��� ������������� �������.

� shell ��� �������� ������: mknod channel p, ��� channel - ����� ���
�������� �������, � - ������� ������ (mknod ������������ � ���
dev-������).

������ ����� ����� ���������� ������, ������� �� ������ ������ � FIFO.

������:

�) eat < channel & #�� ����, ��� ��� ���� � ������ ��� ������, ��
��������� ���� �� �����������

�) ls -l > channel; wait #����� � ����� � ��������, ���� ����������
cat

��� �������� FIFO � ��:

#include<sys/types.h>

#include<sys/stat.h>

int mkfifo (const char* pathname, mode_t mode);

mode - ����� �������, �� ������� ����� �������� umask � ��������
��������� ������.

������:

mkfifo ("/tmp/fifo", 0644);

int fd = open ("/tmp/fifo", O_RDONLY);

int fd = open ("/tmp/fifo", O_RDONLY | O_NONBLOCK); - �������������

� ��������� FIFO ������ �� ������������� ������.

�������������

��������

                               write()

                                read()

������������

p[1]

p[0]

�������� ������� ��������� ���������� �� ������ � ����� ���������.

������������ ������� ��������� ���������� �� ������ � ������ ���������
�� ������.

sh

                                  sh

                                fd[0]

                                fd[1]

                                  sh

                                fd[0]

                                fd[1]

fork()

fork()

exec()

                                 who

                                fd[0]

                                fd[1]

                                exec()

                                pipe()

                                  wc

                                fd[0]

                                fd[1]

wc

fd[0]

pipe()

who

fd[1]


% IPC. Сокеты
\chapter{IPC. Сокеты. Как писать сетевое ПО.}

Раздел написан по материалам цикла статей "Cетевое программирование в Unix" ( "Сетевые Решения", 2003(12), 2004(1), 2004(2), 2004(3), см http://nestor.minsk.by/sr/abc/114.html ).

\section{Начальные сведения}

\emph{Сокеты (sockets)} впервые появились в начале 80-х в составе 4.2BSD как программный интерфейс к стеку TCP/IP. С тех пор их часто называют "BSD сокеты". И до и после были попытки создавать сетевые API, но сокеты прижились лучше всего. Возможно потому что при их создании учитывались характерные особенности Unix как среды программирования: файла, как основы взаимодействия процессов и принципа "keep it simple stupid" (KISS).

Кстати, в русскоязычной литературе и сленге имеется множество вариантов перевода и произношения термина "socket": сОкет, сокЕт, гнездо (!!!) и т.п. словотворчество. Весь этот зоопарк обозначает одно и то же.

В основе сокетов лежит модная концепция "клиент-сервер". Сервер реализует некоторый сервис, клиент его использует. Два процесса, используя сеть, обмениваются данными. При создании соединения, с каждой из сторон создается сокет. Протоколы TCP/IP отвечают за передачу между ними данных. Чтобы определить кто и куда передает информацию, бы ли введены два понятия: \emph{адрес и порт}. 

\emph{Адрес} - сетевой адрес компьютера, \emph{порт} - это номер сокета на конкретном компьютере. Причем каждый порт связан с конкретным процессом и поэтому операционная система знает кому отдавать пришедшие данные. \emph{Сокет - это пара адрес:порт}.

В итоге - передача данных использует 2 сокета, по одному на каждого участника обмена информацией. Между ними устанавливается виртуальный канал и нет больше об этом заботы в чешуйчатой зеленой голове программиста. Что вписано в сокет на одной стороне, то появится на другой, и наоборот.

\verb+Для иллюстрации: поглядеть 'netstat -natp' (в linux)+

Работа с сокетом идет через файловый дескриптор, что очень характерно для Unix.
Однако есть 2 вида передачи информации: с предварительным соединением и без. Мы подробно прожуем и проглотим оба вида передачи. Так же очень основательно будет рассмотрен серверный и клиентский режимы работы.

\subsection{Всё есть файл}

Работа с сокетом ведется через дескриптор, являющийся также и дескриптором файла. То есть уже открытое соединение ничем не отличается с точки зрения программиста от открытого файла или терминала. Разработчик работает с сокетом теми же методами, которыми он работает с обычным (regular) файлом. Несколько отклоняясь от темы, замечу, что то же самое мы наблюдаем в отношении и других специальных типов файлов (pipe, fifo, block и character devices).

Это подымает термин «файл» на совершенно новый уровень понимания: "мы работаем с этим как с файлом и плевать, что там в реализации". Файл может быть физическим устройством, сокетом, каналом, каталогом, текстовым файлом, но мы только открываем их по разному. Чтение/запись и управление режимами работы - через одни и те же системные вызовы.

Потрясающий по своей красоте и выразительности пример: запись компакт диска по сети в 1 строку.
\begin{verbatim}
$cat 4.9-i386-disc1.iso|ssh user@host.localdomain cdrecord -
\end{verbatim}
Здесь используется сразу многие типы файлов, объединенных командой shell в единое целое.
\verb+cat+ читает файл с диска и выводит на экран. \verb+Оболочка+ перенаправляет вывод \verb+cat+ через \verb+pipe+ на вход команды \verb+ssh+.

\verb+ssh+ запускает на удаленном хосте команду записи дисков. Причем таким образом чтобы cdrecord кушал данные со стандартного ввода.

\verb+ssh+ доставляет ему этот ввод через сокет. Еще один неявный момент - \verb+cdrecord+ использует файл блочного устройства резака для записи полученного образа компакта. Подсчитаем: regular file, pipe, socket, block device - итого 4 различных вида файлов.

Нашему примеру можно придать изящный финальный штрих - избавится от временного файла в лице образа компакта и создавать его на лету:
\begin{verbatim}
$mkisofs /home/user2/for_write|ssh user@host.localdomain cdrecord -
\end{verbatim}

Нет временных файлов, только безостановочное движение потока данных из одного состояния в другое, подкрепленное могучей концепцией "бывают разные файлы, но мы работаем с ними одинаково, просто пишем и читаем".

\subsection{Передача с предварительным соединением и без}

Есть две модели передачи информации посредством сокетов. С установлением соединения и без оного.

Мы в основном будем останавливаться на первой.
\begin{enumerate}
\item Наиболее важные отличия сonnection-oriented соединения (назовем его условно TCP):
\begin{itemize}
\item требует установления соединения перед передачей;
\item гарантирует доставку данных (создание виртуального канала);
\item передача идет потоком данных. TCP/IP гарантирует то что данные дойдут так, как посылались;
\item основан на протоколе TCP.
\end{itemize}
\item connectionless (условно назовем его UDP):
\begin{itemize}
\item не требует соединения;
\item ничего не гарантирует;
\item передача идет пакетами, которые могут пропадать, дублироваться и т.п. Контроль за целостностью информации целиком в руках программиста;
\item основан на протоколе UDP.
\end{itemize}
\end{enumerate}

Итого:

TCP имеет бОльшие накладные расходы чем UDP. Но UDP проще организован, и если нужна высокая скорость и компьютеры стоят в одном сегменте сети, то UDP очень хорошо подойдет. В свою очередь, TCP больше подходит для работы по ненадежным каналам.

Прибегнув к скользкой аналогии,\newline
UDP - выстрел вслепую, пули могут пропасть, а могут и размножится по дороге.\newline
TCP - телефон: что сказано с обеих сторон, то и услышано. Конечно если ворона не села на провод и не перекусила его.

\section{Основные системные вызовы}

Аннотированный список системных вызовов, тем или иным образом качающихся сокетов (в списке есть как socket-specific, так и общие вызовы, связанные с файлами)
\begin{itemize}
\item accept(2) - принять соединение ( используется сервером);
\item bind(2) - связать сокет с конкретным адресом (обычно используется только сервером);
\item connect(2) - соединиться с удаленным сервером (клиентский);
\item close(2) - закрыть файл или сокет;
\item listen(2) - начать прослушивание сокета (серверный);
\item read(2) - чтение данных из файла;
\item recv(2) - чтение из сокета;
\item select(2) - проверка изменения статуса открытых файлов;
\item send(2) - послать данные через сокет;
\item shutdown(2) - закрыть соединение;
\item socket(2) - создать сокет;
\item write(2) - запись данных в файл.
\end{itemize}

\emph{Примечание:} в Solaris, унаследовавшей реализацию сокетов от System V Release 4, сокеты не входят в ядро, а реализованы библиотекой. Поэтому man по ним содержится в 3-м разделе.

\section{Простейший TCP-клиент}

Как ни странно - чаще всего в его роли выступает небезызвестный \verb+telnet+. Утилита \verb+telnet+ есть во всех мне известных реализациях стека tcp/ip. Набираем \verb+telnet <host> <port>+ и наслаждаемся прямым общением с сервером. Благо классические TCP-based протоколы - текстовые.
\begin{verbatim}
[tcpclient.cpp]
1 #include <sys/types.h>
2 #include <sys/socket.h>
3 #include <netinet/in.h>
4 #include <arpa/inet.h>
5 #include <string.h>
6 #include <unistd.h>
7 #include <stdio.h>
8 int main() {
9 struct sockaddr_in addr;
10 memset(&addr,0,sizeof(addr));
11 addr.sin_family=AF_INET;
12 addr.sin_port=htons(1212);
13 addr.sin_addr.s_addr=inet_addr("127.0.0.1");
14 int sock;
15 sock=socket(PF_INET,SOCK_STREAM,0);
16 connect(sock,(sockaddr *)&addr,sizeof(addr));
17 char buf[80+1];
18 memset(buf,0,81);
19 read(sock,buf,80);
20 puts(buf);
21 shutdown(sock,0);
22 close(sock);
23 return 0;
24 }
\end{verbatim}

Эта программка соединяется с localhost, порт 1212, считывает последовательность байт и выводит ее на экран.

А теперь самое вкусное - обгладывание костей, любезно предоставленных нашим информационным спонсором. Строки 1-4: заголовочные файлы. Содержат объявления функций и типов данных для работы через сетевые сокеты. 5 и 7 строки - библиотечные функции Си. 6 строка - заголовочный файл стандартных системных вызовов.

9, 10, 11-13 строки - объявление, зачистка и заполнение структуры, содержащей адрес соединения. Обращаю особое внимание на то, как инициализируется порт и адрес назначения. Адрес и порт хранятся в особом формате, называемом \emph{network byte order}. Прямое присваивание обычно ничего не даст, потому что архитектура i386 и network byte order не совпадают. Макрос \verb+htons(3)+ занимается преобразованием числа в сетевой вид. Функция \verb+inet_addr(3)+ преобразует строковое значение IP-адреса в числовое. \verb+AF_INET+ из 11 строки - указание типа сокетов (сетевые в нашем случае).

И наконец, ключевой момент. 15 строка - создание сокета (TCP-сокета естественно) и 16 строка - соединение с сервером. В \verb+connect(2)+ используется указатель на адрес , потому что размер структуры с адресом может варьироваться в зависимости от типа сокета (сетевой или локальный).

Далее идет считывание (19 строка) и вывод на экран (20 строка) полученного.

21 и 22 строки - закрыть соединение и освободить файловый дескриптор, им использованный.

\section{TCP-сервер}

В качестве примера использована реализация сервиса daytime.
Суть протокола daytime: выдать время и закрыть соединение со стороны сервера.
\begin{verbatim}
[daytime.cpp]
1 #include <sys/types.h>
2 #include <sys/socket.h>
3 #include <netinet/in.h>
4 #include <arpa/inet.h>
5 #include <string.h>
6 #include <time.h>
7 #include <unistd.h>
8 #include <stdio.h>
9 char * daytime() {
10 time_t now;
11 now=time(NULL);
12 return ctime(&now);
13 }
14 int main() {
15 struct sockaddr_in addr;
16 memset(&addr,0,sizeof(addr));
17 addr.sin_family=AF_INET;
18 addr.sin_port=htons(1212);
19 addr.sin_addr.s_addr=INADDR_ANY;//inet_addr("127.0.0.1");
20 int sock, c_sock;
21 sock=socket(PF_INET,SOCK_STREAM,0);
22 bind(sock,(struct sockaddr *)&addr,sizeof(addr));
23 listen(sock,5);
24 for (;;) {
25 c_sock=accept(sock,NULL,NULL);
26 char buf[81];
27 memset(buf,0,81);
28 strncpy(buf,daytime(),80);
29 write(c_sock,buf,strlen(buf));
30 shutdown(c_sock,0);
31 close(c_sock);
32 puts("answer tcp");
33 }
34 return 0;
35 }
\end{verbatim}

Пойдем по порядку. Сервер отличается от клиента режимом работы. Он должен:
\begin{enumerate}
\item занять адрес и порт (22 строка), \verb+bind(2)+;
\item включить прослушивание порта (23 строка), \verb+listen(2)+;
\item принимать и обрабатывать соединения (24-33 строки), \verb+accept(2)+.
\end{enumerate}

Подчеркнем отличия от клиента: в 19 строке указывается специальная константа \verb+INADDR_ANY+. Она обозначает, что соединения будут приниматься на всех интерфейсах. Можно задать и конкретный адрес (в стиле TCP-клиента). Так работают многие сервисы, которым можно задать адрес привязки в конфиге (apache, squid).

С 24 по 33 строку у нас бесконечный цикл. До первого \verb-Ctrl+C-, естественно. Сервер до полной и окончательной победы занимается приемом соединений.

Очень интересная особенность: \emph{каждое входящее соединение порождает свой сокет} (см. 25 строку). Контрольный сокет (\verb+sock+) и клиентский сокет (\verb+c_sock+) - разделены. Обмен данными идет через \verb+c_sock+. Соединения принимаются на контрольный (\verb+sock+).

Есть два важных следствия разделения на контрольный сокет и сокеты соединений:
\begin{itemize}
\item  упрощение работы программиста - нет необходимости отслеживать, кто и куда прислал данные, разные по сути понятия - выделены;
\item возможность распараллеливания обработки соединений (подробнее об этом расскажу позже).
\end{itemize}
В примере используется простая последовательная обработка. Пока предыдущее соединение не обработано - входящие соединения ждут своей очереди. Это имеет смысл только для простых случаев - когда время обработки запроса мало и задержка для вновь поступающих невелика.

9-13 строки - получение текущего времени. Эта функция используется в строке 28 для формирования результата.

\section{UDP}
Вот мы и подошли к финальной части букваря, к протоколу UDP и методам работы с ним.
Начнем традиционно, с клиента, и закончим, как обычно, сервером.
Напоминаю, что UDP - это модель передачи данных без образования соединения и без проверки корректности передачи. Посему если информация пропала - никто не виноват. Но если канал передачи надежный - почему бы и не использовать UDP?
Данные передаются так называемыми датаграммами (datagram) без сохранения порядка при переносе к получателю.

\subsection{UDP-клиент}

К сожалению стандартных и широко распространенных программок типа telnet не обнаружено. Поэтому в этой роли выступает накарябаный на колене за вечер корявый шедевр. Встречайте:
\begin{verbatim}
[udp_client.cpp]
1 #include <sys/types.h>
2 #include <sys/socket.h>
3 #include <netinet/in.h>
4 #include <arpa/inet.h>
5 #include <string.h>
6 #include <unistd.h>
7 #include <stdio.h>

8 int main() {
9 struct sockaddr_in server, client={AF_INET,INADDR_ANY,INADDR_ANY};

10 memset(&server,0,sizeof(server));

11 server.sin_family=AF_INET;
12 server.sin_port=htons(1212);
13 server.sin_addr.s_addr=inet_addr("127.0.0.1");

14 int sock;

15 sock=socket(PF_INET,SOCK_DGRAM,0);
16 bind(sock,(sockaddr *)&client,sizeof(client));

17 char buf[81];
18 memset(buf,0,81);
19 strcpy(buf,"request");
20 sendto(sock,&buf,strlen(buf),0,(sockaddr *)&server,sizeof(server));
21 memset(buf,0,81);
22 recvfrom(sock,buf,80,0,NULL,0);
23 puts(buf);

24 return 0;
25 }
\end{verbatim}

Эта мини-программа посылает слово "request" на порт 1212 по адресу 127.0.0.1 (localhost) и читает, что же ответил сервер.
с 1 по 14 строки - стандартная сокетная обвязка, общая для TCP и UDP. У нас клиент, поэтому в 12 и 13 задаем порт и адрес назначения.

В 15 обратите особое внимание на константу \verb+SOCK_DGRAM+ - мы задаем тип сокета как датаграмный.

16 - уже специфична. Мы явно выполняем привязку адреса и порта к созданному сокету. 9 строка явно указывает, что программисту фиолетово, какой порт занять и через какой из сетевых интерфейсов отсылать данные. Ядро поймет это указание правильно и выдаст порт случайным образом.

20 и 22 строки - следующие специфичные для UDP элементы. Системные вызовы \verb+sendto(2)+ и \verb+recvfrom(2)+ предназначены для отправки и получения сообщений в/из сокета. Их можно использовать даже если сокет находится в несоединенном состоянии. Кстати, несмотря на природу \verb+SOCK_DGRAM+, здесь можно использовать \verb+connect(2)+ и заменить \verb+sendto(2)+ на \verb+write(2)+ и \verb+recvfrom(2)+ на \verb+read(2)+. Будет сделана виртуальная привязка сокета к пункту назначения, но соединение не будет устанавливаться, т.к. сокет остается по прежнему \verb+SOCK_DGRAM+.

\subsection{UDP-сервер}

Опять же daytime сервер в DGRAM-реинкарнации. Ждет датаграмму, при приходе оной извлекает адрес отправителя и посылает в ответ текущее время сервера.
\begin{verbatim}
[udp_server.cpp]

1 #include <sys/types.h>
2 #include <sys/socket.h>
3 #include <netinet/in.h>
4 #include <arpa/inet.h>
5 #include <string.h>
6 #include <time.h>
7 #include <unistd.h>
9 #include <stdio.h>

10 char * daytime() {
11 time_t now;
12 now=time(NULL);
13 return ctime(&now);
14 }

15 int main() {
16 struct sockaddr_in addr;

17 memset(&addr,0,sizeof(addr));

18 addr.sin_family=AF_INET;
19 addr.sin_port=htons(1212);
20 addr.sin_addr.s_addr=INADDR_ANY;

21 int sock, c_sock;

22 sock=socket(PF_INET,SOCK_DGRAM,0);
23 bind(sock,(struct sockaddr *)&addr,sizeof(addr));
24 for (;;) {
25 struct sockaddr from;
26 unsigned int len=sizeof(from);
27 char buf[81];
28 memset(buf,0,81);
29 recvfrom(sock,&buf,80,0,&from,&len);
30 printf("udp incoming:%s",buf);

31 memset(buf,0,81);
32 strncpy(buf,daytime(),80);

33 sendto(sock,buf,strlen(buf),0,&from,len);
34 puts("answer udp");
35 }

36 return 0;
37 }
\end{verbatim}

Все меньше и меньше остается незнакомых элементов в программах. Будь я графоманом и/или любителем гонораров - расписывал бы каждую строку :).

В этом примере мы заострим внимание всего на двух строках - 29 и 33. При вызове \verb+recvfrom(2)+ системным вызовом заполняются параметры 5 и 6. Адрес отправителя и размер структуры для его хранения. Так сервер узнает о существовании клиента и о наличии у него желания пообщаться. И в 33 строке идет подача данных в ответ.

\subsection{Объедки анализа}

Наблюдательные дети наверное заметили, что клиент и сервер у UDP практически идентичны по набору используемых вызовов. Отличие всего в двух маленьких детальках, а дьявол всегда прячется в деталях.

Первая деталька: присваивание адреса и порта сокету: сервер явно указывает порт, клиент - саботирует. Ведь чтобы обратиться к серверу - за ним должен быть зафиксирован порт.

Вторая деталька: порядок вызовов \verb+recvfrom(2)+ и \verb+sendto(2)+. Клиент сначала отсылает, потом принимает, сервер, наоборот, принимает а затем отсылает. Клиент-сервер as is: клиент начинает, сервер реагирует на действия.

\section{Модели организации серверов}

В жизни программиста, пишущего для интерфейса сокетов, через некоторое время неотвратимо наступает переломный момент. Написание приложения, обрабатывающего несколько подключений одновременно, или работающего под серьезной нагрузкой. Или хитрого клиента, который одновременно общается с несколькими серверами.

Существует N-ое количество способов организации нетривиальных серверов и клиентов. Наиболее распространенных из них я и коснусь. Материал будет обильно нашпигован информацией из W. Richard Stevens "Unix. Network Programming. Networking API" и ru.unix.prog FAQ. Поэтому не буду отсылать на конкретные источники, а просто последовательно излагать.

\subsection{"естественный" и "правильный" способы организации сервера}

Чем же руководствоваться при выборе подходящего способа организации сервера? Ответ - здравым смыслом, конечно. Непременно сначала подумать, а не бросаться создавать отстойные приложения.

Первый критерий: подумайте, как и сколько времени обрабатывается 1 соединение.

Если у приложения простой протокол запрос-ответ с минимальной обработкой и задержкой на сервере, простой последовательный сервер будет в самый раз. Клиенты просто не заметят разницы :)

\subsection{Последовательный сервер}

Последовательный - это когда все запросы обрабатываются друг за другом последовательно (см сервис daytime):
\begin{verbatim}
24 for (;;) {
25 c_sock=accept(sock,NULL,NULL);
26 char buf[81];
27 memset(buf,0,81);
28 strncpy(buf,daytime(),80);
29 write(c_sock,buf,strlen(buf));
30 shutdown(c_sock,0);
31 close(c_sock);
32 puts("answer tcp");
33 }
\end{verbatim}

Пока текущий не обработан, следующий запрос на соединение стоит в очереди. Размер очереди указывается в \verb+listen(2)+.
Очень просто и, как ни странно, дьявольски эффективно. Никаких дополнительных накладных расходов на управление соединениями. Но случись задержка в обработке сервером запроса - и остальные клиенты нервно курят в стороне. Длительные операции - увы - не для него.

Как же маштабируемость? Ответ - нуль. Понапихали 4 процессора в машину? Ничего не улучшится :). Все одно, последовательно пилит на одном, ОС бессильна что-либо изменить.
Итого: эффективен, но для простых вещей.

\subsection{один процесс == один клиент (простой и prefork)}

Мысль начинающего сетевого программиста общается сама с собой по очевидной схеме:\newline
- Ага: надо много клиентов одновременно обрабатывать...\newline
- Давайте каждому соединению создадим новый процесс!\newline
- Да, это круто, пельмень!

Так вылазит первая, самая очевидная схема ("естественная"): сервер использует несколько процессов, каждый из которых обслуживает по одному клиенту.

Преимущества этой модели очевидны:
\begin{enumerate}
\item простая как ванильная сушка;
\item хорошо маштабируется с ростом числа процессов;
\item ошибка в 1 процессе не ведет к отказу всей программы.
\end{enumerate}

Но тут есть и отрицательные стороны. Процесс - это достаточно тяжелый объект OC. При большом количестве клиентов мы получаем значительную загрузку системы в целом вплоть до отказа. Переключения контекста и связанная с ними черновая работа ОС вполне могут подсадить на задницу процессор любой мощности (при действительно большом количестве клиентов).

\begin{verbatim}
[fork_server.cpp]

1 for (;;) {
2 c_sock=accept(sock,NULL,NULL);
3 if ( !fork() ) {
4 puts("incoming tcp\n");
5 char time_str[81];
6 memset(time_str,0,81);
7 strncpy(time_str,daytime(),80);
8 write(c_sock,time_str,strlen(time_str));
9 puts("answer tcp\n");
10 shutdown(c_sock,0);
11 close(c_sock);
12 exit(0);
13 }
14 }
\end{verbatim}

Как же это работает? После прихода соединения (строка 2), запускается новый процесс через \verb+fork(2)+ (строка 3). Головной поток выполнения снова вернулся к фазе \verb+accept(2)+, новый же процесс обрабатывает соединение и завершает работу. Почему \verb+fork(2)+ в \verb+if+? Выкурите \verb+man 2 fork+ до полного просветления.

Кстати, приведенный пример - не единственный возможный способ реализации модели «1 процесс == 1 клиент». Существует также улучшенный (и более сложный вариант) организации под названием \emph{prefork}. Суть его следующая: после фазы \verb+listen(2)+ мы запускаем N процессов. В каждом из них - последовательный сервер (см пункт 4.1.1). Вся толпа из N серверов занимается обслуживанием одного и того же принимающего порта. Механизм прост: если одновременно несколько процессов делают \verb+accept(2)+ на одинаковый порт, то ОС погружает всех в спячку до прихода соединения. После прихода соединения - осуществляется т.н. "побудка" и один из процессов побеждает в соц соревновании за сокет. Остальные снова уходят в спячку. Управление этим режимом полностью осуществляется операционной системой.
Плюсы - все уже запущено, только начать обрабатывать. Минус - при количестве процессов >200 начинаются потери в производительности (проверено на людях). Система начинает терять время на обдумывании, кому из 200 процессов отдать на заклание обработку. Как вариант - можно сделать блокировку \verb+accept(2)+ семафором или блокировкой. Тогда только 1 процесс делает \verb+accept(2)+ и разруливание "побудки" не нужно.

\subsection{один поток == один клиент}

Идея проста: меняешь в предыдущем разделе процессы на потоки и защищаешь общие данные. Эффективность подобного решения - величина неизвестная. В разных версиях различных ОС потоки реализованы по разному. Но в общем случае считается, что поток весит меньше чем процесс, и переключение контекста между потоками менее накладно.
\begin{verbatim}
1 for (;;) {
2 pthread_mutex_lock(&mutex_tcp);
3 c_sock=accept(sock,NULL,NULL);
4 pthread_mutex_unlock(&mutex_tcp);
5 pthread_create(&tid,NULL,&deliver_tcp,&c_sock);
6 }
\end{verbatim}

Здесь применена защита \verb+accept(2)+ исключающим семафором (как иллюстрация к идее о prefork, описанной в предыдущем разделе, примененная и к потокам). 2 и 4 строки - блокировка и раз-блокировка мутексом. В строке 5 мы запускаем поток на выполнение функции \verb+deliver_tcp+ с параметром \verb+c_sock+.

\subsection{Однопроцессная Finite State Machine и мультиплексирование}

«Потоки, как и объектно ориентированное программирование - это такая "серебряная пуля". Человеку, которому лень думать над тем, что собственно надлежит сделать, разбиение программы на потоки или использование объектов кажется естественным.
А потом он начинает наступать на не очевидные грабли. Потому что на самом деле это не пуля, а крылатая ракета на жидком водороде. Дальнобойность и убойная сила - поразительные, но обслуживания требует ох какого квалифицированного. И стоит дорого.
Поэтому там, где можно обойтись пулей из автомата (в данном случае - конечного) надо обходиться пулей. А крылатой ракетой стрелять только тогда, когда после анализа других тактических вариантов стало ясно, что здесь ничего другое не поможет.» (Виктор Вагнер, из переписки в ru.unix.рrog.)

В среде специалистов наиболее уважаемым вариантом является реализация c использованием FSM (по русски - на конечном автомате). Конечный автомат - это старая математическая абстракция, описывающая машину с некоторым количеством состояний и переходов между ними.

Ключевая особенность Unix, делающая возможным работу FSM-механизма - средства опроса состояния сокета (через select, poll, kevent/kqueue или epoll).

Сервер опрашивает состояние контрольного сокета и сокетов пришедших соединений. В случае активности на каком-нибудь из них - производит необходимые действия и возвращается в состояние опроса. Если объяснение признано невнятным, вот вам атомный пример.
\begin{verbatim}
Однопоточный TCP и UDP echo-server с использованием опроса состояния сокетов через select. Приведен с сокращениями:

...
1 struct sockaddr_in addr;
/* здесь инициализация addr */

2 int sock, c_sock, u_sock;
/* здесь включение tcp-сервера на sock и udp-сервера на u_sock */

/* определение наборов дескрипторов файлов и их максимального размера */
3 fd_set rfds, afds;
4 int nfds=getdtablesize();
5 FD_ZERO(&afds);
6 FD_SET(sock,&afds);
7 FD_SET(u_sock,&afds);

8 for (;;) {
9 memcpy(&rfds, &afds, sizeof(rfds));
10 select(nfds, &rfds, NULL,NULL,NULL);

11 if ( FD_ISSET(sock, &rfds) ) {
12 c_sock=accept(sock,NULL,NULL);
13 puts("incoming tcp");
14 FD_SET(c_sock,&afds);
15 continue;
16 }

17 for (int fd=0; fd<nfds; ++fd)
18 if ( fd == u_sock && FD_ISSET(u_sock,&rfds) ) {
19 struct sockaddr from;
20 unsigned int len=sizeof(from);
21 char buffer[81];
22 memset(buffer,0,81);
23 int size=recvfrom(u_sock,&buffer,80,0,&from,&len);
24 sendto(u_sock,buffer,size,0,&from,len);
25 printf("answer udp:%s",buffer);

26 } else if ( fd != sock && FD_ISSET(fd,&rfds) ) {
27 char buffer[81];
28 memset(buffer,0,81);
29 int len=read(fd,buffer,80);
30 if (len <=0) {
31 puts("connection closed");
32 close(fd);
33 FD_CLR(fd, &afds);
34 continue;
35 }

36 write(fd,buffer,len);
37 printf("answer tcp:%s",buffer);
38 } // if and for
39 } // for
\end{verbatim}

Echo-сервер делает весьма прозаическую вещь: все, что к нему приходит, отсылается обратно отправителю. Как в tcp, так и в udp. Данная реализация работает одновременно с множественными TCP-соединениями и с присылаемыми UDP-пакетами в одном единственном процессе.

Насладимся же разбором исходников! :) C строки 3 по 7 появляется новый персонаж нашего повествования: набор дескрипторов файлов (\verb+fd_set+). Его зовут на помощь, когда надо оперировать не 1 сокетом за раз, а целой пачкой. Для начала заполним его контрольным TCP- и UDP-сокетом. Это начальное состояние нашей FSM - \verb+afds+, в котором определены два сокета - \verb+sock+ и \verb+u_sock+.

10 строка - вот и опрос состояния сокетов. В начале их у нас два - \verb+sock+ и \verb+u_sock+. По мере выполнения итераций цикла их число может меняться. \verb+select(2)+ изменяет значения \verb+rfds+ - потому в строке 9 мы восстанавливаем его каждый раз из \verb+afds+.

А вот дальше - черная работа. Проверяем, кто из сокетов дернулся:\newline
- sock? Надо принимать соединение и добавлять новый сокет к afds (строки 11-16);\newline
- \verb+u_sock+? Вычитать UDP пакет и вернуть его содержимое отправителю (строки 18-25);\newline
- один из клиентских TCP-сокетов? Прочитать новый кусочек данных и вернуть их назад (строки 26-38).\newline

Теперь немножко о select(2), пожилом, но полным сил мастодонте Unix API.

Параметры 2, 3, 4 - указатели на наборы дескрипторов файлов. Каких файлов - неважно (см. 1 часть цикла "Сетевое программирование в Unix"). Можете одновременно опрашивать stdout, socket и pipe - система с удовольствием схавает и отработает. Соответственно наборы обозначают дескрипторы "доступные для чтения", "доступные для записи", "те, в которых случилась ошибка". Вполне могут отличаться друг от друга, если вам надо сотворить нетривиальную вещь. Часть из них может быть NULL.

Последний, 5-й, параметр - это время, после которого select закончит работать, если ничего не произошло с дескрипторами из наборов. Когда время NULL - ждет вечно, до наступления событий.

И 1-й параметр - это число, равное значению максимального дескриптора из набора +1. Такой параметр выполняет простую функцию - ограничивает количество просматриваемых дескрипторов указанным числом. Чтобы не зацепить лишние, не используемые программой открытые файлы (это замедлит работу). Вобщем select(2) смотрит 0,1,2,3,..n дескрипторы в поисках активности на них. Это то слабое место, в которое обоснованно тыкают пальцем фанаты других способов опроса состояния файла.

Что же можно сказать об этой модели в общем?
Наиболее эффективна с точки зрения использования CPU.
При наличии долгоиграющих блокирующих операций может быть затыки с параллельным приемом других соединений. В этом случае медленные операции можно вынести в отдельные потоки/процессы и решить тем самым затык. Или поработать операционной системой - разбить большую операцию на куски и выполнять ее, попутно отвлекаясь на остальную работу.
Как уже стало очевидно, модель с FSM требует очень тщательного программирования.

\subsection{Смешанные модели}

Вот мы и изучили все базовые типы серверов. Можно заняться их скрещиванием.
Например устойчива к сбоям модель "многопроцессность+FSM". Или производительна (на Linux) "многопоточность+FSM".
Cнова повторюсь. Нет "самой лучшей модели", есть наиболее подходящая к вашим условиям.

\section{Unix-domain сокеты или локальные сокеты}

Адрес сокетов в домене UNIX - имя файла. При вызове bind файл создаётся в файловой системе.
\begin{verbatim}
Пример сервера (AF_UNIX)
1. sockfd = socket (AF_UNIX, SOCK_DGRAM, 0);
2. unlink ("./echo.server");
3. bzero (&serv_addr, sizeof(serv_addr));
4. strcpy(serv_addr.sun_path, "./echo.server");
5. serv_addr.sun_family = AF_UNIX;
6. saddrlen = sizeof
   (serv_addr.sun_family)+strlen(serv_addr.sun_path);
7. bind (sockfd, (struct sockaddr *) &serv_addr, saddrlen);
\end{verbatim}

\begin{verbatim}
1. sockfd = socket (AF_UNIX, SOCK_DGRAM, 0);
2. bzero (&serv_addr, sizeof(serv_addr));
3. strcpy(serv_addr.sun_path, "./echo.server");
4. serv_addr.sun_family = AF_UNIX;
5. saddrlen = sizeof
   (serv_addr.sun_family)+strlen(serv_addr.sun_path);
6. strcpy(clnt_addr.sun_path, "./tmp/clnt.xxxx");
\end{verbatim}


% System V IPC.
\chapter{System V IPC}

� ��������� System V IPC ��������� 3 ������ ��������������:
\begin{itemize}
\item ������� ���������
\item ����������� ������
\item ��������
\end{itemize}

��������������� ����� �������� ��������� � ������� ���������� �������� ��� ������ � ����.
�������� � ������������ ����. ����� �������� - �� ������� ������������� ���������� ��������.

������� ��������� �������� IPC - ��� ������� � �� ����� �������� ������. �� ���������� ��� ����� �� �������� �������, ������������� �� ����� ���������� ��� ����� � ������� ����������� �������. ����������� ������ ������ syscalls. ���������� ���� user-space ������� ��� ��������� � �������� (ipcs, ipcrm).

��������� ��������� System V IPC - �� �����, �� � ��� �� ����� ���� ���������  ������������������� (\ref{select}, \ref{poll}). ��� ��������� ������������� ����� ��� 1 ��������� ������������ ��� ��������� � ������/������� ��/� ����/���������. 

\section{�������������� � �����}

������ ��������� IPC (�������, ��������, ����������� ������) ���������� � ���� ������������� ������������� ���������������. 

��� �������� ��������� IPC ���� ������ ���� ������. 

��� ��� ���� ���������� ������ ���� \verb+key_t+. \verb+key_t+ �������� � \verb+<sys/types.h>+ ��� ����� (�� ������� ���� - 32���������). ���� ����������� � ������������� �����.

������ ������������ �������� \verb+ftok+ �� ���� � ����� � ������������� �������������� :
\begin{verbatim}
#include <sys/types.h>
key_t ftok(const char * pathname, int proj_id);
\end{verbatim}

���������� ��������� ���� ��� ����������� ������� ������� � ������� � ����� IPC ���������:
\begin{enumerate}
\item ������ ������ ����� ��������� IPC �������� \verb+IPC_PRIVATE+ � ��������� ���� ����-���� ��� ������� (�������� � ����).  ����� ����� �������������� ��� ���������� "������-�������", ����� ���� ��������� ��� fork ����������� ��������.
\item ����� ����� ������������ ����, ����� ���� ��� �����. ������ ������ ��������� IPC �������� ���� ����. ������ �������� ������. �������� - ���� ��� ����� �������������� ���-�� ���.
\item ������ � ������ ����� ���������� �� ���� � ����� � ������������� ������� (�� 0 �� 255) � ������� \verb+ftok()+ ��� �������� ���� � �������������� � ����.
\end{enumerate}

\section{����� �������}

System V IPC ��������� ��������� \verb+ipc_perm+ � ������ ���������� IPC. ��� ���������� 
\emph{��������� � ����� �������} (���������� ������� � �����).

\begin{verbatim}
struct ipc_perm {
	uid_t uid; // uid ���������
	gid_t gid; // uid ������
	uid_t cuid; // uid ��������� 
	gid_t cgid; // gid ���������
	mode_t mode; // ������ �������
	ulog seq; // ����� ������������� (TODO, ������� �����)
	key_t key; // ����
};
\end{verbatim}

��� ���� ����� \verb+seq+ �������� ��� �������� ��������� IPC. � ������� ������� \verb+msgctl, +\verb+semctl+, \verb+shmctl+ �� ����� �������� \verb+uid+, \verb+gid+ � \verb+mode+ �������.

\begin{table}[hbtp]
\caption{����� ������� System V IPC ��������}
\begin{tabular}{ l l l l  }
	Permission & Message Queue & Semaphore & Shared Memory \\
	\hline
	user-read  & \verb+MSG_R+ & \verb+SEM_R+ & \verb+SHM_R+ \\
	user-write & \verb+MSG_W+ & \verb+SEM_A+ & \verb+SHM_W+ \\
	\hline
	group-read  & \verb+MSG_R>>3+ & \verb+SEM_R>>3+ & \verb+SHM_R>>3+ \\
	group-write & \verb+MSG_W>>3+ & \verb+SEM_A>>3+ & \verb+SHM_W>>3+ \\
	\hline
	other-read  & \verb+MSG_R>>6+ & \verb+SEM_R>>6+ & \verb+SHM_R>>6+ \\
	other-write & \verb+MSG_W>>6+ & \verb+SEM_A>>6+ & \verb+SHM_W>>6+ \\
	\hline
\end{tabular}
\label{sysvipc_perm}
\end{table}

\section{������� ���������}

\emph{������� ���������} - ��� ��������� ������ ���������, ���������� � ���� � ������������ ��������������� ������� (Queue ID).

������� ����� ������� ��� ������� ��� ������������ ����� \verb+msgget()+.

����� ��������� ����������� � ����� ������� ����� \verb+msgsnd()+. ������ ��������� ����� ��� (������������� ����� �����), �����, � ������ ���� �����, ������� ���������� � \verb+msgsnd()+ ��� ���������� � �������.

��������� ����������� �� ������� ����� \verb+msgrcv+. �� ����� ��������� ��������� ����������� �� �� ����.

\subsection{�������� �������}
\begin{verbatim}
#include <sys/types.h>
#include <sys/ipc.h>
#include <sys/msg.h>

int msgget(key_t key, int flag);
\end{verbatim}


\backmatter
\addcontentsline{toc}{part}{Литература}

\begin{thebibliography}{99}
\bibitem{Rob} �. �����������. ������������ ������� UNIX. ������ ������� - BHV, 2005.
\bibitem{Nemeth} �. �����, �. ������� � ��. UNIX: ����������� ���������� ��������������. ��� ��������������. 3-� ���. - BHV, 2002
\bibitem{Esr} ���� �������. ��������� ���������������� ��� Unix (The Art of Unix Programming). - �������, 2005.
\bibitem{Glass} ���� �����, ���� �����. Unix ��� ������������� � �������������. - BHV, 2004.
\bibitem{Moli} �. ����. Unix/Linux: ������ � �������� ����������������. - �����-�����, 2004.
\bibitem{Ker} �. ��������, �. ����. UNIX. ����������� ���������. ������, 2003
\bibitem{Friedle} ��. �����. ���������� ���������. ���������� ������������. - �����, 2001. 
\bibitem{Robbins} ������� �������. Linux. ���������������� � ��������. - �����-�����, 2005. 
\bibitem{Pet} �. ������. LINUX. �� ��������� � ����������. - ���, 2000.
\bibitem{NetProgVol1} �. �������.  UNIX: ���������� ������� ����������. - �����, 2003.
\bibitem{NetProgVol2} �. �������.  UNIX: �������������� ���������. - �����, 2002.
\bibitem{Asp} ASPLinux. ����������� ������������. - ASPLinux, 2001.
\bibitem{Bah} �. ���. ����������� ������������ ������� UNIX. http://www.lib.ru.
\bibitem{Sol} �. ��������. Sed � awk. ������� �������.
\bibitem{shell} ���������������� ��  shell (Unix).
\bibitem{bootle} ���� �����-����. �������� � Unix. - ����, 1995.
\bibitem{Sage} Rassel S. Sage. ������ ���������������� ������ � UNIX. http://www.citforum.ru.
\bibitem{Mak} ��. ��������. UNIX. - ���������, 1996.
\bibitem{text} �. ������. Linux. ��������� ������. ����������� ����������. - �����, 2001 
\bibitem{cvsbook} Free Software Foundation. CVS book. 1993-2004.
\bibitem{ProgPract} �. ��������, �. ����. �������� ����������������. - �������, 2004

\end{thebibliography}


\chapter{GNU Free Documentation License}
%\label{label_fdl}

 \begin{center}

       Version 1.2, November 2002


 Copyright \copyright 2000,2001,2002  Free Software Foundation, Inc.
 
 \bigskip
 
     59 Temple Place, Suite 330, Boston, MA  02111-1307  USA
  
 \bigskip
 
 Everyone is permitted to copy and distribute verbatim copies
 of this license document, but changing it is not allowed.
\end{center}


\begin{center}
{\bf\large Preamble}
\end{center}

The purpose of this License is to make a manual, textbook, or other
functional and useful document "free" in the sense of freedom: to
assure everyone the effective freedom to copy and redistribute it,
with or without modifying it, either commercially or noncommercially.
Secondarily, this License preserves for the author and publisher a way
to get credit for their work, while not being considered responsible
for modifications made by others.

This License is a kind of "copyleft", which means that derivative
works of the document must themselves be free in the same sense.  It
complements the GNU General Public License, which is a copyleft
license designed for free software.

We have designed this License in order to use it for manuals for free
software, because free software needs free documentation: a free
program should come with manuals providing the same freedoms that the
software does.  But this License is not limited to software manuals;
it can be used for any textual work, regardless of subject matter or
whether it is published as a printed book.  We recommend this License
principally for works whose purpose is instruction or reference.


\begin{center}
{\Large\bf 1. APPLICABILITY AND DEFINITIONS}
\addcontentsline{toc}{section}{1. APPLICABILITY AND DEFINITIONS}
\end{center}

This License applies to any manual or other work, in any medium, that
contains a notice placed by the copyright holder saying it can be
distributed under the terms of this License.  Such a notice grants a
world-wide, royalty-free license, unlimited in duration, to use that
work under the conditions stated herein.  The \textbf{"Document"}, below,
refers to any such manual or work.  Any member of the public is a
licensee, and is addressed as \textbf{"you"}.  You accept the license if you
copy, modify or distribute the work in a way requiring permission
under copyright law.

A \textbf{"Modified Version"} of the Document means any work containing the
Document or a portion of it, either copied verbatim, or with
modifications and/or translated into another language.

A \textbf{"Secondary Section"} is a named appendix or a front-matter section of
the Document that deals exclusively with the relationship of the
publishers or authors of the Document to the Document's overall subject
(or to related matters) and contains nothing that could fall directly
within that overall subject.  (Thus, if the Document is in part a
textbook of mathematics, a Secondary Section may not explain any
mathematics.)  The relationship could be a matter of historical
connection with the subject or with related matters, or of legal,
commercial, philosophical, ethical or political position regarding
them.

The \textbf{"Invariant Sections"} are certain Secondary Sections whose titles
are designated, as being those of Invariant Sections, in the notice
that says that the Document is released under this License.  If a
section does not fit the above definition of Secondary then it is not
allowed to be designated as Invariant.  The Document may contain zero
Invariant Sections.  If the Document does not identify any Invariant
Sections then there are none.

The \textbf{"Cover Texts"} are certain short passages of text that are listed,
as Front-Cover Texts or Back-Cover Texts, in the notice that says that
the Document is released under this License.  A Front-Cover Text may
be at most 5 words, and a Back-Cover Text may be at most 25 words.

A \textbf{"Transparent"} copy of the Document means a machine-readable copy,
represented in a format whose specification is available to the
general public, that is suitable for revising the document
straightforwardly with generic text editors or (for images composed of
pixels) generic paint programs or (for drawings) some widely available
drawing editor, and that is suitable for input to text formatters or
for automatic translation to a variety of formats suitable for input
to text formatters.  A copy made in an otherwise Transparent file
format whose markup, or absence of markup, has been arranged to thwart
or discourage subsequent modification by readers is not Transparent.
An image format is not Transparent if used for any substantial amount
of text.  A copy that is not "Transparent" is called \textbf{"Opaque"}.

Examples of suitable formats for Transparent copies include plain
ASCII without markup, Texinfo input format, LaTeX input format, SGML
or XML using a publicly available DTD, and standard-conforming simple
HTML, PostScript or PDF designed for human modification.  Examples of
transparent image formats include PNG, XCF and JPG.  Opaque formats
include proprietary formats that can be read and edited only by
proprietary word processors, SGML or XML for which the DTD and/or
processing tools are not generally available, and the
machine-generated HTML, PostScript or PDF produced by some word
processors for output purposes only.

The \textbf{"Title Page"} means, for a printed book, the title page itself,
plus such following pages as are needed to hold, legibly, the material
this License requires to appear in the title page.  For works in
formats which do not have any title page as such, "Title Page" means
the text near the most prominent appearance of the work's title,
preceding the beginning of the body of the text.

A section \textbf{"Entitled XYZ"} means a named subunit of the Document whose
title either is precisely XYZ or contains XYZ in parentheses following
text that translates XYZ in another language.  (Here XYZ stands for a
specific section name mentioned below, such as \textbf{"Acknowledgements"},
\textbf{"Dedications"}, \textbf{"Endorsements"}, or \textbf{"History"}.)  
To \textbf{"Preserve the Title"}
of such a section when you modify the Document means that it remains a
section "Entitled XYZ" according to this definition.

The Document may include Warranty Disclaimers next to the notice which
states that this License applies to the Document.  These Warranty
Disclaimers are considered to be included by reference in this
License, but only as regards disclaiming warranties: any other
implication that these Warranty Disclaimers may have is void and has
no effect on the meaning of this License.


\begin{center}
{\Large\bf 2. VERBATIM COPYING}
\addcontentsline{toc}{section}{2. VERBATIM COPYING}
\end{center}

You may copy and distribute the Document in any medium, either
commercially or noncommercially, provided that this License, the
copyright notices, and the license notice saying this License applies
to the Document are reproduced in all copies, and that you add no other
conditions whatsoever to those of this License.  You may not use
technical measures to obstruct or control the reading or further
copying of the copies you make or distribute.  However, you may accept
compensation in exchange for copies.  If you distribute a large enough
number of copies you must also follow the conditions in section 3.

You may also lend copies, under the same conditions stated above, and
you may publicly display copies.


\begin{center}
{\Large\bf 3. COPYING IN QUANTITY}
\addcontentsline{toc}{section}{3. COPYING IN QUANTITY}
\end{center}


If you publish printed copies (or copies in media that commonly have
printed covers) of the Document, numbering more than 100, and the
Document's license notice requires Cover Texts, you must enclose the
copies in covers that carry, clearly and legibly, all these Cover
Texts: Front-Cover Texts on the front cover, and Back-Cover Texts on
the back cover.  Both covers must also clearly and legibly identify
you as the publisher of these copies.  The front cover must present
the full title with all words of the title equally prominent and
visible.  You may add other material on the covers in addition.
Copying with changes limited to the covers, as long as they preserve
the title of the Document and satisfy these conditions, can be treated
as verbatim copying in other respects.

If the required texts for either cover are too voluminous to fit
legibly, you should put the first ones listed (as many as fit
reasonably) on the actual cover, and continue the rest onto adjacent
pages.

If you publish or distribute Opaque copies of the Document numbering
more than 100, you must either include a machine-readable Transparent
copy along with each Opaque copy, or state in or with each Opaque copy
a computer-network location from which the general network-using
public has access to download using public-standard network protocols
a complete Transparent copy of the Document, free of added material.
If you use the latter option, you must take reasonably prudent steps,
when you begin distribution of Opaque copies in quantity, to ensure
that this Transparent copy will remain thus accessible at the stated
location until at least one year after the last time you distribute an
Opaque copy (directly or through your agents or retailers) of that
edition to the public.

It is requested, but not required, that you contact the authors of the
Document well before redistributing any large number of copies, to give
them a chance to provide you with an updated version of the Document.


\begin{center}
{\Large\bf 4. MODIFICATIONS}
\addcontentsline{toc}{section}{4. MODIFICATIONS}
\end{center}

You may copy and distribute a Modified Version of the Document under
the conditions of sections 2 and 3 above, provided that you release
the Modified Version under precisely this License, with the Modified
Version filling the role of the Document, thus licensing distribution
and modification of the Modified Version to whoever possesses a copy
of it.  In addition, you must do these things in the Modified Version:

\begin{itemize}
\item[A.] 
   Use in the Title Page (and on the covers, if any) a title distinct
   from that of the Document, and from those of previous versions
   (which should, if there were any, be listed in the History section
   of the Document).  You may use the same title as a previous version
   if the original publisher of that version gives permission.
   
\item[B.]
   List on the Title Page, as authors, one or more persons or entities
   responsible for authorship of the modifications in the Modified
   Version, together with at least five of the principal authors of the
   Document (all of its principal authors, if it has fewer than five),
   unless they release you from this requirement.
   
\item[C.]
   State on the Title page the name of the publisher of the
   Modified Version, as the publisher.
   
\item[D.]
   Preserve all the copyright notices of the Document.
   
\item[E.]
   Add an appropriate copyright notice for your modifications
   adjacent to the other copyright notices.
   
\item[F.]
   Include, immediately after the copyright notices, a license notice
   giving the public permission to use the Modified Version under the
   terms of this License, in the form shown in the Addendum below.
   
\item[G.]
   Preserve in that license notice the full lists of Invariant Sections
   and required Cover Texts given in the Document's license notice.
   
\item[H.]
   Include an unaltered copy of this License.
   
\item[I.]
   Preserve the section Entitled "History", Preserve its Title, and add
   to it an item stating at least the title, year, new authors, and
   publisher of the Modified Version as given on the Title Page.  If
   there is no section Entitled "History" in the Document, create one
   stating the title, year, authors, and publisher of the Document as
   given on its Title Page, then add an item describing the Modified
   Version as stated in the previous sentence.
   
\item[J.]
   Preserve the network location, if any, given in the Document for
   public access to a Transparent copy of the Document, and likewise
   the network locations given in the Document for previous versions
   it was based on.  These may be placed in the "History" section.
   You may omit a network location for a work that was published at
   least four years before the Document itself, or if the original
   publisher of the version it refers to gives permission.
   
\item[K.]
   For any section Entitled "Acknowledgements" or "Dedications",
   Preserve the Title of the section, and preserve in the section all
   the substance and tone of each of the contributor acknowledgements
   and/or dedications given therein.
   
\item[L.]
   Preserve all the Invariant Sections of the Document,
   unaltered in their text and in their titles.  Section numbers
   or the equivalent are not considered part of the section titles.
   
\item[M.]
   Delete any section Entitled "Endorsements".  Such a section
   may not be included in the Modified Version.
   
\item[N.]
   Do not retitle any existing section to be Entitled "Endorsements"
   or to conflict in title with any Invariant Section.
   
\item[O.]
   Preserve any Warranty Disclaimers.
\end{itemize}

If the Modified Version includes new front-matter sections or
appendices that qualify as Secondary Sections and contain no material
copied from the Document, you may at your option designate some or all
of these sections as invariant.  To do this, add their titles to the
list of Invariant Sections in the Modified Version's license notice.
These titles must be distinct from any other section titles.

You may add a section Entitled "Endorsements", provided it contains
nothing but endorsements of your Modified Version by various
parties--for example, statements of peer review or that the text has
been approved by an organization as the authoritative definition of a
standard.

You may add a passage of up to five words as a Front-Cover Text, and a
passage of up to 25 words as a Back-Cover Text, to the end of the list
of Cover Texts in the Modified Version.  Only one passage of
Front-Cover Text and one of Back-Cover Text may be added by (or
through arrangements made by) any one entity.  If the Document already
includes a cover text for the same cover, previously added by you or
by arrangement made by the same entity you are acting on behalf of,
you may not add another; but you may replace the old one, on explicit
permission from the previous publisher that added the old one.

The author(s) and publisher(s) of the Document do not by this License
give permission to use their names for publicity for or to assert or
imply endorsement of any Modified Version.


\begin{center}
{\Large\bf 5. COMBINING DOCUMENTS}
\addcontentsline{toc}{section}{5. COMBINING DOCUMENTS}
\end{center}


You may combine the Document with other documents released under this
License, under the terms defined in section 4 above for modified
versions, provided that you include in the combination all of the
Invariant Sections of all of the original documents, unmodified, and
list them all as Invariant Sections of your combined work in its
license notice, and that you preserve all their Warranty Disclaimers.

The combined work need only contain one copy of this License, and
multiple identical Invariant Sections may be replaced with a single
copy.  If there are multiple Invariant Sections with the same name but
different contents, make the title of each such section unique by
adding at the end of it, in parentheses, the name of the original
author or publisher of that section if known, or else a unique number.
Make the same adjustment to the section titles in the list of
Invariant Sections in the license notice of the combined work.

In the combination, you must combine any sections Entitled "History"
in the various original documents, forming one section Entitled
"History"; likewise combine any sections Entitled "Acknowledgements",
and any sections Entitled "Dedications".  You must delete all sections
Entitled "Endorsements".

\begin{center}
{\Large\bf 6. COLLECTIONS OF DOCUMENTS}
\addcontentsline{toc}{section}{6. COLLECTIONS OF DOCUMENTS}
\end{center}

You may make a collection consisting of the Document and other documents
released under this License, and replace the individual copies of this
License in the various documents with a single copy that is included in
the collection, provided that you follow the rules of this License for
verbatim copying of each of the documents in all other respects.

You may extract a single document from such a collection, and distribute
it individually under this License, provided you insert a copy of this
License into the extracted document, and follow this License in all
other respects regarding verbatim copying of that document.


\begin{center}
{\Large\bf 7. AGGREGATION WITH INDEPENDENT WORKS}
\addcontentsline{toc}{section}{7. AGGREGATION WITH INDEPENDENT WORKS}
\end{center}


A compilation of the Document or its derivatives with other separate
and independent documents or works, in or on a volume of a storage or
distribution medium, is called an "aggregate" if the copyright
resulting from the compilation is not used to limit the legal rights
of the compilation's users beyond what the individual works permit.
When the Document is included in an aggregate, this License does not
apply to the other works in the aggregate which are not themselves
derivative works of the Document.

If the Cover Text requirement of section 3 is applicable to these
copies of the Document, then if the Document is less than one half of
the entire aggregate, the Document's Cover Texts may be placed on
covers that bracket the Document within the aggregate, or the
electronic equivalent of covers if the Document is in electronic form.
Otherwise they must appear on printed covers that bracket the whole
aggregate.


\begin{center}
{\Large\bf 8. TRANSLATION}
\addcontentsline{toc}{section}{8. TRANSLATION}
\end{center}


Translation is considered a kind of modification, so you may
distribute translations of the Document under the terms of section 4.
Replacing Invariant Sections with translations requires special
permission from their copyright holders, but you may include
translations of some or all Invariant Sections in addition to the
original versions of these Invariant Sections.  You may include a
translation of this License, and all the license notices in the
Document, and any Warranty Disclaimers, provided that you also include
the original English version of this License and the original versions
of those notices and disclaimers.  In case of a disagreement between
the translation and the original version of this License or a notice
or disclaimer, the original version will prevail.

If a section in the Document is Entitled "Acknowledgements",
"Dedications", or "History", the requirement (section 4) to Preserve
its Title (section 1) will typically require changing the actual
title.


\begin{center}
{\Large\bf 9. TERMINATION}
\addcontentsline{toc}{section}{9. TERMINATION}
\end{center}


You may not copy, modify, sublicense, or distribute the Document except
as expressly provided for under this License.  Any other attempt to
copy, modify, sublicense or distribute the Document is void, and will
automatically terminate your rights under this License.  However,
parties who have received copies, or rights, from you under this
License will not have their licenses terminated so long as such
parties remain in full compliance.


\begin{center}
{\Large\bf 10. FUTURE REVISIONS OF THIS LICENSE}
\addcontentsline{toc}{section}{10. FUTURE REVISIONS OF THIS LICENSE}
\end{center}


The Free Software Foundation may publish new, revised versions
of the GNU Free Documentation License from time to time.  Such new
versions will be similar in spirit to the present version, but may
differ in detail to address new problems or concerns.  See
http://www.gnu.org/copyleft/.

Each version of the License is given a distinguishing version number.
If the Document specifies that a particular numbered version of this
License "or any later version" applies to it, you have the option of
following the terms and conditions either of that specified version or
of any later version that has been published (not as a draft) by the
Free Software Foundation.  If the Document does not specify a version
number of this License, you may choose any version ever published (not
as a draft) by the Free Software Foundation.


\begin{center}
{\Large\bf ADDENDUM: How to use this License for your documents}
\addcontentsline{toc}{section}{ADDENDUM: How to use this License for your documents}
\end{center}

To use this License in a document you have written, include a copy of
the License in the document and put the following copyright and
license notices just after the title page:

\bigskip
\begin{quote}
    Copyright \copyright  YEAR  YOUR NAME.
    Permission is granted to copy, distribute and/or modify this document
    under the terms of the GNU Free Documentation License, Version 1.2
    or any later version published by the Free Software Foundation;
    with no Invariant Sections, no Front-Cover Texts, and no Back-Cover Texts.
    A copy of the license is included in the section entitled "GNU
    Free Documentation License".
\end{quote}
\bigskip
    
If you have Invariant Sections, Front-Cover Texts and Back-Cover Texts,
replace the "with...Texts." line with this:

\bigskip
\begin{quote}
    with the Invariant Sections being LIST THEIR TITLES, with the
    Front-Cover Texts being LIST, and with the Back-Cover Texts being LIST.
\end{quote}
\bigskip
    
If you have Invariant Sections without Cover Texts, or some other
combination of the three, merge those two alternatives to suit the
situation.

If your document contains nontrivial examples of program code, we
recommend releasing these examples in parallel under your choice of
free software license, such as the GNU General Public License,
to permit their use in free software.

%---------------------------------------------------------------------

\end{document}
