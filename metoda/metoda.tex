% some formating
\documentclass[11pt,a4paper,twoside]{book}

\usepackage[colorlinks,hyperindex,plainpages=false,pdfencoding=auto,unicode]{hyperref}
\usepackage{keyval}
\hypersetup{pdfauthor={Vlad Shakhov}}
\def\pdfBorderAttrs{/Border [0 0 0] } % No border arround Links
\makeatletter
\newcounter{chapter.H}
\setcounter{chapter.H}{0}
\makeatother

\usepackage{cmap}
\usepackage[utf8]{inputenc}
\usepackage[T2A]{fontenc}
\usepackage[russian,english]{babel}
\usepackage{amssymb}
\pagenumbering{arabic}

\usepackage{polyglossia}   %% загружает пакет многоязыковой вёрстки
\setdefaultlanguage[spelling=modern]{russian}  %% устанавливает главный язык документа
\setotherlanguage{english} %% объявляет второй язык документа

\setmainfont{CMU Serif}           %% задаёт основной шрифт документа
\setsansfont{CMU Sans Serif}      %% задаёт шрифт без засечек
\setmonofont{CMU Typewriter Text} %% задаёт моноширинный шрифт

\title{Учебное пособие по курсу Операционные Системы и Среды (ОСиС)}
\author{Влад 'mend0za' Шахов}
\date{последнее обновление - \today}

\begin{document}

\frontmatter

\maketitle

Copyright \copyright Владимир 'mend0za' Шахов, 2002-2007
\bigskip

Каждый имеет право воспроизводить, распространять и/или вносить изменения в настоящий Документ в соответствии с условиями GNU Free Documentation License, Версией 1.2 или любой более поздней версией, опубликованной Free Software Foundation;

данный документ содержит следующий Текст, помещаемый на первой странице обложки "Учебное пособие по курсу ОСиС";

данный документ не содержит неизменяемых секций;

Копия настоящей Лицензии включена в раздел под названием "GNU Free Documentation License".

\bigskip

\tableofcontents

\mainmatter

%введение
\chapter{����� �������� � �� UNIX}

\section{������ �����������}

��� ������� - ����� 100\% �������. ������ ����� ����� �� �������� ���������� ������. ��� ������� � ���, ��� ����� ������ ��������� ���� ������ � ������ � �������, ��� ������� �������� ������� ���� ���� �������� � ���� ������� �� ������������ ����������, ��� ������ ������ ���. ��� ����� ��� ���������� ������ ������� ����� �������� ����.

��� ������� ������� ������������� ��� ��������� 3 ����� ������������� �����������. ���� ������������ ������� � ����� (����� ����) ���������� �� ��������� ������ ���� � ����� � ������� ������ ����� �� � ����� ��������������.

�������� ������ ����� ���� - ���������� ������������, ���������� ���������������� �����������, ������������ � ����������������� Unix-������ (������������� - �������� �����������������). 

���������������� �������� GNU Free Documentation License (��. ���������� \ref{fdl}). �������� �����������, �������������� � ���������, � ��������� ��������������� ���������. 
  
\section{��������}

��� ����� Unix? ��� ��������� ������������ ������ (��), ���������� ������� ������������ � ���������� � �������������. 

Unix ��� ������� ���������� � ������ 70-� ����� � ����������� �� ��� ���.

�������� ����������� �������� UNIX: Linux, BSD (FreeBSD, NetBSD, OpenBSD), AIX, HPUX, Solaris, SCO. 

��������� ����������� ���������, �������������� ����������� ��������� UNIX:
\begin{itemize}
\item[-] POSIX - Portable Operating System Interface
\item[-] ANSI C (c89 � �99) 
\item[-] Opengroup\footnote{Opengroup - �����������, ������� ���������� ������ ���������� �� Unix-������� (www.opengroup.org). �������� �������� ����� Unix.} Single Unix Specification Version 3 (����� SUSv3). 
\end{itemize}
����������. ����� ��� ������ Unix �� ����� ������������� ��� Unix-�������� ������������ �������, ���� �� ������� ���������� �������.

\subsection{�������� �����}

\begin{itemize}
\item ��� �� �� - ��������� ���������� � �������� ��.
\item ������������� ��������������������� ��.
\item ������� ���������� - ������� ����� ��������� �������� ���������� ��������� � ��� ����������� �����������.
\item �������, �� ������ ���������������� ��������� (��������� ������).
\item E����� ����������� �������� �������. ����� ��������� �������� ������� �������������� ������ � ������, ����������, ���������, ������, ���� � ���� � ����������� ������.
\item ������� ���������� ������������ �����������.
\end{itemize}

\subsection{��������� �������}

������������ ����������� UNIX �������������:
\begin{enumerate}
\item ���� - ��������� ��������� ���������� � ���������� ���������� ������� ����� ����� (��������� ������).
\item ��������� ��������� (���������� �����, �����������, �������), ���������� ��������� (���������, �������, ����������� � �.�.). 
\end{enumerate}

\subsubsection{������� ����}

\begin{itemize}
\item \emph{������������� �������} - �������� � ������ ��
\item \emph{���������� ���������� � ��������} 
\item \emph{���������� �������} - ����������� ��������� ������������ �� ���������� ������, ���������� ������������� ������ ����������
\item \emph{���������� �������} - ��������� ������� �������� �������, ������ ��������� � ������ 
\item \emph{����� ������� ����� ����������} 
	\begin{enumerate}
	\item �������������� ������ ������ ����������
 	\item � ������ ����� ����� �������� ������
 	\item � ����� ����� ���������� � ���������� ������� ���������
	\end{enumerate}
\item \emph{����������� ��������� (API)} - ������������ ������ � ������������ ���� �� ������� ��������� ������������ ����� ��������� ������, ����������� � ���� ���������� ������� �� ��.
\end{itemize}

\subsubsection{��������� ������}

���� ��������� ��������� ������������ �� ����������. ��� ����� �������, �� ������ ��������� ����� ����, ��������� ���������� �� ����������� ���������� � �������� �� ��.

\emph{��������� ������} - ��� �������, ���������� ����������� ����������� ��������� ���������� �� ������ ���������� �� �������� ������������. ��� ������������, ��������� ����� - ��� ������� (������������ �� ��), ������� �� �������� � ����� ���������. ��� �������������� �������� �������������� ����� ��������� ������.

\subsubsection{���������� ����}

\begin{enumerate}
\item�������� ����������. ������������ ��������������� ������ � ������:
	\begin{itemize}
	\item�������� ���� ������� � �����;
	\item������/������ �����;
	\item���������� � �������� �����;
	\item��������������� ��������  � ������������ �����������, ��������������� ������� ���������� �����/������.
	\end{itemize}
\item���������� ���������� ����������.
 
	���������� �� ���������� ��������� ��������� ���� ��� ��������� ��������� (�����). \\
���������� ������������:
	\begin{itemize}
	\item�������� � �������� ���������
	\item������������� ��������� �������� (������, �������������� ��������) ����� ����������
	\item������������� ���������
	\item������������� ��������������
	\end{itemize}

  ����������� ������ ����  \emph{����������� ���������} ��������� ��������� ��������� � ����������� �� �������. 
\item���������� �����/������. ��������� ������� �������� ������� � ���������� ���������� ���������� ��� ������� � ������������ ����������� (������, ������, ����������). ������������ ����������� ������ � ��������������� � ���������� ���������.
\end{enumerate}

\subsection{�� UNIX ��� ������������}
\subsubsection{������������}

� ������ ������ �� UNIX ��������������� ��� ������������� �������. ����� ������ ������, ����� ����� � �������, ����� �� ���������� ��������� (�������) ���� \emph{���} (account name, �����) � \emph{������} (password).

�������, ������������������ � ������� �, �������������, ������� ������� ���, ���������� \emph{������������������ ������������� �������}.

����������� ����� ������������� ������ ��������� ������������� �������. ������������ �� ����� �������� ���� ������� ���, �� ����� ���������� ��� �������� ������. ������ ��������� � ��������� ����� � �������������� ����.

��� ������������ ��� ��� ����� �������� � �������. �������� ������� ����� ����������� ���������. � ������� ������������������� ������������ ���� \emph{�������� �������}. � ���� �� ����� ������ ������. � ������ ��������� ������ ������ ���������.

\subsubsection{��������� ������������}

������������ ��������� - \emph{��������� ������}. ����� ����� � ������� ��� ������������ �������������� ������ ����� �� \emph{��������� ��������}. ����� �������� \emph{shell (eng. - ��������)}, ��� ��� ��� �������� ������� ���������� ���� �������. �������� - ��� ������������� ������� (���������� � �������) �  �������� ������ ���������� ������ shell scripts, ����������� ������ ������� ���������. 

\subsubsection{GUI � Unix}

����������� ��������� (GUI) �� �������� ����������� � Unix � ��������������� �� �����. ����������� ����������� ������� �������� ����������� ��� ��� �������.

%командная строка
\chapter{Командная строка}

\section{Основные принципы и команды}

Как следует из названия, основное назначение командной оболочки - ввод и исполнение команд. Для ввода команд служит т.н. \emph{командная строка}, содержащая приглашение к вводу.

В примерах приглашение командной строки будем обозначать \verb+$+.

После ввода имени и пароля, пользователь получает приглашение на ввод. После окончания работы, он выходит, набирая \verb+exit+ или \verb+logout+. 

Большинству команд можно передавать дополнительную информацию. Это делается с помощью \emph{аргументов и ключей}.\\
Общий вид команды:
\verb+команда [-ключи] аргумент1 ... аргументN+

Простейшая команда - \verb+echo+. Она просто выводит на экран свои параметры.

\emph{Ключ} - 1 или более букв, перед которой ставится \verb+-+(минус или дефис). Обычно ключи управляют режимами работы команды.

\emph{Аргумент} - строка, передаваемая команде.
\begin{verbatim}
Пример 
$kill -9 1023
kill имя команды (завершение процесса)
-9  режим (безусловное завершение процесса)
1023 номер процесса (завершаемый процесс) 
$ - символ приглашения ввода (в разных системах - разный)
\end{verbatim}

После ввода команды, система производит поиск в каталогах программ, заданных переменной \emph{PATH} \label{path}. Если команда найдена, то она будет запущена. Иногда, если PATH не содержит нужного каталога, нужно указать полный путь (см. \ref{fullpath}) к программе.

\emph{Примечание.} Текущий каталог НЕ ВХОДИТ в PATH. Программы из него запускаютсяь следующим образом:\\
\verb+./program+
 
\section{Встроенная справка}

Любая Unix система обладает развитой системой справки. Она называется \verb+man+ (от англ. слова manual). 

Справка запускается следующим образом: \verb+man статья+\footnote{Будет показана \emph{только первая} доступная статья}. Для перемещения используются клавиши \verb+"вверх", "вниз"+. Для выхода нажмите \verb+q+.

\verb+man+ делится на несколько разделов, которые пронумерованы\footnote{Здесь приведены только те разделы, которые будут активно использоватьсяв рамках курса ОСиС. Описания других разделов вы сможете найти в книгах \cite{Rob} и \cite{Pet}}:
\begin{itemize}
\item 1 -  Команды,  которые  могут  быть   запущены пользователем
\item 2 -  Системные вызовы (функции, исполняемые ядром)
\item 3 -  Библиотечные вызовы (функции из библиотек различных языков программирования)
\item 5 -  Форматы файлов и соглашения
\end{itemize}

Запись типа \verb+bash(1)+ обозначает, что справка о команде \verb+bash+ находиться в разделе 1 (команды). Бывает что справки в различных разделах называются одинаково. Тогда для получения нужной надо явно указать раздел.

\begin{verbatim}
Примеры использования man
$ man echo
 
Получение справки по самой команде man
$ man man

Справка из конкретного раздела
$ man 2 open

Все статьи с таким названием из всех разделов
$ man -a mount
\end{verbatim}

Некоторые команды содержат свою собственную систему помощи\footnote{в основном это касается команд написанных проектом GNU}, Она вызывается с помощью ключа \verb+--help+ или вызовом без параметров
\begin{verbatim}
Примеры
$ file
Usage: file [-bciknvzL] [-f namefile] [-m magicfiles] file...
Usage: file -C [-m magic]

$ split --help
Usage: split [OPTION] [INPUT [PREFIX]]
Output fixed-size pieces of INPUT to PREFIXaa, PREFIXab, ...; default
PREFIX is `x'.  With no INPUT, or when INPUT is -, read standard input.

  -b, --bytes=SIZE        put SIZE bytes per output file
  -C, --line-bytes=SIZE   put at most SIZE bytes of lines per output file
  -l, --lines=NUMBER      put NUMBER lines per output file
  -NUMBER                 same as -l NUMBER
      --verbose           print a diagnostic to standard error just
                            before each output file is opened
      --help              display this help and exit
      --version           output version information and exit

SIZE may have a multiplier suffix: b for 512, k for 1K, m for 1 Meg.

Report bugs to <bug-textutils@gnu.org>.

\end{verbatim}


\section{Перемещение по файловой системе} 


Файл в Unix - это основа всего. Существует высказывание "Unix - это файлы".

Имя файла может состоять из любых символов, которые можно ввести с клавиатуры. Точка не имеет особого значения - имя файла может содержать ее или нет\footnote{можно создать файл или каталог с именем из одних точек}. Обычно пользователи присваивают расширения своим файлам, чтобы различать их тип. Система \emph{не связывает типы файлов с их расширениями}. Например файл \verb+example.txt+ может быть исполняемым (программой), а может и не быть. А \verb+/bin/sh+ как правило исполняемый.

Однако у точки есть специальные значения 
\begin{itemize}
\item точка в начале файла указывает на то, что файл - скрытый, например \verb+.profile+
\item файл \verb+.+ - ссылка на текущий каталог (см. пример относительных путей)
\item файл \verb+..+ - ссылка на родительский каталог
\end{itemize}

Если вы не знаете тип файла, примените команду \verb+file+.
\begin{verbatim}
$ file /home/work/OSIS/LAB/RUsak/demon1.c 
/home/work/OSIS/LAB/RUsak/demon1.c: ASCII C program text, with very long lines

$ file /bin/bash
/bin/bash: ELF 32-bit LSB executable, Intel 80386, version 1, dynamically linked
\end{verbatim}

Unix различает строчные и заглавные буквы в именах. Например \verb+name1.txt.125+ и \verb+NAME1.TXT.125+ - это разные файлы.

Файловая система ОС Unix является единой. Это значит что, все файлы находятся в рамках одной логической структуры - \emph{дерева каталогов}. В Unix отсутствует понятие "диск" и "буква устройства". 

Например - файлы с дискеты обычно находятся в \verb+/mnt/floppy+, но могут быть и в любом другом месте.

Началом дерева является корневой каталог (корень), обозначаемый \verb+/+ . 

Разделителем каталогов является прямой слэш  \verb+/+ .

\emph{Путь} - последовательный список каталогов, который нужно пройти, чтобы достигнуть файла или каталога.

Есть 2 вида путей: \emph{абсолютный} и \emph{относительный}.\\
\emph{Абсолютный}\label{fullpath} - полный путь относительно корневого каталога.
\begin{verbatim}
Примеры абсолютных путей:
/etc/init.d/apache
/bin/sh
/usr/local/share
\end{verbatim}
\emph{относительный} - путь относительно текущего каталога \label{relativepath}
\begin{verbatim}
Примеры относительных путей:
./a.out
laba1/text.cpp
../index.html
\end{verbatim}

\subsection{Команды перемещения}

\begin{itemize}
\item \verb+cd+ - перемещение между каталогами. Может использоваться в 2-х вариантах:
\begin{verbatim}
перемещение по указанному пути
$cd  /usr/local 

вернуться в домашний каталог
$cd
\end{verbatim} 

\item \verb+pwd+ - показать текущий каталог
\begin{verbatim}
$pwd 
/home/user/OSiS/metoda
\end{verbatim} 

\item \verb+ls+ - показать содержимое каталога(файлы и подкаталоги). Имеет множество ключей и режимов работы
\begin{verbatim}
Без параметров - вывод содержимого текущего каталога
$ ls
literatura.aux  Makefile    metoda.dvi  metoda.tex  part1.aux  part2.aux
literatura.tex  metoda.aux  metoda.log  metoda.toc  part1.tex  part2.tex

C аргументом - вывод содержимого этого каталога
$ ls /usr
bin   doc  games    kerberos  libexec  man                sbin   src  X11R6
dict  etc  include  lib       local    OpenOffice.org1.0  share  tmp

С ключом "-l" - полный формат вывода (с доп информацией)
$ ls -l /home/user/OSiS
итого 710
drwxrwxr-x    2 user     user         1024 Июн 11 14:11 Lectures
-rw-rw-r--    1 user     user       123686 Июн 12 12:53 lectures.rar
-rw-rw-rw-    1 user     user         1789 Июн 11 15:44 lhr10.log
-rw-r--r--    1 user     user            0 Июн 11 15:44 lkypc.pdf
-rw-rw-r--    1 user     user       438112 Апр 20  1999 lshort.dvi
-rw-r--r--    1 user     user       112747 Июн 12 12:53 lshrtdvi.zip
drwxrwxr-x    2 user     user         1024 Июн 14 18:50 metoda
-rw-rw-r--    1 user     user        40960 Июн 11 14:06 metoda.doc
\end{verbatim} 

\end{itemize}

\subsection{Подключение других устройств(дисковода, CD-ROM)}

Чтобы подключить устройство (дисковод, CD-ROM, раздел вичестера, сетевой диск), нужно указать место, куда будет отображаться его содержимое. Это место называется \emph{точка монтирования}. Точка монтирования - это обычный каталог. После подключения (в терминах Unix - \emph{монтирования}) каталог будет содержать файлы, расположенные на устройстве. Список подключенных устройств и монтирование - команда \verb+mount(8)+.

Обычно поключаемые устройства отображаются на каталог \verb+/mnt+. Например \verb+/mnt/cdrom+, \verb+/mnt/floppy+.

Отключение (\emph{размонтирование}) производит отсоединение устройства от дерева каталогов. Команда \verb+umount(8)+.

\section{Копирование, удаление, перемещение файлов и каталогов}
\begin{itemize}
\item \verb+cp+ - копирование
\begin{verbatim}
копирование одного файла в другой
$ cp src.file /tmp/dest.file1

копирование нескольких файлов в другой каталог
$ cp *.tex metoda.dvi /mnt/floppy

копирование каталогов
$ cp -r metoda /archive/old.Docs
\end{verbatim}

\item \verb+rm+ - удаление файлов и \verb+rmdir+ - удаление каталогов\footnote{работает только для \emph{пустых} каталогов}
\begin{verbatim}
простое удаление
$ rm part1.aux intro.temp

удаление всех файлов из каталога 
$ rm /tmp/*

удаление каталога
$ rmdir oLD.stupid.dir
\end{verbatim}

\item \verb+mv+ - переместить(переименовать)
\begin{verbatim}
переместить один файл в другой
$ mv src.file /tmp/dest.file1

перемещение нескольких файлов в другой каталог
$ mv *.tex metoda.dvi /mnt/floppy

перемещение каталогов
$ mv metoda /archive/old.Docs
\end{verbatim}

\item \verb+mkdir+ - создать каталог
\begin{verbatim}
cоздать 1 или более каталогов
$ mkdir /tmp/dir.12345
$ mkdir empty.DIR DiRecTorY
\end{verbatim}
\end{itemize}

\section{Информация о системе}

\begin{itemize}
\item \verb+ps+ - список запущенных процессов
\item \verb+who+ - список пользователей, работающих в системе 
\item \verb+date+ - текущая дата и время
\item \verb+w+ - общая информация о системе
\end{itemize}
\begin{verbatim}
Примеры:
Процессы на текущей консоли
$ ps
 3642 pts/1    00:00:00 bash
 4548 pts/1    00:00:00 ps

Процессы конкретного пользователя
$ ps -u user
 1766 ?        00:00:04 xterm
 1853 pts/2    00:00:13 vim
 3642 pts/1    00:00:00 bash
 4553 pts/1    00:00:00 ps

Все процессы
$ ps -ef (для BSD и Linux - ps ax )
UID        PID  PPID  C STIME TTY          TIME CMD
user      1766  1176  0 06:45 ?        00:00:04 xterm -title Terminal
user      1768  1766  0 06:45 pts/2    00:00:00 bash
user      1853  1768  0 06:56 pts/2    00:00:13 vim part2.tex
	<пропущено - список 52 строки>

$ date
Сбт Июн 15 10:34:17 EEST 2002

$ who
root     tty1     Jun 15 10:24
work     tty2     Jun 15 10:24
user     pts/1    Jun 15 09:18

$ w
 10:35am  up  4:14,  3 users,  load average: 0.09, 0.04, 0.01
USER     TTY      FROM              LOGIN@   IDLE   JCPU   PCPU  WHAT
root     tty1     -                10:24am 11:25   0.02s  0.02s  -bash 
work     tty2     -                10:24am 11:11   0.03s  0.03s  -bash 
user     pts/1    -                 9:18am  1.00s  0.14s  0.01s  w 
\end{verbatim}

\section{Общение между пользователями}
\begin{itemize}
\item \verb+write пользователь+ - послать сообщение\footnote{ ctrl+d воспринимается системой как конец ввода}
\begin{verbatim}
$ write stud11
набрать текст сообщения
нажать <ctrl+d>
\end{verbatim}
\item \verb+talk пользователь+ - двухсторонний чат 
\item \verb+mail+ - электронная почта
\begin{verbatim}
Послать письмо
$  mail stud7 //локальному пользователю
Subject: test
набрать текст сообщения
нажать <ctrl+d>

$ mail user@tut.by //удаленному пользователю
.....

$ mail // просмотреть свою почту
.....
\end{verbatim}
\end{itemize}

\section{Просмотр, создание, объединение файлов}

\begin{itemize}
\item \verb+cat+ - вывод содержимого файла на экран 
\item \verb+more+ - разбиение входного файла на страницы
\item \verb+less+ - просмотр файлов (выход по клавише \verb+q+)
\end{itemize}

\begin{verbatim}
Примеры
Вывод на экран файла
$ cat file

Копирование файла
$ cat file >file2

Объединение 2-х файлов в третий
$ cat file1 file2 >end.file

Постраничный вывод файла на экран
$ cat large.file |more 

Ввод файла с клавиатуры
$ cat >new.text
<набирается текст>
<ctrl+d>

\end{verbatim}

\section{Объединение команд} 

Идеологически, Unix - это собрание большого количества небольших утилит, выполняющих какую-то одну локальную задачу. Но делающими свою задачу лучше всего, во всех возможных вариантах и режимах. 

\emph{Фильтры} - это программы, предназначенные для обработки текста тем или иным способом.

Часто они применяются в связке с другими командами, образуя \emph{конвейеры}. Конвейер обозначается символом \verb+|+. Его значение следующее: програма слева от конвейера передает свой вывод на вход программы справа от конвейера\footnote{конвейер может применяться несколько раз}. Простейший пример: \verb+$ cat myFILE|more+\footnote{То что выводит cat посылается на вход more. А more выступает здесь в качестве простого фильтра.}. В Unix этот метод называют \emph{перенаправление}. Подробнее о перенаправлении - в \ref{redirect}.

В общем случае, фильтры читают свой \emph{стандартный ввод} и пишут на \emph{стандартный вывод}. Если не было перенаправления, то вводом считается клавиатура, а выводом - экран.
\begin{center}
Часто употребимые фильтры
\end{center}
\begin{itemize}
\item \verb+grep+\footnote{Существует целое семейство grep-команд : grep, egrep, fgrep} - поиск в файле по образцу
\\ \verb+Пример: $ ls /usr/include | grep "stdlib.h" +
\item \verb+sort+ - сортировка содержимого
\item \verb+wc+ - статистика по файлу (кол-во букв, байт, строк и т.п.)
\item \verb+head+ и \verb+tail+ - показать начало(head) и конец(tail) файла
\item \verb+tee+ - одновременный вывод в файл и на экран
\\ \verb+Пример: $ ls -1 /tmp | tee all.tempfiles+
\end{itemize}

\section{Приемы эффективной работы}

Оболочка \verb+bash+ обладает 3 базовыми средствами автоматизации\footnote{Подробнее об автоматизации BASH - в \cite{Asp}, \cite{Pet}, man bash}, делающими работу в командной строке простой и легкой:
\begin{enumerate}
\item Автодополнение путей и команд

\emph{Использование}: набрать 1 или более начальных символов команды и нажать \verb+TAB+. Если символов хватает для определения команды, то недостающие будут добавлены автоматически. Если существует более 1 подходящей команды, то при повторном нажатиии \verb+TAB+ на экран высветиться список возможных команд. Можно тогда добавить несколько букв и однозначно определить команду.
\begin{verbatim}
$ la<TAB>
lambda      last        lastb       lastlog     latex       latex2html
$ last<TAB>
last     lastb    lastlog  
$ lastb<ENTER>
lastb: /var/log/btmp: No such file or directory
Perhaps this file was removed by the operator to prevent logging lastb info.
\end{verbatim}
Для поиска комманд используется переменная PATH (см. \ref{path}). 

Точно так же работает автодополнение для путей: вводиться кусочек пути и после нажатия <TAB> происходит дополнение пути.\\

\emph{Примечание}. Автодополнение не работает для ключей комманд (\verb+ls --he<TAB>+ - будет безрезультатным) и для аргументов комманды \verb+man+.
\item История команд

Для просмотра истории - \verb+history+. Для обращения к конкретному пункту истории - \verb+!номер+. Можно прокручивать историю с помощью клавиш \verb+"вверх"+ и \verb+"вниз"+. 
\item Редактирование командной строки

Возможно с помощью клавиш \verb+"вправо"+ и \verb+"влево"+. На начало строки - \verb-ctrl+a-, на конец - \verb-ctrl+e-.
\end{enumerate}

%оболочка (shell)
\chapter{Shell (��������)}
\section{������� ��������}

\emph{�������� (��������� ��������������, ����������, shells)} ������������ ����� ������������� ������� ����� ������������� � ��. ��� ����������� ��������� ������, ��������� �������������� ���������� ������, ������ � �������� �������.

Shell  ��� ������� ���������� ���������. ��� �� �������� ������ ����, � ������� ����� ���� �������� �� ����� ������, ��������, �� ������� ��� ��������� ��������. �� �������� ����� ���� �������� ������ �������� (��� ����� ��), ��� ���� �������������� �����������.

���� shell: sh, csh, ksh, zsh, tcsh, ash � ������.

� ����� ����� ��������������� \emph{Bourne Shell - sh}. ��������, ����������� � Bourne Shell ���������� ��� ��� ������ Unix. �� ������������ ������� ����� �������������� \verb+bash+ (Bourne Again SHell).

\section{Bourne Shell}

\emph{�������� (������)} �������� ������������ ����� ��������� ����, ������� ������ ���������� ������������������ ��������. �������� ����� ��������� ����� ������������������ ������ (��� ���������� ������ ��������, ��� � ������� ������ UNIX, � ����������� ��� ��� ���), ������� �������� ��� ������ ���������� ����� ���������.

��� Shell ����� �������, ��� ��� � ��������� � ���� ����������������.

������ ������������ � ��������� ����� ����� �� ���������� �� ���������� �������� ����������� �������.

�������� ������ �������: ��� ������� ������� �� ��������� ������ ����������� ����� ��������������, ��� �������� ������ ������ ������ (���-����� ������������ ��� ������ ��� �������).

\emph{������� �������}: �� �����, bash ���.\\
��� ������� �� �����, ���� ������� ������� ����� ������� X (eXecutable - �����������). ���������  �� ��������� ����� ��������� � \ref{attrib} 


\subsection{��������� ��������}

\verb+#+ - ��, ��� ������� �� ���(� ��� ����� � ������ \verb+#+), �������� ������������. ����������� ����� �������� ��� ������ ��� ��������� �� ��������.\\
\verb+\+ - ����������, ��� ������ ����������� �� ��������� ������ �����.

����� �������� ��������� ������ � 1 ������, ����������� \verb+;+ \label{tz}.

\section{����������}

�������� ���������� - ������, ������� ���������� �������������.\\
\verb+V1 = 5; v2 = "string"+ \footnote{�������� ���������� ����� ���������� �� ������� � ������}\\
���������� ����� �����  ����������� ��������, ������� ������������ ��������.\\
\verb+V3 = `pwd`+

\begin{center}
��������� ��������
\end{center} 
\begin{itemize}
\item \verb+$���_����������+ - � ��� ����� ������������� �������� ����������.
\item \verb+$(���_����������)+ -  �������� ���������� �� ����������� ��������.
\begin{verbatim}
������: 
$ echo result = $(v1)2 +
result = 52+
\end{verbatim}
\end{itemize}

� shell ���������� ��� \emph{����������������} ����������\footnote{������ ������ ���������� � ���������������� ���������� ����� ����� � man bash}:
\begin{itemize}
	\item \verb+HOME+ - �������� ������� ������������
	\item \verb+PATH+ - ���� ������ ����������� �������� 
	\item \verb+MAIL+ - ������ ��� ����� � ������ ������������
	\item \verb+PS1,PS2+ - ��������� � ��������� ����������� shell (������ \verb+$+, ������� ������� � �������� - ��� ��������� �����������).
\end{itemize}

� shell ���������� ��� ����������, ������� ������������ ��������� �� ���� ���������� ��������. ��� ��� ���������� \emph{����������} ����������:
\begin{enumerate}
	\item \verb+$0, $1, $2, ... ,$9+  �������� ����������, ������������ ������� �� ��������� ������. 
	\item \verb+$0+  - ��� ������ �������.\\
		\emph{����������}: ������� ����� ����������������  �������� ��� ������� �� \verb+$0+ � ���������� �������� ���������.
	\item \verb+$#+ - ����� ����������, ���������� �������;
	\item \verb+$*+ - ��� ���������, ���������� �������. ������������ ����� ������ �����, ����������� � �������.
\end{enumerate}

���������� ��� ���� ������� ��� ������������ ����������:
\begin{itemize}
	\item \verb+' '+ - ����������������  �����������,  ��������: \verb+v4='$v1'+  ��������  \verb+$v1+, � �� 5.
	\item \verb+" "+ - ����������� ����� ������������� �������� \verb+\+ � \verb+$+. ��������: \verb+v5=$v1+. ���������� \verb+v5+ ���������� 5.
		������ ��� �������������� �������� ������������. 
		\begin{verbatim}
		�������� ������������
        		v6=string
        		v7="string"
        		v8='string'
		���� ���������� ���������. ������� ������ ���� ���������� ������.
		\end{verbatim}
	\item \verb+` `+ - ���������� ������� ������ ������. ��������� ���������� ������� ����� �������� ����������.\\
	\verb+������: $list=`ls -a`+
\end{itemize}

�� ��������� ��� ���������� ��������, �� ���� ����������, ���� ����������� ������. ����� ������� �� ����������� (��� ������� shell), ���� ������ �� ��� ������ export. \verb+��������: export v1+

��� ������ �������������� ���������� ����������� ����� ������ ������.

��� �������� ���������� ������������ \verb+unset ������_����������+. \verb+������: $ unset $v1 $v3 $v4+

������� \verb+set+ ������� ������ ���� ������������� ���������� shell.


\section{��������������� �����/������}.

\label{redirect}������ ���������, ���������� �� shell, �������� ��� �������� ������ �����/������, ������� �� ��������� ������������� � ����������. ������ �������� ��������(\emph{�����������}):
\begin{itemize}
\item 0 ����������� ����� �����, ������������� � �����������
\item 1 ����������� ����� ������, ������������� � �������
\item 2 ����������� ����� ������, ������������� � �������
\end{itemize}

����������� ������ Unix ���������� ������ ����������� ������, ������� ��� ���� ������ ����� ������������ ���������������.
\begin{center}
���� ���������������
\end{center}
\begin{itemize}
	\item \verb+>file+ - ����� ������ ���������������� � ����. \verb+������: cat file1>file2+
	\item \verb+>>file+ - ������ �� ������ ������ ����������� � ����.
	\item \verb+<file+ -  ��������� ������ ��� ������������ ����� �� �����.
	\item \verb+p1|p2+ - �������� ������ ��������� �1 �� ���� ��������� �2 (\emph{�������� ��� ������������� �����}). \verb+������: cat spisok | wc l+
	\item \verb+n>file+ - ������������ ������ � ������� \verb+n+ � ����.
	\item \verb+n>>file+ - ������������ ������ � ������������ \verb+n+ � ����, �� ������ ����������� � ����� �����.
	\item \verb+n>&m+  - ���������� ������ � ������������� \verb+n+ � \verb=m=. 
	\item \verb+<<str+ - ����������� \emph{"���� �����"}. ���������� ����������� ����� ����� �� ��������� ������ \verb+str+ �� ����� � ����� �������� ��� �� ���� ���������.
\end{itemize}
\begin{verbatim}
�������: 
1)ls -al | wc 1>&2 1>>wc.out
2)run 2>/dev/null  ���������� ������ ������.
\end{verbatim}


\section{�������(wildcard's, �������������� �������)}

�������� ��������� ������ ����������� ����. \emph{�������������� ������} ���������� ��������� �� ����� ������, ���� ���-�� � �������� �������� ��� ������. ��� ������� � �������, ����� ������ ����� ��� ���������� ������� ��������� ������ �� ������������� �������.
\begin{itemize}
\item \verb+*+  - �������� ����� ���������� �������� (����� ���� � 0) , � ����� �����.
\item \verb+?+  - �������� ����� ������ � ����� �����.
\item \verb+[�������]+ -  ������ ����� ������ �� ���������.\verb+[a-c1-3]+ ���� ��� �  \verb+[abc123]+
\item \verb+\�+  - ������ ������ � ��������� (\emph{����������}) \label{slash}, ���� �  ��� ����������(\verb+\,',",`,# � �.�.+).
\end{itemize}
\begin{verbatim}
�������: 
$ ls [a-d]* //��� �����, ������������ �� a,b,c,d
$ ls x*y // ���, ������������ �� x � ����������� �� y
$ ls *\ ? // ������������� ������ - ������
\end{verbatim}

\section{�������� ���������� ������}

��������� ������� ����������� - \emph{�������� ����������}. ��� ���� ������: ����� ���� �������� ������� �� ���������� ���������� ����������. ���� ��� ���������� ����������� ������ \verb+';'+ (��. \ref{tz}). �� ��� ������������� \verb+';'+ ������������������ ������ ������ �����������, ��� ����������� �� ����������� ������ ��������� ������ (������ � ���). 

��� ��������� �������� ������, � Bourne Shell ���������� ��������� �����������: 
\begin{itemize}
\item \verb+p1&&p2+  ����������� �1, ���� ������ (��� �������� 0\footnote{���� �������� ��� shell - �������������� �� (��. \ref{shellreturncode})}), �� ����������� �2
\item \verb+p1||p2+ - ����������� �1, ���� �������� (��� �������� �� 0), �� ����������� �2
\item \verb+p1&+ - �1 ����������� � ������� ������, � shell �� ���� ��������� ������ �1 (��. \ref{jobcontrol}, �������� �������). �������� ����� ������� ����������� �� ����.
\item \verb+(p1;p2;)+ - ������� ����������� ��������������� � ����� ��������.  
\item \verb+{p1;p2;}+ - ������� ����������� ��������������� � ������� shell.
\end{itemize}
\begin{verbatim}
�������: 
1)(ps; who) | more
2)mount | wc -l > mounts.number &
\end{verbatim}

\emph{����������}: �������� � ������� ������ �� ����� ������������ ����������� ���� � �����, ������� �� ���� ��������������. ����� ������ �������� ��� ���������.


\section{�������� ���������}

���������
\begin{verbatim}
if   ������� 		
then 
else 	
fi 					 
\end{verbatim}

� shell true (0)  � false (�� 0) ����� �������� �������� �� ��������� � ��.\label{shellreturncode}. \footnote{����� �������� ���� �������� ��������� ������ ������� � man � ������ EXIT STATUS ��� DIAGNOSTIC ��� RETURN CODE}

�������� ��������� ����� ���������� � �������, �������� ;.

�������� ����� ���� ��������� �������. ����� ������������ \verb+test+ � �����������. �������� ����������� ��������:
\begin{itemize}
\item test s ����  - �������� �� ������ ����� �������� �� 0
\item test r ����  - �������� �� ���� ��� ������
\item test f ����  - ���������� �� ���� � �������� �� �� �������
\item test d ����  - ���������� �� ���� � �������� �� �� ���������
\end{itemize}

����� ���������� ��� \verb+test+, ��������� �� �� �������� � \verb+[ ... ]+.

\emph{����������}. ����� \verb+[, ], if+ ����������� ������ ������ �������!

\begin{verbatim}
������: ��������� ������ ������������:
1)if test f $HOME/file.txt
then
            echo �� ����!
     fi
2)if [ -f $HOME/file.txt]
then
            echo �� ����!
     fi
3)  test f $HOME/file.txt && echo �� ����!
\end{verbatim}

\subsection{��������� �����}
 
\begin{itemize}
\item \verb+������1 = ������2+  �������� �� ���������
\item \verb+������1 != ������2+  �������� �� �� �����
\item \verb+-n $����������+  true, ���� ������ ����� ��������� �����
\end{itemize}
\begin{verbatim}
�������:
1)if [ $v1 = abc ]; then; echo ����������!
2)if [ -n $empty ]; then; echo �������������
\end{verbatim}

\subsection{��������� �����}

����������� �������� \verb+$x+(�������� ����������) ��� �����\footnote{��� ����������� ��������� �������� � �����������}:
\begin{enumerate}
\item \verb+$x eq $y+   true, ���� ��������� �����
\item \verb+$x ne $y+  true, ���� ��������� �� �����
\item \verb+$x gt $y+  true, ���� �������� x ������ �������� y 
\item \verb+$x ge $y+  true, ���� �������� x ������ ���� ����� �������� y
\end{enumerate}

\verb+������:  if [ $# eq 2 ]; then; echo 2 ��������� +

\subsection{������� ���������}

\begin{enumerate}
\item \verb+!���������+	- ���������
\item \verb+���������1 a ���������2+	- ���������� �
\item \verb+���������1 o ���������2+	- ���������� ���
\end{enumerate}
\begin{verbatim}
������: 
1)if [ !\( $x eq $y\) ]
2)if [ $a ne 3 a $b lt $c ]
3)if [ $x = $y a \( $n lt 0 o $m gt 30\) ]
\end{verbatim}
\emph{����������}: ������ ������������ (��. \ref{slash}), ��� ��� ��� ����� ����������� ����� ��� ������ (���������������� ���������� � ����� ���������� shell).


\section{�����}

� ����� shell ���� ��������� ����� ������. ����� ����������� �� ���:\verb+for+, \verb+while+ .
\begin{itemize}
\item ���� \verb+for+\footnote{BASH ������������ ����� ����� � ����� ��: for ((i=1; $i<10; $i++)) - c 2 ������������ � ������������ ��������. } ����������� ������� ���, ������� ���� � ������. var ��������������� ��������� �������� �� ������. ������ ����� ������������� �������, ��� ����� ������� (`�������`) ��� � ������� ��������.
\begin{verbatim}
for ���������� in ������
do
....
done
\end{verbatim}
\item ���� \verb+while+ �����������, ���� ������� �� ������ ������.
\begin{verbatim}
while �������
do
...
done
\end{verbatim}
\end{itemize}
\begin{verbatim}
�������:
while sleep 60
do
 who | grep mary
done

for user in `who`
do
  echo � ���� $user �����!
done

for i in * ; do echo $i; done  #���������� ls.
\end{verbatim}

�������� �����  ����� ���������� �� ��������� ������� ��� �������� ';'. ��� ����� ������ ������������.

\emph{����������}. ���� ������� � ��������� ������ ����, �������� ��������� ��� ������ �� ������� ������ (�������) �� ��� ������� \verb+ENTER+ �������� ��������� ������� ��������� ������� �� ��������� ������.

���������� \emph{����������}(��. \ref{internalcmd}) ������� \verb+break+ ���  ������ �� �����.

\section{�������}

��� �������������� ������� ������������ ����� ���������� �������:
\begin{verbatim}
���_������� ()
{
   �������
}
\end{verbatim}
��������� � �������� ���������� - ��� � �������.
\begin{verbatim}
������: ���������� � ����������� ��� ��������
mcd ()
{
		cd $*
		PS1 = `pwd`
}
\end{verbatim}

������� ����������� �� \emph{����������} (� ��������) \label{internalcmd} � \emph{�������}. ������ ���������� ������� �� ������� �������� ������ ��������.
 
���������������� ���������� �������: \verb+cd+, \verb+pwd+, \verb+echo+, \verb+exit+, \verb+set+, \verb+unset+.


\section{���������� �������������� ��������}

\emph{������ ���������}. � shell ����������� ������ ������������� ����������\footnote{��� ��������� ����� � ������� ���������� ����� ������������ ��������� bc}!

\verb+�xpr ������+  ����������� ������ � �����. \verb+��������: expr 23+.

����������� ��������: \verb=+, -, *, /, %= (������� �� ������). �� ��������� �������.
\begin{verbatim}
�������: 
1)a = `expr $a + 3`
2)b = `expr 2 \* 3` - ������ \ �������� ����������� �������� *.
\end{verbatim}
\emph{����������}. ����� � �������� ����������� ���������.

% vi
\chapter{��������� �������� VI}

��������� ��� Unix ������� �� 2 ������ - ��������� ���������� �����(vi, emacs, joe, ed) � ����-��������������� (mcedit, kwriter, kword).

��������� ���������� ����� ������ �������� � \emph{���������� (���������)} ������. ��� �������� � ��� ����������� ������� ������ ����������� ������, ������������ �� ������������� ����������� ������. ���� � ���� � ���, ��� �������, �� ������������.

�������� \verb+vi+ ������������ ��� ����������� � ����� Unix-�������� �������\footnote{� ���� �������� �� ������ � �������� Single Unix Specification}. ���������� ��������� ���������� ���������� �� vi: vim, elvis. 

����������� ����� vi (vim � �������) �������� ����� ������� �����������������, ������� �� ���������� �����������. �������� vi ���������� ���������� ��� �����-�������������, ������� ������ �������� �� ����� ����� ���������� � ����������� ��������. ��� �������� � ��� ����� ����������� �� ������� ��������, ���������-��������, ����� ����������.

\emph{����������}. ����� �� ����� ������������� �������� vim. ������ ��� ��������� ������� ����� ����� ��������� � ����� vi-����������� ���������.

\section{������ ������}

� vi ���������� ��� ������������� ��������� ������ ������:
\begin{itemize}
\item[-] ��������� ����� (command mode)
\item[-] ����� ����� (edit mode)
\item[-] ����� ����������� �������������� (ex mode)
\end{itemize}

\emph{��������� �����} ���������� �� ��������� ��� ������� vi. � ���� ������ ������� ������ \emph{�� ��������} � ����� ��������, � ���������������� ��� ���������� ������� ����������� �� ������ � ��������������. ������� ������� ���������� ������ ���� ������ (��� � DOS/Windows) �� � ���� �� ��������.

\emph{����������}. ���� �� �� ������, � ����� ������ ����������, �� ������� ������� \verb+ESC+ ��� �������� � ��������� �����.

�������� ������ � ��������� ������ ����������. ��� ����� ����� ������� � \emph{����� �����}. ��� ����� ������ ������� (���������� ������!) \verb+a+ (�� append - ����� ������� ������� �������)  � \verb+i+ (�� insert - ����� ������� �������� �������). � ������ ����� ������� ������ �������� � ����� ������� ��������, �������� ��������� ����� ����� ��� ������������� ������������.

������� � ��������� ����� �������������� �������� ������� \verb+escape+.

��� �������� � ����������� (�������) ������������ \emph{ex-�����}. �� ���������� �������� \verb+:+ ���������� ������. ����� ����� ������ ������� ex-������. ��������:
\begin{itemize}
\item ������� ������������ ���� (\verb+:e ���_�����+) 
\item �������� ���� � ������� ������� (\verb+:r ���_�����+)
\item �������� ���� (\verb+:w+), � ��� ����� ��� ������ ������ (\verb+:w ���_�����+)
\item ����� �� ������������ ����� (\verb+:q+)\label{viexit}
\item ����� � ��������������� ����������� ����� (\verb+:x+)
\end{itemize}

����� �� ���������� � ex-������, �� � ������ ����� ���� ������ ���������� \verb+:+.

�������� ���������� ������ ex-������. �������� \verb+:wq+. 

������� ex-������ ������������ �� ���������� �������� ������� \verb+Enter+ ����� ���� ���������� ������� � ��������� �����.

\emph{����������}. ������� ��������� ����� ���� (�������� \verb+:e+) ��� ��������� ������ ��������� (�������� \verb+:q+) ��� ������������� ������ ����� ������� ������.

\section{��������� ������}

�������� ������� ����� ��������� ex-������� \verb+:help+.

���������� ������� �������� �������� \verb+|������|+. ������� �� ��� ���������� ����� \verb+:help ������+.

����� �������� �������� ������� �� vim - \verb+vimtutor+. � ��� ������� ����� ������� �������� ������ ������������� vim.

\section{������ � ��������� ���������}

\verb+Vi (vim)+ ����� ���� ������� �� ��������� ������, � ������ ����� ��� ��� ��������. ���� ������� ��� �����, �� �������� ��������� ���\footnote{���� ���� �� ����������, �� ��������� �����}.

\verb+������: $ vi ~/texts/newtext.txt+

������� vim ��� ����� ����� ������� �������� vim � ������� ��������.

��� ������ �� ��������� ������� \verb+:q+ ��� \verb+:wq+ (��. "������ ������" \ref{viexit}). 


\section{����������� �� ������}

� ��������� ������ ���������� ��������� �������\footnote{������ �������� � ������� �� ����������, �� �� ����� ���������� �� ���}:
\begin{itemize}
\item \verb+h+ - ������ ����� �� 1 ������
\item \verb+l+ - ������ ������ �� 1 ������
\item \verb+j+ - ������ ���� �� 1 ������
\item \verb+l+ - ������ ����� �� 1 ������
\end{itemize}

�����, ���� ����������� �������, ����������� � ������� ������.
\begin{itemize}
\item \verb+w � W+ - ����������� ������ �� "��������� �����"\footnote{��������� �����, ���������� ��������, ������� ����������, +, -} � �.�. "������� �����"\footnote{����������� ���������� ��������} 
\item \verb+b � B+ - ����������� ����� �� "��������� �����" � "������� �����" 
\item \verb+0 � $+ - �� ������ � �� ����� ������
\item \verb+( � )+ - �� ������ ����������� � ��� ����� 
\end{itemize}


������, ��� ������ ������ vi ���������� ������� ������ ��������� - � ������ � ������� ��������� ����� ������� (\verb+e+ � \verb+E+, \verb+w+ � \verb+W+); �������� ������ ������� �� ���� ��� �� ��������� �������� ������.

������� ��������� vi ����� �������������� � ���������� �����������.

�������� ������� \verb+5h+ ���������� ������ �� 5 �������� ����� (������ ������ � ������� �������), � ������� \verb+3B+ �� 3 "�������" ����� �����. 

��� ����������� �� ���������� ������, ����� ������������ ��������� ������� ex-������: \verb+:N+, ��� N - ����� ������.


\section{���� � �������������� ������}

��� �������� ������ ���������� ������� � ����� �����.

��� ����� ������ ��������� �������:
\begin{itemize}
\item \verb+i � I+ - ���� � ������� ������� ��� � ������ ������
\item \verb+a � A+ - ���� ����� ������� ��� � ����� ������
\end{itemize}

����� ����� �������� � � ������ �����, ��������� \verb+DEL+ � \verb+BACKSPACE+, �� ����� ������� ������������ ������� ��������������.

������� �������������� ������������� ��� ��������� ������������� ������ ��� �������� � ����� �����:
\begin{itemize}
\item \verb+x+ - �������� ���������� �������
\item \verb+dd+ - �������� ������
\item \verb+dw+ - �������� �����
\item \verb+d)+ - �������� �����������
\end{itemize}

��� � ������� �����������, ������� �������������� ����� ������������ � ���������� �����������. ��� ������� \verb+5dd+ ������ ������� ������ � ��� 4 ������ ���� ��, � \verb+3dw+ ������ ��� ����� ������ �������.

\section{����������� � �������}

� vim ��� ���� ����� ���������� ��������� ����� - ���������, Visual Selection (��. \verb+man vim+). ������ � ����������� ������� �� ����� �������� ������������ ���������:
\begin{itemize}
\item \verb+p+ - �������� �� ������.
\item \verb+yy+ - ����������� ������ � �����
\item \verb+yw+ - ����������� ������� ����� � �����
\item \verb+y)+ - ����������� �����������
\item \verb+y}+ - ����������� �����
\end{itemize}

\emph{����������}.� ����� ����� �������� ��� ��������� � ������� ������ x, dd, dw � �� ��������. ����� ������� ��� ������� ����� ������� ��� ����������� � ���������.

��� ����������������� ������� ����� �������������� � ��������� ����������. ��������: \verb+3p+ - 3 ���� �������� ���������� ������.

\section{����� ��������}

�������� �������� ��������� ������ �������������� ����� ���� �������� �������� \verb+u+ (���������� �� undo). ��������� ������� - ������ ����������� ��������, � ��� �����. ��� �������� (redo) �������� ���������� �������� ������������ \verb-control+r-.

\section{����� � ������}

��� ������ �� ������ ������ ������� \verb+/+ (������ ����). ��� ����� ����� ������� � ��������� ������ � ������ ������ ���������� ������ \verb+/+, ����� �������� �� ������ ������ ������� ��� ������. ��� ����� ���� ��������� ������ ��� \emph{���������� ���������} (��. \ref{regexp} ). ����� ������� \verb+ENTER+ � ������ ����� ���������� \footnote{��� ����������� ������ ��� vim} ��� ��������� ��������� ������ ������ � ������ �������� � ������� ���������� ��������� ��������� ���� �� ������.

��� ������ ��������� ��������� ������ ������, ���������� ������� \verb+n+ (���� �� ������) � \verb+N+ (����� �� ������).

��� ������ � ������ ��������� ����������, � ��� ����� � � �������������� ���������� ���������, ������������� ������� ex-������ \verb+:s+(substitute). ������ �������: \newline
\verb+:#s/pattern/string/�����+\newline
��� \verb+#+ - �������� ����� (����� \verb+,+ ��� \verb+;+ - ��. \verb+:help cmdline-ranges+).

\emph{����������}. ����� � ����������� vi �� ��������������.

����� ����������� �����: \verb+c+ - ������������� ������ ������, \verb+g+ - ������ ���� ��������� � ������.

\emph{����������}. ����� � ������ � vi �������� ������ ��� ������������������ ��������, ������������ 1 ������. ���������� ������������������ �������� ���� ������ ������������ 1 ������.
 

\section{����� ������� ������}

�������� vim ����� ������� ������, ��� ��� �� ��������� ����������� �������� � �������, �� ������ �� ���������.

�� vi ����� ��������� ������� ��������� � ������� ������� ex-������ \verb+:! cmdlline+:
\begin{verbatim}
������
:! ls -l

:! man bash
\end{verbatim}

��� �������� ������� �������, ��� ��� ��� ���������� ��������� ������� � �������� �� ���������. ��� ������� ��������� ������ �� ����� ����� �������� �� ������ �� ������� ������������� ������� \verb+ENTER+.


% regexp и sed
\chapter{���������� ���������. sed}

\label{regexp}
\emph{���������� ��������� (regular expression ��� regexp)} -  ����������� ������ ��������, ������� �������� ��� ������ ����������� ����������. ����� ������ ��� ������ �������� ������� ����. 

���������� ������� �������� �����, �� ���������� ��������� ����� �������� � ���������� �������, ���������� ������ �������. ��� UNIX-���������, �������������� ����� � ������, ���������� ���������� ���������. ���� ����� ��� ����� ������� ���������� ����������, �������, ��� ��� ������������� ����������� ���������. 

\emph{���������� ���������} - ������, ������ � ����������� �������� ��������� �������. ������������� ������� ���������� ��������� ���� �� ���� ���������� ����������� ���� ����������������, ��������������� ��� �������� � ������� ������. ��� �������������� ��������� �� ������� ���������� ������� ��� ����� ���������������� ���������� ��������� �������� ���������, �������, �������� ��������� ������ ������ ���� � ��������� ������������ ����� �������� ��� ����.

���������� ��������� ��������� �������� \emph{������������ (�������� ��� wildcards)}. 

��������� ��������� ���������� ���������� ��������� � ������ ���� (grep, egrep). �� ���� ����� ���������� ��������� ������������ ������ ����������� �������� �����������, �. �. "�������". 

\section{��������� ���������� ���������}

���������� ��������� ������� �� ���� ����� ��������. ����������� ������� ���������� \emph{�������������}. ��� ��������� ������� (�� ���� ������� �����), ���������� \emph{����������}.

���������� ��������� ����� ������������� ��� ��������������� ����, � ������� �������� ��������� ������� ����, � ����������� - ������� �������������� ���������. ����� �� ������������ �������� ������������ ���������������� ���������� � ������� �����������, ���������� ��������� �����.

��� �������: ���������� ������� \verb+grep+\footnote{������� ��������� grep ������������� ��� ������ ������ �� �������� ���������� ���������}. ��� ������� ��������� \verb+grep+ ���������� ���������� ��������� � ������ ��������������� ������. ��� ������������ regexp � ������ ������� ����� � ������� ������ �� ������, � ������� ���� ������� ����������.
\begin{verbatim}
$ grep 'cat' file1.text 
\end{verbatim}
���� � ����� ��������� ($\ulcorner cat \lrcorner$ \footnote{��� grep ������������� ��������� regexp � �������, ��� ��� ��������� ����������� ����� ��� �������� ����������� �������� � ��������� ����� �������� �����������}) �� ������������ �����������, ��� ���������� ������������ � ��������� "�������� ������ ������". ����� ������� � �������� ��� ������ �����, ���������� ��� ������� ������ ����� \verb+c, a+ � \verb+t+. ����� ��� ����� �������� ������, � ������� ����������� ����� (� �������) \verb+vacation+. ���� ���� � ������ ��� ����� \verb+cat+, ������������������ ���� \verb+c+ \verb+a+ \verb+t+ � ����� \verb+vacation+ ��� ����� ��������� ������� ���������. ���������� ������ ������� ��������� ��������.

\section{������� ��� ���������� ���������}

���������� ����� ��� ������������� ������� ��� ���������� ���������:
\begin{enumerate}
\item ������������ �������� ���� ����������, ������� ���������� ������.
\item �������������� (��. \ref{requant}) ������ �������� �����������. ���� ��������� ������� ����� ��������� ���������� ����� ���, �������� ������ �������� ����� ������������ ����� ����������.
\end{enumerate}

�������� ������ ����������� ����������� ���������:
\begin{itemize}
\item ���������� ��������� ������ ��������� ��� ��� ����� � ����� �����
\item ���������� ��������� ������ ���� �������� � �����������
\item ��� ������ ���� ����������� ( ������ ��������� � ���������� ��� ������������ � ����������� �� ����������� ������)
\end{itemize}

\section{�������� ���������� ���������}

� ������ ���������� ���������� ��������� ��������� ������ �������, ������� ������ ������������ � ������ �����������, �������������� �����������, ����� �����������.

� �������, �������� ���������� ��������� � sed, perl � grep ����� ������������ ����� ������� ����� �����. ����� ����, ��������� �������� grep ���� ����� ������������ ������ ��������. 

����� ��������� ����� �������������� �������� \verb+sed+, � ��������� ������� �� ������ ���������� ���������� ���������.
 
\section{�����������}

���������� ��������� ����� ������������, ����������� ������ �������. �������� ��������� �� ��� �������� � ������ ������ ��������� (��� ������� �� ���������).

\subsection{������ � ����� ������}

$\ulcorner\verb+^+\lrcorner$ (������) �  $\ulcorner \$ \lrcorner$(������) ������������ ����� ������ � ����� ����������� ������.

�������:
\begin{enumerate}
\item \(\ulcorner \verb+^cat+ \lrcorner\) ������� ��� ������, � ������ ������� ���������� \verb+cat+.
\item \(\ulcorner \verb+^cat$+ \lrcorner\) ������� ��� ������, ������� ������� ������ �� \verb+cat+
\item \(\ulcorner \verb+^$+ \lrcorner\) ������ ������
\end{enumerate}

����������� $\ulcorner \verb+^+ \lrcorner$ � $\ulcorner \$ \lrcorner$ � ���, ��� ��� ��������� � ������������ \emph{��������} ������, � �� � ��������� ������.

\subsection{���������� ������}

\subsubsection{���������� � ����� �������� �� ���������� ���������}

��� ������ ����������� $\ulcorner[\ldots]\lrcorner$, ���������� \emph{���������� �������} (character class), ����� ����������� �������, ������� ����� ���������� � ������ ������� ������. 

�������:\label{examplegray}
\begin{enumerate}
\item $\ulcorner gr[ea]y\lrcorner$. ��� ���������� "����� ������ \verb+g+, �� ������� ���� \verb+r+, �� ������� ������� \verb+e+ ��� \verb+a+ � ��� ��� ����������� �������� \verb+y+".
\item $\ulcorner[Ss]eparate\lrcorner$. ��������� ����� �������� � ������ �����
\end{enumerate}

���������� �������� � ������ ����� ���� �����. ��������, ����� $\ulcorner[123456]\lrcorner$ ��������� � ����� �� ������������� ����.

� ��������� (������) ����������� ������ \emph{���������� ����������� ������} \verb+-+ ���������� �������� ��������; ��� ��������� $\ulcorner[1-6]\lrcorner$ ������������ ����������� �������. ������ $\ulcorner[0-9]\lrcorner$ � $\ulcorner[a-z]\lrcorner$ ������ ������������ ��� ������ ���� � �������� ������� �������� ��������������.

���������� ����� ����� ��������� ��������� ����������, ������� ����� $\ulcorner[0123456789abcdefABCDEF]\lrcorner$ ������������ � ���� $\ulcorner[0-9a-fA-F]\lrcorner$. ��������� ����� ����� ���������� ������ � ����������: $\ulcorner[0-9\verb+_!.?+A-Z]\lrcorner$ (�������� �� ����� �������, ������� � ������� �������� � ������� �������������, �����, ���������������� � ��������������� ������).

\emph{���������� 1}. ����� ��������� ������� ����������� ������ ������ ����������� ������ - � ��������� ������� �� ��������� � ������� ������� � ������.

\emph{���������� 2}. �������, ������������ ������ �������������� ������������ (� �� �������) ������ ������ � �� ��� ���������, ��������� ��������.

\emph{���������� 3}. ����� �� ���������������� ��� ����������, ���� �� ���������� �� ������ ������� ������, �������� $\ulcorner [\verb+-./+] \lrcorner$.

\subsubsection{�������������� ���������� ������}

���� ������ $\ulcorner[\ldots]\lrcorner$ ������������ ������ $\ulcorner[\verb+^+\ldots]\lrcorner$, ����� ��������� � ������ ��������� \emph{�� ���������} � ����������� ������. ������: $\ulcorner q[\verb+^+u]\lrcorner$.

�������  \verb+^+ ����������� ������ - ������ ����, ����� ����������� �������, ������������� ������, ������������� �������, �� �������� � ����.

\emph{����������}. ��������������� ����� �������� "���������� � ��������� �� ��������� � ������", � �� "������������ � ���������, ��������� � ������". ������� ��������������� ����� ������ ������������� ��� ����������� ����� ������ ��� �������� ������, ����������� ��� �������, \emph{�����} �������������.

\subsection{���� ������������ ������}

���������� $\ulcorner . \lrcorner$ (�����) ������������ ����� ����������� ����� ������ ��� ����������� ������, ����������� \emph{���} �������. ����������� � ��� �������, ����� � ��������� �������� ����������� ��������� ����� ���������� ������������ �������.

������: ����� ���� ����� ����, ������� ����� ���� �������� � ������� \verb+07/04/76+, \verb+07-06-76+ ��� \verb+07.06.76+. ����� ������� ������� - $\ulcorner 07.04.76 \lrcorner$. �� ����� ��������� ����� ��������� � �� ������� \verb+19 207304 7639+. ��������� $\ulcorner 07[-./]04[-./]76 \lrcorner$ ������������ ����� ������ ����������, �� ��� ������� ������ � ����������.

��� ���������� ���������� ��������� ����� ����������� ���� �� ��������� � �����c��� �� ���� ������ ������. ���� �� �������, ��� � ������ $\ulcorner 07.04.76 \lrcorner$ ��������� �� ������� ������������� ����������, �� ���� ��������� ������ ����� ���������������. 

���������. \emph{������ �������� ������ - ������ ������, �������������� ����������� ������������� ���������� ���������}

\subsection{�����}
\subsubsection{���� �� ���������� ���������}

\label{reor}
����� ������� ������ $\ulcorner \verb+\|+ \lrcorner$ \footnote{��� Perl � egrep - $\mid$ } ���������� "���". �� ��������� ���������� ��������� ���������� ��������� � ����, ����������� � ����� �� ���������-�����������. 

��������, $\ulcorner Erik \lrcorner$ � $\ulcorner Bobby \lrcorner$ - ��� ������ ���������, �  $\ulcorner Erik\verb+\|+Bobby \lrcorner$ - ���� ���������, ����������� � ����� �� ���� �����. ������������, ������������ ���� ��������, ���������� \emph{��������������} (alternatives).

����������� ������ ������ �������� ��������������� (�� ���� ���������� ����� ������ �����������).

�������� � ������� 1 �� \ref{examplegray} $\ulcorner gr[ae]y \lrcorner$. ��� ��������� ����� �������� ����� � ���� $\ulcorner gray \backslash \mid grey \lrcorner$ ��� ���� $\ulcorner \verb+gr\(a\|e\)y+ \lrcorner$\footnote{��� �������� sed. ��� egrep ��� perl ��� ����� gr(a$\mid$e)y.}. ����� ������� ������ \verb+\(+ � \verb+\)+ �������� ����������� ������ �� ���������� ���������. ��� ������ $\ulcorner gra\mid ey \lrcorner$ ����� �������� "$\ulcorner gra \lrcorner$ ��� $\ulcorner ey \lrcorner$".

��������� ������ ������ ����� ���� ��� ������ �������, �� "�������" ��� �������������� ��� ������ �����.
 
\emph{����������}. �� ������� ����������� ������ � ���������� �������. ����� $\ulcorner abc \lrcorner$ � ����������� ������ $\ulcorner \verb+\(a\|b\|c\)+ \lrcorner$ ���������� ���������� ���� � �� ��, �� ��� �� ��� ������ ������. ���������� ����� ��������� ����� � ����� ��������, ����� �� ������� ��� �������� �� ��� ������ ���������� ��������. � ������ �������, ����������� ������ ����� ��������� ������������ ������������ ������, ���������� �� ��������� ���� � ������ ������ ������: $\ulcorner \verb+\(1.000.000\|million\|thousand*thousand\)+ \lrcorner$. � ������� �� ���������� �������, ����������� ������ �� ����� ���������������.

\subsection{������� ����}

���� �� ���������������� ������� ����������� � ���, ��� ������� ����� ����������� ������ ������ ����. ��� ������ �������� ������ � ����� ����� ������������ \emph{����������������������} $\ulcorner \backslash< \lrcorner$ � $\ulcorner \backslash> \lrcorner$

��� � ����� \verb+^+ � \verb+$+, ��� ���������������������� �� ������������ ���������� ��������.

�������:
\begin{enumerate}
\item $\ulcorner \verb=\<cat\>= \lrcorner$ - ����� ��������� ����� \verb+cat+
\item $\ulcorner \verb=\<free= \lrcorner$ - ����� �����, ������������ � \verb+free+, � ������� \verb+freeware+
\item $\ulcorner \verb=ed\>= \lrcorner$- ����� �����, ��������������� �� \verb+ed+
\end{enumerate}

\emph{����������} ���� �� ���� ������� $\ulcorner < \lrcorner$ � $\ulcorner > \lrcorner$ ������������� �� ��������. ��� ����������� ������ ����� ������ � ��������� � �������� ������ \verb+\+.  

\section{��������������} 

\label{requant}
\emph{��������������} ���������� ���������� ����������� �������������� ��������. ���� �� ����, �������������� �� �������� ��������� �������� � ������, �� ������������ ����� ������� ��� ��������� � �������, ���������, ������� ��� ����� ����������� ���� ������ ��� ���������.

�������������� ��������������� ��������� ������������� ���������� � �������� ����� ���������� ��� ����� ������� �����.

\subsection{�������������� ��������}

���������� $\ulcorner \verb+\?+ \lrcorner$ \footnote{��� Perl, egrep - ?.} (�������������� ����) �������� "�������������� ������". �� ��������� ����� �������, ������� ����� ���������� � ������ ������� ������, �� ������� �������� �� ��������� ��� ��������� ����������. �������������� ���� ���������� \emph{������} � �������, �������������� ��������������� ����� ���.

������: $\ulcorner \verb+colou\?r+ \lrcorner$  

������2; ����� ��� ���� ����� ����, ���������� ��������� ���� ������. �� ����������, ��� ����� ��������� ���: \verb+4+ ��� \verb+4th+ ��� \verb+fourth+ - $\ulcorner fourth\mid 4\mid 4th \lrcorner$. ������ �������� ��������� ����� ��������� �� $\ulcorner \verb+4\(th\)\?+ \lrcorner$. ������� $\ulcorner \verb+fourth\|4\(th\)\?+ \lrcorner$.

����� �������, ������������� $\ulcorner \verb+\?+ \lrcorner$ ����� �������������� � � ���������� � �������.

\subsection{����������}

\begin{itemize}
\item ���������� $\ulcorner \verb=\+= \lrcorner$ \footnote{��� perl � egrep $+$}���������� "���� ��� ��������� ����������� ��������������� �������������� ���������".
\item ���������� $\ulcorner * \lrcorner$ ���������� "����� ���������� ����������� �������� (� ��� ����� � �������)".
\end{itemize}

����� ������, $\ulcorner * \lrcorner$ �������� "����� ������� ����������� ������� ��������, �� ��� ������������� �������� � ��� ���". ����������� $\ulcorner \verb=\+= \lrcorner$ ����� ������� �����, �� ��� ���������� ���� �� ������ ���������� ������������� ����������� ��������.

�������:
\begin{enumerate}
\item $\ulcorner \verb=^[0-9]\+= \lrcorner$ - ������, ������������ � ����� ��� ����� ���� 
\item $\ulcorner \verb=^[0-9]*=\$ \lrcorner$ -  ������, ���������� � ���� ������ �����(����� ���� � ������) 
\item $\ulcorner .* \lrcorner$ - ����� ���������� ����� ��������
\item $\ulcorner * \lrcorner$ � $\ulcorner + \lrcorner$ ����� ��������� �� ��������: $\ulcorner \verb=\(th\)i\+= \lrcorner$ - ���� � ����� ��������� ���� \verb+th+ ������
\item $\ulcorner _\sqcup \verb=\+= \lrcorner$ - ���� ��� ����� ��������
\end{enumerate} 

\subsection{��������}

����������� ���� $\ulcorner \ldots \verb+\{min,max\}+ \lrcorner$\footnote{� egrep � perl ��� �������� ��� \dots\{min,max\}} ���������� \emph{������������} ���������������.

��������, ��������� $\ulcorner \ldots \verb+\{3,12\}+ \lrcorner$ ��������� �� 12 ���, ���� ��� ��������, �� ����� ������������ � ����� 3 ������������. ������ \verb+\{0,1\}+ ������������ ����������� \verb+\?+, \verb+\{1,\}+ - \verb=+=.

\section{������� ������ � �������� �c����}

�� ��� ������� � ����� ������������ ������� ������:
\begin{itemize}
\item ����������� ������� �������� \verb+|+ (��. \ref{reor})
\item ����������� �������� ��� ���������� ��������������� (��. \ref{requant})
\end{itemize}

���������� ��� ���� ���������� ������� ������. ������� ������ ����� "����������" �����, ������� ������ � ���������� � ��� �������������.

\emph{�������� ������} ��������� ������ ����� �����, ������� ��������� � ������ ������� � �������������� ����� ����������� ���������, ������ �� ������ ��������� ��������� ���� ����� \emph{����������}.

������� ������ "����������" �����, � ����������� ���������� $\ulcorner \backslash 1 \lrcorner$ ������������ ���� ����� (����� �� �� �� ���) � ���������� ����� ����������� ���������.

� ��������� ����� �������� ��������� ��� ������� ������ � ��������� �� ��������� ����� � ������� $\ulcorner \backslash 1 \lrcorner$ , $\ulcorner \backslash 2 \lrcorner$, $\ulcorner \backslash 3 \lrcorner$ � �.�. ���� ������ ���������� � ������������ � ���������� ������� ����������� ������ ������ ������.

������. ����� ��� ���� ����� ������������� �����. ���� �������� ���������� �����, �� ����� �������� ��� � ������, ��������, $\ulcorner the\ the \lrcorner$. �� ��� ���� ���� ��������� ����� ������� ����������. ��� ���� ����� ���� "����������" �����, � ����� ������� ������ �� �� �����. ������� $\ulcorner the \lrcorner$ ���������� ���������� ��� ����������� ����� - $\ulcorner \verb=[A-Za-z]\+= \lrcorner$ � �������� ��� � ������� ������� -  $\ulcorner \verb=\([A-Za-z]\+\)= \lrcorner$. ������� ��������� ��� �������� - $\ulcorner _\sqcup \verb=\+= \lrcorner$. ������ ��������� ���������� ��������� � ������� �������� ������  $\ulcorner \verb=\([A-Za-z]\+\)= _\sqcup \verb=\+\1= \lrcorner$. � ��������� - ��������� ������� ����\footnote{����� ����� ������� �� ������ ������������� �����, �� � ���������, ����� �����, ����������� �����, �������� ������ ��� ����������.} -  $\ulcorner \verb=\<\([A-Za-z]\+\)= _\sqcup \verb=\+\1\>= \lrcorner$.

\section{�������������}

����� �������� � ��������� ������, ������� ��������� � ������������, ���������� ��������� \emph{�������������}. ������������� ����������� � ������� ������� \verb+\+.

��������: ���������� "�����" $\ulcorner . \lrcorner$ ��������� � ����� ��������. ����� �������� ������� �����, ���� �������� $\ulcorner \verb+\.+ \lrcorner$, ������� ���������� "��������������" (escaped) ������.

������������� ����� ����������� �� ����� ������������ �������������, ����� ������������ ���������� �������.



\section{sed}

\emph{sed (sequential ��� stream editor)  ��������������� (��������) �������� ������}. �� ������ ��� ���������� ������� � �������������� ������. ���������� sed - ��� ����������� ��������� ������. ����� ���� ������� ����, ����� ��������� � ����������� �����������.

����� ������ ��� ������������� � �������� �������� ��� �������� ��������� �������.

����� ����� ����������� GNU sed.

sed ����� ������������ ����� ��������� ���������:
\begin{verbatim}
sed [-n] [-e] '������� ��������������' �������_����
\end{verbatim}
\begin{center}�\end{center}
\begin{verbatim}
sed [-n] -f �������� �������_�����
\end{verbatim}

���� ������������ ������ ������.

���������:
\begin{itemize}
\item -f cmdfile  ��������� �������� �� �����
\item -n  ������������ ������, ����� ���� ����������� �� �������� 
\end{itemize}

���� ������ ���������, �� ��� ����������� \verb+;+.

������� �����: ������������� ������� �����. ���� �� ��������� ��� �����, �� sed ����� �������� �� ����������� ������. ��������� ��������� � ����������� ����� � ������ ���������������� � ���� ��� ��������. ���� ������� ������ ���������, �� ��� ������������ � ���� �����, � ������� � ���� ����� ������. 

������ � ������ �������������. ���� sed ����������� � ���������� ������, �� ������ ����� ����� ������������. ���� ������ ���� �������� 200 �����, �� ������� ������ ������ ���������� ����� ����� 201. 

\emph{����������}. ������� ����� \emph{�� ����������}.

\begin{center}
����� ������:
\end{center}
������� ����� (stdin) ����������� � ������� �������� (pettern buffer), ����� ����� � ������ ��������������� ����������� �������, � ����� ��������� ��������� ��� ���������� ��� ���������� ���������.

\emph{��������:} ������ �� ������������� ������������� ��������� � �������� �����. ��� �������� � ��������������� �����������.

\subsection{����� ��� �������}

\verb+[�����1[, �����2]] ������� [���������]+

\emph{�������}: ������������ ����� ����� �������. ������������ ������������ ��������. �������� \verb+'p'+.

\emph{�����}: ����� ���� ����� ������, ���������� ���������, \$ (��������� ������).

���� � ������ �������� ���������� ���������, �� ��� ������ ��� ������, ��������������� ����������� ���������. ���������� ��������� ������� � \verb+/+ (������ ����), �� ���� \verb+/regexp/+.

���� �� ������ ������, �� �������������� ��� ������ ������.

\emph{�������� ��������} - ��� ���� �������, ����������� \verb+","+ � ���������� ��� ������, ������� �� ������, ��������������� ������� ������, �� ������, ��������������� ������� ������ ������������. ���� ������ �����  ������ �������, �� �������������� ������ ������ ������, ��������������� ������� ������.

��� ���������� ������� \verb+!+ ����� ������ ����� �������� �� ���������������: �������������� ��� ������, �� ������� � ���������.

�������: \verb+1,4; 1,$; 2,6!+.

\subsection{������� sed}

������� ������ ����� ������������ �������.

\subsubsection{������} 

\verb+s/regexp/replacement/flags+. 

\verb+s+ - ����� ������� (������, ����������� - substitute), \verb+regexp+ - ������ ������, �� ���� ��, ��� ��������� �� \verb+replacement+. 

�����:
\begin{itemize}
\item \verb+g+  �������� ��� ���������
\item \verb+w+ file  �������� ��������� � ����
\item \verb+p+  ����� ������ ������� ������ �� ����� (������ ������������ � ������ sed \verb+-n+).      
\end{itemize}
\begin{verbatim}
�������: 
1. $ sed 's/sun/moon/g' myfile
2. $ sed '1,4 !s/sun/moon/g' myfile
3. $ echo ������� ������� | sed -n 's/\(��\)\(���\)/���<&>\2\1/p'
# ���������: �����<�����>����� �������
4. $ sed '/^Example/,/ED$/s/first/second/g' 
# ���� ��������� ����������� �����, �� ������������� ��� ���������.
5. $ sed 's/Sunday/Monday/gw' changes
# ��� Sunday ����������
\end{verbatim}

\subsubsection{�������� �����}
\verb+d+
\begin{verbatim}
�������: 
1. $ sed '4,5 d' file
2. $ sed '/sun/ !d' file.txt 
# �������� ���� �����, ����� ���������� sun.
3. $ sed '/sun/,/moon/ d' myfile
# ��������� �������� �� ������ ������, ���������� sun
�� ������ ������, ���������� moon
\end{verbatim}

\subsubsection{����� �� �����}
\verb=p=  

������ ������������ � \verb+sed -n+ (����� ������ ����� ���������� ��� ����).
\begin{verbatim}
�������: 
$ sed -n '/stroka/i !p'.
$ sed -n ' 1,4 p'
# ������� ������ � 1 �� 4 ������������
\end{verbatim}

\subsubsection{���������� ��������} 

\verb+y/source_chars/dest_chars/+  

������ �������� �� �������� "���� � ������" (������ ������ ���� ����� �����).
������:\verb+ $ sed 'y/abc/ABC/' file+

\subsubsection{������ � ����} \verb+w file+ - ����� ����� � ����.

\subsubsection{������� �����} \verb+r file+ - ������� � �������� ����� �����. ���� ��� ���, �� ����������� ���� ������� ����� (��� ������).

\subsubsection{������� �����} 
\begin{itemize}
\item \verb+����� a+  �������� �� �������������� �������
\item \verb+����� i+  ������� �� ��������� ������
\end{itemize}

\emph{����������}: a � i ��������� ������������ ������ ���� �����.
\begin{verbatim}
������:
$ cat script
  3 a\
	����� ���������\
	������
$ who | sed -f script
 root
 stud1
 stud10
        ����� ���������
        ������
 stud11
\end{verbatim}                    

������ ������������� \verb+"\"+ ���������, ����� ������ ��� ������� ����� ������ ����� ����������.


% файлы и файловая система
\chapter{�������� ������� �� UNIX}

� ����� ������ ������������ � �� UNIX ���������� ��� ���� ��������: ����� � ��������.

 ��� ������ �������� � ���� ������, ������ � ������������ ����������� �������������� ����� ������/������  � ����������� �����. 

��� ������� ��������� ���� ��������� ��������������� ����������� ����, ������� ����� �������� � �������� ��� ����������.

�� ����� ���������� �������  ����� ��������� ��� ������ ������ � ����. � ������ �������, ��� ���������������� �� ������������ ����������� ��������������� ���������. 

����� �������, ������� �������� ������� � ��������� ����� �������������.

\section{������� �������� � �������� �������}

� UNIX ����� ������������ � ���� \emph{����������� ���������} (������), ���������� \emph{�������� ��������} (FS ��� file system).

\emph{������ ���� ����� ���}, ������������ ��� ������������ � ������ FS.

������ ������ �������� \emph{�������� �������} (root directory), ������� ��� "/".

����� ���� ������, ����� "/", �������� \emph{���� - ������ ���������, ������� ���� ������, ����� ������� �����}.
��� ��������� �������� ������������ ���������� � ������ ������ ���������, ������ �������� �������� ������� "/". ����� �������, ������ ��� ������ ����� ���������� � "/". ������ ��� ����� �� �������� �������������� ���������� (HDD, CD-ROM ��� ���������� ���������� � ����), �� ������� �� ���������� ���������. ������ "/" �������� ������������ � ��������� ���������.

������ ���� ����� ��������� � ��� \emph{����������} (���������� � ��������� ������������ - \emph{inode}), ���������� ��� �������������� ����� � ����������� �� ��������� �������� ��� ���.
 	
���������� ������ \emph{����� �������}, \emph{���������-������������} � \emph{���������-������}, ��������� �� �������� �����, �������� ������. \emph{� ���������� ���} �������� �� \emph{����� �����}.

\section{���� ������}

� UNIX ���������� ����� ����� ������, ������������� �� �������� � ��������� ��� ���������� �������� ��� ����:

\subsection{������� ���� (regular file)}

��� �������� ����� ��� ������, ���������� ������ � ��������� �������. ��� �� ��� ������ ������������������ ����. ������������� ����������� ������������ ���������� �������. \newline
������: ��������� ����, �������� ������, ����������� ����. �� ����� ������������� ��������� \verb+cat ���+ �  \verb+less ���+.	

\subsection{������� (directory)}

��� ����, ���������� ����� ����������� � ��� ������, � ����� ��������� �� ���������� ���� ������, ����������� �� ����������� �������� ��� ����.
 
�������� ���������� ��������� ����� � ������ �������� �������, ��� ��� ��� ���� �� �������� ���������� � ����� ���������������. �������� �������� ������.

������:  
\begin{displaymath}
\verb+����� inode+ 
\left(
\begin{array}{ll}
	\verb+10245 .+ \\ 
	\verb+12432 ..+ \\ 
	\verb+ 8672 file1.txt+ \\ 
	\verb+12567 first+ \\ 
	\verb+19678 report+ 
\end{array} 
\right)
\verb+��� �����+ 
\end{displaymath}

��� ������ � ���������� ������������ �������: \verb+ls+ � ������� \verb+-a+ � \verb+-l+, \verb+cd+, \verb+mkdir+, \verb+rm+, \verb+rmdir+, \verb+mv+.

������ ��� ����� � ������ ������ �������� �������� ������������ ������ ����� ������ ����� � ��� ����������. ������ ������� \emph{��� ����� � �������� �������� ������}. ��� ���������  ��� � �������� ��������� � ��������� ������������ �, ��� �����, � �����������.

\subsection{����������� ���� ���������� (special device file)}

������������ ������ � ����������� ����������. ��������� ���������� � ������� ����� ���������. ������ � ����������� ���������� ����� ��������, ������/������ � ����������� ���� ����������. \emph{���������� �����} ��������� ���������������� ����� ������� (�����������), � \emph{�������} - ����� �������� ������������ ����� - �������. � ��������� ����������� ������ �������� ��� ����� ����������, ��� � ����� ������� �����.

��� �������� ������ ��������� ������������ ������� \verb+mknod+.

\subsection{FIFO ��� ����������� ����� (named pipe)}

������������ ��� ����� ����� ����������. �������� ����� ���������� ��� �������� ������� �������������� �������������� (��. \ref{fifo}). 

\section {����� (������)}


\subsection{������� ������}

����� ����� ����� � ��� ������� ���������� \emph{������� �������} (hard link). 
����� ������ ������� � ����������� �, ��������������, � ������� �����, � �� �����, ��� ���� ���������� ���������� �� ����, ��� ��� �������� � �������� �������. ����� ������� ��������� ������ ����� ����� ��������� ���� � �������� �������.
\begin{verbatim}
������: 
$ pwd
/home/stud1
$ln first /home/stud2 second 
# �������� ������� ������.
\end{verbatim}
��� ������� ������ �� ���� ��������� �����������.  
 
����� \verb+first+ � \verb+second+ ����� ��������� ������ ������ � �������� �������. ���������, ��������� � ����� �� ���� ������, �������� � ������, ��� ��� ��� ��������� �� ���� � �� �� ������. ���� ��� �������� ������ � ������ ������� ��� ����� ��� ����� ������ �������.

\begin{displaymath}
\left(
\begin{array}{lcl}
	 \verb+/home/stud1+ 	&  & \verb+/home/stud2+ \\
 	\verb+10245 .+		&  & \verb+12563 .+ \\ 
	 \verb+12432 ..+ 	&  & \verb+12432 ..+ \\
	 \verb+8672  file1.txt+ &  & \verb+12672 a.out+ \\
	 \verb+12567 first+ 	& \longrightarrow \verb+12567(inode)+ \longleftarrow & \verb+12567 second+ \\
	 \verb+19678 second+ 	&  \downarrow  & \verb+9675  dir1+ \\
				& \verb+������ �����+ &
\end{array}
\right)
\end{displaymath}

���� ���������� � ������� �� ��� ���, ���� ���������� ���� �� ���� ������� �����, ����������� �� ����, �� ���� ���� � ���� ���� ���� �� ���� ���. ��������, ������� �������� ����� \verb+second+ �� ������� ������. �� ����� ������� ����� \verb+first+. 

� ������ ������� \verb+ls -l+ ������ ������� ���������� ���������� ������� ������ �����.

����� �������, ������� ����� �� ����������� � ������� ���� ������, � �������� ������������ ������ ����� ����� ����� � ��� �����������.

������� ������ ����� ������� �������� \verb+ln+ (link).

\subsection{������������� ������}

������ ��� ����� - ������������� �����, ����������� �������� ���������� ����, � ������� �� �������, ������������ ��������.
������������� ������ �������� � ���� ��� �����, �� ������� ���������, � �� ��� ������.

���������� ������������ ������ ��������. ������ \verb+symfirst+ - ����� ����� �����, �� ������� ��������� ������������� �����.
�� �������� � \verb+symfirst+ �� ���, ��� � ������� ������: ��� ��������� � ���� �������� ������ \verb+first+.

\subsection{����� (socket)}
������������ ��� �������������� ��������������. ����� ��������� ����������� � ��������������� ���� (��. \ref{socket}.

\section{��������� �������� �������}

��� Unix-������� ����� ������� ������� ������������ � ���������� ������ � ���������. ������������� ������������ ���� ������ � ��������� ��������� � UNIX-�������� �� ��������� ������ � �������. ��������� ��������� ����� � ���������� � ������.

�������� ������� \verb+"/"+ �������� ������� FS. ��� ��������� ����� � �������� ������������� � ������ ���������, ����������� �������� ���������.

\emph{���������� ��� ������ ���} ����� ���������� ������ ��������������� ����� � ��������� �������� �������. ���������� � \verb+"/"+ (� �������� ��������) � �������� ������ ���� ������������, ������� ����� ������, ����� ������� �����.

\emph{������������� ���} ���������� ��������������� ����� ����� ������� �������. ������� �� ���������� � \verb+"/"+.

\emph{�������-������} - ��� ���, ������� �������� ������ �������. ��� ����� (\verb+..+) ��� ��� �������� ������ ��������� � ��������, ����������� ������� �������. �������� ������� �� ����� ������. �������, ����������� � ������ ��������, ���������� \emph{���������-�������� ��� ������������}. � �������� �������� ����� ���������� �� ����� \verb+"."+. ��������, \verb+./file1+.

\emph{�������� ��� ��������� ���������} ���������� �������, ������ ���������� ������� ������������ � � ������� �� ����� ������� ���� ����� � ���������.
� ������ ��������� �������� ������������ ����� ���������� �� ����� \verb+~+ (������). ��������, \verb+~/file.txt+.

\subsection{�������� ��������}
\begin{enumerate}
\item \verb+/bin+ - �������� ����� ������������� ����� � �������.
\item \verb+/dev+ - �������� ����������� ����� ���������, ���������� ����������� ������� � ������������ �����������. ����� ��������� �����������, ������������ ���������� �� �����. ��������, \verb+/dev/dsk+ - ������ � ������.
\item \verb+/etc+ - ��������� ���������������� ����� � �������. ������ ������� ������ ��������� � \verb+/sbin+ � \verb+/usr/sbin+.
\item \verb+/lib+ - ���������� �� � ������ ������ ����������������. ����� ��������� - � \verb+/usr/lib+. 
\item \verb+/lost+found+ - "������� ���������� ������", �� ���� ���������� ���� ��� ��� ����, �� ������������ �� �����.
\item \verb+/mnt+ - ��� ���������� ���������� (������������) ���������� �������� ������ � �������� ��� ��������� ������ ���������.
\item \verb+/home+ - �������� �������������.
\item \verb+/usr+ 
\begin{itemize} 
 \item \verb+/usr/bin+ - �������;
 \item \verb+/usr/include+ - ������������ ����� ��;
 \item \verb+/usr/man+ - ���������� �������;
 \item \verb+/usr/local+ - �������������� ���������;
 \item \verb+/usr/share+ - �����, ����������� ����� ����������  �����������.	
\end{itemize}
\item \verb+/var+ - ��������� ����� ��������� ��������� (������, �����, ��������).
\item \verb+/tmp+ - ������� ��������� ������. ������ ������ �� ������ ��� ���� ������������� �������.
\end{enumerate}

\section{�������� ������}

\subsection{��������� ������}

������� ���������� ������������ ������ ������������� �������.
������������ ����� ���� ������ ���������� �����, ���� �� ������� �������� ���������, � ��������� - ���������������.

\verb+/etc/passwd+ - ������ ���� ������������� � �� ��������� �����;
\verb+/etc/group+ - ������ ���� ����� � �� �������������� �������������.
� UNIX ����� ���� ����� ���� ����������:
\begin{enumerate}
\item ���������-������������
\item ���������-������.
\end{enumerate}
��� ���� ��������-������������ �� ����������� ����������� ���������-������. 

������� \verb+ls -l+ ������� ���������� � ���������� � ������ � ��������� �������. ��� ��������� ���������� ������������ �������:\newline
\verb+chown �����_����. ���_�����+. ��������: \verb+chown sys something.doc+.\newline
\verb+chgrp  �����_����. ���_�����+. ��������: \verb+chgrp adm something.doc+.

������� ���������-������������ ����� ���� ������� ��������, ���� ������������� (root). ������� ���������-������ ����� ���� ��������-������������ ��� ������, � ������� �� ��� ����������� (POSIX), ���� �������������.

\subsection{����� ������� � ������}

� ������� ����� ���������� ��������, ���������� ������� �������.
� UNIX ���������� ��� ������� ���� �������:
\begin{enumerate}
\item	\verb+u+ (user) ��� ���������-������������
\item	\verb+g+ (group) ��� ���������-������         
\item  	\verb+o+ (other) ��� ���� ��������� 
\item 	\verb+�+ (all - ���������� 3 ���������� ������). ��� ���� ������� �������������
\end{enumerate}

� ������ �� ���� ������� ����������� ��� �������� ����� �������:
\begin{enumerate}
 \item \verb+r+ (read) ����� �� ������           
 \item \verb+w+ (write) ����� �� ������           
 \item \verb+x+ (execute) ����� �� ����������  
\end{enumerate}

� ������ ������� ������ ������� \verb+ls -l+ ����� ����������� ������������� �����.
\begin{verbatim}
������: 
$ ls -l 
 - r w - r - - r w x     1   stud1    students  ...  example.program
 0 1 2 3 4 5 6 7 8 9
0 - ���  �����: - �������; d �������; l ������������� ������; 
    c,b ����������/������� ���� ���������.
1-3 - ����� ������� ��� ���������-������������.
4-6 - ����� ������� ��� ���������-������.
7-9 - ����� ������� ��� ���������.
\end{verbatim}

����� ����� �������� ��������-������������ �(���) �������������.
��� ��������� ���� ������� ������������ ������� \verb+chmod+:
\begin{displaymath}
\verb+chmod+ \quad
\left[ \begin{array}{c}
 u \\ g \\ o \\ a
\end{array} \right]
\left[ \begin{array}{c}
+ \\ - \\ =
\end{array} \right]
\left[ \begin{array}{c}
r \\ w \\ x 
\end{array} \right]
\quad  \verb+�����+ \qquad
\begin{array}{l}
+\ \verb+�������� ����� � �������+ \\
-\ \verb+������ ����� �� �������+ \\
=\ \verb+�������� ����� � ��������� �����+
\end{array}
\end{displaymath}

\begin{verbatim}
������:
$ chmod a+w text  
# �������� ���������� ������ ���� �������������;
$ chmod go=r text 
# ���������� ������ ���� ����� �� ������ ��� ���� ����� ���������-������������;
$ chmod g+x-r program 
# �������� ��� ������ ����� �� ���������� � ������ � ��� ����� ������;
$ chmod u+w, og+r-w text2;
\end{verbatim}

�������� ����� ������� ���� ����� �������� ������ � ������������ ������� ���������.
% ������� ��������� 	                                          

\verb+������: chmod 666 *+.

\subsection{�������� ���� �������}

��� ������� ����� - ��������: ����� �� ������ ����, ����� ��������� ����, ����� �� ������, ����� ����� ����������� ���� ��������, � ����� �� ����������, ����� ��������� ��������� ��� ������.

\emph{����������}. ��� ��������� ������� ������� ���������� ���������� ������� r, ����� ��������� ������������� ��� ��������� ��������� ����� �������.

��� ��������� � ������������� ������ ������������� ���� ������� ���������� ��-�������.

����� ������������� ������ ��������� � ������, �� ������� ��� ���������. �� ����� ������ ����� \verb+777+ (���� ���) � ��� �� ����� ��������. 

��� ��������� \verb+r+ ��������� �������� ����� (� ������ �����) ������, ����������� � ������ ��������. \verb+X+ ��������� "���������" �������, �� ���� ��������� � ����������  � �������� ������ ���������� � ��������.
\begin{verbatim}
������:  
$ chmod u+r-x dir1
$ ls dir1      - ����������
$ ls -l dir1  - Permission denied
$  cd dir1     - Permission denied (���� �).
\end{verbatim}

\verb+r+ � \verb+x+ ��� �������� ��������� ���������� (���� �� ������� �������).
\begin{verbatim}
������:  
$ mkdir dark_dir
$ chmod a-r+w dark_dir
$ ls dark_dir  -�����������
$ ls -l        -���  
$ cat file1       
# yes (������� ���� ��� �����, ����� ���������� � ����).
\end{verbatim}

������� w ������ ���� ���������� ��� ����, ����� ����� ���� �������� �������: ��������� � ������� �����.
��� �������� ����� �� �������� ���������� ����� ������������� ������� w ��� ��������, � ������� �� ���������, � ����� ����� ��� ���� �� �����������.

\subsection{������������������ �������� ����}

\begin{enumerate}
	\item ���� �� ������������� (root), ������ ��������. ����� �� �����������.
	\item ���� �������� ������������� ����������, ���� �������� ��� ����. � ������������ � ���� ��� ����������� ���������� �������� ��� ���.
	\item ���� �������� ������������� �������������, �������� � ������, ��������� ������, ���� �������� ��� ����. ��������������, �� ���� �������� ����������, ���� ���.
	\item ���������� ��� ���� ��������� �������������.
\end{enumerate}
\begin{verbatim}
������: 
----rwr--  2  stud1   students ... file1  
stud1 � ������� ����� ��������, �� ��, ��� ��������, 
����� � ����� ������ ������� ����� �������. 
\end{verbatim}

\subsection{�������������� �������� �����}

��� ������� ������:
\begin{itemize}
	\item \verb+t+ - \emph{"sticky bit"} (��� �������)- ��������� ����� ������������ ����� � ������ ����� ���������� (���������� ��������)
	\item \verb+s+ - set UID, \emph{SUID} - ���������� ����� � ��������, ��� � ����������� �����, � �� ��� � ������������, ������������ ��������� (�� ���������)
	\item \verb+s+ - set GID, \emph{SGID} - �� �� ��� ������
	\item \verb+1+ - ������������ - � ������ ������ ������� � ������ ����� �������� ������ ���� ������
\end{itemize}

��� ���������:
\begin{itemize}
	\item \verb+t+ - ������������ ����� ������� ������ �� ����� � ��������, �������� ������� ��� ����� ����� �� ������;
	\item \verb+s+ ��� ����������� ������ ������-�������� ����������� �� ��������-������ (� �� �� ��������� ������ ������������, ���������� ����).
\end{itemize}

�������������� �������� ����� ��������������� � ������� \verb+chmod+.

 
% процессы
\chapter{��������}

\emph{�������} - ��� ��������� ������������� ���������.

��������� - ������������ ������, ���� �� ���������, ��������� ����� ���� ����������� ����.

��� ������� ���������  �� ���������� ��  ������ ������� \emph{���������} ��� ����� ���������� ������, ���� ��������� ������� ������, ����������� ������� � ����������� �����/������ � ��������� ��������� ��������.

������� ������� �� ����������, ����������� �����������, ������ � ���������� � ����������� ������, �����, ��� ����������� ������, �������� ����� � ������ ��������.

��������� ����� �������� ����� ������ ��������. ������������ ����� ��������� ��������� ����������� ����� ���������. ��������, ���������� BASH - ������������ ����� �������������. ����� ������� UNIX - ������������� ��.

���������� �������� ����������� � ������ ���������� ������ ����������, ������� ������� �� �������� ���������� ������ ���������� ������� ��������. ������� ��������������� �� ������ ������� � ������, �� ��� �� �������� ����� ������ � ����. 

�������� ����������� ���� �� �����. � �� �� �����, �������� ����� ����������� ������������ ���� � ������ ������� � ������� ������� �������������� �������������� \emph{(IPC)}.

���� IPC:
\begin{enumerate}
\item �������
\item ������
\item ����������� ������
\item ��������
\item ���������
\item �����.
\end{enumerate}

\section{���� ���������}

\subsection{��������� ��������}

\emph{��������� ��������} �������� ������ ���� � ������ ����������� � ����������� ������. ��������� �������� �� ����� ��������������� �� �������� � ���� ����������� ������ � ����������� ������ ������� ��� ������������� ���� �������.

\begin{verbatim}
�������:
1)��������� ��������
2)��������� ������
� ������.
\end{verbatim}

����������� ���������� � ������ ��������� ��������� ��������� � ����, ����� �������, ��� ����� ���������� � �������� � ������, �� ��������� ����� (����). 

������� \verb+init+ ����� ����� ������� � ���������, ���� �� ����������� �� �����. �� ����������� ���� ��������� �������. \verb+init+ ����������� ������ ����� �������� ���� � ��������� ��� ��������� ����������������� ������ �� ����������.

\subsection{������}

\emph{������} - ��������������� ��������, ����������� ������� �������, � ����������� � ������� ������. ��� �� ������� �� � ����� ���������������� ������� � �� ����� ��������������� ����������� �������������. ������������ ������ ��������� ���������: \begin{itemize}
\item ������
\item �������� �������
\item ������������� �������
\item �����
\item web-�������
\item ����.
\end{itemize}

\subsection{���������� ��������}

\emph{���������� ��������} - ��� ��������� ��������. ��� �������, ����������� � ������ ����������������� ������. 

\verb+������: ls, BASH.+

���������������� �������� ����� ����������� ��� � �������������, ��� � � ������� ������, �� ����� �� ����� (����������) ���������� ������� ������ ������������. ��� ������ �� ������� ��� ���������������� �������� ����� ����������.

\emph{����������}: ������������� �������� ���������� ������� ����������, �, ���� ����� ������� �� �������� ����������, ������������ �� ����� �������� � ������� ������������. (����� �������, ����� ���� ����� ������� ������ ��������� �� ����� �������������� ��������.)

\section{�������� ���������}

�������� ��������� �� ���������� ��������� ������� ��������.

�������� ��������� ��������: \verb+ps -ef+.

\subsection{������������� ��������}

������������� �������� - \emph{Process ID (PID)} - ������ ������� ����� ���������� �������������, ����������� ���� ������� ��������� ��������.
��� �������� ������ ��������, ���� ����������� ��� ��������� ��������� �������������. ���������� PID - �� ������������, �� ���� PID ������ �������� ������, ��� PID ��������, ���������� ����� ���. 

����  PID ������ ������������� ��������, ��������� ������� ������� ����������� ��������� � ���� �����������.

����� ������� ��������� ������ - ���� ����������� ������� �� PID.

\subsection{������������ �������}

������������� ������������� �����c�� -Parent Process ID (PPID) -  PID ��������, ����������� ������.

\subsection{��������� ��������}
��������� �������� (nice number) - ������������� ��������� ��������, ����������� ������������� ��� ����������� ����������� �������. ��� ������ �����, ��� ������ ��������� (nice - ��������, �� ���� ��� ����� "��������" �������, ��� ������ �� ��������� CPU).

����������� ������������� �������� - ��������� ����������: ����������� ���������� ����� �� ����� ����������. ������������� - ���������, �� ����� ���������� ��������������� ��� ������������� � ������� nice.

\subsection{������������ �����}
������������ ����� (TTY) - �������� ��� ��������������, ��������������� � ���������. 

\emph{����������}. ������ �� ����� ���������������� ���������. 

\subsection{�������������� �������������}

�������� (RID) �  ����������� (EUID) �������������� ������������. RID - ������������� ������������, ������������ ���� �������. EUID ������ ��� ����������� ���� ������� ��������� ��������� �������� (� ������ ������� � �������� �������.)
������ RID=EUID, �� ���� ������� ����� �� �� �����, ��� � ������������, ����������� ���. RID!=EUID, ����� �� ��������� ���������� ��� SUID. ����� EUID=UID, �� ���� ������� �������� �� �� �����, ��� � � ��������� ������������ ����� (��������, �������������).

\subsection{�������������� �����}

�������� (RGID) � ����������� (EGID) �������������� ������. RGID=GID ��������� ������ ������������, ������������ �������. EGID ������ ��� ����������� ���� ������� ������������ �� ������ ������� ������. �� ��������� RGID=EGID, ����� SGID, �������������� �� �������, ����� EGID=GID ������-��������� �������.

\section{��������� ���� ���������}

������� � UNIX ��������� ��������� ������� \verb+fork(2)+. 

�������, ��������� ����� fork(2), ���������� \emph{������������}, � ����� ��������� - \emph{��������}. ����� ������� �������� ������ ������ ����������� ��� ��������. 

\emph{����������}. ����� ������� ����� �� �� ���������� � ������, ��� � ��������. ����� ����, ���������� ������������� � ��������� �������� � ����� � ��� �� �����������, ��������� �� ��������� ������� fork. ������������ �� ������� - ������������� PID.

������ ������� ����� ������ ��������, �� ����� ����� ��������� ��������.

��� ������� ������, �� ���� �������� ����� ���������, ������� ������ ������� ����� \verb+exec(3)+. ��� ���� ����� ������� �� �����������, � ����������� ��� ������ �������� ��������� ���������� ����� ����������� ���������. ��� �� ����� ����������� �������� ���������� ���������, ���������� ����������� ������� �����/������ � ������, � ����� ��������� ��������.

� UNIX ������ �� ���������� ����� ��������� ����� ������ � ����������� ������ ��������. ����� �������, ������� ������� ��������� fork, �������� �������� �������, ������� ����� ��������� exec, ��������� ������� ������������ �������. ����� ��������� ������� ���������� fork-and-exec. 

������ ��������, ����� ���������� ������ ������ fork ��� ������������ exec. � ���� ������ ����������� ��� ������������� � ��������� ��������� ������ ��������� ���������� ��������� ��� ������������� � ��������� ���������. (fork ���������� PID ������������ �������� � ������������ � ���� - � ��������.)

��� �������� ��������� ����� ����� fork. ������ �������������� ���� �� fork-and-exec, ���� � ������� exec.
������������ ���� ��������� �������� init ��� �������������� ���������.


\section{�������}

������� �������� �������� �������� ����������� � ������������� ������-���� �������. ������ ����� ���� �� ������ �������� ������� ��� �� ���� �� ������-���� ��������.
 
������� - ���������� ����� IPC. 

��������, ��� ������� �� ���� �������� ���������� ������ SIGFPE, � ��� ������� Ctrl+C �� ��������� �������� �������� ���������� ������ SIGINT.

��� �������� �������� ������������ ������� \verb+kill+: \\
\verb+$ kill sig_no pid+, ��� \\
sig\_no - ����� ��� ���������� �������� �������;\\
pid - ������������� ��������, �������� ���������� ������.

������������ ����� �������� ������� ������ ��� ���������, ���������� ������� �� ��������, �� ���� RID � EUID ��������� � UID ������������. ������������� (root) ����� �������� ������� ���� ���������.

\begin{verbatim}
������: ������� ������� ��������, ������ ��� ����������� � ������� ������
$ back_fone_prog &
$ kill $! (�� ��������� ���������� SIGTERM, ����� 15).
\end{verbatim}

��� ��������� ������� ������� ����� ����������� ��������� �������:
\begin{enumerate}
\item ������������ ������.\newline
	\emph{���������:} �� ������� ������������ ��������� ��������� �������, ��������, SIGFPE.
\item �������� �� ���������. ������ ��� ���������� ������.
\item ����������� ������ � �������������� ���������� ���. ��������, �������� SIGINT �������� ������� ��� tmp-����� � ��������� ��������� ����������.
\end{enumerate}

\emph{����������:} SIGKILL � SIGSTOP ������ �� �����������, �� ������������.

�������� ��������, ����� ������� �� ��������� �� SIGKILL:
\begin{enumerate}
\item ��������-�����. ���������� �� ����������, �� �������� ������ � ��������� ������� ���������.
\item ��������, ��������� ����������� ������� NFS. ��������, �������� ������� ������ � ���� ���������� ����������, ������� ��� ����������. �������� ������ �������� SIGINT ��� SIGQUIT. 
\item �������, ��������� ���������� �������� � �����������, ��������, ��������� ����� ��� �������������������� ������� CD-ROM �� ��������� �����.
\end{enumerate}

������� ������������ �� ������ ��� ���������� ������ ���������, �� � ����� ����� ������������� �������� ��� ����������. (��� �� ��������� � SIGKILL � SIGSTOP, ������ ��� �� ������ �����������.)
� �������, ��������� ������ - proxy servers, smtp (pop, imap), ����, bind ��� ��������� ������� SIGHUP ������ ���������� ���� ���������������� ����� � ������������. 

% Unix - среда программирования
\chapter{����� ���������������� Unix}

\section{Unix-way ����������������}

\section{Unix ��� ������ ����� ���������� (IDE)}
�� ������������ � ru.unix.prog  ������� ��������� �� ru.unix.prog FAQ :
\begin{quote}
 >> Q: ����� ���� IDE (integrated development environments) ��� Unix?  H�\\
 >> ����� ����������, ����� ��������������, �������� � ������ - ���� ���\\
 >> ������?\\
 
\dots

UNIX ��� �� ���� �������� Integrated Development Environment.

� "�������" IDE ���� ��������-����������, ������� �������� � ������ ������
������� �������, � � ������ ������ -- ���� ���������� ������ ������� �� DLL
��� ����� ������� � ��������.

� UNIX ����� ����������-������������ �������� shell (Emacs ���������
shell'�� � ������ ������).  ��� ���������� ������ ������� ����������
���������� ���������� ����������� ����������� ������, ����� ��� make, cc,
ld, � �. �.

������������ � ���� ����� ��, ��� ������������ �������������� �������
������� ������� ����� "��������" ���������.

H�������, ������� "����������� �����������" ���� ����� ����������� �
������� make, �� ����� ����� ����� ������������, ������, cook, ��� ��
������������� ����� GNU Make � BSD Make �� �����.  ����� ����� �������� �
������������� ����������, ������������, etc.  ����� ����, ��� �� ���� shell
�������� "�������� ������� �������", � ����� ����� ���� �������.

����� ����, ��� ��� ������������ ������� ����������� �������������, �� IDE
"Unix" ������������ ����� ������� �� ��������������� ������� �������
�������, ��������, ��������� ������������� ����, ��������� ������������ �
�. �.
\end{quote}

������� �������, ��������� ������ Unix (shell) � �������� IDE ��� Unix. �������� ������ ��������� �� ������������� �� ���������� �������� ������ ������������� (������������). ����� ���������, �������������� ��� ����� IDE (����������, ��������, �����������, ��������, ���������, ������� ������ � ������������ �������, ������� �������� ������) ����� ���� ������� �� ������. ��� ���������� �� ������� ����� ������� \emph{���������������� �������� Unix}.

\section{�������������� ������ � �������}

����� UNIX-������� �������� ���������� �� (��), ���������� libc  � ����.

��� ������ UNIX ������������� ������ ������������, ������������ ����� ������ � ���� ��, ����� ������� ���������� ������ �������� ������ � ������� ������� UNIX. ��� ����� ����� ���������� \emph{���������� �������� (system calls)}.

��������� ����� ���������� �������, ����������� ����� �� �� ����� ��������, ������������ �����. Syscall �������� ����������� ������ ������� ������ �������������� ���������� ��������� � �����.

���� ���������� ���������������� - ��. ����������� ����������� Unix - ��������� ����������� �� ������������. ����� ����, ����� ���������� � ��������� ������ ���������� �� program space. ��������� ����������� ������ ��� ��������� ��������� ��������� � ���� (������������-��������� �� �����). 

���������� \emph{libc} - ��� ����� ����������� � ��������� ������� � ��������� �������, ���������� ������ ��������� �������. ��� ������������ �������� ����� \emph{��������� �������} � \emph{������������ ��������} ���� � ���, ��� ��� ��������������� � �����. ��������� ����� ����� ������ � ������������ ���� � ��� �����������. ������������ ������� ����������� � ������������ �������� (���� ������� ����� � ������ ��������� ������ � ���� ����������).


\section{�������� ���������� �������� ��� Unix}

����� �������, ���� �������. ��� ��� ��� (�� ������� �������� ������������ ���)

�� 30 ��� ������������� ������ Unix �������������� ���� ��������: �����, �������� � ����, ������������ ����� ����������� �������������. �������� - ��� ����� ����������� ������������, ���������� ����� � ����������� ���� ����� ��������-��� ����������.

�� Tao Of The Unix Programming (by Eric S. Raymond):
\begin{itemize}
\item ������� �����������: ������ ������� �����, ����������� ������ ������������.
\item ������� �������: ������� ����� ��� ������������.
\item ������� ����������: ������������ ��������� ����� ��� ����� ����������������� � ������� �����������.
\item ������� ����������: ��������� ��������� �� ���������� � ���������� �� �������.
\item ������� ��������: �������������� ������. ����������� ������� ����������� ������ �����, ����� ��� ����� �� ��������.
\item ������� �����������: ������ ������� ���������, ������ ���� ����, ��� ������ ������ �� �������.
\item ������� ������������: ������ ��������, ����� ������� �������� � ������� ��������� �����.
\item ������� ����������: ���������� - ��������� ������� � ��������.
\item ������� �������������: ������� ������ � ������, � ����������� ������ ������ ���� �������� � �����.
\item ������� ����������� ���������: ����� �������������� ����������, ������� �� ��� ����� ����� ��������������.
\item ������� ������: ����� ��������� ������ ������� ������, ��� ������ �������.
\item ������� ��������������: ����� ��������� ������ ���������, ��� ������ ����� � ����� �� ���� ������. ��������� �����, ��������� ��� ��������.
\item ������� ��������: ����� ������������ - �������, ��������� ���, ��������� ��� ����� ������ ����� �� ��������� (�������������).
\item ������� ��������� ����: ��������� ������� �����������. ������ ��������� ��� ��������� ������ �������� ������, ����� ��������.
\item ������� �����������: ������� ��������� ����� ���������� ����. ��������� ��� ��������, ������ ��� ��������������.
\item ������� ������������: �� ��������� ���� ���������� �� "����������� ������ �������"
\item ������� �������������: ������������ � �������� �� �������. ��� ����� ��������� ������� ��� �� �������.
\end{itemize}

��������� Unix � ����� �����.

\emph{keep It Simple, Stupid - ������ ��� �������, ������.}



% Инструментальные средства 
\chapter{���������������� �������� ������������}

\section{���������� ��}

���������� ����� �� (C compliler ��� cc) - ������������ ����� �������. � ���� ���������� ���������� ����� ������ Unix ��� ������� �� ����� ��������� ������������. ���� � ������� ������� ������� �������� �� ��\footnote{����� ����, ��� ���� �� ��� ������ ��� ���������� ���� Unix. ������ ����� �� �������� ����� ������� �������������� Unix.}.

���������� ���������� ��� ������� ��������� ������. �� ���������� �������� \verb+cc+. ���������� ������� ���������� ��������� ������������ C� \footnote{� SUSv3(POSIX) �������� ����� ������������ ������ � ����������, ������� ������ ������������ ���������� C�. ����� ������ ������� ��� ��������� �����������. �����������, ��� ������ �� ��� ����� � ���� �������������� ���������.} (gcc � Linux � FreeBSD, ����������� ����������� � ����������� ������������ ������ Unix), ������� \verb+cc+ ����� ��������� �� ���������� �� ��������� ��� ����� �������. 

\begin{verbatim}
������� ����� �����������
$�� foo.c
�������������� � ������ (��������) ��������� �� 'foo.c' 
�������� ����������� ���� 'a.out'

$cc -c foo.c
������ �������������. ����� ������� ��������� ������ 'foo.o'

$cc -o exec_foo foo.c
��������� ����������� ���� 'exec_foo' (������ 'a.out')

$c� foo.c bar.o
���������� 2 ��������� ����� � ����������� ('a.out')
\end{verbatim}


\section{make}

\emph{Make} - ����������� ��������, ����������� ��� ������ ����������� ��������. �������� ������������� ���������� ��� ������� ����� �������������� ��������� � ��������� ������ � ������ ������������.

����� ������: make ������ ���� � ��������� ������� (makefile) �, ������������� ��� ����������, ��������� ������������ ��������.

Makefile - ��������� ����, ��������� ��������� ����� ������� ������� � ��������, ����������� ��� ��� ������.

\subsection{������}

Make �������� �������� ��������� ������ � ����������� �������� \verb+make+. ��� ������� \verb+make+ ��������� ������� ������ \verb+makefile+, \verb+Makefile+ � ���� � ��������� �� ��������� ������ �� ��������� Make-������. ����� ���� ������� ����� Makefile ������������ ������ \verb+-f+.

\subsection{������ � ������������� make-������.}

�������� ������� - \emph{������� (rules)}.

\begin{verbatim}
����� ���:
<���� 1> <���� 2> ?<���� n>:<����������� 1> <�����-�� 2>?<�����-�� n>
        <������� 1>
        <������� 2>
        ?
        <������� n>
\end{verbatim}

\emph{���� (target)} - ����� �������� ���������, ������ ���������� �������� ������ � �������. ���� ����� ���� ������ �����.

\emph{����������}. ����� ��������� ����������� ���������, ����� \verb+make+ ������� �� �� �����.

\begin{verbatim}
������1: ���� ��� ��� �����
iEdit: main.o Editor.o
         gcc main.o Editor.o -o iEdit
������ ���������, ��� ����� �������� ����������� ���� �� ��������� �������.
\end{verbatim}

���� ����� ���� ������ ���������� ��������, ����� ������� ���������, ��� ����������� ��������� ��������.
\begin{verbatim}
������2: ���� ��� ��� ��������
clean:
        rm *.o iEdit
\end{verbatim}

����� ���� �������� \emph{������������} (phony targets) ��� ������������ (pseudo targets).

\emph{����������� (dependency)} - ��� '�������� ������', ����������� ��� ���������� ��������� � ������� ����. ��� ��������������� ������� ���������� ����. ������������ ����� ���� ��� ����� ��� ��� ��������. � �������1 main.o � Editor.o - �����������. ����� ������ ������������, ����� ����� ���� ������� iEdit.
\begin{verbatim}
������3:
clean_all: clean_obj
        rm iEdit
clean_obj:
        rm *.o
 ��� ���������� �lean_all ���������� ��������� �������� clean_obj.
\end{verbatim}

\emph{�������} - ��������, ������� ���� ��������� ��� ���������� ��� ���������� ����. ����� �������� ������ ���� ������ ��������� (��� 9). ��� make ���������� �������.

�������� makefile, ������� �������� ��������� ������, � ������� ������� ���� ��������� ���� � �����������.
\begin{verbatim}
������4: 
1. 	iEdit: main.o Editor.o
2.              gcc main.o Editor.o -o iEdit
3.       main.o: main.cpp
4. 	        gcc -c main.cpp
5.	Editor.o: Editor.cpp
6.	        gcc - Editor.cpp
7.	clean:
8.	        rm *.c
\end{verbatim}

����� ������ - ���������� ������� ���� (default goal). ����  ���� - ��� �������� (�����������), �� ����������� ��������. ���� ������� ���� - ��� �����, �� make ������ ����� ������ ������. ������� ���� ������ �������� ��� �������� make: make iEdit, make clean. ���� make ���������� ��� ����������, �� � �������� ������� ������� ������ ����������� ����. (� �������  ��� iEdit). ������ ������ ���� \emph{all} ��� ���� �� ���������.

�������� ������:
\begin{enumerate}
	\item ����� ������� ����
	\item ���������� ����
	\item ��������� ������
	\item ��������� ������������
\end{enumerate}

���������� ���� - ��������� ����������� � ����� ����������, ���� �� ��������� �������.
��� ������ make iEdit ����������, ��� ������� ���� - iEdit. ������� �� ���������� - ������ 1,2. ����������� ������� iEdit, ����������, ��� ������� �� main.o � Editor.o. ��� ���� ������������ ���������� ������� (3,4) � (5,6). main.o ������� �� main.cpp. ���� ��� ��� ���������� �����, �� ���������� ���� .���, �� ����������� ����������. ���������� � ��� Editor.o. ��� clean ������������ ��� � make ����� ��������� � ����������.

\emph{������������ ������} - ����������������� ������ ��, ��� ���� ��������. ��� ������ .� � .��� ������ ����������� ��� ����������� .h �����.

\subsection{���������� make.}

����������: \verb+��� = ������+ (����� � ���������). 

��������� �������� ����������: \verb+$(���)+. �������� - ��������� ������, ����� ��������� ������ �� ������ ����������.
\begin{verbatim}
������: 
obj_list = main.o Editor.o 
# ����������; 

$(obj_list) 
# ��������� ��������

1)dir_list = . .. src/include
all:
        echo $(dir_list) 
2)optimize_flags  = -03
compile_flags = $(optimize_flags) -pipe
all:
        echo $(compile_flags)
���������: -03 -pipe
3)program_name = iEdit
obj_list = main.o Editor.o TextLine.o
$(program_name) : $(obj_list)
        gcc $(obj_list) -o $(program_name)
\end{verbatim}

\emph{����������}. �������� ���������� ����������� � ������ �������������.

����� ������������ ����������:
\begin{enumerate}
\item \verb+CC+ - ������� ���������� �� ���������.
\item \verb+CFLAGS+ - ��������� ����������
\item \verb+LDFLAGS+ - ��������� �������� ��������� ������
\end{enumerate}

\subsubsection{�������������� ����������}

\emph{}
\begin{itemize}
\item \verb+$^+ - ������ ������������, ���������� ���������
\item \verb+$@+ - ��� ���� (�����). ���� � ��� ��������� ����� (�� \ref{PatternRules}), ��� ���������� ��������� �������� ��� ����, ��� ������� ����������� ������ � ��������� ������
\item \verb+$<+ - ��� ������ �����������
\end{itemize}
\begin{verbatim}
������:
$(program_name):$(obj_name)
        gcc $^ -o $@
\end{verbatim}

\subsection{��������� �������}
\label{PatternRules}
��������� ������� (implicit ��� pattern rules) ����������� � ������ ������.

���������: 
\begin{verbatim}
.<����������_������_�����.> .<����������_������_�����>:
        <������� 1>
        <������� 2>
        ?
        <������� n>	
\end{verbatim}
\begin{verbatim}
������:
.cpp .o:
      gcc -c $^
\end{verbatim}


\section{������� ���������� ��������. CVS}

����� ����� ��� ���������� �������� ������ ������ ��������. ������� ��������� ������ ��������.  � ���������� ����������� ��� �� ������������� �������� ��������� � ����������. ��� ����� ������ ������� ���������� ��������. �� ������������ ������ ����� ������� CVS, RCS, Monotone, Arch, BitKeeper � SourceSafe. 

\verb+CVS+ - \emph{Conhurent Versions Systems} (������� ���������� �������������� ��������). 

\subsection{�����������}

\emph{����������� CVS (��� ���������)} ������ ������ ����� ���� ������ � ��������� ��� ����������� CVS,  ������� ��� ���������

������ �� ������� �� ��������� ������ � ������ � CVS ��������. ������������ ������� CVS ��� ��������� ����� � "������� �������" � ����� ������ ��� ��� ������. 
����� �������� ��������� - ���� ������ ��������� � �����������. ����� �����, � ��������� ����������� ���������� � ��������� ����������, ������� �������� ��������� � ������ �������� ����������.

��� ��������, ����� �� ����������� ������������ - ����������� ���������� ��������� CVSROOT, ���� ���� ����������� � ������� ����� -d.

�������:
\begin{verbatim}
$ CVSROOT=/var/cvs; export CVSROOT
$ cvs checkout module/project
���
$ cvs -d /var/cvs module/project
\end{verbatim}

����� ��������� ������������ - ����� ����� ������������ �������� (�������). ��� ��� ���������� ������� ����� � (������) - ������ �������.
\begin{verbatim}
$ cvs -d server1:/root checkout sdir1
\end{verbatim}

��������� �� ���� - � \ref{cvsbook} ��� � � ������� �������� �� http://opennet.ru

\subsection{��������� ������� ����� ����������}


\subsection{���������� ����������� � ���������������}

\subsection{������������ ������ ��� ��������}



\section{���������� �� (libc)}

libc �������� 2 �����: \emph{��������� ������} � \emph{������������ �������}. 

��������� ������ ����������, ��� ������� ����� �� (���������� �� ����������� ���������� � ����). � UNIX ������ ��������� ����� ����� ��������������� ������� (��� �������) � ��� �� ������, ���������� � ����������� ���������� ��. ������� �� ���������� ��������� �������������� ���������� � ����� ���������������� ���� ����. ����� �������, ������������ ��� - ������ ��������, ����������� ���������� ��������� � ����.

������� ������ �������� - ����� ����� ����������, �� �� �������� ���������� ��������. ������� ������ ���������� � ��������� ������ - ������ ����� ���������������� UNIX.

� ������� �� ������ ���������, libc ��������� � ������ �����������, ���������� �� ��.

���������� � ��������� ������� � �������� ���������� � 2 � 3 �������� \verb+man+ �������������. � ��������� �������� ��������� ����� ��������� �������, ������� ��������� ������� ����� ���� ����������� ��� ������������ � ����� ������� � ��� ��������� ������ � ������. 

libc ��������� �������� � ���� ����������, ������������ � ANSI C (stdio, math, assert). ��� ���������: ���� �� �������� ������� ������� ��������� ����������� - ��� �������� �� �� ANSI C. ����� ��������� ����� ��������������� � �������� �� ���� unix-��������.

��������� ���������� libc ����� ����������� � ������, ����������� ����������� � �������������� ��������������.


% IPC.Процессы и сигналы
\chapter{�������� � �������.IPC}

� UNIX �������� ����������� � ����������� �������� ������������ � ����������� ���� �� �����, ����� ������� ������� � ��������
����������� ������� ��������� ���� �� �����.

�� ���������� ������������� �������������� ���������. ��� ����� ���������:
\begin{itemize}
\item ���������� �������� ��������������
\item ��������� ������������� ������� ������ �������� �� ������
\end{itemize}

�������������� ������ ��������� ������:
\begin{enumerate}
\item �������� ������
\item ���������� ������������� ������
\item ���������
\end{enumerate}

������ �������� �������������� ���������� ��������� � ������ ������������� ������� ������.

\section{���� IPC}
\begin{itemize}
 \item �������
 \item ������
 \item FIFO (����������� ������)
 \item ��������� (������� ���������)
 \item ��������
 \item ����������� ������
 \item ������
\end{itemize}
       
\section{�������}

���������� ��� IPC. ��������� ���������� ������� ��� ������ ��������� � ����������� ���������� �������.

\emph{������ ���������} - ����� ������� ����������� ������������ ������ ���������. � ������ ������ ���� ���� ���������� �������������. ����� ������ - �������, PID �������� ��������� � ID ������. ������ ������ ����������� ��������� �� ��������.

������� ����� �������� ������ � ������� ����.

\emph{����������� ��������} - ������� ����� ���� ������ � ����������, ������� ���������� �����������. ��� �������� ������ ����� ���� � ��� �� ����������� ��������.

����������� ���� ���������� \verb+/dev/tty+ ������ � ����������� ���������� ��������. ������� ��� ����� ���������������� �������������� ������� �� ����������� ������������ �������, ������� ����� ���� ��������� ��� ������ ���������.

\emph{������} - �������� ������ ������������ ��������� ��� ����������� ���������� �������.

���� ��� ���� � ������������� ��������: ��������� (�����������), �������� � ���������. � ���������� - �������� ��������.

\emph{����������}: ������� �� ����� �������������, �� ���� � ����� ���������� ������ ������� ���������� ��������� ����� ������ ����������� �������.

������� ���������� ��������:
\begin{itemize}
\item ������ �������� (��������, ������� �� 0)
\item ������������ ���������� (������� ������ Del, Ctrl+Z, Ctrl+C, ���������� ���������)
\item ������ ��������
\item ���������� ��������� (��� ��������� ���������������)
\item ����� (���������� ��������� ����)
\item ����������� (������� ����������� ���������� � ���������� ����������)
\item ������.
\end{itemize}

��� ��������� ����� ��������� ��� ��������:
\begin{enumerate}
\item ��������� ������� �� ������ (���������)
\item ������������ ������� - ������������ ��������� �� ����� ���������� ����������� �������� ����
\item ������� �������.
\end{enumerate}

\emph{����������:} ������� �� ����� ��������������� ���������� ����������.

������� ������� ��������� ������������� ��� (��������, SIGINT), ������� ���������, ��� ���� ������ ������������ ������ ����� ����.  ����� �������� ���������� � \verb+<signal.h>+.

\subsection{������� ��������}

\begin{verbatim}
#include<sys/types.h>
#include<signal.h>

int kill (pid_t pid, int sig);
\end{verbatim}

\verb+������: kill (7421, SIGTERM)+.

������� ����� �������� ������� ������ ����.

��������� ������ PID (\ref{pid}) - \verb+pid=getpid();+ 

��������� ������ PPID (\ref{ppid}) - \verb+ppid=getppid();+

�����������: EUID (\ref{euid}) ��� RID (\ref{rid}) ��������, ����������� ������, ������ ��������� � EUID � RID ��������-��������. ��� �������������� ����� ����������� ���.

��� ��������� ������ kill ���������� -1 � ���������� errno ������������� ��������: EPERM (������ ������� ������ ��������), ESRCH (������ �������� ���) , EINVAL (sig �������� �������� ����� �������).

\subsubsection{����� ��������� PID}

Pid==0 - ������ ���������� ���� ��������� ������, � ������� ����������� �������, ��������� ������;

Pid==-1 � ���� EUID �� ��������������, �� ���������� ���� ���������, RID ������� ����� EUID ����������� ��������, ������� � ��� (���� ��� RID=EUID);

Pid==-1 � EUID ��������������, �� ������ ���������� ���� ���������, ����� ��������� ���������;

Pid<0 � �� ����� -1 - ���������� ���� ���������, ������������� ������ ������� ����� �� ������ PID, ������� ��������� �������, ���� �� ����� ������ � ��� ������.

\subsubsection{������� ������� ������ ����}

\begin{verbatim}
#include<signal.h>

int raise (int sig)
\end{verbatim}
����������� �������� ���������� ������. � ������ ������ ���������� 0. ��������, \verb+raise (SIGKILL)+.

\subsection{�������}

\verb+setitimer+ - ��������� ������ �������� (3 ����).

\begin{verbatim}
#include<unistd.h>

unsigned int alarm (unsigned int secs);
\end{verbatim}

\verb+secs+ - ����� � ��������, �� ������� ��������������� ������. ����� ��������� ������� �������� ���������� \verb+SIGALRM+.

\verb+������: alarm(60).+

\emph{���������� �������} \verb+alarm(0).+

����� ������� �� �������������. ����� ���������� �������� ����������.

\begin{verbatim}
#include<unistd.h>

int pause (void);
\end{verbatim} 
���������������� ���������� �������� �� ��������� ������ �������, ����� ������������ ������ � \verb+alarm+.

\subsection{���������� � ��������� ����������}

����������� �������� �������� ���������� ���������� (normal termination). ������ �� ����� ��������� ������� \verb+exit+.

��������� ������� ( SIGABRT, SIGBUS, SIGQUIT, SIGILL � ������) ���������� ��������� ���������� �� ������� �������� ����������, ��������� � �.�. � core (dump).

\subsection{����� �������� (�� ��������, ���������)}

���������� � \verb+<signal.h>+
\begin{itemize}
\item SIGABRT - ���������� �������� (abort), ���������� �������� ��� ������ ���������� ������ abort. Core dump.
\item SIGALRM - ������ �������.
\item SIGBUS - ���������� ������ �� ����. ��������� ����������.
\item SIGCHLD - ������� ��� ���������� ��������� ��������. ������������.
\item SIGCONT - ����������� ������ �������������� �������� (�������� ��� SIGSTOP).
\item SIGHUP - ������������ �����, ���������� ���������, ������������ � ������������ �����, ��� ���������� ��������� ��� ��� ���������� ������ ������ ������ ������ ������.
\item SIGILL - ������������ ������� ����������. Core dump.
\item SIGINT - ���������� ���������� ��������� (�trl+C). ���������� ���� ��������� ������.
\item SIGKILL - ����������� ����������� ��������. �� ���������������.
\item SIGPIPE - ������� ������ � ����� ��� �����, ��� �������� ����������� ������� ��� �������� ������.
\item SIGPROF - ������ �������������� �������.
\item SIGQUIT - ���������� ���������. ����� �� SIGINT, �� ���������� ���������.
\item SIGSEGV - ������������ ����� ������. ��������� �����.
\item SIGSTOP - ������ ��������. ���������� ���������. ������ �����������.
\item SIGSYS - ��������� ��������� �����.
\item SIGTERM - ����������� ������ ����������. ������������ ��� ����������� ���������� ��������.
\item SIGTSTP - ������������ ������ ��������� (�trl+Z). ����� �� SIGSTOP, �� ����� �����������.
\item SIGTTIN - ������� ����� � ��������� ������� ���������. ��������� ��������.
\item SIGTTOU - ������� ������ �� �������� ������� ���������. ��������� ��������.
\item SIGURG - ����������� � ����� ������ ������� ������������ ������.
\item SIGUSR1, SIGUSR2 - ��������������� ��� ���������������� �����. �� ��������� - ������ �� ����������.
\item SIGVTALRM - ����������� ������.
\item SIGXCPU - ���������� ������ ������������� �������. Core dump.
\item SIGXFSZ - ���������� ������ �� ������ �����.
\end{itemize}
            
\subsection{������ ��������}
\label{sigset}
\emph{����� ��������} - ��� ������ ��������, ������� ���������� �������� ���������� ������.

��� \verb+sigset_t+ � \verb+<signal.h>+, ��� ������ ��������� ����������� ���� ��������, ������������ � �������.

����� �������� - ���� �� �������, ������ ��������, ���� �� ������� ������, �������� �����������.

\subsubsection{������������� ������}

\begin{verbatim}
#include <signal.h>

int sigemptyset(sigset_t *set);
int sigfillset(sigset_t *set);
\end{verbatim}

\subsubsection{���������� � �������� ��������}

\begin{verbatim}
int sigaddset(sigset_t *set, int signo);

int sigdelset(sigset_t *set, int signo);
\end{verbatim}

\subsubsection{������� �������� ������ � �������}
\begin{verbatim}
sigset_t mask1, mask2;

sigemptyset(&mask1);

sigaddset(&mask1, SIGINT);
sigaddset(&mask1, SIGQUIT);

sigfillset(&mask2);
sigdelset(&mask2, SIGCHLD);
\end{verbatim}

\subsection{���������� ��������}

����� ����������� ������ ����� ������ ��������� ��������:

\begin{verbatim}
#include<signal.h>

int sigaction(int signo, const struct sigaction *act, struct sigaction
*oact);
\end{verbatim}

\verb+signo+ - ������, ��� �������� �������� ��������.

\verb+act+ - ���������� ����������.

\verb+oact+ - ���� �� NULL, �� � ��� ��������� ���������� ������ ����������.

\subsection{������ ��������� sigaction}

\begin{verbatim}
struct sigaction {
void (*sa_handler)(int); //������� �����������
sigset_t sa_mask; //�������, ����������� �� ����� ��������� �������
int sa_flags; //�����, �������� �� ��������� �������
viod (*sa_sigaction)(int, siginfo_t*, void*);
};
\end{verbatim}

����������:

\verb+sa_handler:+
\begin{enumerate}
\item \verb+SIG_DFL+ - ��������� ��������� �� ���������.
\item \verb+SIG_IGN+ - ��������� �������������. �� ����� ���������� ��� SIGSTOP � SIGKILL.
\item ����� �������, ����������� �������� ���� \verb+int+ (\verb+sa_handler=f1+). ��� ����� ���������� ��� ��������� �������, � signo ���������� ��� ��������.
\end{enumerate}

���������� ���������� ������� �� ������ ����� ���������, � ����� �������� �� ��� ���������� ����� ���������� � �����, � ������� ���� ��������.
\begin{itemize}
\item \verb+sa_mask+ - ������� �� ����� ������ ����� �������������� �� ����� ���������� �������-����������� (\verb+sa_handler+).
\item \verb+sa_flags+ - ��������� ��������� �������:
\end{itemize}

\begin{enumerate}
\item \verb+SA_RESETHAND+ - ����� �������� �� ����������� ������� ���������� �� ��������� \verb+SIG_DFL+.
\item \verb+SA_SIGINFO+ - ����������� ���������� �������������� ���������� � ������ \verb+sa_handler+ ������������ \verb+sa_sigaction+.
\item \verb+SA_RESTART+ - ������ ����������� �������� ���������� ������.
\end{enumerate}

\subsection{���������� ������� (���������� ������)}

\begin{verbatim}
#include<signal.h>

void (*signal(int sig, void(*disp)(int))) (int);
\end{verbatim}
\verb+sig+ - ����� �������.

\verb+disp+ - \verb+SIG_DFL, SIG_IGN+ ��� �������-����������.

\verb+������: signal(SIGINT, SIG_IGN);+

\subsection{������������ ��������}

\verb+int sigprocmask(int how, const sigset_t *set, sigset_t *oset);+
\begin{itemize}
\item \verb+how+ - �����������, ����� �������� ���� ���������:
	\begin{enumerate}
		\item \verb+SIG_MASK+ - ���������� ������������ �������� �� ������;
		\item \verb+SIG_UNBLOCK+ - ������ ������������ �������� �� ������;
		\item \verb+SIG_BLOCK+ - ���������� ������ � ������� �����������.
	\end{enumerate}
\item \verb+set+ - ����� ��������.
%\item \verb+oset+ - ������ �����. ���� �� ����� NULL, �� � ���� ��������� ��������.
\end{itemize}

\begin{verbatim}
������:

sigset_t set1;

sigfillset(&set1);

sigprocmask(SIG_SETMASK, &set1, NULL);

//����������� ������� - �������������

sigprocmask(SIG_UNBLOCK, &set, NULL);
\end{verbatim}

\emph{����������:} ����� ����������� ������ � ������� ������������ ����.



% IPC.Каналы и FIFO
%\chapter{IPC. ������ � FIFO}

\section{������}

\emph{����������� �����} (��� ������ �����\footnote{� ������� ������������ ����� ����� ��������� ������������ �����}) ������ ��� ������������ ������������� �����, ����������� ���� ������� � ������.

������ �������� ���� � ������ shell. ��� ����� ���������� ��� UNIX.

\verb+������: who | wc -l+

\subsection{�������� � ��������}
\begin{verbatim}
#include<unistd.h>

int pipe(int filedes[2]);
\end{verbatim}

\verb+filedes+ - ������ �������� ������������ �� ���� ����� �����: \verb+filedes[0]+ - ���������� ��� ������ �� ������, � \verb+filedes[1]+ - ��� ������ � �����.

���������� �������� -1, ���� ��������� ������ ��������.

\verb+int close(filedes);+

���������� 0 � ������ ������. ������ �� ���������.

\subsection{������/������ ����� �������������� �������� ����/�����}

\begin{verbatim}
#include<unistd.h>

ssize_t read(int filedes, void* buffer, size_t n);

ssize_t write(int filedes, const void* buffer, size_t n);
\end{verbatim}
\verb+filedes+ - �������� ����������;

\verb+buffer+ - ��������� �� ������ ��� ���������, �/�� ������� ����������
������;

\verb+n+ - ���������� ����, ������� ����� ���������/��������;

\verb+size_t+ - ���������� ������� �����������/���������� ����.

\begin{verbatim}
������:

#include<unistd.h>
#include<stdio.h>

main()
{
	const int MSGSIZE 16
	char* msg "Hello world #0";
	char inbuf[MSGSIZE];
	int p[2], j;
	pid_t pid;

	if (pipe(p) == -1) exit(1);

	if (pid=fork())
	{
		close(p[0]);
		write(p[1], msg, MSGSIZE);
		break;
	} else 
	{
		close(p[1]);
		read(p[0], inbuf, MSGSIZE);
		printf("%s\n", inbuf);
		wait(NULL);
	};
	exit(0);
}
\end{verbatim}

\emph{���������:} ��� ������ fork() ����������� ����������� ��������� ���������� �� ��������, ������� ������ ������������ ��� ����� ������ ����� ������������ ����������.

����������� ������ ������ �� POSIX - 512 ����. �������� ������� ���, �� ���� ����� ����� ���� ������������ �������. ��� ����������� ���������� ����� ����� ������������ ������ ������ ��� ������.

\subsection{����� ������/������ � �������� ������}

���� write ���������� ��� ������, � ������� ���� �����, �� ������ ���������� � ���������� ���������� �������.

���� ��� write ��������� ������������, �� ������ �� ���� �� ��� ���, ���� �� ����������� �����.

���� write ����� ������ ����, ��� ����������� ���� � ������ �����, �� ������������ ������������ ����� ������, � ����� ��������������� �� ������������ ����� ��� ���������� ������.

������ write ����� ���������� �������� (��������), � ������ ���������� �� ���� ����������� ��������. �� ��� ������� ������ ������ ������ ���������� ��������.

���������: ���� ��������� ��������� ����� � �����, �� ����� ���� ������������ ����������.

��� read �����������, ������ �� �����. ���� �� ����, �� read ����������� �� ��� ���, ���� � ������ �� �������� ������. ���� ������ ����, ���������� ������� �� read, ���� ���� ���� ��������� ������ ������, ��� �����������.

\subsection{�������� �������}

2 ������:
\begin{enumerate}
\item ���������� �� ������:

���� ���������� ������ ��������, � ������� ����� ������ �� ������, �� ������ �� ����������. ���� ����� ��������� �� ����������, � ����� ��� ���� ����, �� ����� �������� ������� ������� ���� ������.

���������������� � ��������� �������� ��������� ������ � ������ �� read � ������� ���������.  

\item ���������� �� ������:

���� ���� ������ ��������, �� ������ �� ����������. ���� ��������� ������ ���, �� ���������� ������ SIGPIPE ���� ���������, ��������� ������ � �����. ���� ������� �� ������������� ���, �� �� ���������� ���� ������, ���� �� ���� �������� ��������� ��������������, �� ����� ����������� write ������ �������� -1 � ���������� errno ����� ��������� �������� EPIPE.
\end{enumerate}

\subsection{������ � ������ ��� ������������}

��� ������������� read � write ������ ����� ���������� ������������, ������� ����� ���� ������������. ��������, ��� ��������� ��������� ������ ��� ������ ���������� ������� �� ��� ���, ���� � ����� �� ��� �� �������� ������.

���������� ��� �������: \verb+fstat+ � \verb+fcntl+. ������ �����.

\verb+fstat+ - � ����� �� ������������ ����� ���������� ������� ���������� �������� � ������, �� ��� ����� ���������� ���������.

\verb+fcntl+ - ������ ��� ���������� ��� ��������� �������, ����� ��������� ������ �������.
\begin{verbatim}
#include<sys/types.h>
#include<unistd.h>
#include<fcntl.h>

int fcntl(int filedes, int cmd, ...);
\end{verbatim}
filedes - �������� ����������;

cmd - ����������� ������� (�������� �� ������ � ����� <fcntl.h>);

��� �������� ��������� ������ �� ��������� � ������� �� cmd.
\begin{verbatim}
������: ������ ����������

fcntl(filedes, F_SETFL, O_NONBLOCK);
\end{verbatim}

\subsection{������������� select}

���� ��������� ����� ������������ ��� ������ � ���������� �������
������������. ������������ ����� ��� ������� ������, ����������, FIFO
� �������.

������: ��������� ������� � ���������� �������� (��������). �� ������
���������� � ���������, ����� ������������ � ���������� � ������� ����
����������, ������ ���������, ��� ����� �� ���������. ���� ����������
��������� ����� � ��������� �������, �� ������ ������ ����� ��� ����
����� ������� ��� ��������� � ���������� ������� (��������, ��
�����������).

\begin{verbatim}
���������:
#include<sys/time.h>

int select (int nfds, fd_set* readfds, fd_set* writefds, fd_set*
errorfds, struct timeval * timeout);
\end{verbatim}

���������:

\emph{nfds} - ������� �����������, ������� ������������ ������� ��� �������.
��������, ���� 0,1,2 - ����������� � ������� ��� ��� ����� �
������������� 3 � 4, �� nfds=5. ���� �������� ����� ������ ������ ���
����� ��������� \verb+FD_SETSIZE+ �� \verb+<sys/time.h>+. ��� ����� �������������
����� ������������, ������� ����� ������������ select.

\emph{readfds, writefds, errorfds} - ��������� �� ������� �����, � �������
������ ��� ������������� ��������� �����������. ���� ��� �������, ��
��� �������� ������� select � ����� �����������. ����� ����������
������ ��������.

\emph{readfds} - ����������, ��� �������� ������ ������� ����������� ������;

\emph{writefds} - ����������, ��� �������� ������ ������� ����������� ������;

\emph{errordfds} - ����������, ��� �������� ������ ������� �����������
��������� ������ ��� �������������� �������� (��������, ������������
������ �� ����).

���������� ����������� ��� fd_set � ������� (��� �������, �
����������� �� ����������) ��� ������ � �������.

timeout - ��������� �� ��������� timeval:

#include<sys/time.h>

struct timeval {

long tv_sec; //�������

long tv_usec; //������������

};

 1. ���� ��������� timeval ����� NULL, �� select ����� ������������ ��
    ��������� ������������� �������;
 2. ���� ���� ��������� �������, �� select ���������� �����������;
 3. ���� ���� ���������, �� ������� ���������� ����� ��������
    ���������� ������ � �����������.

������� ��� ������ � fd_set:

#include<sys/time.h>

void FD_ZERO (fd_set* fdset); - ������������� ������� �����

void FD_SET (int fd, fd_set* fdset); - ��������� ���� �����������

void FD_IZSET (int fd, fd_set* fdset); - ��������, ���������� �� ���

void FD_CLR (int fd, fd_set* fdset); - ������� ��� ����������� � �����

select ����������:

-1 � ������ ������;

0, ���� ����� ��������� ��������;

����� - ���������� ������������ ������������.

�����������: ��� �������� ����������������� ������� ����� readfds,
writefds, errorfds � �������� ���� �� ���������� � ������� ���������
��������� �����������. � ��������� ����������� ���������� ����� �
�������� timeout, ��� ��� ��� ��������� ���������� ���-�� ���������.

������: �� ��� ����������� (�� pipe)

#include<sys/types.h>

#include<unistd.h>

#include<fcntl.h>

int fd1, fd2, fd3;

fdset readset;

fd1 = open("file1", O_RDONLY);

fd2 = open("file2", O_RDONLY);

FD_ZERO(&readset);

FD_SET(fd1, &readset);

FD_SET(fd2, &readset);

int res = select (5, &readset, NULL, NULL, NULL);

if (res > 0)

{

if FD_ISSET(fd1, &readset) printf("������ fd1");

if FD_ISSET(fd2, &readset) printf("������ fd2");

};

                     ������ � exec (����������).

who | wc -l

0x08 graphic
0x08 graphic
0x08 graphic
0x08 graphic
0x08 graphic
0x08 graphic

0x08 graphic
0x08 graphic
0x08 graphic
0x08 graphic
0x08 graphic
0x08 graphic
0x08 graphic

0x08 graphic
0x08 graphic

0x08 graphic
0x08 graphic
0x08 graphic

0x08 graphic
0x08 graphic
0x08 graphic
0x08 graphic
0x08 graphic
0x08 graphic

0x08 graphic
0x08 graphic
0x08 graphic

0x08 graphic

                      FIFO (����������� ������).

� ������� �� ������ - ��������� � �� ��� ��������� ��� ��������
������� UNIX. FIFO ����� ������, ���������, ����� �������.
�����������, �����������, ���������, ��� ������� ����.

read � write ����� ���� ��� � ��� ������������� �������.

� shell ��� �������� ������: mknod channel p, ��� channel - ����� ���
�������� �������, � - ������� ������ (mknod ������������ � ���
dev-������).

������ ����� ����� ���������� ������, ������� �� ������ ������ � FIFO.

������:

�) eat < channel & #�� ����, ��� ��� ���� � ������ ��� ������, ��
��������� ���� �� �����������

�) ls -l > channel; wait #����� � ����� � ��������, ���� ����������
cat

��� �������� FIFO � ��:

#include<sys/types.h>

#include<sys/stat.h>

int mkfifo (const char* pathname, mode_t mode);

mode - ����� �������, �� ������� ����� �������� umask � ��������
��������� ������.

������:

mkfifo ("/tmp/fifo", 0644);

int fd = open ("/tmp/fifo", O_RDONLY);

int fd = open ("/tmp/fifo", O_RDONLY | O_NONBLOCK); - �������������

� ��������� FIFO ������ �� ������������� ������.

�������������

��������

                               write()

                                read()

������������

p[1]

p[0]

�������� ������� ��������� ���������� �� ������ � ����� ���������.

������������ ������� ��������� ���������� �� ������ � ������ ���������
�� ������.

sh

                                  sh

                                fd[0]

                                fd[1]

                                  sh

                                fd[0]

                                fd[1]

fork()

fork()

exec()

                                 who

                                fd[0]

                                fd[1]

                                exec()

                                pipe()

                                  wc

                                fd[0]

                                fd[1]

wc

fd[0]

pipe()

who

fd[1]


% IPC. Сокеты
\chapter{IPC. ������. ��� ������ ������� ��.}

������ ������� �� ���������� ����� ������ "C������ ���������������� � Unix" ( "������� �������", 2003(12), 2004(1), 2004(2), 2004(3), �� http://nestor.minsk.by/sr/abc/114.html ).

\section{��������� ��������}

\emph{������ (sockets)} ������� ��������� � ������ 80-� � ������� 4.2BSD ��� ����������� ��������� � ����� TCP/IP. � ��� ��� �� ����� �������� "BSD ������". � �� � ����� ���� ������� ��������� ������� API, �� ������ ��������� ����� �����. �������� ������ ��� ��� �� �������� ����������� ����������� ����������� Unix ��� ����� ����������������: �����, ��� ������ �������������� ��������� � �������� "keep it simple stupid" (KISS).

������, � ������������� ���������� � ������ ������� ��������� ��������� �������� � ������������ ������� "socket": �����, �����, ������ (!!!) � �.�. ���������������. ���� ���� ������� ���������� ���� � �� ��.

� ������ ������� ����� ������ ��������� "������-������". ������ ��������� ��������� ������, ������ ��� ����������. ��� ��������, ��������� ����, ������������ �������. ��� �������� ����������, � ������ �� ������ ��������� �����. ��������� TCP/IP �������� �� �������� ����� ���� ������. ����� ���������� ��� � ���� �������� ����������, �� �� ������� ��� �������: \emph{����� � ����}. 

\emph{�����} - ������� ����� ����������, \emph{����} - ��� ����� ������ �� ���������� ����������. ������ ������ ���� ������ � ���������� ��������� � ������� ������������ ������� ����� ���� �������� ��������� ������. \emph{����� - ��� ���� �����:����}.

� ����� - �������� ������ ���������� 2 ������, �� ������ �� ������� ��������� ������ �����������. ����� ���� ��������������� ����������� ����� � ��� ������ �� ���� ������ � ���������� ������� ������ ������������. ��� ������� � ����� �� ����� �������, �� �������� �� ������, � ��������.

\verb+��� �����������: ��������� 'netstat -natp' (� linux)+

������ � ������� ���� ����� �������� ����������, ��� ����� ���������� ��� Unix.
������ ���� 2 ���� �������� ����������: � ��������������� ����������� � ���. �� �������� ������� � ��������� ��� ���� ��������. ��� �� ����� ������������ ����� ���������� ��������� � ���������� ������ ������.

\subsection{�� ���� ����}

������ � ������� ������� ����� ����������, ���������� ����� � ������������ �����. �� ���� ��� �������� ���������� ����� �� ���������� � ����� ������ ������������ �� ��������� ����� ��� ���������. ����������� �������� � ������� ���� �� ��������, �������� �� �������� � ������� (regular) ������. ��������� ���������� �� ����, ������, ��� �� �� ����� �� ��������� � ��������� � ������ ����������� ����� ������ (pipe, fifo, block � character devices).

��� �������� ������ ����� �� ���������� ����� ������� ���������: "�� �������� � ���� ��� � ������ � �������, ��� ��� � ����������". ���� ����� ���� ���������� �����������, �������, �������, ���������, ��������� ������, �� �� ������ ��������� �� �� �������. ������/������ � ���������� �������� ������ - ����� ���� � �� �� ��������� ������.

����������� �� ����� ������� � ��������������� ������: ������ ������� ����� �� ���� � 1 ������.
\begin{verbatim}
$cat 4.9-i386-disc1.iso|ssh user@host.localdomain cdrecord -
\end{verbatim}
����� ������������ ����� ������ ���� ������, ������������ �������� shell � ������ �����.
\verb+cat+ ������ ���� � ����� � ������� �� �����. \verb+��������+ �������������� ����� \verb+cat+ ����� \verb+pipe+ �� ���� ������� \verb+ssh+.

\verb+ssh+ ��������� �� ��������� ����� ������� ������ ������. ������ ����� ������� ����� cdrecord ����� ������ �� ������������ �����.

\verb+ssh+ ���������� ��� ���� ���� ����� �����. ��� ���� ������� ������ - \verb+cdrecord+ ���������� ���� �������� ���������� ������ ��� ������ ����������� ������ ��������. ����������: regular file, pipe, socket, block device - ����� 4 ��������� ���� ������.

������ ������� ����� ������� ������� ��������� ����� - ��������� �� ���������� ����� � ���� ������ �������� � ��������� ��� �� ����:
\begin{verbatim}
$mkisofs /home/user2/for_write|ssh user@host.localdomain cdrecord -
\end{verbatim}

��� ��������� ������, ������ ��������������� �������� ������ ������ �� ������ ��������� � ������, ������������� ������� ���������� "������ ������ �����, �� �� �������� � ���� ���������, ������ ����� � ������".

\subsection{�������� � ��������������� ����������� � ���}

���� ��� ������ �������� ���������� ����������� �������. � ������������� ���������� � ��� �����.

�� � �������� ����� ��������������� �� ������.
\begin{enumerate}
\item �������� ������ ������� �onnection-oriented ���������� (������� ��� ������� TCP):
\begin{itemize}
\item ������� ������������ ���������� ����� ���������;
\item ����������� �������� ������ (�������� ������������ ������);
\item �������� ���� ������� ������. TCP/IP ����������� �� ��� ������ ������ ���, ��� ����������;
\item ������� �� ��������� TCP.
\end{itemize}
\item connectionless (������� ������� ��� UDP):
\begin{itemize}
\item �� ������� ����������;
\item ������ �� �����������;
\item �������� ���� ��������, ������� ����� ���������, ������������� � �.�. �������� �� ������������ ���������� ������� � ����� ������������;
\item ������� �� ��������� UDP.
\end{itemize}
\end{enumerate}

�����:

TCP ����� ������� ��������� ������� ��� UDP. �� UDP ����� �����������, � ���� ����� ������� �������� � ���������� ����� � ����� �������� ����, �� UDP ����� ������ ��������. � ���� �������, TCP ������ �������� ��� ������ �� ���������� �������.

��������� � ��������� ��������,\newline
UDP - ������� �������, ���� ����� ��������, � ����� � ����������� �� ������.\newline
TCP - �������: ��� ������� � ����� ������, �� � ��������. ������� ���� ������ �� ���� �� ������ � �� ���������� ���.

\section{�������� ��������� ������}

�������������� ������ ��������� �������, ��� ��� ���� ������� ���������� ������� (� ������ ���� ��� socket-specific, ��� � ����� ������, ��������� � �������)
\begin{itemize}
\item accept(2) - ������� ���������� ( ������������ ��������);
\item bind(2) - ������� ����� � ���������� ������� (������ ������������ ������ ��������);
\item connect(2) - ����������� � ��������� �������� (����������);
\item close(2) - ������� ���� ��� �����;
\item listen(2) - ������ ������������� ������ (���������);
\item read(2) - ������ ������ �� �����;
\item recv(2) - ������ �� ������;
\item select(2) - �������� ��������� ������� �������� ������;
\item send(2) - ������� ������ ����� �����;
\item shutdown(2) - ������� ����������;
\item socket(2) - ������� �����;
\item write(2) - ������ ������ � ����.
\end{itemize}

\emph{����������:} � Solaris, �������������� ���������� ������� �� System V Release 4, ������ �� ������ � ����, � ����������� �����������. ������� man �� ��� ���������� � 3-� �������.

\section{���������� TCP-������}

��� �� ������� - ���� ����� � ��� ���� ��������� �������������� \verb+telnet+. ������� \verb+telnet+ ���� �� ���� ��� ��������� ����������� ����� tcp/ip. �������� \verb+telnet <host> <port>+ � ������������ ������ �������� � ��������. ����� ������������ TCP-based ��������� - ���������.
\begin{verbatim}
[tcpclient.cpp]
1 #include <sys/types.h>
2 #include <sys/socket.h>
3 #include <netinet/in.h>
4 #include <arpa/inet.h>
5 #include <string.h>
6 #include <unistd.h>
7 #include <stdio.h>
8 int main() {
9 struct sockaddr_in addr;
10 memset(&addr,0,sizeof(addr));
11 addr.sin_family=AF_INET;
12 addr.sin_port=htons(1212);
13 addr.sin_addr.s_addr=inet_addr("127.0.0.1");
14 int sock;
15 sock=socket(PF_INET,SOCK_STREAM,0);
16 connect(sock,(sockaddr *)&addr,sizeof(addr));
17 char buf[80+1];
18 memset(buf,0,81);
19 read(sock,buf,80);
20 puts(buf);
21 shutdown(sock,0);
22 close(sock);
23 return 0;
24 }
\end{verbatim}

��� ���������� ����������� � localhost, ���� 1212, ��������� ������������������ ���� � ������� �� �� �����.

� ������ ����� ������� - ������������ ������, ������� ��������������� ����� �������������� ���������. ������ 1-4: ������������ �����. �������� ���������� ������� � ����� ������ ��� ������ ����� ������� ������. 5 � 7 ������ - ������������ ������� ��. 6 ������ - ������������ ���� ����������� ��������� �������.

9, 10, 11-13 ������ - ����������, �������� � ���������� ���������, ���������� ����� ����������. ������� ������ �������� �� ��, ��� ���������������� ���� � ����� ����������. ����� � ���� �������� � ������ �������, ���������� \emph{network byte order}. ������ ������������ ������ ������ �� ����, ������ ��� ����������� i386 � network byte order �� ���������. ������ \verb+htons(3)+ ���������� ��������������� ����� � ������� ���. ������� \verb+inet_addr(3)+ ����������� ��������� �������� IP-������ � ��������. \verb+AF_INET+ �� 11 ������ - �������� ���� ������� (������� � ����� ������).

� �������, �������� ������. 15 ������ - �������� ������ (TCP-������ �����������) � 16 ������ - ���������� � ��������. � \verb+connect(2)+ ������������ ��������� �� ����� , ������ ��� ������ ��������� � ������� ����� ������������� � ����������� �� ���� ������ (������� ��� ���������).

����� ���� ���������� (19 ������) � ����� �� ����� (20 ������) �����������.

21 � 22 ������ - ������� ���������� � ���������� �������� ����������, �� ��������������.

\section{TCP-������}

� �������� ������� ������������ ���������� ������� daytime.
���� ��������� daytime: ������ ����� � ������� ���������� �� ������� �������.
\begin{verbatim}
[daytime.cpp]
1 #include <sys/types.h>
2 #include <sys/socket.h>
3 #include <netinet/in.h>
4 #include <arpa/inet.h>
5 #include <string.h>
6 #include <time.h>
7 #include <unistd.h>
8 #include <stdio.h>
9 char * daytime() {
10 time_t now;
11 now=time(NULL);
12 return ctime(&now);
13 }
14 int main() {
15 struct sockaddr_in addr;
16 memset(&addr,0,sizeof(addr));
17 addr.sin_family=AF_INET;
18 addr.sin_port=htons(1212);
19 addr.sin_addr.s_addr=INADDR_ANY;//inet_addr("127.0.0.1");
20 int sock, c_sock;
21 sock=socket(PF_INET,SOCK_STREAM,0);
22 bind(sock,(struct sockaddr *)&addr,sizeof(addr));
23 listen(sock,5);
24 for (;;) {
25 c_sock=accept(sock,NULL,NULL);
26 char buf[81];
27 memset(buf,0,81);
28 strncpy(buf,daytime(),80);
29 write(c_sock,buf,strlen(buf));
30 shutdown(c_sock,0);
31 close(c_sock);
32 puts("answer tcp");
33 }
34 return 0;
35 }
\end{verbatim}

������ �� �������. ������ ���������� �� ������� ������� ������. �� ������:
\begin{enumerate}
\item ������ ����� � ���� (22 ������), \verb+bind(2)+;
\item �������� ������������� ����� (23 ������), \verb+listen(2)+;
\item ��������� � ������������ ���������� (24-33 ������), \verb+accept(2)+.
\end{enumerate}

���������� ������� �� �������: � 19 ������ ����������� ����������� ��������� \verb+INADDR_ANY+. ��� ����������, ��� ���������� ����� ����������� �� ���� �����������. ����� ������ � ���������� ����� (� ����� TCP-�������). ��� �������� ������ �������, ������� ����� ������ ����� �������� � ������� (apache, squid).

� 24 �� 33 ������ � ��� ����������� ����. �� ������� \verb-Ctrl+C-, �����������. ������ �� ������ � ������������� ������ ���������� ������� ����������.

����� ���������� �����������: \emph{������ �������� ���������� ��������� ���� �����} (��. 25 ������). ����������� ����� (\verb+sock+) � ���������� ����� (\verb+c_sock+) - ���������. ����� ������� ���� ����� \verb+c_sock+. ���������� ����������� �� ����������� (\verb+sock+).

���� ��� ������ ��������� ���������� �� ����������� ����� � ������ ����������:
\begin{itemize}
\item  ��������� ������ ������������ - ��� ������������� �����������, ��� � ���� ������� ������, ������ �� ���� ������� - ��������;
\item ����������� ����������������� ��������� ���������� (��������� �� ���� �������� �����).
\end{itemize}
� ������� ������������ ������� ���������������� ���������. ���� ���������� ���������� �� ���������� - �������� ���������� ���� ����� �������. ��� ����� ����� ������ ��� ������� ������� - ����� ����� ��������� ������� ���� � �������� ��� ����� ����������� ��������.

9-13 ������ - ��������� �������� �������. ��� ������� ������������ � ������ 28 ��� ������������ ����������.

\section{UDP}
��� �� � ������� � ��������� ����� �������, � ��������� UDP � ������� ������ � ���.
������ �����������, � �������, � ��������, ��� ������, ��������.
���������, ��� UDP - ��� ������ �������� ������ ��� ����������� ���������� � ��� �������� ������������ ��������. ������ ���� ���������� ������� - ����� �� �������. �� ���� ����� �������� �������� - ������ �� � �� ������������ UDP?
������ ���������� ��� ����������� ������������ (datagram) ��� ���������� ������� ��� �������� � ����������.

\subsection{UDP-������}

� ��������� ����������� � ������ ���������������� ���������� ���� telnet �� ����������. ������� � ���� ���� ��������� ����������� �� ������ �� ����� ������� ������. ����������:
\begin{verbatim}
[udp_client.cpp]
1 #include <sys/types.h>
2 #include <sys/socket.h>
3 #include <netinet/in.h>
4 #include <arpa/inet.h>
5 #include <string.h>
6 #include <unistd.h>
7 #include <stdio.h>

8 int main() {
9 struct sockaddr_in server, client={AF_INET,INADDR_ANY,INADDR_ANY};

10 memset(&server,0,sizeof(server));

11 server.sin_family=AF_INET;
12 server.sin_port=htons(1212);
13 server.sin_addr.s_addr=inet_addr("127.0.0.1");

14 int sock;

15 sock=socket(PF_INET,SOCK_DGRAM,0);
16 bind(sock,(sockaddr *)&client,sizeof(client));

17 char buf[81];
18 memset(buf,0,81);
19 strcpy(buf,"request");
20 sendto(sock,&buf,strlen(buf),0,(sockaddr *)&server,sizeof(server));
21 memset(buf,0,81);
22 recvfrom(sock,buf,80,0,NULL,0);
23 puts(buf);

24 return 0;
25 }
\end{verbatim}

��� ����-��������� �������� ����� "request" �� ���� 1212 �� ������ 127.0.0.1 (localhost) � ������, ��� �� ������� ������.
� 1 �� 14 ������ - ����������� �������� �������, ����� ��� TCP � UDP. � ��� ������, ������� � 12 � 13 ������ ���� � ����� ����������.

� 15 �������� ������ �������� �� ��������� \verb+SOCK_DGRAM+ - �� ������ ��� ������ ��� �����������.

16 - ��� ����������. �� ���� ��������� �������� ������ � ����� � ���������� ������. 9 ������ ���� ���������, ��� ������������ ���������, ����� ���� ������ � ����� ����� �� ������� ����������� �������� ������. ���� ������ ��� �������� ��������� � ������ ���� ��������� �������.

20 � 22 ������ - ��������� ����������� ��� UDP ��������. ��������� ������ \verb+sendto(2)+ � \verb+recvfrom(2)+ ������������� ��� �������� � ��������� ��������� �/�� ������. �� ����� ������������ ���� ���� ����� ��������� � ������������� ���������. ������, �������� �� ������� \verb+SOCK_DGRAM+, ����� ����� ������������ \verb+connect(2)+ � �������� \verb+sendto(2)+ �� \verb+write(2)+ � \verb+recvfrom(2)+ �� \verb+read(2)+. ����� ������� ����������� �������� ������ � ������ ����������, �� ���������� �� ����� ���������������, �.�. ����� �������� �� �������� \verb+SOCK_DGRAM+.

\subsection{UDP-������}

����� �� daytime ������ � DGRAM-������������. ���� ����������, ��� ������� ���� ��������� ����� ����������� � �������� � ����� ������� ����� �������.
\begin{verbatim}
[udp_server.cpp]

1 #include <sys/types.h>
2 #include <sys/socket.h>
3 #include <netinet/in.h>
4 #include <arpa/inet.h>
5 #include <string.h>
6 #include <time.h>
7 #include <unistd.h>
9 #include <stdio.h>

10 char * daytime() {
11 time_t now;
12 now=time(NULL);
13 return ctime(&now);
14 }

15 int main() {
16 struct sockaddr_in addr;

17 memset(&addr,0,sizeof(addr));

18 addr.sin_family=AF_INET;
19 addr.sin_port=htons(1212);
20 addr.sin_addr.s_addr=INADDR_ANY;

21 int sock, c_sock;

22 sock=socket(PF_INET,SOCK_DGRAM,0);
23 bind(sock,(struct sockaddr *)&addr,sizeof(addr));
24 for (;;) {
25 struct sockaddr from;
26 unsigned int len=sizeof(from);
27 char buf[81];
28 memset(buf,0,81);
29 recvfrom(sock,&buf,80,0,&from,&len);
30 printf("udp incoming:%s",buf);

31 memset(buf,0,81);
32 strncpy(buf,daytime(),80);

33 sendto(sock,buf,strlen(buf),0,&from,len);
34 puts("answer udp");
35 }

36 return 0;
37 }
\end{verbatim}

��� ������ � ������ �������� ���������� ��������� � ����������. ���� � ���������� �/��� ��������� ��������� - ���������� �� ������ ������ :).

� ���� ������� �� �������� �������� ����� �� ���� ������� - 29 � 33. ��� ������ \verb+recvfrom(2)+ ��������� ������� ����������� ��������� 5 � 6. ����� ����������� � ������ ��������� ��� ��� ��������. ��� ������ ������ � ������������� ������� � � ������� � ���� ������� ����������. � � 33 ������ ���� ������ ������ � �����.

\subsection{������� �������}

�������������� ���� �������� ��������, ��� ������ � ������ � UDP ����������� ��������� �� ������ ������������ �������. ������� ����� � ���� ��������� ���������, � ������ ������ �������� � �������.

������ ��������: ������������ ������ � ����� ������: ������ ���� ��������� ����, ������ - ����������. ���� ����� ���������� � ������� - �� ��� ������ ���� ������������ ����.

������ ��������: ������� ������� \verb+recvfrom(2)+ � \verb+sendto(2)+. ������ ������� ��������, ����� ���������, ������, ��������, ��������� � ����� ��������. ������-������ as is: ������ ��������, ������ ��������� �� ��������.

\section{������ ����������� ��������}

� ����� ������������, �������� ��� ���������� �������, ����� ��������� ����� ����������� ��������� ���������� ������. ��������� ����������, ��������������� ��������� ����������� ������������, ��� ����������� ��� ��������� ���������. ��� ������� �������, ������� ������������ �������� � ����������� ���������.

���������� N-�� ���������� �������� ����������� ������������� �������� � ��������. �������� ���������������� �� ��� � � �������. �������� ����� ������� ���������� ����������� �� W. Richard Stevens "Unix. Network Programming. Networking API" � ru.unix.prog FAQ. ������� �� ���� �������� �� ���������� ���������, � ������ ��������������� ��������.

\subsection{"������������" � "����������" ������� ����������� �������}

��� �� ����������������� ��� ������ ����������� ������� ����������� �������? ����� - ������� �������, �������. ���������� ������� ��������, � �� ��������� ��������� ��������� ����������.

������ ��������: ���������, ��� � ������� ������� �������������� 1 ����������.

���� � ���������� ������� �������� ������-����� � ����������� ���������� � ��������� �� �������, ������� ���������������� ������ ����� � ����� ���. ������� ������ �� ������� ������� :)

\subsection{���������������� ������}

���������������� - ��� ����� ��� ������� �������������� ���� �� ������ ��������������� (�� ������ daytime):
\begin{verbatim}
24 for (;;) {
25 c_sock=accept(sock,NULL,NULL);
26 char buf[81];
27 memset(buf,0,81);
28 strncpy(buf,daytime(),80);
29 write(c_sock,buf,strlen(buf));
30 shutdown(c_sock,0);
31 close(c_sock);
32 puts("answer tcp");
33 }
\end{verbatim}

���� ������� �� ���������, ��������� ������ �� ���������� ����� � �������. ������ ������� ����������� � \verb+listen(2)+.
����� ������ �, ��� �� �������, ���������� ����������. ������� �������������� ��������� �������� �� ���������� ������������. �� ������� �������� � ��������� �������� ������� - � ��������� ������� ������ ����� � �������. ���������� �������� - ��� - �� ��� ����.

��� �� ���������������? ����� - ����. ���������� 4 ���������� � ������? ������ �� ��������� :). ��� ����, ��������������� ����� �� �����, �� ��������� ���-���� ��������.
�����: ����������, �� ��� ������� �����.

\subsection{���� ������� == ���� ������ (������� � prefork)}

����� ����������� �������� ������������ �������� ���� � ����� �� ��������� �����:\newline
- ���: ���� ����� �������� ������������ ������������...\newline
- ������� ������� ���������� �������� ����� �������!\newline
- ��, ��� �����, ��������!

��� ������� ������, ����� ��������� ����� ("������������"): ������ ���������� ��������� ���������, ������ �� ������� ����������� �� ������ �������.

������������ ���� ������ ��������:
\begin{enumerate}
\item ������� ��� ��������� �����;
\item ������ ������������� � ������ ����� ���������;
\item ������ � 1 �������� �� ����� � ������ ���� ���������.
\end{enumerate}

�� ��� ���� � ������������� �������. ������� - ��� ���������� ������� ������ OC. ��� ������� ���������� �������� �� �������� ������������ �������� ������� � ����� ������ �� ������. ������������ ��������� � ��������� � ���� �������� ������ �� ������ ����� ��������� �� ������� ��������� ����� �������� (��� ������������� ������� ���������� ��������).

\begin{verbatim}
[fork_server.cpp]

1 for (;;) {
2 c_sock=accept(sock,NULL,NULL);
3 if ( !fork() ) {
4 puts("incoming tcp\n");
5 char time_str[81];
6 memset(time_str,0,81);
7 strncpy(time_str,daytime(),80);
8 write(c_sock,time_str,strlen(time_str));
9 puts("answer tcp\n");
10 shutdown(c_sock,0);
11 close(c_sock);
12 exit(0);
13 }
14 }
\end{verbatim}

��� �� ��� ��������? ����� ������� ���������� (������ 2), ����������� ����� ������� ����� \verb+fork(2)+ (������ 3). �������� ����� ���������� ����� �������� � ���� \verb+accept(2)+, ����� �� ������� ������������ ���������� � ��������� ������. ������ \verb+fork(2)+ � \verb+if+? �������� \verb+man 2 fork+ �� ������� ������������.

������, ����������� ������ - �� ������������ ��������� ������ ���������� ������ �1 ������� == 1 ������. ���������� ����� ���������� (� ����� ������� �������) ����������� ��� ��������� \emph{prefork}. ���� ��� ���������: ����� ���� \verb+listen(2)+ �� ��������� N ���������. � ������ �� ��� - ���������������� ������ (�� ����� 4.1.1). ��� ����� �� N �������� ���������� ������������� ������ � ���� �� ������������ �����. �������� �����: ���� ������������ ��������� ��������� ������ \verb+accept(2)+ �� ���������� ����, �� �� ��������� ���� � ������ �� ������� ����������. ����� ������� ���������� - �������������� �.�. "�������" � ���� �� ��������� ��������� � ��� ������������ �� �����. ��������� ����� ������ � ������. ���������� ���� ������� ��������� �������������� ������������ ��������.
����� - ��� ��� ��������, ������ ������ ������������. ����� - ��� ���������� ��������� >200 ���������� ������ � ������������������ (��������� �� �����). ������� �������� ������ ����� �� �����������, ���� �� 200 ��������� ������ �� �������� ���������. ��� ������� - ����� ������� ���������� \verb+accept(2)+ ��������� ��� �����������. ����� ������ 1 ������� ������ \verb+accept(2)+ � ������������ "�������" �� �����.

\subsection{���� ����� == ���� ������}

���� ������: ������� � ���������� ������� �������� �� ������ � ��������� ����� ������. ������������� ��������� ������� - �������� �����������. � ������ ������� ��������� �� ������ ����������� �� �������. �� � ����� ������ ���������, ��� ����� ����� ������ ��� �������, � ������������ ��������� ����� �������� ����� ��������.
\begin{verbatim}
1 for (;;) {
2 pthread_mutex_lock(&mutex_tcp);
3 c_sock=accept(sock,NULL,NULL);
4 pthread_mutex_unlock(&mutex_tcp);
5 pthread_create(&tid,NULL,&deliver_tcp,&c_sock);
6 }
\end{verbatim}

����� ��������� ������ \verb+accept(2)+ ����������� ��������� (��� ����������� � ���� � prefork, ��������� � ���������� �������, ����������� � � �������). 2 � 4 ������ - ���������� � ���-���������� ��������. � ������ 5 �� ��������� ����� �� ���������� ������� \verb+deliver_tcp+ � ���������� \verb+c_sock+.

\subsection{�������������� Finite State Machine � �������������������}

�������, ��� � �������� ��������������� ���������������� - ��� ����� "���������� ����". ��������, �������� ���� ������ ��� ���, ��� ���������� �������� �������, ��������� ��������� �� ������ ��� ������������� �������� ������� ������������.
� ����� �� �������� ��������� �� �� ��������� ������. ������ ��� �� ����� ���� ��� �� ����, � �������� ������ �� ������ ��������. �������������� � ������� ���� - �������������, �� ������������ ������� �� ������ ������������������. � ����� ������.
������� ���, ��� ����� �������� ����� �� �������� (� ������ ������ - ���������) ���� ���������� �����. � �������� ������� �������� ������ �����, ����� ����� ������� ������ ����������� ��������� ����� ����, ��� ����� ������ ������ �� �������.� (������ ������, �� ��������� � ru.unix.�rog.)

� ����� ������������ �������� ��������� ��������� �������� ���������� c �������������� FSM (�� ������ - �� �������� ��������). �������� ������� - ��� ������ �������������� ����������, ����������� ������ � ��������� ����������� ��������� � ��������� ����� ����.

�������� ����������� Unix, �������� ��������� ������ FSM-��������� - �������� ������ ��������� ������ (����� select, poll, kevent/kqueue ��� epoll).

������ ���������� ��������� ������������ ������ � ������� ��������� ����������. � ������ ���������� �� �����-������ �� ��� - ���������� ����������� �������� � ������������ � ��������� ������. ���� ���������� �������� ���������, ��� ��� ������� ������.
\begin{verbatim}
������������ TCP � UDP echo-server � �������������� ������ ��������� ������� ����� select. �������� � ������������:

...
1 struct sockaddr_in addr;
/* ����� ������������� addr */

2 int sock, c_sock, u_sock;
/* ����� ��������� tcp-������� �� sock � udp-������� �� u_sock */

/* ����������� ������� ������������ ������ � �� ������������� ������� */
3 fd_set rfds, afds;
4 int nfds=getdtablesize();
5 FD_ZERO(&afds);
6 FD_SET(sock,&afds);
7 FD_SET(u_sock,&afds);

8 for (;;) {
9 memcpy(&rfds, &afds, sizeof(rfds));
10 select(nfds, &rfds, NULL,NULL,NULL);

11 if ( FD_ISSET(sock, &rfds) ) {
12 c_sock=accept(sock,NULL,NULL);
13 puts("incoming tcp");
14 FD_SET(c_sock,&afds);
15 continue;
16 }

17 for (int fd=0; fd<nfds; ++fd)
18 if ( fd == u_sock && FD_ISSET(u_sock,&rfds) ) {
19 struct sockaddr from;
20 unsigned int len=sizeof(from);
21 char buffer[81];
22 memset(buffer,0,81);
23 int size=recvfrom(u_sock,&buffer,80,0,&from,&len);
24 sendto(u_sock,buffer,size,0,&from,len);
25 printf("answer udp:%s",buffer);

26 } else if ( fd != sock && FD_ISSET(fd,&rfds) ) {
27 char buffer[81];
28 memset(buffer,0,81);
29 int len=read(fd,buffer,80);
30 if (len <=0) {
31 puts("connection closed");
32 close(fd);
33 FD_CLR(fd, &afds);
34 continue;
35 }

36 write(fd,buffer,len);
37 printf("answer tcp:%s",buffer);
38 } // if and for
39 } // for
\end{verbatim}

Echo-������ ������ ������ ������������ ����: ���, ��� � ���� ��������, ���������� ������� �����������. ��� � tcp, ��� � � udp. ������ ���������� �������� ������������ � �������������� TCP-������������ � � ������������ UDP-�������� � ����� ������������ ��������.

���������� �� �������� ����������! :) C ������ 3 �� 7 ���������� ����� �������� ������ �������������: ����� ������������ ������ (\verb+fd_set+). ��� ����� �� ������, ����� ���� ����������� �� 1 ������� �� ���, � ����� ������. ��� ������ �������� ��� ����������� TCP- � UDP-�������. ��� ��������� ��������� ����� FSM - \verb+afds+, � ������� ���������� ��� ������ - \verb+sock+ � \verb+u_sock+.

10 ������ - ��� � ����� ��������� �������. � ������ �� � ��� ��� - \verb+sock+ � \verb+u_sock+. �� ���� ���������� �������� ����� �� ����� ����� ��������. \verb+select(2)+ �������� �������� \verb+rfds+ - ������ � ������ 9 �� ��������������� ��� ������ ��� �� \verb+afds+.

� ��� ������ - ������ ������. ���������, ��� �� ������� ��������:\newline
- sock? ���� ��������� ���������� � ��������� ����� ����� � afds (������ 11-16);\newline
- \verb+u_sock+? �������� UDP ����� � ������� ��� ���������� ����������� (������ 18-25);\newline
- ���� �� ���������� TCP-�������? ��������� ����� ������� ������ � ������� �� ����� (������ 26-38).\newline

������ �������� � select(2), �������, �� ������ ��� ���������� Unix API.

��������� 2, 3, 4 - ��������� �� ������ ������������ ������. ����� ������ - ������� (��. 1 ����� ����� "������� ���������������� � Unix"). ������ ������������ ���������� stdout, socket � pipe - ������� � ������������� ������� � ����������. �������������� ������ ���������� ����������� "��������� ��� ������", "��������� ��� ������", "��, � ������� ��������� ������". ������ ����� ���������� ���� �� �����, ���� ��� ���� ��������� ������������� ����. ����� �� ��� ����� ���� NULL.

���������, 5-�, �������� - ��� �����, ����� �������� select �������� ��������, ���� ������ �� ��������� � ������������� �� �������. ����� ����� NULL - ���� �����, �� ����������� �������.

� 1-� �������� - ��� �����, ������ �������� ������������� ����������� �� ������ +1. ����� �������� ��������� ������� ������� - ������������ ���������� ��������������� ������������ ��������� ������. ����� �� �������� ������, �� ������������ ���������� �������� ����� (��� �������� ������). ������ select(2) ������� 0,1,2,3,..n ����������� � ������� ���������� �� ���. ��� �� ������ �����, � ������� ����������� ������ ������� ������ ������ �������� ������ ��������� �����.

��� �� ����� ������� �� ���� ������ � �����?
�������� ���������� � ����� ������ ������������� CPU.
��� ������� ������������� ����������� �������� ����� ���� ������ � ������������ ������� ������ ����������. � ���� ������ ��������� �������� ����� ������� � ��������� ������/�������� � ������ ��� ����� �����. ��� ���������� ������������ �������� - ������� ������� �������� �� ����� � ��������� ��, ������� ���������� �� ��������� ������.
��� ��� ����� ��������, ������ � FSM ������� ����� ����������� ����������������.

\subsection{��������� ������}

��� �� � ������� ��� ������� ���� ��������. ����� �������� �� ������������.
�������� ��������� � ����� ������ "�����������������+FSM". ��� ��������������� (�� Linux) "���������������+FSM".
C���� ���������. ��� "����� ������ ������", ���� �������� ���������� � ����� ��������.



% System V IPC.
\chapter{System V IPC}

К семейству System V IPC относятся 3 метода взаимодействия:
\begin{itemize}
\item очереди сообщений
\item разделяемая память
\item семафоры
\end{itemize}

Характеризуются общим способом адресации и схожими системными вызовами для работы с ними.
Хранятся в пространстве ядра. После создания - не требуют существования создавшего процесса.

Область видимости структур IPC - вся система и не имеют счётчика ссылок. Не адресуются как имена на файловой системе, соотвественно не могут управлятся как файлы с помощью стандартных вызовов. Управляются только своими syscalls. Используют свои user-space утилиты для просмотра и удаления (ipcs, ipcrm).

Поскольку структуры System V IPC - не файлы, то к ним не может быть применено  мультиплексирование (\ref{select}, \ref{poll}). Это усложняет использование более чем 1 структуры одновременно или совместно с вводом/выводом из/в файл/устройтво. 

\section{Идентификаторы и ключи}

Каждая структура IPC (очереди, семафоры, разделяемая память) адресуется в ядре положительным целочисленным идентификатором. 

При создании структуры IPC ключ должен быть указан. 

Все три типа адресуются ключём типа \verb+key_t+. \verb+key_t+ определён в \verb+<sys/types.h>+ как целое (по меньшей мере - 32разрядное). Ключ переводится в идентификатор ядром.

Обычно генерируется функцией \verb+ftok+ по пути к файлу и произвольному идентификатору :
\begin{verbatim}
#include <sys/types.h>
key_t ftok(const char * pathname, int proj_id);
\end{verbatim}

Существуют различные пути для обеспечения встречи клиента и сервера в одной IPC структуре:
\begin{enumerate}
\item Сервер создаёт новую структуру IPC указывая \verb+IPC_PRIVATE+ и сохраняет ключ куда-либо для клиента (например в файл).  Также может использоваться при отношениях "предок-потомок", тогда ключ передаётся при fork порождённому процессу.
\item Через общий заголовочный файл, задав ключ как число. Сервер создаёт структуру IPC указывая этот ключ. Клиент получает доступ. Проблема - ключ уже может использоваться кем-то ещё.
\item Клиент и сервер могут согласится на путь к файлу и идентификатор проекта (от 0 до 255) и вызвать \verb+ftok()+ для перевода пути и идентификатора в ключ.
\end{enumerate}

\section{Права доступа}

System V IPC связывает структуру \verb+ipc_perm+ с каждой структурой IPC. Она определяет 
\emph{владельца и права доступа} (аналогично доступу к файлу).

\begin{verbatim}
struct ipc_perm {
	uid_t uid; // uid владельца
	gid_t gid; // uid группы
	uid_t cuid; // uid создателя 
	gid_t cgid; // gid создателя
	mode_t mode; // режимы доступа
	ulog seq; // число использований (TODO, неясный смысл)
	key_t key; // ключ
};
\end{verbatim}

Все поля кроме \verb+seq+ задаются при создании структуры IPC. С помощью вызовов \verb+msgctl, +\verb+semctl+, \verb+shmctl+ мы можем поменять \verb+uid+, \verb+gid+ и \verb+mode+ позднее.

\begin{table}[hbtp]
\caption{Права доступа System V IPC объектов}
\begin{tabular}{ l l l l  }
	Permission & Message Queue & Semaphore & Shared Memory \\
	\hline
	user-read  & \verb+MSG_R+ & \verb+SEM_R+ & \verb+SHM_R+ \\
	user-write & \verb+MSG_W+ & \verb+SEM_A+ & \verb+SHM_W+ \\
	\hline
	group-read  & \verb+MSG_R>>3+ & \verb+SEM_R>>3+ & \verb+SHM_R>>3+ \\
	group-write & \verb+MSG_W>>3+ & \verb+SEM_A>>3+ & \verb+SHM_W>>3+ \\
	\hline
	other-read  & \verb+MSG_R>>6+ & \verb+SEM_R>>6+ & \verb+SHM_R>>6+ \\
	other-write & \verb+MSG_W>>6+ & \verb+SEM_A>>6+ & \verb+SHM_W>>6+ \\
	\hline
\end{tabular}
\label{sysvipc_perm}
\end{table}

\section{Очереди сообщений}

\emph{Очередь сообщений} - это связанный список сообщений, хранящийся в ядре и определяемый идентификатором очереди (Queue ID).

Очередь можно создать или открыть уже существующую через \verb+msgget()+.

Новые сообщения добавляются в конец очереди через \verb+msgsnd()+. Каждое сообщение имеет тип (положительное целое число), длину, и данные этой длины, которые передаются в \verb+msgsnd()+ при добавлении в очередь.

Сообщения принимаются из очереди через \verb+msgrcv+. Мы можем принимать сообщения основываясь на их типе.

\subsection{Создание очереди}
\begin{verbatim}
#include <sys/types.h>
#include <sys/ipc.h>
#include <sys/msg.h>

int msgget(key_t key, int flag);
\end{verbatim}


\backmatter
\addcontentsline{toc}{part}{Литература}

\begin{thebibliography}{99}
\bibitem{Rob} �. �����������. ������������ ������� UNIX. ������ ������� - BHV, 2005.
\bibitem{Nemeth} �. �����, �. ������� � ��. UNIX: ����������� ���������� ��������������. ��� ��������������. 3-� ���. - BHV, 2002
\bibitem{Esr} ���� �������. ��������� ���������������� ��� Unix (The Art of Unix Programming). - �������, 2005.
\bibitem{Glass} ���� �����, ���� �����. Unix ��� ������������� � �������������. - BHV, 2004.
\bibitem{Moli} �. ����. Unix/Linux: ������ � �������� ����������������. - �����-�����, 2004.
\bibitem{Ker} �. ��������, �. ����. UNIX. ����������� ���������. ������, 2003
\bibitem{Friedle} ��. �����. ���������� ���������. ���������� ������������. - �����, 2001. 
\bibitem{Robbins} ������� �������. Linux. ���������������� � ��������. - �����-�����, 2005. 
\bibitem{Pet} �. ������. LINUX. �� ��������� � ����������. - ���, 2000.
\bibitem{NetProgVol1} �. �������.  UNIX: ���������� ������� ����������. - �����, 2003.
\bibitem{NetProgVol2} �. �������.  UNIX: �������������� ���������. - �����, 2002.
\bibitem{Asp} ASPLinux. ����������� ������������. - ASPLinux, 2001.
\bibitem{Bah} �. ���. ����������� ������������ ������� UNIX. http://www.lib.ru.
\bibitem{Sol} �. ��������. Sed � awk. ������� �������.
\bibitem{shell} ���������������� ��  shell (Unix).
\bibitem{bootle} ���� �����-����. �������� � Unix. - ����, 1995.
\bibitem{Sage} Rassel S. Sage. ������ ���������������� ������ � UNIX. http://www.citforum.ru.
\bibitem{Mak} ��. ��������. UNIX. - ���������, 1996.
\bibitem{text} �. ������. Linux. ��������� ������. ����������� ����������. - �����, 2001 
\bibitem{cvsbook} Free Software Foundation. CVS book. 1993-2004.
\bibitem{ProgPract} �. ��������, �. ����. �������� ����������������. - �������, 2004

\end{thebibliography}


\include{fdl}
\end{document}
