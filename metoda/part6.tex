\chapter{Файловая система ОС UNIX}

С точки зрения пользователя в ОС UNIX существует два типа объектов: файлы и процессы.

 Все данные хранятся в виде файлов, доступ к периферийным устройствам осуществляется через чтение/запись  в специальные файлы. 

При запуске программы ядро загружает соответствующий исполняемый файл, создает образ процесса и передает ему управление.

Во время выполнения процесс  может считывать или писать данные в файл. С другой стороны, вся функциональность ОС определяется выполнением соответствующих процессов. 

Таким образом, понятия файловой системы и процессов тесно взаимосвязаны.

\section{Базовые сведения о файловой системе}

В UNIX файлы организованы в виде \emph{древовидной структуры} (дерева), называемой \emph{файловой системой} (FS или file system).

\emph{Каждый файл имеет имя}, определяющее его расположение в дереве FS.

Корнем дерева является \emph{корневой каталог} (root directory), имеющий имя "/".

Имена всех файлов, кроме "/", содержат \emph{путь - список каталогов, которые надо пройти, чтобы достичь файла}.
Все доступное файловое пространство объединено в единое дерево каталогов, корнем которого является каталог "/". Таким образом, полное имя любого файла начинается с "/". Полное имя файла не содержит идентификатора устройства (HDD, CD-ROM или удаленного компьютера в сети), на котором он фактически находится. Символ "/" является разделителем в структуре каталогов.

Каждый файл имеет связанные с ним \emph{метаданные} (хранящиеся в индексных дескрипторах - \emph{inode}), содержащие все характеристики файла и позволяющие ОС выполнять операции над ним.
 	
Метаданные хранят \emph{права доступа}, \emph{владельца-пользователя} и \emph{владельца-группу}, указатели на дисковые блоки, хранящие данные. \emph{В метаданных нет} сведений об \emph{имени файла}.

\section{Типы файлов}

В UNIX существует шесть типов файлов, различающихся по строение и поведению при выполнении операций над ними:

\subsection{Обычный файл (regular file)}

Это наиболее общий тип файлов, содержащий данные в некотором формате. Для ОС это просто последовательность байт. Интерпретация содержимого производится прикладной задачей. \newline
Пример: текстовый файл, двоичные данные, исполняемый файл. Их можно просматривать командами \verb+cat имя+ и  \verb+less имя+.	

\subsection{Каталог (directory)}

Это файл, содержащий имена находящихся в нем файлов, а также указатели на метаданные этих файлов, позволяющие ОС производить операции над ними.
 
Каталоги определяют положение файла в дереве файловой системы, так как сам файл не содержит информации о своем местонахождении. Каталоги образуют дерево.

Пример:  
\begin{displaymath}
\verb+Номер inode+ 
\left(
\begin{array}{ll}
	\verb+10245 .+ \\ 
	\verb+12432 ..+ \\ 
	\verb+ 8672 file1.txt+ \\ 
	\verb+12567 first+ \\ 
	\verb+19678 report+ 
\end{array} 
\right)
\verb+Имя файла+ 
\end{displaymath}

Для работы с каталогами используются команды: \verb+ls+ с ключами \verb+-a+ и \verb+-l+, \verb+cd+, \verb+mkdir+, \verb+rm+, \verb+rmdir+, \verb+mv+.

Первые два байта в каждой строке каталога являются единственной связью между именем файла и его содержимым. Именно поэтому \emph{имя файла в каталоге называют связью}. Оно связывает  имя в иерархии каталогов с индексным дескриптором и, тем самым, с информацией.

\subsection{Специальный файл устройства (special device file)}

Обеспечивает доступ к физическому устройству. Различают символьные и блочные файлы устройств. Доступ к устройствам происходит путем открытия, чтения/записи в специальный файл устройства. \emph{Символьные файлы} позволяют небуферизованный обмен данными (посимвольно), а \emph{блочные} - обмен пакетами определенной длины - блоками. К некоторым устройствам доступ возможен как через символьные, так и через блочные файлы.

Для создания файлов устройств используется команда \verb+mknod+.

\subsection{FIFO или именованный канал (named pipe)}

Используется для связи между процессами. Подробно будет рассмотрен при описании системы межпроцессного взаимодействия (см. \ref{fifo}). 

\section {Связь (ссылка)}


\subsection{Жесткая ссылка}

Связь имени файла с его данными называется \emph{жесткой ссылкой} (hard link). 
Имена жестко связаны с метаданными и, соответственно, с данными файла, в то время, как файл существует независимо от того, как его называют в файловой системе. Такая система позволяет одному файлу иметь несколько имен в файловой системе.
\begin{verbatim}
Пример: 
$ pwd
/home/stud1
$ln first /home/stud2 second 
# создание жесткой ссылки.
\end{verbatim}
Все жесткие ссылки на файл абсолютно равноправны.  
 
Файлы \verb+first+ и \verb+second+ будут отличатся только именем в файловой системе. Изменения, внесенные в любой из этих файлов, затронут и другой, так как они ссылаются на одни и те же данные. Даже при переносе файлов в другой каталог все равно они будут жестко связаны.

\begin{displaymath}
\left(
\begin{array}{lcl}
	 \verb+/home/stud1+ 	&  & \verb+/home/stud2+ \\
 	\verb+10245 .+		&  & \verb+12563 .+ \\ 
	 \verb+12432 ..+ 	&  & \verb+12432 ..+ \\
	 \verb+8672  file1.txt+ &  & \verb+12672 a.out+ \\
	 \verb+12567 first+ 	& \longrightarrow \verb+12567(inode)+ \longleftarrow & \verb+12567 second+ \\
	 \verb+19678 second+ 	&  \downarrow  & \verb+9675  dir1+ \\
				& \verb+Данные файла+ &
\end{array}
\right)
\end{displaymath}

Файл существует в системе до тех пор, пока существует хотя бы одна жесткая связь, указывающая на него, то есть пока у него есть хотя бы одно имя. Например, простое удаления файла \verb+second+ не удаляет данные. Их можно достать через \verb+first+. 

В выводе команды \verb+ls -l+ вторая колонка показывает количество жестких связей файла.

Таким образом, жесткая связь не принадлежит к особому типу файлов, а является естественной формой связи имени файла с его метаданными.

Жесткие ссылки можно создать командой \verb+ln+ (link).

\subsection{Символическая ссылка}

Особый тип связи - символическая связь, позволяющая косвенно адресовать файл, в отличие от жесткой, обращающейся напрямую.
Символическая ссылка содержит в себе имя файла, на который ссылается, а не его данные.

Физическое расположение файлов различно. Размер \verb+symfirst+ - длина имени файла, на который ссылается символическая связь.
ОС работает с \verb+symfirst+ не так, как с обычным файлом: при обращении к нему появятся данные \verb+first+.

\subsection{Сокет (socket)}
Используются для межпроцессного взаимодействия. Будут подробнее рассмотрены в соответствующей теме (см. \ref{socket}.

\section{Структура файловой системы}

Все Unix-системы имеют сходную систему расположения и именования файлов и каталогов. Использование общепринятых имен файлов и структуры каталогов в UNIX-подобных ОС облегчает работу и перенос. Нарушение структуры ведет к нарушениям в работе.

Корневой каталог \verb+"/"+ является основой FS. Все остальные файлы и каталоги располагаются в рамках структуры, порождаемой корневым каталогом.

\emph{Абсолютное или полное имя} файла определяет точное местонахождение файла в структуре файловой системы. Начинается с \verb+"/"+ (в корневом каталоге) и содержит полный путь подкаталогов, которые нужно пройти, чтобы достичь файла.

\emph{Относительное имя} определяет местонахождение файла через текущий каталог. Никогда не начинается с \verb+"/"+.

\emph{Каталог-предок} - это тот, который содержит другой каталог. Две точки (\verb+..+) как имя каталога всегда относятся к каталогу, содержащему текущий каталог. Корневой каталог не имеет предка. Каталог, находящийся в другом каталоге, называется \emph{каталогом-потомком или подкаталогом}. К текущему каталогу можно обратиться по имени \verb+"."+. Например, \verb+./file1+.

\emph{Домашним или начальным каталогом} называется область, котрая выделяется каждому пользователю и в которой он может хранить свои файлы и программы.
К своему домашнему каталогу пользователь может обратиться по имени \verb+~+ (тильда). Например, \verb+~/file.txt+.

\subsection{Основные каталоги}
\begin{enumerate}
\item \verb+/bin+ - наиболее часто употребляемые файлы и утилиты.
\item \verb+/dev+ - содержит специальные файлы устройств, являющиеся интерфейсом доступа к периферийным устройствам. Может содержать подкаталоги, группирующие устройства по типам. Например, \verb+/dev/dsk+ - доступ к дискам.
\item \verb+/etc+ - системные конфигурационные файлы и утилиты. Иногда утилиты отсюда выносятся в \verb+/sbin+ и \verb+/usr/sbin+.
\item \verb+/lib+ - библиотеки Си и других языков программирования. Часть библиотек - в \verb+/usr/lib+. 
\item \verb+/lost+found+ - "каталог потерянных файлов", то есть потерявших свое имя при сбое, но существующих на диске.
\item \verb+/mnt+ - для временного связывания (монтирования) физических файловых систем к корневой для получения единой структуры.
\item \verb+/home+ - каталоги пользователей.
\item \verb+/usr+ 
\begin{itemize} 
 \item \verb+/usr/bin+ - утилиты;
 \item \verb+/usr/include+ - заголовочные файлы Си;
 \item \verb+/usr/man+ - справочная система;
 \item \verb+/usr/local+ - дополнительные программы;
 \item \verb+/usr/share+ - файлы, разделяемые между различными  программами.	
\end{itemize}
\item \verb+/var+ - временные файлы сервисных подсистем (печати, почты, новостей).
\item \verb+/tmp+ - каталог временных файлов. Обычно открыт на запись для всех пользователей системы.
\end{enumerate}

\section{Атрибуты файлов}

\subsection{Владельцы файлов}

Группой называется определенный список пользователей системы.
Пользователь может быть членом нескольких групп, одна из которых является первичной, а остальные - дополнительными.

\verb+/etc/passwd+ - список всех пользователей и их первичных групп;
\verb+/etc/group+ - список всех групп и их дополнительных пользователей.
В UNIX любой файл имеет двух владельцев:
\begin{enumerate}
\item владельца-пользователя
\item владельца-группу.
\end{enumerate}
При этом владелец-пользователь не обязательно принадлежит владельцу-группе. 

Команда \verb+ls -l+ выводит информацию о владельцах в третью и четвертую колонки. Для изменения владельцев используются команды:\newline
\verb+chown новый_влад. имя_файла+. Например: \verb+chown sys something.doc+.\newline
\verb+chgrp  новый_влад. имя_файла+. Например: \verb+chgrp adm something.doc+.

Сменить владельца-пользователя может либо текущий владелец, либо администратор (root). Сменить владельца-группу может либо владелец-пользователь для группу, к которой он сам принадлежит (POSIX), либо администратор.

\subsection{Права доступа к файлам}

У каждого файла существуют атрибуты, называемые правами доступа.
В UNIX существует три базовых типа доступа:
\begin{enumerate}
\item 	\verb+u+ (user) для владельца-пользователя
\item 	\verb+g+ (group) для владельца-группы         
\item  	\verb+o+ (other) для всех остальных 
\item 	\verb+а+ (all - объединяет 3 предыдущих класса). Для всех классов пользователей
\end{enumerate}

В каждом из этих классов установлены три основных права доступа:
\begin{enumerate}
 \item \verb+r+ (read) право на чтение           
 \item \verb+w+ (write) право на запись           
 \item \verb+x+ (execute) право на выполнение  
\end{enumerate}

В первой колонке вывода команды \verb+ls -l+ можно просмотреть установленные права.
\begin{verbatim}
Пример: 
$ ls -l 
 - r w - r - - r w x     1   stud1    students  ...  example.program
 0 1 2 3 4 5 6 7 8 9
0 - тип  файла: - обычный; d каталог; l символическая ссылка; 
    c,b символьный/блочный файл устройств.
1-3 - права доступа для владельца-пользователя.
4-6 - права доступа для владельца-группы.
7-9 - права доступа для остальных.
\end{verbatim}

Права может изменять владелец-пользователь и(или) администратор.
Для изменения прав доступа используется команда \verb+chmod+:
\begin{displaymath}
\verb+chmod+ \quad
\left[ \begin{array}{c}
 u \\ g \\ o \\ a
\end{array} \right]
\left[ \begin{array}{c}
+ \\ - \\ =
\end{array} \right]
\left[ \begin{array}{c}
r \\ w \\ x 
\end{array} \right]
\quad  \verb+файлы+ \qquad
\begin{array}{l}
+\ \verb+добавить права к текущим+ \\
-\ \verb+отнять права от текущих+ \\
=\ \verb+обнулить права и присвоить новые+
\end{array}
\end{displaymath}

\begin{verbatim}
Пример:
$ chmod a+w text  
# добавить разрешение писать всем пользователям;
$ chmod go=r text 
# установить только одно право на чтение для всех кроме владельца-пользователя;
$ chmod g+x-r program 
# добавить для группы право на выполнение и отнять у нее право читать;
$ chmod u+w, og+r-w text2;
\end{verbatim}

Возможно также задание прав через числовой формат в восьмеричной системе счисления.
% битовая структура 	                                          

\verb+Пример: chmod 666 *+.

\subsection{Значение прав доступа}

Для обычных фалов - очевидно: право на чтение надо, чтобы прочитать файл, право на запись, чтобы иметь возможность файл изменить, а право на выполнение, чтобы запустить программу или скрипт.

\emph{Примечание}. Для успешного запуска скрипта необходимо установить атрибут r, чтобы командный интерпретатор мог построчно считывать текст скрипта.

Для каталогов и символических связей интерпретация прав доступа проводится по-другому.

Права символических ссылок совпадают с файлом, на который она указывает. На самой ссылке стоит \verb+777+ (всем все) и это не имеет значения. 

Для каталогов \verb+r+ позволяет получить имена (и только имена) файлов, находящихся в данном каталоге. \verb+X+ позволяет "выполнить" каталог, то есть заглянуть в метаданные  и получить полную информацию о каталоге.
\begin{verbatim}
Пример:  
$ chmod u+r-x dir1
$ ls dir1      - выполнится
$ ls -l dir1  - Permission denied
$  cd dir1     - Permission denied (надо х).
\end{verbatim}

\verb+r+ и \verb+x+ для каталога действуют независимо (одно не требует другого).
\begin{verbatim}
Пример:  
$ mkdir dark_dir
$ chmod a-r+w dark_dir
$ ls dark_dir  -выполниться
$ ls -l        -нет  
$ cat file1       
# yes (заранее зная имя файла, можно обратиться к нему).
\end{verbatim}

Атрибут w должен быть установлен для того, чтобы можно было изменять каталог: создавать и удалять файлы.
Для удаления файла из каталога достаточно иметь установленный атрибут w для каталога, в котором он находился, а права файла при этом не учитываются.

\subsection{Последовательность проверки прав}

\begin{enumerate}
	\item если вы администратор (root), доступ разрешен. Права не проверяются.
	\item если операция запрашивается владельцем, идет проверка его прав. В соответствии с ними ему разрешается выполнение операции или нет.
	\item если операция запрашивается пользователем, входящим в группу, владеющую файлом, идет проверка его прав. Соответственно, он либо получает разрешение, либо нет.
	\item аналогично для всех остальных пользователей.
\end{enumerate}
\begin{verbatim}
Пример: 
----rwr--  2  stud1   students ... file1  
stud1 в доступе будет отказано, но он, как владелец, 
может в любой момент сменить права доступа. 
\end{verbatim}

\subsection{Дополнительные атрибуты файла}

Для обычных файлов:
\begin{itemize}
	\item \verb+t+ - \emph{"sticky bit"} (бит липучка)- сохранить образ выполняемого файла в памяти после выполнения (устаревший аттрибут)
	\item \verb+s+ - set UID, \emph{SUID} - установить права у процесса, как у запущенного файла, а не как у пользователя, запустившего программу (по умолчанию)
	\item \verb+s+ - set GID, \emph{SGID} - то же для группы
	\item \verb+1+ - блокирование - в каждый момент времени с файлом может работать только одна задача
\end{itemize}

Для каталогов:
\begin{itemize}
	\item \verb+t+ - пользователь может удалять только те файлы и каталоги, которыми владеет или имеет право на запись;
	\item \verb+s+ для создаваемых файлов группа-владелец наследуется от каталога-предка (а не от первичной группы пользователя, создающего файл).
\end{itemize}

Дополнительные атрибуты также устанавливаются с помощью \verb+chmod+.

