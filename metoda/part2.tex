\chapter{Командная строка}

\section{Основные принципы и команды}

Как следует из названия, основное назначение командной оболочки - ввод и исполнение команд. Для ввода команд служит т.н. \emph{командная строка}, содержащая приглашение к вводу.

В примерах приглашение командной строки будем обозначать \verb+$+.

После ввода имени и пароля, пользователь получает приглашение на ввод. После окончания работы, он выходит, набирая \verb+exit+ или \verb+logout+. 

Большинству команд можно передавать дополнительную информацию. Это делается с помощью \emph{аргументов и ключей}.\\
Общий вид команды:
\verb+команда [-ключи] аргумент1 ... аргументN+

Простейшая команда - \verb+echo+. Она просто выводит на экран свои параметры.

\emph{Ключ} - 1 или более букв, перед которой ставится \verb+-+(минус или дефис). Обычно ключи управляют режимами работы команды.

\emph{Аргумент} - строка, передаваемая команде.
\begin{verbatim}
Пример 
$kill -9 1023
kill имя команды (завершение процесса)
-9  режим (безусловное завершение процесса)
1023 номер процесса (завершаемый процесс) 
$ - символ приглашения ввода (в разных системах - разный)
\end{verbatim}

После ввода команды, система производит поиск в каталогах программ, заданных переменной \emph{PATH} \label{path}. Если команда найдена, то она будет запущена. Иногда, если PATH не содержит нужного каталога, нужно указать полный путь (см. \ref{fullpath}) к программе.

\emph{Примечание.} Текущий каталог НЕ ВХОДИТ в PATH. Программы из него запускаютсяь следующим образом:\\
\verb+./program+
 
\section{Встроенная справка}

Любая Unix система обладает развитой системой справки. Она называется \verb+man+ (от англ. слова manual). 

Справка запускается следующим образом: \verb+man статья+\footnote{Будет показана \emph{только первая} доступная статья}. Для перемещения используются клавиши \verb+"вверх", "вниз"+. Для выхода нажмите \verb+q+.

\verb+man+ делится на несколько разделов, которые пронумерованы\footnote{Здесь приведены только те разделы, которые будут активно использоватьсяв рамках курса ОСиС. Описания других разделов вы сможете найти в книгах \cite{Rob} и \cite{Pet}}:
\begin{itemize}
\item 1 -  Команды,  которые  могут  быть   запущены пользователем
\item 2 -  Системные вызовы (функции, исполняемые ядром)
\item 3 -  Библиотечные вызовы (функции из библиотек различных языков программирования)
\item 5 -  Форматы файлов и соглашения
\end{itemize}

Запись типа \verb+bash(1)+ обозначает, что справка о команде \verb+bash+ находиться в разделе 1 (команды). Бывает что справки в различных разделах называются одинаково. Тогда для получения нужной надо явно указать раздел.

\begin{verbatim}
Примеры использования man
$ man echo
 
Получение справки по самой команде man
$ man man

Справка из конкретного раздела
$ man 2 open

Все статьи с таким названием из всех разделов
$ man -a mount
\end{verbatim}

Некоторые команды содержат свою собственную систему помощи\footnote{в основном это касается команд написанных проектом GNU}, Она вызывается с помощью ключа \verb+--help+ или вызовом без параметров
\begin{verbatim}
Примеры
$ file
Usage: file [-bciknvzL] [-f namefile] [-m magicfiles] file...
Usage: file -C [-m magic]

$ split --help
Usage: split [OPTION] [INPUT [PREFIX]]
Output fixed-size pieces of INPUT to PREFIXaa, PREFIXab, ...; default
PREFIX is `x'.  With no INPUT, or when INPUT is -, read standard input.

  -b, --bytes=SIZE        put SIZE bytes per output file
  -C, --line-bytes=SIZE   put at most SIZE bytes of lines per output file
  -l, --lines=NUMBER      put NUMBER lines per output file
  -NUMBER                 same as -l NUMBER
      --verbose           print a diagnostic to standard error just
                            before each output file is opened
      --help              display this help and exit
      --version           output version information and exit

SIZE may have a multiplier suffix: b for 512, k for 1K, m for 1 Meg.

Report bugs to <bug-textutils@gnu.org>.

\end{verbatim}


\section{Перемещение по файловой системе} 


Файл в Unix - это основа всего. Существует высказывание "Unix - это файлы".

Имя файла может состоять из любых символов, которые можно ввести с клавиатуры. Точка не имеет особого значения - имя файла может содержать ее или нет\footnote{можно создать файл или каталог с именем из одних точек}. Обычно пользователи присваивают расширения своим файлам, чтобы различать их тип. Система \emph{не связывает типы файлов с их расширениями}. Например файл \verb+example.txt+ может быть исполняемым (программой), а может и не быть. А \verb+/bin/sh+ как правило исполняемый.

Однако у точки есть специальные значения 
\begin{itemize}
\item точка в начале файла указывает на то, что файл - скрытый, например \verb+.profile+
\item файл \verb+.+ - ссылка на текущий каталог (см. пример относительных путей)
\item файл \verb+..+ - ссылка на родительский каталог
\end{itemize}

Если вы не знаете тип файла, примените команду \verb+file+.
\begin{verbatim}
$ file /home/work/OSIS/LAB/RUsak/demon1.c 
/home/work/OSIS/LAB/RUsak/demon1.c: ASCII C program text, with very long lines

$ file /bin/bash
/bin/bash: ELF 32-bit LSB executable, Intel 80386, version 1, dynamically linked
\end{verbatim}

Unix различает строчные и заглавные буквы в именах. Например \verb+name1.txt.125+ и \verb+NAME1.TXT.125+ - это разные файлы.

Файловая система ОС Unix является единой. Это значит что, все файлы находятся в рамках одной логической структуры - \emph{дерева каталогов}. В Unix отсутствует понятие "диск" и "буква устройства". 

Например - файлы с дискеты обычно находятся в \verb+/mnt/floppy+, но могут быть и в любом другом месте.

Началом дерева является корневой каталог (корень), обозначаемый \verb+/+ . 

Разделителем каталогов является прямой слэш  \verb+/+ .

\emph{Путь} - последовательный список каталогов, который нужно пройти, чтобы достигнуть файла или каталога.

Есть 2 вида путей: \emph{абсолютный} и \emph{относительный}.\\
\emph{Абсолютный}\label{fullpath} - полный путь относительно корневого каталога.
\begin{verbatim}
Примеры абсолютных путей:
/etc/init.d/apache
/bin/sh
/usr/local/share
\end{verbatim}
\emph{относительный} - путь относительно текущего каталога \label{relativepath}
\begin{verbatim}
Примеры относительных путей:
./a.out
laba1/text.cpp
../index.html
\end{verbatim}

\subsection{Команды перемещения}

\begin{itemize}
\item \verb+cd+ - перемещение между каталогами. Может использоваться в 2-х вариантах:
\begin{verbatim}
перемещение по указанному пути
$cd  /usr/local 

вернуться в домашний каталог
$cd
\end{verbatim} 

\item \verb+pwd+ - показать текущий каталог
\begin{verbatim}
$pwd 
/home/user/OSiS/metoda
\end{verbatim} 

\item \verb+ls+ - показать содержимое каталога(файлы и подкаталоги). Имеет множество ключей и режимов работы
\begin{verbatim}
Без параметров - вывод содержимого текущего каталога
$ ls
literatura.aux  Makefile    metoda.dvi  metoda.tex  part1.aux  part2.aux
literatura.tex  metoda.aux  metoda.log  metoda.toc  part1.tex  part2.tex

C аргументом - вывод содержимого этого каталога
$ ls /usr
bin   doc  games    kerberos  libexec  man                sbin   src  X11R6
dict  etc  include  lib       local    OpenOffice.org1.0  share  tmp

С ключом "-l" - полный формат вывода (с доп информацией)
$ ls -l /home/user/OSiS
итого 710
drwxrwxr-x    2 user     user         1024 Июн 11 14:11 Lectures
-rw-rw-r--    1 user     user       123686 Июн 12 12:53 lectures.rar
-rw-rw-rw-    1 user     user         1789 Июн 11 15:44 lhr10.log
-rw-r--r--    1 user     user            0 Июн 11 15:44 lkypc.pdf
-rw-rw-r--    1 user     user       438112 Апр 20  1999 lshort.dvi
-rw-r--r--    1 user     user       112747 Июн 12 12:53 lshrtdvi.zip
drwxrwxr-x    2 user     user         1024 Июн 14 18:50 metoda
-rw-rw-r--    1 user     user        40960 Июн 11 14:06 metoda.doc
\end{verbatim} 

\end{itemize}

\subsection{Подключение других устройств(дисковода, CD-ROM)}

Чтобы подключить устройство (дисковод, CD-ROM, раздел вичестера, сетевой диск), нужно указать место, куда будет отображаться его содержимое. Это место называется \emph{точка монтирования}. Точка монтирования - это обычный каталог. После подключения (в терминах Unix - \emph{монтирования}) каталог будет содержать файлы, расположенные на устройстве. Список подключенных устройств и монтирование - команда \verb+mount(8)+.

Обычно поключаемые устройства отображаются на каталог \verb+/mnt+. Например \verb+/mnt/cdrom+, \verb+/mnt/floppy+.

Отключение (\emph{размонтирование}) производит отсоединение устройства от дерева каталогов. Команда \verb+umount(8)+.

\section{Копирование, удаление, перемещение файлов и каталогов}
\begin{itemize}
\item \verb+cp+ - копирование
\begin{verbatim}
копирование одного файла в другой
$ cp src.file /tmp/dest.file1

копирование нескольких файлов в другой каталог
$ cp *.tex metoda.dvi /mnt/floppy

копирование каталогов
$ cp -r metoda /archive/old.Docs
\end{verbatim}

\item \verb+rm+ - удаление файлов и \verb+rmdir+ - удаление каталогов\footnote{работает только для \emph{пустых} каталогов}
\begin{verbatim}
простое удаление
$ rm part1.aux intro.temp

удаление всех файлов из каталога 
$ rm /tmp/*

удаление каталога
$ rmdir oLD.stupid.dir
\end{verbatim}

\item \verb+mv+ - переместить(переименовать)
\begin{verbatim}
переместить один файл в другой
$ mv src.file /tmp/dest.file1

перемещение нескольких файлов в другой каталог
$ mv *.tex metoda.dvi /mnt/floppy

перемещение каталогов
$ mv metoda /archive/old.Docs
\end{verbatim}

\item \verb+mkdir+ - создать каталог
\begin{verbatim}
cоздать 1 или более каталогов
$ mkdir /tmp/dir.12345
$ mkdir empty.DIR DiRecTorY
\end{verbatim}
\end{itemize}

\section{Информация о системе}

\begin{itemize}
\item \verb+ps+ - список запущенных процессов
\item \verb+who+ - список пользователей, работающих в системе 
\item \verb+date+ - текущая дата и время
\item \verb+w+ - общая информация о системе
\end{itemize}
\begin{verbatim}
Примеры:
Процессы на текущей консоли
$ ps
 3642 pts/1    00:00:00 bash
 4548 pts/1    00:00:00 ps

Процессы конкретного пользователя
$ ps -u user
 1766 ?        00:00:04 xterm
 1853 pts/2    00:00:13 vim
 3642 pts/1    00:00:00 bash
 4553 pts/1    00:00:00 ps

Все процессы
$ ps -ef (для BSD и Linux - ps ax )
UID        PID  PPID  C STIME TTY          TIME CMD
user      1766  1176  0 06:45 ?        00:00:04 xterm -title Terminal
user      1768  1766  0 06:45 pts/2    00:00:00 bash
user      1853  1768  0 06:56 pts/2    00:00:13 vim part2.tex
	<пропущено - список 52 строки>

$ date
Сбт Июн 15 10:34:17 EEST 2002

$ who
root     tty1     Jun 15 10:24
work     tty2     Jun 15 10:24
user     pts/1    Jun 15 09:18

$ w
 10:35am  up  4:14,  3 users,  load average: 0.09, 0.04, 0.01
USER     TTY      FROM              LOGIN@   IDLE   JCPU   PCPU  WHAT
root     tty1     -                10:24am 11:25   0.02s  0.02s  -bash 
work     tty2     -                10:24am 11:11   0.03s  0.03s  -bash 
user     pts/1    -                 9:18am  1.00s  0.14s  0.01s  w 
\end{verbatim}

\section{Общение между пользователями}
\begin{itemize}
\item \verb+write пользователь+ - послать сообщение\footnote{ ctrl+d воспринимается системой как конец ввода}
\begin{verbatim}
$ write stud11
набрать текст сообщения
нажать <ctrl+d>
\end{verbatim}
\item \verb+talk пользователь+ - двухсторонний чат 
\item \verb+mail+ - электронная почта
\begin{verbatim}
Послать письмо
$  mail stud7 //локальному пользователю
Subject: test
набрать текст сообщения
нажать <ctrl+d>

$ mail user@tut.by //удаленному пользователю
.....

$ mail // просмотреть свою почту
.....
\end{verbatim}
\end{itemize}

\section{Просмотр, создание, объединение файлов}

\begin{itemize}
\item \verb+cat+ - вывод содержимого файла на экран 
\item \verb+more+ - разбиение входного файла на страницы
\item \verb+less+ - просмотр файлов (выход по клавише \verb+q+)
\end{itemize}

\begin{verbatim}
Примеры
Вывод на экран файла
$ cat file

Копирование файла
$ cat file >file2

Объединение 2-х файлов в третий
$ cat file1 file2 >end.file

Постраничный вывод файла на экран
$ cat large.file |more 

Ввод файла с клавиатуры
$ cat >new.text
<набирается текст>
<ctrl+d>

\end{verbatim}

\section{Объединение команд} 

Идеологически, Unix - это собрание большого количества небольших утилит, выполняющих какую-то одну локальную задачу. Но делающими свою задачу лучше всего, во всех возможных вариантах и режимах. 

\emph{Фильтры} - это программы, предназначенные для обработки текста тем или иным способом.

Часто они применяются в связке с другими командами, образуя \emph{конвейеры}. Конвейер обозначается символом \verb+|+. Его значение следующее: програма слева от конвейера передает свой вывод на вход программы справа от конвейера\footnote{конвейер может применяться несколько раз}. Простейший пример: \verb+$ cat myFILE|more+\footnote{То что выводит cat посылается на вход more. А more выступает здесь в качестве простого фильтра.}. В Unix этот метод называют \emph{перенаправление}. Подробнее о перенаправлении - в \ref{redirect}.

В общем случае, фильтры читают свой \emph{стандартный ввод} и пишут на \emph{стандартный вывод}. Если не было перенаправления, то вводом считается клавиатура, а выводом - экран.
\begin{center}
Часто употребимые фильтры
\end{center}
\begin{itemize}
\item \verb+grep+\footnote{Существует целое семейство grep-команд : grep, egrep, fgrep} - поиск в файле по образцу
\\ \verb+Пример: $ ls /usr/include | grep "stdlib.h" +
\item \verb+sort+ - сортировка содержимого
\item \verb+wc+ - статистика по файлу (кол-во букв, байт, строк и т.п.)
\item \verb+head+ и \verb+tail+ - показать начало(head) и конец(tail) файла
\item \verb+tee+ - одновременный вывод в файл и на экран
\\ \verb+Пример: $ ls -1 /tmp | tee all.tempfiles+
\end{itemize}

\section{Приемы эффективной работы}

Оболочка \verb+bash+ обладает 3 базовыми средствами автоматизации\footnote{Подробнее об автоматизации BASH - в \cite{Asp}, \cite{Pet}, man bash}, делающими работу в командной строке простой и легкой:
\begin{enumerate}
\item Автодополнение путей и команд

\emph{Использование}: набрать 1 или более начальных символов команды и нажать \verb+TAB+. Если символов хватает для определения команды, то недостающие будут добавлены автоматически. Если существует более 1 подходящей команды, то при повторном нажатиии \verb+TAB+ на экран высветиться список возможных команд. Можно тогда добавить несколько букв и однозначно определить команду.
\begin{verbatim}
$ la<TAB>
lambda      last        lastb       lastlog     latex       latex2html
$ last<TAB>
last     lastb    lastlog  
$ lastb<ENTER>
lastb: /var/log/btmp: No such file or directory
Perhaps this file was removed by the operator to prevent logging lastb info.
\end{verbatim}
Для поиска комманд используется переменная PATH (см. \ref{path}). 

Точно так же работает автодополнение для путей: вводиться кусочек пути и после нажатия <TAB> происходит дополнение пути.\\

\emph{Примечание}. Автодополнение не работает для ключей комманд (\verb+ls --he<TAB>+ - будет безрезультатным) и для аргументов комманды \verb+man+.
\item История команд

Для просмотра истории - \verb+history+. Для обращения к конкретному пункту истории - \verb+!номер+. Можно прокручивать историю с помощью клавиш \verb+"вверх"+ и \verb+"вниз"+. 
\item Редактирование командной строки

Возможно с помощью клавиш \verb+"вправо"+ и \verb+"влево"+. На начало строки - \verb-ctrl+a-, на конец - \verb-ctrl+e-.
\end{enumerate}
