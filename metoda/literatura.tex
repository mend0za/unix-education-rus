
\begin{thebibliography}{99}
\bibitem{Rob} А. Робачевский. Операционная система UNIX. Второе издание - BHV, 2005.
\bibitem{Nemeth} Э. Немет, Г. Снайдер и др. UNIX: Руководство системного администратора. Для профессионалов. 3-е изд. - BHV, 2002
\bibitem{Esr} Эрик Реймонд. Искусство программирования для Unix (The Art of Unix Programming). - Вильямс, 2005.
\bibitem{Glass} Грэм Гласс, Кинг Эйблс. Unix для программистов и пользователей. - BHV, 2004.
\bibitem{Moli} Б. Моли. Unix/Linux: теория и практика программирования. - Кудиц-образ, 2004.
\bibitem{Ker} Б. Керниган, Р. Пайк. UNIX. Программное окружение. Символ, 2003
\bibitem{Friedle} Дж. Фридл. Регулярные выражения. Библиотека программиста. - Питер, 2001. 
\bibitem{Robbins} Арнольд Роббинс. Linux. Программирование в примерах. - Кудиц-образ, 2005. 
\bibitem{Pet} К. Петцке. LINUX. От понимания к применению. - ДМК, 2000.
\bibitem{NetProgVol1} У. Стивенс.  UNIX: разработка сетевых приложений. - Питер, 2003.
\bibitem{NetProgVol2} У. Стивенс.  UNIX: взаимодействие процессов. - Питер, 2002.
\bibitem{Asp} ASPLinux. Руководство пользователя. - ASPLinux, 2001.
\bibitem{Bah} М. Бах. Архитектура операционной системы UNIX. http://www.lib.ru.
\bibitem{Sol} А. Соловьев. Sed и awk. Учебное пособие.
\bibitem{shell} Программирование на  shell (Unix).
\bibitem{bootle} Стен Келли-Бутл. Введение в Unix. - Лори, 1995.
\bibitem{Sage} Rassel S. Sage. Приемы профессиональной работы в UNIX. http://www.citforum.ru.
\bibitem{Mak} Дж. МакМален. UNIX. - Компьютер, 1996.
\bibitem{text} А. Шевель. Linux. Обработка текста. Специальный справочник. - Питер, 2001 
\bibitem{cvsbook} Free Software Foundation. CVS book. 1993-2004.
\bibitem{ProgPract} Б. Керниган, Р. Пайк. Практика программирования. - Вильямс, 2004

\end{thebibliography}
